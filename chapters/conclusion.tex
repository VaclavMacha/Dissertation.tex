\chapter*{Conclusion}
\addcontentsline{toc}{chapter}{Conclusion}

Binary classification is focused on classification performance on all samples. In Chapter~\ref{chap: binary classification}, we discussed many problems in which the performance on all samples is not desired and only performance on a small amount of samples is important. In this work, we studied the problem of classification at the top, which focuses on these specific problems. In the following list, we provide the main contributions of this work:
\begin{itemize}
  \item \textbf{Introduction of a unified framework for classification at the top:} In Chapter~\ref{chap: binary classification} we showed that many well-known categories of problems, such as ranking, accuracy at the top, or hypothesis testing, are closely related to classification at the top. This leads us to introduce a unified framework for classification at the top in Chapter~\ref{chap: framework}. We showed that several known formulations (\TopPush, \Grill, \tauFPL) fall into our framework and derived some completely new formulations (\PatMat, \PatMatNP). The summary of all presented formulations can be found in Table~\ref{tab: summary formulations}.
  \item \textbf{Introduction of \PatMat and \PatMatNP formulations:} In Chapter~\ref{chap: framework}, we introduced the \PatMat formulation as an alternative to the \TopMeanK formulation. These two formulations differ only in the decision threshold approximation. \PatMat formulation uses surrogate approximation of the true quantile while \TopMeanK approximates the true quantile using mean. We showed, that the threshold used in \PatMat  provides worse approximation of the true threshold, but has better theoretical properties. Similarly, we introduced the \PatMatNP formulation as an alternative to the \tauFPL formulation.
  \item \textbf{Derivation of theoretical properties for a linear classifier:} In Chapter~\ref{chap: linear}, we performed a theoretical analysis of the presented formulations when the linear model is used. We showed that known formulations suffer from certain disadvantages. \TopPush and \tauFPL are sensitive to outliers and \Grill is non-convex. On the other hand, we showed that newly introduced \PatMat and \PatMatNP formulations are both robust and convex. We also proved the global convergence of the stochastic gradient descent for \PatMat and \PatMatNP.
  \item \textbf{Derivation of dual forms and use of non-linear kernels:} In Chapter~\ref{chap: dual}, we showed that all presented formulations (except for \Grill and \GrillNP) can be divided into two families based on the form of the constraints, namely \TopPushK and \PatMat family of formulations. We derived dual forms for \TopPushK and \PatMat family of formulations. Moreover, for both these formulations we show how to incorporate non-linear kernels.
  \item \textbf{Derivation of an efficient algorithm for solving dual forms:} In Chapter~\ref{chap: dual}, we proposed a new coordinate descent algorithm for solving dual forms of \TopPushK and \PatMat family of formulations. The resulting algorithm depends on the used surrogate function. Therefore, we derived the closed-form formulae for selected surrogate functions. Since the algorithm needs a feasible solution for initialization, we also showed how to find such a solution.
  \item \textbf{Introduction of a modified stochastic gradient descent:} In Chapter~\ref{chap: deep}, we study the primal formulations with non-linear models. More precisely, we used neural networks. We showed that when we use a non-linear model, the resulting formulations are non-decomposable. This property is caused by the special threshold constraint in~\eqref{eq: aatp surrogate} and prevents us from using stochastic gradient descent in a standard way. We introduced modified stochastic gradient descent for our formulations. Unfortunatelly, we showed that modified stochastic gradient descent leads to a biased estimate of the true gradient.
  \item \textbf{Introduction of \DeepTopPush formulation:} As mentioned above, the proposed modified stochastic gradient descent leads to a biased estimate of the true gradient. In Chapter~\ref{chap: deep}, we suggested that this can be mitigated by using a large minibatch. However, such an approach is often not possible. For such cases, we proposed \DeepTopPush as an efficient alternative to \TopPush formulation that does not suffer from this issue. For \DeepTopPush, we implicitly removed some optimization variables, created an unconstrained end-to-end network, and used the stochastic gradient descent to train it. We modified the minibatch so that the sampled threshold (computed on a minibatch) is a good estimate of the true threshold (computed on all samples). We showed both theoretically and numerically that this procedure reduces the bias of the sampled gradient.
  \item \textbf{Numerical comparison:} In Chapter~\ref{chap: experiments}, we performed a numerical comparison of all presented formulations. We showed a good performance of our newly introduced formulation \PatMatNP, when used in its primal form. We also showed that \DeepTopPush formulation could be beneficial, especially for very large real-world datasets. Lastly, we demonstrated that standard formulations provide poor results at very low false-positive rates on steganalysis datasets and malware detection datasets, while the formulations proposed in this work outperform them.
\end{itemize}

\section*{Future Work}

The problem of classification at the top is well-defined for binary classification problems. However, there is no straightforward extension of classification at the top for multiclass problems. For instance, consider the problem of malware detection. In Chapter~\ref{chap: experiments}, we introduced malware detection as a problem in which we only want to decide whether some binary file is malware. However, there are plenty of different types of malware families. Therefore it would be better to formulate the problem as a multiclass classification problem, i.e., one class for clean software and one class for each malware family. Any standard multiclass classifier could solve this problem. However, these do not provide the option to maximize the number of correctly classified samples with a prescribed level of false alarms. The most straightforward approach would be to create a two-stage classification. In the first stage, we could use classification at the top to detect malware with the desired level of false-positive rate (false alarms). Then, in the second stage, we could use a standard multiclass classifier to find the family to which the detected malware belongs. The drawback of this approach is that we can only set a global level of false alarms instead of specific levels for each class. The logic is similar to the problem of accuracy at the top. However, further research would be needed since there is no straightforward way to extend accuracy at the top for the multiclass problem.
