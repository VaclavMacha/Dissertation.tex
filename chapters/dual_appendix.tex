\chapter{Appendix for Chapter~\ref{chap: dual}}

In this chapter we provide proofs and additional results for the Chapter~\ref{chap: dual}. In the first part, we introduce concept of conjugate functions. In the second part, we derive dual formulation to the formulations from Table~\ref{tab: summary formulations}. Finally, the last part focuses on how to efficiently solve these dual formulations.

\section{Convex Conjugate}
\begin{definition}[Convex conjugate~\cite{boyd2004convex}]\label{def: conjugate}
  Let~$l \colon \R^n \to \R.$ The function~$l^{\star} \colon \R^n \to \R,$ defined as
  \begin{equation*}
    l^{\star} (\bm{y})
      =  \sup_{\bm{x} \in \domain l} \{\bm{y}^{\top}\bm{x} - l(\bm{x})\}
      = -\inf_{\bm{x} \in \domain l} \{l(\bm{x}) - \bm{y}^{\top}\bm{x}\}.
  \end{equation*}
  is called conjugate function of~$l.$
\end{definition}
Recall the hinge loss and quadratic hinge loss function defined in Notation~\ref{not: surrogates} as follows
\begin{equation*}
  \begin{aligned}
    l_{\text{hinge}}(s) & = \max\Brac[c]{0, 1 + s}, \\
    l_{\text{quadratic}}(s) & = \Brac{\max\Brac[c]{0, 1 + s}}^2.\\
  \end{aligned}
\end{equation*}
The conjugate for the hinge loss can be found in~\cite{shnlev2014accelerated} and has the following form
\begin{equation}\label{eq:Conjugate of hinge loss}
  l_{\text{hinge}}^{\star}(y) =
  \begin{cases}
    -y & \text{if } y \in [0, 1], \\
    \infty & \text{otherwise.}
  \end{cases}  
\end{equation}
Similarly, the conjugate for the quadratic hinge is defuined in~\cite{kanamori2013conjugate} as
\begin{equation}\label{eq:Conjugate of truncated quadratic loss}
  l_{\text{quadratic}}^{\star}(y) =
  \begin{cases}
    \frac{y^2}{4} - y & \text{if } y \geq 0, \\
    \infty & \text{otherwise.}
  \end{cases}
\end{equation}

\section{Dual formulations}

In this section, we show how to derive the dual formulations to the formulations from Table Table~\ref{tab: summary formulations}. 

\subsection{Ranking Problems}

In this section, we derive the dual formulation of \TopPushK. Table~\ref{tab: summary formulations} shows, that \TopPush is a special is a special case of the \TopPushK for~$K = 1.$ Therefore, it is sufficient to show the dual form only for \TopPushK. Firstly, we introduce the alternative form of the \TopPushK.

\begin{lemma}[\TopPushK alternative formulation.]\label{lem: TopPushK primal alternative}
  The problem~\eqref{eq:TopPushK primal} can be equivalently written as follows
  \begin{maxi}{\bm{w}, t, \bm{y}, \bm{z}}{
    \frac{1}{2} \norm{\bm{w}}_{2}^{2}+ C \sum_{i = 1}^{\npos} l(y_i)
    }{\label{eq: TopPushK primal alternative}}{}
    \addConstraint{y_i}{= t + \frac{1}{K} \sum_{j = 1}^{\nneg} z_j - \bm{w}^\top \bm{x}^+_i, \quad}{i = 1, \; 2, \ldots, \; \npos.}
    \addConstraint{z_j}{\geq \bm{w}^\top \bm{x}^-_j - t,}{j = 1, \; 2, \ldots, \; \nneg}
    \addConstraint{z_j}{\geq 0,}{j = 1, \; 2, \ldots, \; \nneg}
  \end{maxi}
\end{lemma}
\begin{proof}
  Firstly, we rewrite the formula for the decision threshold from~\eqref{eq:TopPushK primal}using the Lemma~1 from~\cite{ogryczak2003minimizing}
  \begin{equation*}
    \sum_{j = 1}^{K} s^{-}_{[j]} = \min_{t} \Brac[c]{Kt + \sum_{j = 1}^{\nneg} \max\{0, \; s^-_j - t\}}.
  \end{equation*}
  Substituing this formula into the objective function from~\eqref{eq:TopPushK primal}, we get
  \begin{align*}
    \sum_{i = 1}^{\npos} l\Brac{\frac{1}{K}\sum_{j = 1}^{K} s^{-}_{[j]} - s^+_{i}}
      & = \sum_{i = 1}^{\npos} l\Brac{ \frac{1}{K} \min_{t} \Brac[c]{Kt + \sum_{j = 1}^{\nneg} \max\Brac[c]{0, \; s^-_j - t}} - s^+_{i}} \\
      & = \min_{t} \; \sum_{i = 1}^{\npos} l\Brac{t + \frac{1}{K} \sum_{j = 1}^{\nneg} \max\Brac[c]{0, \; s^-_j - t} - s^+_{i}}.
  \end{align*}
  where the last equality follows from the fact, that the surrogate function is~$l$ is non-decreasing. The max operator can be replaced using auxiliary variable~$\bm{z} \in \R^{\nneg}$ which for all~$j = 1, \; 2, \ldots, \; \nneg$ fullfills~$z _j \geq s^-_j - t$ and at the same time~$z _j \geq 0.$ Moreover, we introduce new variable~$\bm{y} \in \R^{\nneg}$ defined for all~$i = 1, \; 2, \ldots, \; \npos$ as
  \begin{equation*}
    y_i = t + \frac{1}{K} \sum_{j = 1}^{\nneg} z_j - s^+_i.
  \end{equation*}
  Altogether, we get the formulation~\eqref{eq: TopPushK primal alternative}, where we use the fact, that we have linear model and therefore~$s^-_j = \bm{w}^\top \bm{x}^-_j$ for all~$j = 1, \; 2, \ldots, \; \nneg$ and ~$s^+_i = \bm{w}^\top \bm{x}^+_i$ for all~$i = 1, \; 2, \ldots, \; \npos$.
\end{proof}

\pagebreak

\begin{theorem}[Dual formulation of \TopPush and \TopPushK ]\label{thm: TopPushK dual}
  Consider \TopPushK formulation~\eqref{eq: toppush surrogate} with linear model, surrogate function~$l$ and Notation~\ref{not: kernel matrix}. Then the corresponding dual problem has the following form
  \begin{maxi!}{\bm{\alpha}, \bm{\beta}}{
    - \frac{1}{2} \vecab^\top \Kneg \vecab - C \sum_{i = 1}^{\npos} l^{\star}\Brac{\frac{\alpha_i}{C}}
    }{\label{eq: TopPushK dual}}{\label{eq: TopPushK dual L}}
    \addConstraint{\sum_{i = 1}^{\npos} \alpha_i}{= \sum_{j = 1}^{\nneg} \beta_j \label{eq: TopPushK dual c1}}
    \addConstraint{0 \leq \beta_j}{\leq \frac{1}{K} \sum_{i = 1}^{\npos} \alpha_i, \quad j = 1, 2, \ldots, \nneg, \label{eq: TopPushK dual c2}}
  \end{maxi!}
  where~$l^{\star}$ is conjugate function of~$l.$ If~$K = 1,$ the upper bound in the second constrainet vanishes due to the first constraint and we get the dual form of \TopPush.
\end{theorem}
\begin{proof}
  In Lemma~\ref{lem: TopPushK primal alternative} we derived alternative fomrulation of \TopPushK with Lagrangian in the following form
  \begin{align*}
    \mathcal{L}(\bm{w}, t, \bm{y}, \bm{z}; \bm{\alpha}, \bm{\beta}, \bm{\gamma})
     & = \frac{1}{2} \norm{\bm{w}}_{2}^{2}
       + C \sum_{i = 1}^{\npos} l(y_i)
       + \sum_{i = 1}^{\npos} \alpha_i \Brac{t + \frac{1}{K} \sum_{j = 1}^{\nneg} z_j - \bm{w}^\top \bm{x}^+_i - y_i} \\
     & + \sum_{j = 1}^{\nneg} \beta_j \Brac{\bm{w}^\top \bm{x}^-_j - t - z_j}
       + \sum_{j = 1}^{\nneg} \gamma_j z_j,
  \end{align*}
  with feasibility conditions~$\beta_j \ge 0$ and~$\gamma_j \ge 0$ for all~$j = 1, \; 2, \ldots, \; \nneg.$ Then the corresponding dual objective function reads
  \begin{equation*}
    g(\bm{\alpha}, \bm{\beta}, \bm{\gamma})
      = \min_{\bm{w}, t, \bm{y}, \bm{z}} \; \mathcal{L}(\bm{w}, t, \bm{z}; \bm{\alpha}, \bm{\beta}, \bm{\gamma}),
  \end{equation*}
  Since the Lagrangian~$\mathcal{L}$ is separable in primal variables, it can be minimized with respect to each variable separately, i.e., the dual function can be rewritten as follows
  \begin{equation}\label{eq: TopPushK dual function}
    \begin{aligned}
      g(\bm{\alpha}, \bm{\beta}, \bm{\gamma})
        & = \min_{\bm{w}} \; \frac{1}{2} \norm{\bm{w}}_{2}^{2}
          - \bm{w}^{\top} \Brac{\sum_{i = 1}^{\npos} \alpha_i \bm{x}^+_i - \sum_{j = 1}^{\nneg} \beta_j \bm{x}^-_j} \\
        & + \min_{t} \; t \Brac{\sum_{i = 1}^{\npos} \alpha_i - \sum_{j = 1}^{\nneg} \beta_j} \\
        & + \min_{\bm{y}} \; C \sum_{i = 1}^{\npos} \Brac{l(y_i) - \frac{\alpha_i}{C}y_i} \\
        & + \min_{\bm{z}} \; \sum_{j = 1}^{\nneg} \Brac{\sum_{i = 1}^{\npos} \alpha_i - \beta_j - \gamma_j}z_j
    \end{aligned}
  \end{equation}
  From optimality conditions with respect to~$\bm{w}$ we deduce 
  \begin{equation*}
    \bm{w}
        = \sum_{i = 1}^{\npos} \alpha_i \bm{x}^+_i - \sum_{j = 1}^{\nneg} \beta_j \bm{x}^-_j
        = \Matrix{\X^+ \\ - \X^-}^\top \vecab,
  \end{equation*}
  where we use Notation~\ref{not: kernel matrix}. Using this relation, we get the first part of the objective function~\eqref{eq: TopPushK dual L} 
  \begin{equation*}
    \frac{1}{2} \norm{\bm{w}}_{2}^{2} - \bm{w}^{\top} \Brac{\sum_{i = 1}^{\npos} \alpha_i \bm{x}^+_i - \sum_{j = 1}^{\nneg} \beta_j \bm{x}^-_j}
      = - \frac{1}{2} \norm{\bm{w}}_{2}^{2}
      = - \frac{1}{2} \bm{w}^{\top} \bm{w}
      = - \frac{1}{2} \vecab^{\top} \Kneg \vecab,
  \end{equation*}
  where~$\Kneg$ is defined in Notation~\ref{not: kernel matrix}. Optimality condition with respect to~$t$ reads 
  \begin{equation*}
    \sum_{i = 1}^{\npos} \alpha_i - \sum_{j = 1}^{\nneg} \beta_j = 0,
  \end{equation*}
  and implies constrain in~\eqref{eq: TopPushK dual c1}. Similarly, Optimality condition with respect to~$\bm{z}$ reads for all $j = 1, \; 2, \ldots, \; \nneg$ as 
  \begin{equation*}
    \frac{1}{K} \sum_{i = 1}^{\npos} \alpha_i - \beta_j - \gamma_j = 0.
  \end{equation*}
  Plugging the feasibility condition~$\gamma_j \geq 0$ into this equality and combining it with the feasibility conditions~$\beta_j \geq 0$ yields constraint~\eqref{eq: TopPushK dual c2}. Finally, minimization of the Lagrangian with respect to~$\bm{y}$ yields for all $i = 1, \; 2, \ldots, \; \npos$ 
  \begin{equation*}
    C \min_{y_i} \Brac{l(y_i) - \frac{\alpha_i}{C} y_i} = - C l^{\star} \Brac{\frac{\alpha_i}{C}}.
  \end{equation*}
  where the equality follows from Definition~\ref{def: conjugate}. Plugging this back into the Lagrange function yields the second part of the objective function~\eqref{eq: TopPushK dual L}, which finishes the proof for \TopPushK. For \TopPush, we have~$K = 1.$ From~\eqref{eq: TopPushK dual c1} and non-negativity of~$\beta_j$ we deduce, that the upper bound in constraint~\eqref{eq: TopPushK dual c2} is always fullfiled and therefore can be ommited, which finishes the proof. 
\end{proof}

\subsection{Accuracy at the Top}

In Section~\ref{sec: aatp} we derived three problem formulations that fall into our framework~\eqref{eq: aatp surrogate}. Namely: \Grill, \TopMeanK and \PatMat. We focus only on \TopMeanK and \PatMat formulations, since as showed in Chapter~\ref{chap: linear}, these two formulations are convex for linear model.

\begin{theorem}[Dual formulation of \TopMeanK]\label{thm: TopMeanK dual}
  Consider \TopMeanK formulation~\eqref{eq: topmeank} with linear model, surrogate function~$l$ and Notation~\ref{not: kernel matrix}. Then the corresponding dual problem has the following form
  \begin{maxi*}{\bm{\alpha}, \bm{\beta}}{
    - \frac{1}{2} \vecab^\top \Kall \vecab - C \sum_{i = 1}^{\npos} l^{\star}\Brac{\frac{\alpha_i}{C}}
    }{}{}
    \addConstraint{\sum_{i = 1}^{\npos} \alpha_i}{= \sum_{j = 1}^{\nall} \beta_j}
    \addConstraint{0 \leq \beta_j}{\leq \frac{1}{K} \sum_{i = 1}^{\npos} \alpha_i, \quad j = 1, 2, \ldots, \nall,}
  \end{maxi*}
  where~$l^{\star}$ is conjugate function of~$l$ and~$K = \nall \tau.$
\end{theorem}
\begin{proof}
  \TopMeanK formulation is similar to the \TopPushK and therefore also dual formulations are similar. The main difference is, that the decision threshold for \TopMeanK is computed from all socres and not only from the negative ones as for \TopPushK. Due to that, the dual variable~$\bm{\beta}$ has different size and the kernel matrix has slightly different form as can be seen in Notation~\ref{not: kernel matrix}. Besides that dual formulations of \TopMeanK and \TopMeanK are identical and the proof of Theorem~\ref{thm: TopMeanK dual} is almost identical to the proof of Theorem~\ref{thm: TopPushK dual}.
\end{proof}

\begin{theorem}[Dual formulation of \PatMat]\label{thm: PatMat dual}
  Consider \PatMat formulation~\eqref{eq: patmat} with linear model, surrogate function~$l$ and Notation~\ref{not: kernel matrix}. Then the corresponding dual problem has the following form
  \begin{maxi!}{\bm{\alpha}, \bm{\beta}, \delta}{
    - \frac{1}{2} \vecab^\top \Kall \vecab - C \sum_{i = 1}^{\npos} l^{\star}\Brac{\frac{\alpha_i}{C}} - \delta \sum_{j = 1}^{\nall} l^{\star} \Brac{\frac{\beta_j}{\delta}} - \delta \nall \tau
    }{\label{eq: PatMat dual}}{\label{eq: PatMat dual L}}
    \addConstraint{\sum_{i = 1}^{\npos} \alpha_i}{= \sum_{j = 1}^{\nall} \beta_j \label{eq: PatMat dual c1}}
    \addConstraint{\delta }{\ge 0, \label{eq: PatMat dual c2}}
  \end{maxi!}
  where~$l^{\star}$ is conjugate function of~$l.$
\end{theorem}
\begin{proof}
  Let us first realize tha \PatMat formulation~\eqref{eq: patmat} with linear model is equivalent to
  \begin{mini*}{\bm{w}, t, \bm{y}, \bm{z}}{
    \frac{1}{2} \norm{\bm{w}}_{2}^{2}+ C \sum_{i = 1}^{\npos} l(y_i)
    }{}{}
    \addConstraint{\sum_{j = 1}^{\nall} l(z_i)}{\le \nall \tau}{}
    \addConstraint{y_i}{= t - \bm{w}^\top \bm{x}^+_i,}{i = 1, \; 2, \ldots, \; \npos.}
    \addConstraint{z_j}{= \bm{w}^\top \bm{x}_j - t, \quad}{j = 1, \; 2, \ldots, \; \nall}
  \end{mini*}
  Corresponding Lagrangian is in the following form
  \begin{align*}
    \mathcal{L}(\bm{w}, t, \bm{y}, \bm{z}; \bm{\alpha}, \bm{\beta}, \delta)
    & = \frac{1}{2} \norm{\bm{w}}_{2}^{2}
      + C \sum_{i = 1}^{\npos} l(y_i)
      + \sum_{i = 1}^{\npos} \alpha_i (t - \bm{w}^{\top}\bm{x}^+_{i} - y_i) \\
    & + \sum_{j = 1}^{\nall} \beta_j(\bm{w}^{\top}\bm{x}_j - t - z_j)
      + \delta \Brac{\sum_{j = 1}^{\nall} l(z_j) - \nall \tau}.
  \end{align*}
  with feasibility condition~$\delta \ge 0.$ Then the corresponding dual objective function reads
  \begin{equation*}
    g(\bm{\alpha}, \bm{\beta}, \delta)
      = \min_{\bm{w}, t, \bm{y}, \bm{z}} \; \mathcal{L}(\bm{w}, t, \bm{y}, \bm{z}; \bm{\alpha}, \bm{\beta}, \delta),
  \end{equation*}
  Since the Lagrangian~$\mathcal{L}$ is separable in primal variables, it can be minimized with respect to each variable separately, i.e., the dual function can be rewritten as follows
  \begin{align*}
    g(\bm{\alpha}, \bm{\beta}, \delta)
      & = \min_{\bm{w}} \; \frac{1}{2} \norm{\bm{w}}_{2}^{2}
        - \bm{w}^{\top} \Brac{\sum_{i = 1}^{\npos} \alpha_i \bm{x}^+_i - \sum_{j = 1}^{\nall} \beta_j \bm{x}_j} \\
      & + \min_{t} \; t \Brac{\sum_{i = 1}^{\npos} \alpha_i - \sum_{j = 1}^{\nall} \beta_j} \\
      & + \min_{\bm{y}} \; C \sum_{i = 1}^{\npos} \Brac{l(y_i) - \frac{\alpha_i}{C}y_i} \\
      & + \min_{\bm{z}} \; \delta \sum_{j = 1}^{\nall} \Brac{l(z_j) - \frac{\beta_j}{\delta}z_j} \\
      & - \delta \nall \tau.
  \end{align*}
  Note that resulting dual function is very similar to the dual function~\eqref{eq: TopPushK dual function} for \TopPushK, i.e. minimization of the Lagrangian with respect to~$\bm{w}$,~$t$ and~$\bm{y}$ yields similar results. From optimality conditions with respect to~$\bm{w}$ we deduce 
  \begin{equation*}
    \bm{w}
        = \sum_{i = 1}^{\npos} \alpha_i \bm{x}^+_i - \sum_{j = 1}^{\nall} \beta_j \bm{x}_j
        = \Matrix{\X^+ \\ - \X}^\top \vecab,
  \end{equation*}
  where we use Notation~\ref{not: kernel matrix}. Using this relation, we get the first part of the objective function~\eqref{eq: PatMat dual L} 
  \begin{equation*}
    \frac{1}{2} \norm{\bm{w}}_{2}^{2} - \bm{w}^{\top} \Brac{\sum_{i = 1}^{\npos} \alpha_i \bm{x}^+_i - \sum_{j = 1}^{\nall} \beta_j \bm{x}_j}
      = - \frac{1}{2} \norm{\bm{w}}_{2}^{2}
      = - \frac{1}{2} \bm{w}^{\top} \bm{w}
      = - \frac{1}{2} \vecab^{\top} \Kall \vecab,
  \end{equation*}
  where~$\Kall$ is defined in Notation~\ref{not: kernel matrix}. Optimality condition with respect to~$t$ reads 
  \begin{equation*}
    \sum_{i = 1}^{\npos} \alpha_i - \sum_{j = 1}^{\nall} \beta_j = 0,
  \end{equation*}
  and implies constrain in~\eqref{eq: PatMat dual c1}. The optimality condition with respect to~$\bm{y}$ is identical to the one in the proof of Theorem~\ref{thm: TopPushK dual}. Finally, inimization of the Lagrangian with respect to~$\bm{z}$ yields for all $j = 1, \; 2, \ldots, \; \nall$ 
  \begin{equation*}
    \delta \min_{\bm{z}} \; \Brac{l(z_j) - \frac{\beta_j}{\delta}z_j} = - \delta l^{\star} \Brac{\frac{\beta_i}{\delta}},
  \end{equation*}
  where the equality follows from Definition~\ref{def: conjugate}. Plugging this back into the Lagrange function yields the second part of the objective function~\eqref{eq: PatMat dual L}, which finishes the proof.
\end{proof}

\subsection{Hypothesis Testing}

In Section~\ref{sec: Neyman-Pearson} we derived three problem formulations that fall into our framework~\eqref{eq: aatp surrogate}. Namely: \GrillNP, \tauFPL and \PatMatNP. Similarly to the previous section, we focus only on \tauFPL and \PatMatNP. Since \tauFPL is a special case of \TopPushK for~$K = \nneg \tau,$ the dual formulation is identical to the one in~\ref{thm: TopPushK dual}.

\begin{theorem}[Dual formulation of \PatMatNP]\label{thm: PatMatNP dual}
  Consider \PatMatNP formulation~\eqref{eq: patmat np} with linear model, surrogate function~$l$ and Notation~\ref{not: kernel matrix}. Then the corresponding dual problem has the following form
  \begin{maxi*}{\bm{\alpha}, \bm{\beta}, \delta}{
    - \frac{1}{2} \vecab^\top \Kneg \vecab - C \sum_{i = 1}^{\npos} l^{\star}\Brac{\frac{\alpha_i}{C}} - \delta \sum_{j = 1}^{\nneg} l^{\star} \Brac{\frac{\beta_j}{\delta}} - \delta \nneg \tau
    }{}{}
    \addConstraint{\sum_{i = 1}^{\npos} \alpha_i}{= \sum_{j = 1}^{\nneg} \beta_j}
    \addConstraint{\delta }{\ge 0,}
  \end{maxi*}
  where~$l^{\star}$ is conjugate function of~$l.$
\end{theorem}
\begin{proof}
  \PatMatNP formulation is similar to the \PatMat and therefore also dual formulations are similar. The main difference is, that the decision threshold for \PatMatNP is computed from all socres and not only from the negative ones as for \PatMat. Due to that, the dual variable~$\bm{\beta}$ has different size and the kernel matrix has slightly different form as can be seen in Notation~\ref{not: kernel matrix}. Besides that dual formulations of \PatMatNP and \PatMat are identical and the proof of Theorem~\ref{thm: PatMatNP dual} is almost identical to the proof of Theorem~\ref{thm: PatMat dual}.
\end{proof}

\subsection{Summary}

In previous sections we derived dual formulations of formulations from Table~\ref{tab: summary formulations}. We showed that dual formulations of \TopPush, \TopPushK, \TopMeanK and \tauFPL are very similary and can be written in general form summarized in Theorem~\ref{thm: Top dual}. Similarly, dual formulations of \PatMat and \PatMatNP are very similary and can be written in general form summarized in Theorem~\ref{thm: Pat dual}

\section{Computing~$\Delta^{*}$ with truncated quadratic surrogate}\label{sec:Computing Delta with truncated quadratic surrogate}

\subsection{\TopPushK}

\toppushkupdatequadratic*
\begin{proof}[Proof of Theorem~\ref{thm:Update rule TopPushK with quadratic loss} on page~\pageref{thm:Update rule TopPushK with quadratic loss}]
  We will show that for each update rule~\eqref{eq:Update rules} and for fixed~$\bm{\alpha},$~$\bm{\beta}$, problem~\eqref{eq:TopPushK dual quadratic} can be rewritten as a quadratic one-dimensional problem
  \begin{align*}
    \maximize{\Delta}
    & -\frac{1}{2} a(\bm{\alpha}, \bm{\beta}) \Delta^2 - b(\bm{\alpha}, \bm{\beta}) \Delta - c(\bm{\alpha}, \bm{\beta}) \\
    \st
    & \Delta_{lb}(\bm{\alpha}, \bm{\beta}) \leq \Delta \leq \Delta_{ub}(\bm{\alpha}, \bm{\beta}).
  \end{align*}
  where~$a,$~$b,$~$c,$~$\Delta_{lb},$~$\Delta_{ub}$ do not depend on~$\Delta.$ The optimal solution to this problem is
  \begin{equation}\label{eq:Delta_optimal}
    \Delta^* = \clip{\Delta_{lb}}{\Delta_{ub}}{-\frac{b}{a}},
  \end{equation}
  which amounts to~\eqref{eq:Optimal delta}. Before discussing the three update rules~\eqref{eq:Update rules}, we realize that~\eqref{eq:TopPushK dual quadratic constraint 1} is always satisfied after the update. For the three updates we have:
  \begin{itemize}
    \item For update rule~\eqref{eq:Update rule a,a} with~$1\le k, l \le \npos$, constraint~\eqref{eq:TopPushK dual quadratic constraint 3} is satisfied since no~$\beta_j$ was updated and the sum of all~$\alpha_i$ did not change. Constraint~\eqref{eq:TopPushK dual quadratic constraint 2} reads~$-\alpha_k \leq \Delta \leq \alpha_l$ and objective~\eqref{eq:TopPushK dual quadratic objective} can be rewritten as
    \begin{equation*}
      - \frac{1}{2} \Brac[s]{\K_{kk} + \K_{ll} - \K_{kl} - \K_{lk} + \frac{1}{C\vartheta^2}} \Delta^2 - \Brac[s]{s_k - s_l + \frac{1}{2C\vartheta^2}(\alpha_k - \alpha_l)} \Delta + c(\bm{\alpha}, \bm{\beta}).
    \end{equation*}

    \item For update rule~\eqref{eq:Update rule a,b} with~$1 \le k \le \npos$ and~$\npos + 1 \le l \le \npos + \nneg$ we define~$\hat{l} = l - \npos.$ Constraint~\eqref{eq:TopPushK dual quadratic constraint 2} reads~$\Delta \geq - \alpha_k.$ Denoting~$\beta_{\max} = \max_{j \in \{1, 2, \ldots, \nneg \} \setminus \{\hat l\}} \beta_j,$ then for any~$K \geq 2$ constraint~\eqref{eq:TopPushK dual quadratic constraint 3} reads
    \begin{equation}\label{eq: TopPushK dual quadratic a,b - bounds}
      \begin{aligned}
        0 & \leq \beta_{\hat{l}} + \Delta \leq \frac{1}{K} \sum_{i = 1}^{\npos} \alpha_i + \frac{1}{K} \Delta, \\
        0 & \leq \beta_{\max} \leq \frac{1}{K} \sum_{i = 1}^{\npos} \alpha_i + \frac{1}{K} \Delta.
      \end{aligned}
    \end{equation}
    If~$K = 1,$ the upper bounds for~$\beta_j$ may be omitted as discussed in Section~\ref{sec:Computing Delta for TopPushK with truncated quadratic loss}. Combining this with~$\Delta \geq - \alpha_k$ yields the lower and upper bound of~$\Delta.$ Using update rule~\eqref{eq:Update rule a,b}, objective~\eqref{eq:TopPushK dual quadratic objective} can be rewritten as
    \begin{equation*}
      - \frac{1}{2} \Brac[s]{\K_{kk} + \K_{ll} + \K_{kl} + \K_{lk} + \frac{1}{2C\vartheta^2}} \Delta^2 - \Brac[s]{s_k + s_l - \frac{1}{\vartheta} + \frac{1}{2C\vartheta^2} \alpha_k} \Delta + c(\bm{\alpha}, \bm{\beta}).
    \end{equation*}

    \item For update rule~\eqref{eq:Update rule b,b} with~$\npos + 1\le k,l \le \npos + \nneg$ we define~$\hat{k} = k - \npos,$~$\hat{l} = l - \npos.$ Since no~$\alpha_i$ was updated, constraint~\eqref{eq:TopPushK dual quadratic constraint 2} is always satisfied. Moreover, since we update only two coordinates of~$\bm{\beta},$ constraint~\eqref{eq:TopPushK dual quadratic constraint 3} for any~$K \geq 2$ reads
    \begin{equation}\label{eq: TopPushK dual quadratic b,b - bounds}
      \begin{aligned}
        0 \leq \beta_{\hat{k}} + \Delta \leq \frac{1}{K} \sum_{i = 1}^{\npos} \alpha_i, \\
        0 \leq \beta_{\hat{l}} - \Delta \leq \frac{1}{K} \sum_{i = 1}^{\npos} \alpha_i,
      \end{aligned}
    \end{equation}
    As in the previous case, the upper bounds for~$\beta_j$ may be omitted for~$K = 1$. Combining the previous results yields the lower and upper bound of~$\Delta.$ Using update rule~\eqref{eq:Update rule b,b}, objective~\eqref{eq:TopPushK dual quadratic objective} can be rewritten as
    \begin{equation*}
      - \frac{1}{2} \Brac[s]{\K_{kk} + \K_{ll} - \K_{kl} - \K_{lk}} \Delta^2 - \Brac[s]{s_k - s_l} \Delta + c(\bm{\alpha}, \bm{\beta}).
    \end{equation*}
  \end{itemize}
  The proofs follows by plugging these cases into the solution~\eqref{eq:Delta_optimal}.
\end{proof}


\subsection{\PatMat}

Plugging the conjugate~\eqref{eq:Conjugate of truncated quadratic loss} of the truncated quadratic loss~\eqref{eq:Truncated quadratic loss} into \PatMat  dual formulation~\eqref{eq: PatMatNP dual} yields
\begin{subequations}\label{eq:PatMat dual quadratic}
  \begin{alignat}{2}
    \maximize{\bm{\alpha}, \bm{\beta}, \delta}
      & -\frac{1}{2} \Matrix{\bm{\alpha} \\ \bm{\beta}}^\top \K \Matrix{\bm{\alpha} \\ \bm{\beta}} + \frac{1}{\vartheta_1} \sum_{i = 1}^{\npos} \alpha_i - \frac{1}{4 C \vartheta_1^2} \sum_{i = 1}^{\npos} \alpha_i^2 \span \span \notag \\
      & + \frac{1}{\vartheta_2} \sum_{j = 1}^{\nneg} \beta_j - \frac{1}{4 \delta \vartheta_2^2} \sum_{j = 1}^{\nneg} \beta_j^2 - \delta n \tau \span \span \label{eq:PatMat dual quadratic objective} \\
    \st 
      & \sum_{i = 1}^{\npos} \alpha_i = \sum_{j = 1}^{\nneg} \beta_j, \label{eq:PatMat dual quadratic constraint 1} \\
      & \alpha_i \geq 0, && \quad \forall i = 1, 2, \ldots, \npos, \label{eq:PatMat dual quadratic constraint 2} \\
      & \beta_j \geq 0,  && \quad \forall j = 1, 2, \ldots, \nneg, \label{eq:PatMat dual quadratic constraint 3} \\
      & \delta \geq 0, \label{eq:PatMat dual quadratic constraint 4}
  \end{alignat}
\end{subequations}
The following theorem provides a formula for the optimal step~$\Delta^\star$ for the update rule~\eqref{eq:Update rules}. Note that we do not perform a joint minimization in~$(\alpha_k,\beta_l,\delta)$ but perform a minimization with respect to~$(\alpha_k,\beta_l)$, update these two values and then optimize the objective with respect to~$\delta$. 

\begin{theorem}[Update rule for~$\Delta^*$ for \PatMat  with truncated quadratic loss]\label{thm:Update rule PatMat with quadratic loss}
  Consider problem~\eqref{eq:PatMat dual quadratic}. Then the optimal step~$\Delta^\star$ equals to
  \begin{equation}\label{eq:Delta_optimal2}
    \Delta^{*} = \clip{\Delta_{lb}}{\Delta_{ub}}{\gamma},
  \end{equation}
  where there are the following three cases (each correspoding to one update rule in~\eqref{eq:Update rules}):
  \begin{itemize}
    \item If~$1\le k, l \le \npos$, then we have
    \begin{align*}
      \Delta_{lb} & = -\alpha_k, \\
      \Delta_{ub} & = \alpha_l, \\
      \gamma      & = -\frac{s_k - s_l + \frac{\alpha_k - \alpha_l}{2C\vartheta_1^2}}{\K_{kk} + \K_{ll} - \K_{kl} - \K_{lk} + \frac{1}{C\vartheta_1^2}}, \\
      \delta^{*}  & = \delta.
    \end{align*}
    \item If~$1 \le k \le \npos$ and~$\npos + 1 \le l \le n$, then defining~$\hat{l} = l - \npos$ we have
    \begin{align*}
      \Delta_{lb} & = \max\{- \alpha_k, - \beta_{\hat{l}}\}, \\
      \Delta_{ub} & = +\infty, \\
      \gamma      & = -\frac{s_k + s_l  - \frac{1}{\vartheta_1} + \frac{\alpha_k}{2C\vartheta_1^2} - \frac{1}{\vartheta_2} + \frac{\beta_{\hat{l}}}{2\delta\vartheta_2^2}}{\K_{kk} + \K_{ll} + \K_{kl} + \K_{lk} + \frac{1}{2C\vartheta_1^2} + \frac{1}{2\delta\vartheta_2^2}}, \\
      \delta^{*}  & = \sqrt{\delta^2 + \frac{1}{4 \vartheta_2 n \tau}({\Delta^{*}}^2 + 2 \Delta^{*} \beta_{\hat{l}})}.
    \end{align*}
    \item If~$\npos + 1\le k,l \le n$, then defining~$\hat{k} = k - \npos,$~$\hat{l} = l - \npos$ we have
    \begin{align*}
      \Delta_{lb} & = - \beta_{\hat{k}}, \\
      \Delta_{ub} & = \beta_{\hat{l}}, \\
      \gamma      & = -\frac{s_k - s_l + \frac{\beta_{\hat{k}} - \beta_{\hat{l}}}{2\delta \vartheta_2^2}}{\K_{kk} + \K_{ll} - \K_{kl} - \K_{lk} + \frac{1}{\delta \vartheta_2^2}}, \\
      \delta^{*}  & = \sqrt{\delta^2 + \frac{1}{2 \vartheta_2 n \tau}({\Delta^{*}}^2 + \Delta^{*} (\beta_{\hat{k}} - \beta_{\hat{l}}))}.
    \end{align*}
  \end{itemize}
\end{theorem}

\begin{proof}
  In the beginning of this subsection we derived problem~\eqref{eq:PatMat dual quadratic}. As in the proof of Theorem~\ref{thm:Update rule TopPushK with quadratic loss}, we show, that for each of update rules~\eqref{eq:Update rules} and for fixed~$\bm{\alpha},$~$\bm{\beta},$~$\delta,$ this problem can be rewritten as a simple one-dimensional quadratic problem with bound constraints. In this case, however, we have to also consider the third primal variable~$\delta.$ For fixed~$\bm{\alpha}$ and~$\bm{\beta},$, maximizing objective function~\eqref{eq:PatMat dual quadratic objective} with respect to~$\delta$ leads to the
  \begin{align*}
    \maximize{\delta}
      & - (n\tau) \delta - \Brac{\frac{1}{4\vartheta_2^2} \sum_{j = 1}^{\nneg} \beta_j^2} \frac{1}{\delta}, \\
    \st
      & \delta \geq 0.
  \end{align*}
  The solution of this problem equals to
  \begin{equation}\label{eq:PatMat dual quadratic optimal delta}
    \delta^* = \sqrt{\frac{1}{4\vartheta_2^2 n \tau} \sum_{j = 1}^{n} \beta_j^2}.
  \end{equation}
  In the following list, we discuss each of update rules~\eqref{eq:Update rules}:
  \begin{itemize}
    \item For update rule~\eqref{eq:Update rule a,a} and any~$1\le k, l \le \npos$, constraint~\eqref{eq:PatMat dual quadratic constraint 3} is satisfied since no~$\beta_j$ was updated. Constraint~\eqref{eq:PatMat dual quadratic constraint 2} reads~$-\alpha_k \leq \Delta \leq \alpha_l$ while objective~\eqref{eq:PatMat dual quadratic objective} can be rewritten as
    \begin{equation*}
      - \frac{1}{2} \Brac[s]{\K_{kk} + \K_{ll} - \K_{kl} - \K_{lk} + \frac{1}{C\vartheta_1^2}} \Delta^2 - \Brac[s]{s_k - s_l + \frac{1}{2C\vartheta_1^2}(\alpha_k - \alpha_l)} \Delta + c(\bm{\alpha}, \bm{\beta}).
    \end{equation*}
    Since optimal~$\delta$ is given by~\eqref{eq:PatMat dual quadratic optimal delta} and no~$\beta_j$ was updated, the optimal~$\delta$ does not change.

    \item For update rule~\eqref{eq:Update rule a,b} with~$1 \le k \le \npos$ and~$\npos + 1 \le l \le n$ we define~$\hat{l} = l - \npos.$ In this case, constraints~(\ref{eq:PatMat dual quadratic constraint 2},\ref{eq:PatMat dual quadratic constraint 3}) can be written in a simple form~$\Delta \geq \max \{- \alpha_k, - \beta_{\hat{l}}\}$ and~$\Delta$ has no upper bound. Objective~\eqref{eq:PatMat dual quadratic objective} can be rewritten as
    \begin{equation*}
      \begin{split}
        - \frac{1}{2} \Brac[s]{\K_{kk} + \K_{ll} + \K_{kl} + \K_{lk} + \frac{1}{2C\vartheta_1^2} + \frac{1}{2\delta\vartheta_2^2}} \Delta^2 \dots \qquad\qquad \\ 
        - \Brac[s]{s_k + s_l - \frac{1}{\vartheta_1} - \frac{1}{\vartheta_2} + \frac{\alpha_k}{2C\vartheta_1^2} + \frac{\beta_{\hat l}}{2\delta\vartheta_2^2}} \Delta + c(\bm{\alpha}, \bm{\beta}).
      \end{split}
    \end{equation*}
    We know that the optimal~$\delta^*$ is given by~\eqref{eq:PatMat dual quadratic optimal delta}, then
    \begin{equation*}
      \delta^*
      = \sqrt{\frac{1}{4\vartheta_2^2 n \tau} \Brac{\sum_{j\neq \hat l} \beta_j^2 + (\beta_{\hat{l}} + \Delta^\star)^2}}
      = \sqrt{\delta^2 + \frac{1}{4\vartheta_2^2 n \tau} (\Delta^{\star2} + 2\Delta^\star \beta_{\hat{l}})}.
    \end{equation*}

    \item For update rule~\eqref{eq:Update rule b,b} with~$\npos + 1\le k,l \le \npos + \nneg$ we define~$\hat{k} = k - \npos,$~$\hat{l} = l - \npos.$ Since no~$\alpha_i$ was updated, constraint~\eqref{eq:PatMat dual quadratic constraint 2} is always satisfied. Constraint~\eqref{eq:PatMat dual quadratic constraint 3} can be written in a simple form~$-\beta_{\hat{k}} \leq \Delta \leq \beta_{\hat{l}}$ and objective~\eqref{eq:PatMat dual quadratic objective} can be rewritten as
    \begin{equation*}
      - \frac{1}{2} \Brac[s]{\K_{kk} + \K_{ll} - \K_{kl} - \K_{lk} + \frac{1}{2\delta\vartheta_2^2}} \Delta^2 - \Brac[s]{s_k - s_l + \frac{\beta_{\hat{k}} - \beta_{\hat{l}}}{\delta\vartheta_2^2}} \Delta + c(\bm{\alpha}, \bm{\beta}).
    \end{equation*}
    We know that the optimal~$\delta^*$ is given by~\eqref{eq:PatMat dual quadratic optimal delta}, then
    \begin{equation*}
      \delta^*
      = \sqrt{\frac{1}{4\vartheta_2^2 n \tau} \Brac{\sum_{j \notin \{\hat{l}, \hat{k}\}} \beta_j^2 + (\beta_{\hat{k}} + \Delta^\star)^2 + (\beta_{\hat{l}} - \Delta^\star)^2}} 
      = \sqrt{\delta + \frac{1}{2\vartheta_2^2 n \tau} (\Delta^{\star2} + \Delta^\star (\beta_{\hat{k}} - \beta_{\hat{l}}))}.
    \end{equation*}
  \end{itemize}
  The proofs follows by plugging these cases into the solution~\eqref{eq:Delta_optimal2}.
\end{proof}

\section{Computing~$\Delta^{*}$ with hinge loss function}\label{sec:Computing Delta with hinge surrogate}

In this section, we provide the results when the truncated quadratic surrogate is replaced by the hinge surrogate. Since the proofs are identical, we omit them.

\subsection{\TopPushK}

Plugging the conjugate~\eqref{eq:Conjugate of hinge loss} of the hinge loss~\eqref{eq:Hinge loss} into \TopPushK dual formulation~\eqref{eq: TopPushK dual} yields
\begin{subequations}\label{eq:TopPushK dual hinge}
  \begin{alignat}{2}
    \maximize{\bm{\alpha}, \bm{\beta}}
    & - \frac{1}{2} \Matrix{\bm{\alpha} \\ \bm{\beta}}^\top \K \Matrix{\bm{\alpha} \\ \bm{\beta}} + \frac{1}{\vartheta} \sum_{i = 1}^{\npos} \alpha_i \span \span \label{eq:TopPushK dual hinge objective} \\
    \st 
    & \sum_{i = 1}^{\npos} \alpha_i = \sum_{j = 1}^{\nneg} \beta_j, \label{eq:TopPushK dual hinge constraint 1} \\
    & 0 \leq \alpha_i \leq C \vartheta, && \quad \forall i = 1, 2, \ldots, \npos, \label{eq:TopPushK dual hinge constraint 2} \\
    & 0 \leq \beta_j  \leq \frac{1}{K} \sum_{i = 1}^{\npos} \alpha_i, && \quad \forall j = 1, 2, \ldots, \nneg. \label{eq:TopPushK dual hinge constraint 3}
  \end{alignat}
\end{subequations}
This is a convex quadratic problem. Moreover, for~$K = 1,$ the upper limit in~\eqref{eq:TopPushK dual hinge constraint 3} is always satisfied due to~\eqref{eq:TopPushK dual hinge constraint 1} and the problem can be simplified. The following theorem provides a formula for optimal~$\Delta$ for each of update rules~\eqref{eq:Update rules}.

\begin{theorem}[Update rule for~$\Delta^*$ for \TopPushK with hinge loss]\label{thm:Update rule TopPushK with hinge loss}
  Consider problem~\eqref{eq:TopPushK dual hinge}. Then
  \begin{equation*}
    \Delta^{*} = \clip{\Delta_{lb}}{\Delta_{ub}}{\gamma},
  \end{equation*}
  where there are the following cases:
  \begin{itemize}
    \item For any~$1\le k, l \le \npos$ we have
    \begin{align*}
      \Delta_{lb} & = \min\{- \alpha_k,\; \alpha_l - C \vartheta \}, \\
      \Delta_{ub} & = \max\{C \vartheta - \alpha_k,\; \alpha_l \}, \\
      \gamma      & = -\frac{s_k - s_l}{\K_{kk} + \K_{ll} - \K_{kl} - \K_{lk}}.
    \end{align*}
    \item For any~$1 \le k \le \npos$ and~$\npos + 1 \le l \le n$ we define~$\hat{l} = l - \npos$ and~$\beta_{\max} = \max_{j \in \{1, 2, \ldots, \nneg \} \setminus \{\hat l\}} \beta_j.$ Then we have
    \begin{align*}
      \Delta_{lb} & = 
        \begin{cases*}
          \min \Brac[c]{- \alpha_k,\;  -\beta_{\hat{l}}} & K = 1, \\
          \min \Brac[c]{- \alpha_k,\;  -\beta_{\hat{l}}, \; K\beta_{\max} - \sum_{i = 1}^{\npos} \alpha_i} & \textrm{otherwise},
        \end{cases*} \\
      \Delta_{ub} & = 
        \begin{cases*}
            C \vartheta - \alpha_k & K = 1, \\
            \max \Brac[c]{C \vartheta - \alpha_k, \; \frac{1}{K-1}\Brac{\sum_{i = 1}^{\npos} \alpha_i - K \beta_{\hat{l}}}}  & \textrm{otherwise}.
        \end{cases*} \\
      \gamma & = - \frac{s_k + s_l - \frac{1}{\vartheta}}{\K_{kk} + \K_{ll} + \K_{kl} + \K_{lk}}.
    \end{align*}
    \item For any~$\npos + 1\le k,l \le \npos + \nneg$ we define~$\hat{k} = k - \npos,$~$\hat{l} = l - \npos$ and then we have
    \begin{align*}
      \Delta_{lb} & = 
        \begin{cases*}
          - \beta_{\hat{k}} & K = 1, \\
          \min \Brac[c]{- \beta_{\hat{k}},\; \beta_{\hat{l}} - \frac{1}{K} \sum_{i = 1}^{\npos} \alpha_i} & \textrm{otherwise},
        \end{cases*} \\
      \Delta_{ub} & = 
        \begin{cases*}
          \beta_{\hat{l}} & K = 1, \\
          \max \Brac[c]{\frac{1}{K} \sum_{i = 1}^{\npos} \alpha_i - \beta_{\hat{k}},\; \beta_{\hat{l}}} & \textrm{otherwise}.
        \end{cases*} \\
      \gamma & = -\frac{s_k - s_l}{\K_{kk} + \K_{ll} - \K_{kl} - \K_{lk}}.
    \end{align*}
  \end{itemize}
\end{theorem}

\subsection{\PatMat}

Plugging the conjugate~\eqref{eq:Conjugate of hinge loss} of the hinge loss~\eqref{eq:Hinge loss} into \PatMat  dual formulation~\eqref{eq: PatMatNP dual} yields
\begin{subequations}\label{eq:PatMat dual hinge}
  \begin{alignat}{2}
    \maximize{\bm{\alpha}, \bm{\beta}, \delta}
    & -\frac{1}{2} \Matrix{\bm{\alpha} \\ \bm{\beta}}^\top \K \Matrix{\bm{\alpha} \\ \bm{\beta}} + \frac{1}{\vartheta_1} \sum_{i = 1}^{\npos} \alpha_i + \frac{1}{\vartheta_2} \sum_{j = 1}^{n} \beta_j - \delta  n \tau \span \span \label{eq:PatMat dual hinge objective} \\
    \st 
    & \sum_{i = 1}^{\npos} \alpha_i = \sum_{j = 1}^{n} \beta_j, \label{eq:PatMat dual hinge constraint 1} \\
    & 0 \leq \alpha_i \leq C \vartheta_1, && \quad \forall i = 1, 2, \ldots, \npos, \label{eq:PatMat dual hinge constraint 2} \\
    & 0 \leq \beta_j  \leq \delta \vartheta_2, && \quad \forall j = 1, 2, \ldots, n, \label{eq:PatMat dual hinge constraint 3} \\
    & \delta \geq 0, \label{eq:PatMat dual hinge constraint 4}
  \end{alignat}
\end{subequations}
The following theorem provides a formula for optimal~$\Delta$ for each of update rules~\eqref{eq:Update rules}.

\begin{theorem}[Update rule for~$\Delta^*$ for \PatMat  with hinge loss]\label{thm:Update rule PatMat with hinge loss}
  Consider problem~\eqref{eq:PatMat dual hinge}. Then
  \begin{equation*}
      \Delta^{*} = \clip{\Delta_{lb}}{\Delta_{ub}}{\gamma},
  \end{equation*}
  where there are the following cases:
  \begin{itemize}
    \item For any~$1\le k, l \le \npos$ we have
    \begin{align*}
      \Delta_{lb} & = \min\{- \alpha_k,\; \alpha_l - C \vartheta_1\}, \\
      \Delta_{ub} & = \max\{C \vartheta_1 - \alpha_k,\; \alpha_l\}, \\
      \gamma      & = -\frac{s_k - s_l}{\K_{kk} + \K_{ll} - \K_{kl} - \K_{lk}}, \\
      \delta^{*}  & = \delta.
    \end{align*}

    \item For any~$1 \le k \le \npos$ and~$\npos + 1 \le l \le \npos$ we define~$\hat{l} = l - \npos$ and~$\beta_{\max} = \max_{j \in \{1, 2, \ldots, n\} \setminus \{\hat l\}} \beta_j.$ Then we have
    \begin{alignat*}{2}
      \Delta_{lb} & = \max\{- \alpha_k,\; -\beta_{\hat{l}} \}, & \qquad
      \Delta_{ub} & = C \vartheta_1 - \alpha_k
    \end{alignat*}
    and the optimal solution is one of the two following possibilities which maximizes the original objective:
    \begin{enumerate}
      \item If~$\beta_{\hat{l}} + \Delta^{*} \leq \beta_{\max}$, then
      \begin{align*}
        \gamma & = -\frac{s_k + s_l - \frac{1}{\vartheta_1} - \frac{1}{\vartheta_2}}{\K_{kk} + \K_{ll} + \K_{kl} + \K_{lk}}, \\
        \delta^{*} & = \frac{\beta_{\max}}{\vartheta_2}.
      \end{align*}
      \item If~$\beta_{\hat{l}} + \Delta^{*} \ge \beta_{\max}$, then
      \begin{align*}
        \gamma & = -\frac{s_k + s_l - \frac{1}{\vartheta_1} - \frac{1 - n\tau}{\vartheta_2}}{\K_{kk} + \K_{ll} + \K_{kl} + \K_{lk}}, \\
        \delta^{*} & = \frac{\beta_{\hat{l}} + \Delta^{*}}{\vartheta_2}.
      \end{align*}
    \end{enumerate}

    \item For any~$\npos + 1\le k,l \le n$ we define~$\hat{k} = k - \npos,$~$\hat{l} = l - \npos.$ Then we have
    \begin{alignat*}{2}
      \Delta_{lb} & = - \beta_{\hat{k}}, & \qquad
      \Delta_{ub} & = \beta_{\hat{l}},
    \end{alignat*}
    and the optimal solution is one of the three following possibilities which maximizes the original objective:
    \begin{enumerate}
      \item If~$\beta_{\max} \geq \max\{\beta_{\hat{k}} + \Delta^{*}, \beta_{\hat{l}} - \Delta^{*}\}$, then
      \begin{align*}
        \gamma     & = -\frac{s_k - s_l}{\K_{kk} + \K_{ll} - \K_{kl} - \K_{lk}}, \\
        \delta^{*} & = \frac{\beta_{\max}}{\vartheta_2}.
      \end{align*}
      \item If~$\beta_{\hat{k}} + \Delta^{*} \geq \max\{\beta_{\max} , \beta_{\hat{l}} - \Delta^{*}\}$, then
      \begin{align*}
        \gamma     & = -\frac{s_k - s_l + \frac{n\tau}{\vartheta_2}}{\K_{kk} + \K_{ll} - \K_{kl} - \K_{lk}}, \\
        \delta^{*} & = \frac{\beta_{\hat{k}} + \Delta}{\vartheta_2}.
      \end{align*}
      \item If~$\beta_{\hat{l}} - \Delta^{*} \geq \max\{\beta_{\hat{k}} + \Delta^{*}, \beta_{\max}\}$, then
      \begin{align*}
        \gamma     & = -\frac{s_k - s_l - \frac{n\tau}{\vartheta_2}}{\K_{kk} + \K_{ll} - \K_{kl} - \K_{lk}}, \\
        \delta^{*} & = \frac{\beta_{\hat{k}} - \Delta^{*}}{\vartheta_2}.
      \end{align*}
    \end{enumerate}
  \end{itemize}
\end{theorem}