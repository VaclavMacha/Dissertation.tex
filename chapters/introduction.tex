\chapter{Introduction}

The problem of data classification is very important mathematical problem. The goal of classification is to find a relation between a set of objects and a target variable based on some properties of the objects. The properties of the objects are usually called features. There are many problems in research as well as in the real world that can be formulated as classification tasks. We can find applications of data classification across all the fields:
\begin{itemize}
  \item \textbf{Medical Diagnonsis:} In medicine, the classification is often used to improve disease diagnosis. In such a case, the features are medical records such as the patient's blood tests, temperature, or roentgen images. The target variable is if the patient has some disease. As an example, classification is used to process mammogram images and detect cancer~\cite{viale2012current, levy2016breast}.
  \item \textbf{Internet Secutiry:} These days, the internet is a crucial part of our lives. With the increasing usage of the internet, the number of attacks increases as well. An essential part of the defense are intrusion detection systems~\cite{grill2016learning, scarfone2007guide} that search for malicious activities (network attacks) in network traffic. Classification can be used to improve such systems~\cite{giacinto2002intrusion, shanbhag2009accurate}.
  \item \textbf{Marketing:} In marketing, the task can be to classify customers based on their buying interests. Such information can be used to build a personalized recommendation system for customers and therefore increase income~\cite{kaefer2005neural, zhang2007building}.
\end{itemize}
Many other classification problems can be found in almost all fields. Also, there is a vast number of classification algorithms that try to solve these classifications problems. Typically these algorithms consist of two phases:
\begin{itemize}
  \item \textbf{Training Phase:} In the training phase, the algorithm uses training data to build a model. The classification algorithms fall into the category of supervised learning algorithms. It means, that these algorithms must have labeled training data to build the model, i.e. the algorithm must have the knowledge of the target classes. The training data typically consists of pairs (sample, label) and can be described as follows
  \begin{equation}\label{eq: training set}
    \mathcal{D}_{\mathrm{train}} = \Brac[c]{(\bm{x}_i, y_i)}_{i=1}^{n},
  \end{equation}
  where the sample~$\bm{x}_i \in \Xc$ is a vector of features that descibes the object of interes and the label~$y_i \in \{1, 2, \ldots, k\}$ represents target class. Moreover~$n \in \N$ is a number of training samples and~$k \in \N$ is a number of target classes.
  \item \textbf{Testing Phase:} In the testing phase, the trained model from the previous phase is used to assign labels~$\hat{y}_i \in \{1, 2, \ldots, k\}$ to the data from unlabeled testing data
  \begin{equation}\label{eq: test set}
    \mathcal{D}_{\mathrm{test}} = \Brac[c]{\bm{x}_i}_{i=1}^{m},
  \end{equation}
  where~$\bm{x}_i \in \Xc$ and~$m \in \N$ is a number of testing samples. The ultimate goal of all classification algorithms is to classify testing samples with the highest accuracy possible.
\end{itemize}
The previous definitions of training and test set are general for classification problems with multiple classes. However, the main focus of this work is on a special subclass of classification problems with only two target classes: binary classification.
