\documentclass[twoside, a4paper, 11pt]{book}
\usepackage[ddmmyyyy]{datetime}

% ------------------------------------------------------------------------------
% Packages
% ------------------------------------------------------------------------------
% Page setting
\usepackage[explicit]{titlesec}
\usepackage{sectsty}
\usepackage{fancyhdr}

% Text options
\usepackage{lmodern}
\usepackage[T1]{fontenc}
\usepackage[utf8]{inputenc}
\usepackage{xspace}

\usepackage{amsfonts}
\usepackage{dsfont}
\usepackage{pifont}

\usepackage[color=myred!50]{todonotes}

% Graphics and colors
\usepackage{graphicx}
\usepackage{import}
\usepackage{graphics}
\usepackage{xcolor}

\definecolor{myred}{RGB}{150,0,0}  
\definecolor{mygreen}{RGB}{0,150,0}
\definecolor{myblue}{RGB}{0, 101, 189}
\definecolor{myyellow}{RGB}{220, 206, 0}
\definecolor{myorange}{RGB}{255, 153, 51}
\definecolor{mycyan}{RGB}{51, 204, 204}
\definecolor{mypurple}{RGB}{204, 0, 153}

\newcommand{\doccol}{\color{myblue}}

% Hyperrefs
\usepackage{hyperref}
\hypersetup{
  pdfusetitle,
  unicode = true,
  bookmarks = true,
  bookmarksnumbered = false,
  bookmarksopen = true,
  breaklinks = false,
  pdfborderstyle = {},
  backref = false,
  colorlinks = true,
  linkcolor = myblue,
  urlcolor = myred,
  citecolor = mygreen,
}

% enumerate and itemize
\usepackage{enumitem}

% Appendix
\usepackage[title, titletoc]{appendix}

% Captions
\usepackage{caption}
\usepackage{subcaption}

\captionsetup[figure]{position = bottom}
\captionsetup[table]{position = bottom}

% Figures

% Tables
\usepackage{booktabs}
\usepackage{nicematrix}

\renewcommand{\arraystretch}{1.5}

% Algorithms
\usepackage{algorithm}
\usepackage{algorithmicx}
\usepackage{algpseudocode}

% Math
\usepackage{amsmath}
\usepackage{amsthm}
\usepackage{amssymb}
\usepackage{mathtools}
\usepackage{nicefrac}
\usepackage{bm}
\usepackage{thmtools}
\usepackage{thm-restate}
\usepackage{optidef}

% Theorems
\usepackage[framemethod=TikZ]{mdframed}
\usepackage{xifthen}

% Tikz and pfgplots
\usepackage{tikz}
\usepackage{pgfplots}
\usepackage{pgfplotstable}

\usetikzlibrary{shapes}
\usetikzlibrary{arrows}
\usetikzlibrary{automata}
\usetikzlibrary{positioning}
\usetikzlibrary{calc}
\usetikzlibrary{intersections}

\pgfplotsset{compat=newest}
\usepgfplotslibrary{groupplots}
\usepgfplotslibrary{fillbetween}

% ------------------------------------------------------------------------------
% Math declarations
% ------------------------------------------------------------------------------
\newcommand{\Brac}[2][r]{%
  \ifx r#1 \left(       #2 \right)       \else
  \ifx c#1 \left\{      #2 \right\}      \else
  \ifx s#1 \left[       #2 \right]       \else
  \ifx v#1 \left\vert   #2 \right\vert   \else
  \ifx a#1 \left\langle #2 \right\rangle \else
  \ifx t#1 \left\lceil  #2 \right\rceil  \else
  \ifx b#1 \left\lfloor #2 \right\rfloor \else
  \ifx n#1 \left\|      #2 \right\|      \else
  \mathrm{Illegal~option}%
  \fi\fi\fi\fi\fi\fi\fi\fi
}

\newcommand{\clip}[4][s]{
  \ifx s#1 \mathrm{clip}_{\Brac[s]{#2,\; #3}}\Brac{#4} \else
  \ifx u#1 \mathrm{clip}_{\left[#2,\; #3\right)}\Brac{#4} \else
  \ifx l#1 \mathrm{clip}_{\left(#2,\; #3\right]}\Brac{#4} \else
  \mathrm{Illegal~option}%
  \fi\fi\fi
}

\newcommand{\yesmark}{\textcolor{mygreen}{\ding{51}}}%
\newcommand{\nomark}{\textcolor{myred}{\ding{55}}}
\newcommand{\good}[1]{\textcolor{mygreen}{#1}}
\newcommand{\bad}[1]{\textcolor{myred}{#1}}

\newcommand{\R}{\mathbb{R}}
\newcommand{\N}{\mathbb{N}}
\newcommand{\X}{\mathbb{X}}
\newcommand{\Xc}{\mathcal{X}}
\newcommand{\D}{\mathcal{D}}

\newcommand{\I}{\mathcal{I}}
\newcommand{\Itil}{\tilde{\mathcal{I}}}
\newcommand{\Ineg}{\I_{-}}
\newcommand{\Ipos}{\I_{+}}

\newcommand{\Imb}{\I_{\text{mb}}}
\newcommand{\Imbneg}{\I_{\text{mb},-}}
\newcommand{\Imbpos}{\I_{\text{mb},+}}

\newcommand{\nall}{n}
\newcommand{\nneg}{n_{-}}
\newcommand{\npos}{n_{+}}
\newcommand{\ntil}{\tilde{n}}

\newcommand{\nmb}{n_{\text{mb}}}
\newcommand{\nmbneg}{n_{\text{mb},-}}
\newcommand{\nmbpos}{n_{\text{mb},+}}

\newcommand{\K}{\mathbb{K}}
\newcommand{\Kall}{\K^{\pm}}
\newcommand{\Kneg}{\K^{-}}

\newcommand{\updateaa}{\Delta_{\alpha,\alpha}}
\newcommand{\updateab}{\Delta_{\alpha,\beta}}
\newcommand{\updatebb}{\Delta_{\beta,\beta}}

\newcommand{\alphak}{\alpha_{\hat{k}}}
\newcommand{\alphal}{\alpha_{\hat{l}}}
\newcommand{\betak}{\beta_{\hat{k}}}
\newcommand{\betal}{\beta_{\hat{l}}}

\newcommand{\norm}[1]{\Brac[n]{#1}}
\newcommand{\abs}[1]{|#1|}
\newcommand{\inner}[2]{\Brac[a]{#1, \; #2}}
\newcommand{\dd}[1]{\mathop{}\!\mathrm{d}#1}
\newcommand{\minimize}[1]{\ifthenelse{\isempty{#1}}{\operatorname*{minimize}\quad}{\operatorname*{minimize}_{#1}\quad}}
\newcommand{\maximize}[1]{\ifthenelse{\isempty{#1}}{\operatorname*{maximize}\quad}{\operatorname*{maximize}_{#1}\quad}}
\newcommand{\st}{\operatorname{subject\ to}}
\newcommand{\argmin}{\operatorname*{argmin}}
\newcommand{\eps}{{\varepsilon}}

\newcommand{\Imin}{I_{\rm mb}}
\newcommand{\Iminh}{I_{\rm mb}^{\rm enh}}
\newcommand{\nmin}{n_{\rm mb}}
\newcommand{\II}{\mathds{1}}
\newcommand{\Iverson}[1]{\mathds{1}_{\Brac[s]{#1}}}

\newcommand{\EE}{\mathbb{E}}
\newcommand{\PP}{\mathbb{P}}
\newcommand{\bias}{\operatorname{bias}}

\newcommand{\Matrix}[1]{\begin{pmatrix} #1 \end{pmatrix}}
\newcommand{\Set}[2]{\Brac[c]{#1 \; \middle\vert \; #2}}
\newcommand{\domain}{\operatorname*{dom}}

\newcommand{\repeatloop}{\texttt{repeat}\xspace}
\newcommand{\forloop}{\texttt{for}\xspace}

\newcommand{\vecab}{\Matrix{\bm{\alpha} \\ \bm{\beta}}}

% models
\newcommand{\AccatTop}{\emph{Accuracy at the Top}\xspace}
\newcommand{\TopPush}{\emph{TopPush}\xspace}
\newcommand{\TopPushK}{\emph{TopPushK}\xspace}
\newcommand{\tauFPL}{{\emph{$\tau$-FPL}}\xspace}
\newcommand{\TopMeanK}{\emph{TopMeanK}\xspace}
\newcommand{\PatMat}{\emph{Pat}\&\emph{Mat}\xspace}
\newcommand{\PatMatNP}{{\emph{Pat}\&\emph{Mat-NP}}\xspace}
\newcommand{\Grill}{\emph{Grill}\xspace}
\newcommand{\GrillNP}{\emph{Grill-NP}\xspace}
\newcommand{\DeepTopPush}{\emph{DeepTopPush}\xspace}
\newcommand{\TFCO}{\emph{TFCO}\xspace}
\newcommand{\APPerf}{\emph{Ap-Perf}\xspace}
\newcommand{\BaseLine}{\emph{BaseLine}\xspace}

% counts and rates
\DeclareMathOperator{\tp}{tp}
\DeclareMathOperator{\tn}{tn}
\DeclareMathOperator{\fp}{fp}
\DeclareMathOperator{\fn}{fn}
\DeclareMathOperator{\tpr}{tpr}
\DeclareMathOperator{\tnr}{tnr}
\DeclareMathOperator{\fpr}{fpr}
\DeclareMathOperator{\fnr}{fnr}

\DeclareMathOperator{\tps}{\overline{tp}}
\DeclareMathOperator{\tns}{\overline{tn}}
\DeclareMathOperator{\fps}{\overline{fp}}
\DeclareMathOperator{\fns}{\overline{fn}}

\DeclareMathOperator{\accuracy}{acc}
\DeclareMathOperator{\baccuracy}{bacc}
\DeclareMathOperator{\precision}{precision}
\DeclareMathOperator{\recall}{recall}
\DeclareMathOperator{\pratrec}{Precision@Recall}
\DeclareMathOperator{\postop}{pos@top}
% ------------------------------------------------------------------------------
% Math
% ------------------------------------------------------------------------------
\let\originalleft\left
\let\originalright\right
\renewcommand{\left}{\mathopen{}\mathclose\bgroup\originalleft}
\renewcommand{\right}{\aftergroup\egroup\originalright}

% ------------------------------------------------------------------------------
% Theorems
% ------------------------------------------------------------------------------
\mdfdefinestyle{theoremstyle}{
  linecolor = myblue!50,
  backgroundcolor = myblue!10,
  frametitlebackgroundcolor = myblue!50,
  outerlinewidth = 1pt,
  roundcorner = 1pt,
  frametitlerule = true,
  skipabove = 5pt,
  skipbelow = 0pt,
}

\newmdtheoremenv[style = theoremstyle]{thm}{Theorem}[chapter]
\mdtheorem[style = theoremstyle]{theorem}[thm]{Theorem}
\mdtheorem[style = theoremstyle]{lemma}[thm]{Lemma}
\mdtheorem[style = theoremstyle]{corollary}[thm]{Corollary}
\mdtheorem[style = theoremstyle]{proposition}[thm]{Proposition}

% Definition style
\mdfdefinestyle{definitionstyle}{
  linecolor = mygreen!50,
  backgroundcolor = mygreen!10,
  frametitlebackgroundcolor = mygreen!50,
  outerlinewidth = 1pt,
  roundcorner = 1pt,
  frametitlerule = true,
  skipabove = 5pt,
  skipbelow = 0pt,
}
\mdtheorem[style = definitionstyle]{definition}[thm]{Definition}

% Example style
\mdfdefinestyle{examplestyle}{
  linecolor = myyellow!50,
  backgroundcolor = myyellow!10,
  frametitlebackgroundcolor = myyellow!50,
  outerlinewidth = 1pt,
  roundcorner = 1pt,
  frametitlerule = true,
  skipabove = 5pt,
  skipbelow = 0pt,
}
\mdtheorem[style = examplestyle]{example}[thm]{Example}
\mdtheorem[style = examplestyle]{notation}[thm]{Notation}


% Note style
\mdfdefinestyle{notestyle}{
  linecolor = myorange!50,
  backgroundcolor = myorange!10,
  frametitlebackgroundcolor = myorange!50,
  outerlinewidth = 1pt,
  roundcorner = 1pt,
  frametitlerule = true,
  skipabove = 5pt,
  skipbelow = 0pt,
}
\mdtheorem[style = notestyle]{note}[thm]{Note}

% Proof style
\mdfdefinestyle{proofstyle}{
  linecolor = myred,
  linewidth = 1pt,
  topline = false,
  bottomline = false,
  rightline = false,
  innertopmargin = 0pt,
  innerbottommargin = 0pt,
  innerrightmargin = 0pt,
  skipabove = 0pt,
  skipbelow = 0pt,
}

\renewcommand{\qedsymbol}{{\color{myred} $\blacksquare$}}

\renewenvironment{proof}[1][\proofname]
{ 
  \begin{mdframed}[style=proofstyle]%
  \noindent{\color{myred}\textit{\textbf{#1:}}}\par\nobreak\noindent\ignorespaces%
}
{%
  \nobreak\noindent\ignorespaces\qed
  \end{mdframed}
}

% ------------------------------------------------------------------------------
% Optional arguments
% ------------------------------------------------------------------------------
\newcommand{\NumberOfPages}[1]{\newcommand\@NumberOfPages{#1}}
\newcommand{\Year}[1]{\newcommand\@Year{#1}}
\newcommand{\AcademicYear}[1]{\newcommand\@AcademicYear{#1}}
\newcommand{\Acknowledgment}[1]{\newcommand\@Acknowledgment{#1}}
\newcommand{\Declaration}[1]{\newcommand\@Declaration{#1}}

\newcommand{\AuthorAffCZE}[1]{\newcommand\@AuthorAffCZE{#1}}
\newcommand{\AuthorAffENG}[1]{\newcommand\@AuthorAffENG{#1}}

\newcommand{\Supervisor}[1]{\newcommand\@Supervisor{#1}}
\newcommand{\SupervisorAffCZE}[1]{\newcommand\@SupervisorAffCZE{#1}}
\newcommand{\SupervisorAffENG}[1]{\newcommand\@SupervisorAffENG{#1}}

\newcommand{\SupervisorSpec}[1]{\newcommand\@SupervisorSpec{#1}}
\newcommand{\SupervisorSpecAffCZE}[1]{\newcommand\@SupervisorSpecAffCZE{#1}}
\newcommand{\SupervisorSpecAffENG}[1]{\newcommand\@SupervisorSpecAffENG{#1}}

\newcommand{\TitleCZE}[1]{\newcommand\@TitleCZE{#1}}
\newcommand{\DegreeProgrammeCZE}[1]{\newcommand\@DegreeProgrammeCZE{#1}}
\newcommand{\FieldCZE}[1]{\newcommand\@FieldCZE{#1}}
\newcommand{\AbstractCZE}[1]{\newcommand\@AbstractCZE{#1}}
\newcommand{\KeywordsCZE}[1]{\newcommand\@KeywordsCZE{#1}}

\newcommand{\TitleENG}[1]{\newcommand\@TitleENG{#1}}
\newcommand{\DegreeProgrammeENG}[1]{\newcommand\@DegreeProgrammeENG{#1}}
\newcommand{\FieldENG}[1]{\newcommand\@FieldENG{#1}}
\newcommand{\AbstractENG}[1]{\newcommand\@AbstractENG{#1}}
\newcommand{\KeywordsENG}[1]{\newcommand\@KeywordsENG{#1}}

\newcommand{\subtitle}[1]{\renewcommand\@subtitle{#1}}
\newcommand\@subtitle{}
\renewcommand\@date{}

% ------------------------------------------------------------------------------
% Title page
% ------------------------------------------------------------------------------
\renewcommand*{\maketitle}{
  \pagenumbering{Roman}
  \hypersetup{pageanchor=false}
  \begin{titlepage}
    \raggedleft
    \begin{minipage}[b][\textheight]{0.15\textwidth}
      \includegraphics[width=\textwidth]{images/logocvut.pdf} \par
      \vfill
      \includegraphics[width=\textwidth]{images/logofjfi.pdf} \par
    \end{minipage}
    \hspace{0.05\textwidth}
    \doccol \rule{2pt}{\textheight}
    \hspace{0.05\textwidth}
    \begin{minipage}[b][\textheight]{0.7\textwidth}
      \begin{minipage}[b]{\textwidth}
        \centering \doccol \Large
        \textbf{Czech Technical University in Prague} \par
        \textbf{Faculty of Nuclear Sciences and} \par
        \textbf{Physical Engineering}
      \end{minipage} \par
      \vfill
      \begin{minipage}[b]{\textwidth}
        \centering \doccol \huge \textsc{Doctoral Thesis}
      \end{minipage} \par
      \vspace{1cm}
      \begin{minipage}[b]{\textwidth}
        \centering  \doccol \huge \textbf{\@title}
      \end{minipage} \par
      \vfill
      \begin{minipage}[b]{0.95\textwidth}
        Prague, \@Year
        \hfill
        \@author
      \end{minipage}
    \end{minipage}
  \end{titlepage}
  \cleardoublepage

  % Thanks ...
  \thispagestyle{plain}
  \noindent
  \begin{minipage}[t]{\textwidth}\vspace{0pt}%
    \doccol \Large \textbf{Acknowledgements:}
  \end{minipage} \par
  \vspace{0.5cm}
  \noindent
  \begin{minipage}[t]{\textwidth}\vspace{0pt}%
    \@Acknowledgment
  \end{minipage} \par
  \vfill
  \noindent
  \begin{minipage}[t]{\textwidth}\vspace{0pt}%
    \doccol \Large \textbf{Declaration:}
  \end{minipage} \par
  \vspace{0.5cm}
  \noindent
  \begin{minipage}[t]{\textwidth}\vspace{0pt}%
    \@Declaration
  \end{minipage} \par
  \vspace{1.5cm}
  \noindent
  \begin{minipage}[t]{0.7\textwidth}\vspace{0pt}%
    V Praze dne \@date
  \end{minipage}
  \begin{minipage}[t]{0.3\textwidth}\vspace{0pt}%
    \centering ~ \par \textbf{\dotfill} \par \@author
  \end{minipage}
  \clearpage

  % Czech bibliography entry
  \thispagestyle{plain}
  \section*{Bibliografický záznam}
  \noindent
  \begin{minipage}[t]{0.25\textwidth}\vspace{0pt}%
    \doccol \textbf{Autor:}
  \end{minipage}
  \hspace{0.05\textwidth}
  \begin{minipage}[t]{0.7\textwidth}\vspace{0pt}%
    \@author, \par
    \@AuthorAffCZE
  \end{minipage} \par
  \vspace{0.5cm}
  \noindent
  \begin{minipage}[t]{0.25\textwidth}\vspace{0pt}%
    \doccol \textbf{Název práce:}
  \end{minipage}
  \hspace{0.05\textwidth}
  \begin{minipage}[t]{0.7\textwidth}\vspace{0pt}%
    \textbf{\@TitleCZE}
  \end{minipage} \par
  \vspace{0.5cm}
  \noindent
  \begin{minipage}[t]{0.25\textwidth}\vspace{0pt}%
    \doccol \textbf{Studijní program:}
  \end{minipage}
  \hspace{0.05\textwidth}
  \begin{minipage}[t]{0.7\textwidth}\vspace{0pt}%
    \@DegreeProgrammeCZE
  \end{minipage} \par
  \vspace{0.5cm}
  \noindent
  \begin{minipage}[t]{0.25\textwidth}\vspace{0pt}%
    \doccol \textbf{Studijní obor:}
  \end{minipage}
  \hspace{0.05\textwidth}
  \begin{minipage}[t]{0.7\textwidth}\vspace{0pt}%
    \@FieldCZE
  \end{minipage} \par
  \vspace{0.5cm}
  \noindent
  \begin{minipage}[t]{0.25\textwidth}\vspace{0pt}%
    \doccol \textbf{Školitel:}
  \end{minipage}
  \hspace{0.05\textwidth}
  \begin{minipage}[t]{0.7\textwidth}\vspace{0pt}%
    \@Supervisor, \par
    \@SupervisorAffCZE
  \end{minipage} \par
  \vspace{0.5cm}
  \noindent
  \begin{minipage}[t]{0.25\textwidth}\vspace{0pt}%
    \doccol \textbf{Školitel specialista:}
  \end{minipage}
  \hspace{0.05\textwidth}
  \begin{minipage}[t]{0.7\textwidth}\vspace{0pt}%
    \@SupervisorSpec, \par
    \@SupervisorSpecAffCZE 
  \end{minipage} \par
  \vspace{0.5cm}
  \noindent
  \begin{minipage}[t]{0.25\textwidth}\vspace{0pt}%
    \doccol \textbf{Akademický rok:}
  \end{minipage}
  \hspace{0.05\textwidth}
  \begin{minipage}[t]{0.7\textwidth}\vspace{0pt}%
    \@AcademicYear
  \end{minipage} \par
  \vspace{0.5cm}
  \noindent
  \begin{minipage}[t]{0.25\textwidth}\vspace{0pt}%
    \doccol \textbf{Počet stran:}
  \end{minipage}
  \hspace{0.05\textwidth}
  \begin{minipage}[t]{0.7\textwidth}\vspace{0pt}%
    \@NumberOfPages
  \end{minipage} \par
  \vspace{0.5cm}
  \noindent
  \begin{minipage}[t]{0.25\textwidth}\vspace{0pt}%
    \doccol \textbf{Klíčová slova:}
  \end{minipage}
  \hspace{0.05\textwidth}
  \begin{minipage}[t]{0.7\textwidth}\vspace{0pt}%
    \@KeywordsCZE
  \end{minipage} \par
  \vfill
  \clearpage

  % Czech abstract
  \thispagestyle{plain}
  \section*{Abstrakt}
  \noindent
  \@AbstractCZE
  \vfill
  \clearpage

  % English bibliography entry
  \thispagestyle{plain}
  \section*{Bibliographic Entry}
  \noindent
  \begin{minipage}[t]{0.25\textwidth}\vspace{0pt}%
    \doccol \textbf{Author:}
  \end{minipage}
  \hspace{0.05\textwidth}
  \begin{minipage}[t]{0.7\textwidth}\vspace{0pt}%
    \@author, \par
    \@AuthorAffENG
  \end{minipage} \par
  \vspace{0.5cm}
  \noindent
  \begin{minipage}[t]{0.25\textwidth}\vspace{0pt}%
    \doccol \textbf{Title of Dissertation:}
  \end{minipage}
  \hspace{0.05\textwidth}
  \begin{minipage}[t]{0.7\textwidth}\vspace{0pt}%
    \textbf{\@TitleENG}
  \end{minipage} \par
  \vspace{0.5cm}
  \noindent
  \begin{minipage}[t]{0.25\textwidth}\vspace{0pt}%
    \doccol \textbf{Degree Programme:}
  \end{minipage}
  \hspace{0.05\textwidth}
  \begin{minipage}[t]{0.7\textwidth}\vspace{0pt}%
    \@DegreeProgrammeENG
  \end{minipage} \par
  \vspace{0.5cm}
  \noindent
  \begin{minipage}[t]{0.25\textwidth}\vspace{0pt}%
    \doccol \textbf{Field of Study:}
  \end{minipage}
  \hspace{0.05\textwidth}
  \begin{minipage}[t]{0.7\textwidth}\vspace{0pt}%
    \@FieldENG
  \end{minipage} \par
  \vspace{0.5cm}
  \noindent
  \begin{minipage}[t]{0.25\textwidth}\vspace{0pt}%
    \doccol \textbf{Supervisor:}
  \end{minipage}
  \hspace{0.05\textwidth}
  \begin{minipage}[t]{0.7\textwidth}\vspace{0pt}%
    \@Supervisor, \par
    \@SupervisorAffENG
  \end{minipage} \par
  \vspace{0.5cm}
  \noindent
  \begin{minipage}[t]{0.25\textwidth}\vspace{0pt}%
    \doccol \textbf{Supervisor Specialist:}
  \end{minipage}
  \hspace{0.05\textwidth}
  \begin{minipage}[t]{0.7\textwidth}\vspace{0pt}%
    \@SupervisorSpec, \par
    \@SupervisorSpecAffENG 
  \end{minipage} \par
  \vspace{0.5cm}
  \noindent
  \begin{minipage}[t]{0.25\textwidth}\vspace{0pt}%
    \doccol \textbf{Academic Year:}
  \end{minipage}
  \hspace{0.05\textwidth}
  \begin{minipage}[t]{0.7\textwidth}\vspace{0pt}%
    \@AcademicYear
  \end{minipage} \par
  \vspace{0.5cm}
  \noindent
  \begin{minipage}[t]{0.25\textwidth}\vspace{0pt}%
    \doccol \textbf{Number of Pages:}
  \end{minipage}
  \hspace{0.05\textwidth}
  \begin{minipage}[t]{0.7\textwidth}\vspace{0pt}%
    \@NumberOfPages
  \end{minipage} \par
  \vspace{0.5cm}
  \noindent
  \begin{minipage}[t]{0.25\textwidth}\vspace{0pt}%
    \doccol \textbf{Keywords:}
  \end{minipage}
  \hspace{0.05\textwidth}
  \begin{minipage}[t]{0.7\textwidth}\vspace{0pt}%
    \@KeywordsENG
  \end{minipage} \par
  \vfill
  \clearpage

  % English abstract
  \thispagestyle{plain}
  \section*{Abstract}
  \noindent
  \@AbstractENG
  \vfill
  \cleardoublepage

  \hypersetup{pageanchor=true}
  \tableofcontents
  \cleardoublepage
  \mainmatter
}

% Title style
\allsectionsfont{\doccol}
\newcommand{\titlebox}[1]{\parbox[b][][b]{\textwidth}{#1}}

% Part and chapter style
\titleformat{\chapter}[display]
  {\doccol \bfseries \huge}
  {\filleft {\fontsize{1.5cm}{1cm}\selectfont\thechapter}}
  {0ex}
  {\titlebox{#1}}
  [{\titlerule[2pt]}]

\titleformat{\part}[display]
   {\doccol \Huge \bfseries \filcenter}
   {\partname{} \thepart}
   {0em}
   {{\titlerule[4pt]} #1}

\assignpagestyle{\part}{plain}

% pga sep
\newskip\linepagesep\linepagesep10pt\relax
\def\vfootline{\begingroup \doccol \rule[-990pt]{2pt}{1000pt} \endgroup}

% fancy page style
\pagestyle{fancy}
\renewcommand{\chaptermark}[1]{\markboth{\thechapter \ #1}{}}
\renewcommand{\sectionmark}[1]{\markright{\thesection \ #1}}
\renewcommand{\footrulewidth}{0pt}
\renewcommand{\headrulewidth}{2pt}
\renewcommand{\headrule}{\hbox to\headwidth{\doccol\leaders\hrule height \headrulewidth \hfill}}

\fancyhf{}
\fancyhead[RO]{\doccol \textbf{\nouppercase{\leftmark}}}
\fancyhead[LE]{\doccol \textbf{\nouppercase{\rightmark}}}
\fancyfoot[RO]{\doccol \vfootline \hskip \linepagesep \textbf{\thepage}}
\fancyfoot[LE]{\doccol \textbf{\thepage} \hskip \linepagesep \vfootline}

% plain page style
\fancypagestyle{plain}{
  \renewcommand{\headrulewidth}{0pt}
  \fancyhf{}
  \fancyfoot[RO]{\doccol \vfootline \hskip \linepagesep \textbf{\thepage}}
  \fancyfoot[LE]{\doccol \textbf{\thepage} \hskip \linepagesep \vfootline}
}

% empty page style
\def\cleardoublepage{
  \clearpage
    \if@twoside
      \ifodd\c@page\else
      \hbox{}
      \thispagestyle{plain}
      \newpage
    \fi
  \fi
}


% ------------------------------------------------------------------------------
% Affiliation
% ------------------------------------------------------------------------------
\title{General Framework for Classification at the Top}
\subtitle{Dissertation}

\author{Ing. Václav Mácha}
\branch{Matematické inženýrství}
\academicyear{2022/2023}
\date{1. prosince 2022}
\supervisor{doc. Ing Václav Šmídl, Ph.D.}
\supervisorspec{Mgr. Lukáš Adam, Ph.D.}

\acknowledgment{Thanks thanks thanks thanks thanks thanks thanks thanks thanks thanks thanks thanks thanks thanks thanks thanks thanks thanks thanks thanks thanks thanks thanks thanks thanks thanks}

\titleCZE{Title title title title title title}

\thesistype{Disertační práce}

\abstractCZE{Abstract abstract abstract abstract abstract abstract abstract abstract abstract abstract abstract abstract abstract abstract abstract abstract abstract abstract abstract abstract abstract abstract abstract abstract abstract abstract abstract abstract abstract abstract abstract abstract abstract abstract abstract abstract abstract abstract abstract abstract abstract abstract abstract abstract abstract abstract abstract abstract abstract abstract abstract abstract abstract abstract abstract abstract abstract abstract abstract abstract abstract abstract abstract abstract abstract abstract abstract abstract abstract abstract abstract abstract abstract abstract abstract abstract}

\keywordsCZE{Keywords keywords keywords keywords keywords keywords keywords keywords keywords keywords keywords keywords keywords}

\titleENG{General Framework for Classification at the Top}

\abstractENG{
  In standard binary classification, the goal is to classify all samples with the lowest possible error. However, in many applications, the error for one class is more serious than the other. Especially if the classes are not balanced. A prototypical example is cancer detection. Classifying a sick patient as healthy is definitely a more serious error in this case than the other way around. However, we still want to minimize both errors since trying to cure a healthy patient is also not ideal. Therefore the ultimate goal is to minimize the number of sick patients classified as healthy with some constraint on the number of healthy patients classified as sick. This formulation belongs to the class of problems where the performance is evaluated only on a small number of relevant (top) samples. We call this class a classification at the top, and it is the main object of this work. 

  Many well-known categories of problems, such as ranking, accuracy at the top, or hypothesis testing, are closely related to classification at the top. In this work we introduce a unified framework for classification at the topand show that several known formulations fall into the framework. We also propose completely new formulations (\PatMat, \PatMatNP) that also fall into the framework. We provide a theoretical analysis of the framework for the primal form when a linear model is used. We also discuss the theoretical properties of individual formulations and potential pitfalls that some formulations may encounter. Besides that, we show the convergence of the stochastic gradient descent for selected formulations (\PatMat, \PatMatNP) even though the gradient estimate is inherently biased. Moreover, we deriv dual forms of selected formulations and show how to incorporate non-linear kernels into these forms. We also derive an efficient coordinate descent algorithm to solve them. Finally, we also study the primal formulations with non-linear models. We show that when we use a non-linear model, the resulting formulations are non-decomposable. This property prevents us from using stochastic gradient descent in a standard way. We introduced modified stochastic gradient descent and show that this modification leads to a biased estimate of the true gradient. To mitigate this issue, we propose a new formulation \DeepTopPush. We demonstrate the performance of proposed formulations on visual recognition datasets and a real-world application on steganalysis and malware detection.
}

\keywordsENG{Binary classification, ranking, accuracy at the top, Neyman-Pearson, hypothesis testing}


% ------------------------------------------------------------------------------
% Document
% ------------------------------------------------------------------------------
\begin{document}

\maketitle

\chapter{Introduction to Binary Classification}

\todo[inline]{change font (??? libertine) and font size (??? 11pt)}

The problem of data classification is very important mathematical problem. The goal of classification is to find a relation between a set of objects and a target variable based on some properties of the objects. The properties of the objects are usually called features. There are many problems in research as well as in the real world that can be formulated as classification tasks. We can find applications of data classification across all the fields:
\begin{itemize}
  \item \textbf{Medical Diagnonsis:} In medicine, the classification is often used to improve disease diagnosis. In such a case, the features are medical records such as the patient's blood tests, temperature, or roentgen images. The target variable is if the patient has some disease. As an example, classification is used to process mammogram images and detect cancer~\cite{viale2012current, levy2016breast}.
  \item \textbf{Internet Secutiry:} These days, the internet is a crucial part of our lives. With the increasing usage of the internet, the number of attacks increases as well. An essential part of the defense are intrusion detection systems~\cite{grill2016learning, scarfone2007guide} that search for malicious activities (network attacks) in network traffic. Classification can be used to improve such systems~\cite{giacinto2002intrusion, shanbhag2009accurate}.
  \item \textbf{Marketing:} In marketing, the task can be to classify customers based on their buying interests. Such information can be used to build a personalized recommendation system for customers and therefore increase income~\cite{kaefer2005neural, zhang2007building}.
\end{itemize}
Many other classification problems can be found in almost all fields. Also, there is a vast number of classification algorithms that try to solve these classifications problems. Typically these algorithms consist of two phases:
\begin{itemize}
  \item \textbf{Training Phase:} In the training phase, the algorithm uses training data to build a model. The classification algorithms fall into the category of supervised learning algorithms. It means, that these algorithms must have labeled training data to build the model, i.e. the algorithm must have the knowledge of the target classes. The training data typically consists of pairs (sample, label) and can be described as follows
  \begin{equation*}\label{eq: training set}
    \D_{\mathrm{train}} = \Brac[c]{(\bm{x}_i, y_i)}_{i=1}^{n},
  \end{equation*}
  where the sample~$\bm{x}_i \in \R^d$ is a~$d$-dimensional vector of features that describes the object of interes and the label~$y_i \in \{1, 2, \ldots, k\}$ represents target class. Moreover~$n \in \N$ is a number of training samples and~$k \in \N$ is a number of target classes.
  \item \textbf{Testing Phase:} In the testing phase, the model is used to assign labels~$\hat{y}_i \in \{1, 2, \ldots, k\}$ to the data from testing set which was not known during the training phase
  \begin{equation*}\label{eq: test set}
    \D_{\mathrm{test}} = \Brac[c]{(\bm{x}_i, y_i)}_{i=1}^{m},
  \end{equation*}
  where~$\bm{x}_i \in \R^d,$~$y_i \in \{1, 2, \ldots, k\}$ and~$m \in \N$ is a number of testing samples. The ultimate goal of all classification algorithms is to classify testing samples with the highest accuracy possible.
\end{itemize}
The previous definitions of training and test set are general for classification problems with multiple classes. However, the main focus of this work is on a special subclass of classification problems with only two target classes: binary classification. The binary classification is a special case of classification in which the number of classes is~$k=2.$ These two classes are usually referred to as negative and positive classes and the positive class is the one that we are more interested in. If we go back to the mammogram example, the positive class would represent cancer. The positive class is usually encoded using label~$1$ and the negtative class using label~$0$ (for neural networks) or~$-1$ (for SVM-like algorithms~\cite{cortes1995support}).

\begin{notation}[Dataset]\label{not: dataset}
  In the rest of the work,, we follow the notation used for neural networks, i.e. we use~$1$ as positive label and~$0$ as negative label. Moreover, by dataset of size~$n \in \N$ we mean set in the form
  \begin{equation*}
    \D = \Brac[c]{(\bm{x}_i, y_i)}_{i=1}^{n},
  \end{equation*}
  where~$\bm{x}_i \in \R^d$ represents samples,~$d \in \N$ its dimension and~$y_i \in \{0, 1\}$ represents corresponding labels. To simplify future notation, we denote set of all indices of dataset~$\D$ as~$\I = \Ineg \cup \Ipos,$ where
  \begin{equation*}
    \begin{aligned}
      \Ineg & = \Set{i}{i \in \{1, 2, \ldots, n\} \; \land \; y_i = 0}, \\
      \Ipos & = \Set{i}{i \in \{1, 2, \ldots, n\} \; \land \; y_i = 1}.
    \end{aligned}
  \end{equation*}
  We also denote the number of negative samples in~$\D$ as~$\nneg = \Brac[v]{\Ineg}$ and the number of positive samples in~$\D$ as~$\npos = \Brac[v]{\Ipos},$ i.e. total number of samples is~$n = \nneg + \npos.$ 
\end{notation}

The goal of any classification problem is to classify given samples with the highest possible accuracy or in other words with the lowest possible error. In the case of binary classification, there are two types of error: positive samples classified as negative and vice versa. Formally, using the Notation~\ref{not: dataset}, the minimization of these two types of errors can be written as follows
\begin{mini}{\bm{w}, t}{
    \lambda_1 \sum_{i \in \Ineg} \Iverson{s_i \geq t} + \lambda_2 \sum_{i \in \Ipos} \Iverson{s_i < t}
  }{\label{eq: Binary classification}}{}
  \addConstraint{s_i}{= f(\bm{x}_i; \bm{w}), \quad}{i \in \I,}
\end{mini}
where~$\lambda_1, \lambda_2 \in \R,$ the function~$f \colon \R^d \to \R$ and~$\Iverson{\cdot{}}$ is Iverson function that is used to counts misclassified samples and is defined as
\begin{equation}\label{eq: iverson}
  \Iverson{x} = \begin{cases}
    0 & \quad \text{if } x \text{ is false}, \\
    1 & \quad \text{if } x \text{ is true}. \\
  \end{cases}
\end{equation}
Moreover, the vector~$\bm{w} \in \R^d$ represents trainable parameters (weights) of the model~$f$ and~$t \in R$ is a decision threshold. The parameters~$\bm{w}$ are determined from training data during the training phase of classification algorithm. Although the decision threshold~$t$ can also be determined from the training data, in many cases it is fixed. For example, for many algorithms the classification score~$s_i$ given by the model~$f$ represents the probability that the sample~$\bm{x}_i$ belongs to the positive class. Therefore, the decision threshold is set to~$t = 0.5$ and the sample is classified as positive if its classification score is larger than this threshold. In Notation~\ref{not: classifier}, we summarize the notation that is used in the rest of the work. 

\begin{notation}[Classifier]\label{not: classifier}
  By classifier, we always mean pair of model~$f$ and corresponding decision threshold~$t \in \R$. By model, we mean a function $f \colon \R^d \to \R$ which maps samples~$\bm{x}$ to its classification scores~$s$, i.e. for all~$i \in \I$ the classification score is defined as
  \begin{equation*}
    s_i = f(\bm{x}_i; \; \bm{w}),
  \end{equation*}
  where~$\bm{w}$ represents trainable parameters (weights) of the model. Predictions are defined ~$i \in \I$ in the following way
  \begin{equation*}
    \hat{y}_i = \begin{cases}
      1 & \quad \text{if } s_i \geq t, \\
      0 & \quad \text{otherwise.}
    \end{cases}
  \end{equation*}
\end{notation}

\todo[inline]{Add description of differrent binary classification problems such as SVm, logistric regression ...}

\section{Performance Evaluation}

In the previous section we defined general binary classification problem~\ref{eq: Binary classification}. However, we did not discuss yet how to measure the performance of the resulting classifier. In this section, we will introduce basic approaches that are used to measure the performance of binary classifiers.

\subsection{Confusion Matrix}
Based on the prediction~$\hat{y}_i$ and an actual label~$y_i$ of the sample~$\bm{x}_i,$ each sample can be assigned to one of the following categories
\begin{itemize}
  \item \textbf{True negative:}~$\bm{x}_i$ is negative and is classified as negative, i.e.~$y_i = 0 \; \land \; \hat{y}_i = 0.$
  \item \textbf{False positive:}~$\bm{x}_i$ is negative and is classified as positive, i.e.~$y_i = 0 \; \land \; \hat{y}_i = 1.$
  \item \textbf{False negative:}~$\bm{x}_i$ is positive and is classified as negative, i.e.~$y_i = 1 \; \land \; \hat{y}_i = 0.$
  \item \textbf{True positive:}~$\bm{x}_i$ is positive and is classified as positive, i.e.~$y_i = 1 \; \land \; \hat{y}_i = 1.$
\end{itemize}
Using these four categories, we can construct a so-called confusion matrix (sometimes also called contingency table)~\cite{fawcett2006introduction} that represents the results of predictionS for all samples from the given dataset~$\D$. An illustration of the confusion matrix is shown in Figure~\ref{fig: confusion matrix}. If we denote vector classification scores given by model~$f$ as~$\bm{s} \in \R^n,$ where~$s_i = f(\bm{x}_i; \bm{w})$ for all~$i \in \I,$ we can compute all fields of the confusion matrix as follows
\begin{equation}\label{eq: confusion counts}
  \begin{aligned}
    \tp(\bm{s}, t) & = \sum_{i \in \Ipos}\Iverson{s_i \geq t}, & \quad
    \fn(\bm{s}, t) & = \sum_{i \in \Ipos}\Iverson{s_i < t}, \\
    \tn(\bm{s}, t) & = \sum_{i \in \Ineg}\Iverson{s_i < t}, & \quad
    \fp(\bm{s}, t) & = \sum_{i \in \Ineg}\Iverson{s_i \geq t}.
  \end{aligned}
\end{equation}

\begin{figure}
  \centering
  \begin{NiceTabular}{cccccc}[cell-space-limits = 7pt]
    && \Block[draw=black, line-width=2pt, rounded-corners]{1-2}{
      \textbf{Predicted label}
    } \\
    && $\hat{y} = 0$
    &  $\hat{y} = 1$
    && \Block{1-1}{\textbf{Row total:}} \\
    \Block[draw=black, line-width=2pt, rounded-corners]{2-1}{
      \rotate \textbf{Actual} \\ \textbf{label}
    }
    & $y = 0$
    & \Block[draw=mygreen, fill=mygreen!50, rounded-corners]{1-1}{
      true \\ negatives \\ (\textbf{tn})
    }
    & \Block[draw=myred, fill=myred!50, rounded-corners]{1-1}{
      false \\ positives \\ (\textbf{fp})
    }
    & $\rightarrow$
    & \Block[draw=black, rounded-corners]{1-1}{all \\ negatives \\ ($\nneg$)} \\
    & $y = 1$
    & \Block[draw=myred, fill=myred!50, rounded-corners]{1-1}{
      false \\ negatives \\ (\textbf{fn})
    }
    & \Block[draw=mygreen, fill=mygreen!50, rounded-corners]{1-1}{
      true \\ positives \\ (\textbf{tp})
    }
    & $\rightarrow$
    & \Block[draw=black, rounded-corners]{1-1}{all \\ positives \\ ($\npos$)} \\
    && $\downarrow$
    &  $\downarrow$ \\
    \Block{1-2}{\textbf{Column} \\ \textbf{total:}}
    && \Block[draw=black, rounded-corners]{1-1}{all predicted \\ negatives}
    & \Block[draw=black, rounded-corners]{1-1}{all predicted \\ positives}
  \end{NiceTabular}
  \caption{Representation of the confusion matrix for the binary classification problem, where the negative class has label~$0$ and the positive class has label~$1.$ The true (target) label is denoted as~$y$ and predicted label is denoted as~$\hat{y}.$}
  \label{fig: confusion matrix}
\end{figure}

\noindent In the following text, we will sometimes use simplified notation~$\tp = \tp(\bm{s}, t)$ (and similar notation for other counts) for example to define classification metrics. In such cases, the vector of classification scores and decision threshold is fixed and is known from the context. Using the simplified notation we can simply define true-positive, false-positive, true-negative and false-negative rates as follows
\begin{equation}\label{eq: confusion rates}
  \begin{aligned}
    \tpr & = \frac{\tp}{\npos}, & \quad
    \fnr & = \frac{\fn}{\npos}, & \quad
    \tnr & = \frac{\tn}{\nneg}, & \quad
    \fpr & = \frac{\fp}{\nneg}. \\
  \end{aligned}
\end{equation}
Figure~\ref{fig: scores and rates} show the relation between classification rates and the decision threshold. The blue and red curves represent theoretical distribution of the scores of negative and positive samples samples respectively. The position of the decision threshold determines the values of the classification rates. The higher the value of the decision threshold, the smaller the false-positive rate, but at the same time the higher the false-negative rate. Similarly, the smaller the value of the decision threshold, the higher the false-positive rate and the smaller the false-negative rate. Ideally, classification without errors is the goal, but it is not usually possible and therefore we have to try to find some trade-off between false positive and a false negative rate. There is no universal truth, which error is worse. For example, we may want to detect cancer from some medical data. In this case, it is probably better to classify a healthy patient as sick than the other way around. On the other hand, in the computer security we do not want an antivirus program that makes a lot of false-positive alerts since it will be disruptive for the user. If we get look at the general definition of the binary classification problem~\eqref{eq: Binary classification}, we can see, that the objective function is in fact just the weighted sum of false positive and false negative samples, i.e. we can use the notation~\eqref{eq: confusion rates} and rewrite the problem~\eqref{eq: Binary classification} to the following form
\begin{mini}{\bm{w}, t}{
    \lambda_1 \cdot \fp(\bm{s}, t) + \lambda_2 \cdot \fn(\bm{s}, t)
  }{\label{eq: Binary classification counts}}{}
  \addConstraint{s_i}{= f(\bm{x}_i; \bm{w}), \quad}{i \in \I.}
\end{mini}
The parameters~$\lambda_1, \; \lambda_2 \in \R$ are used to specify which error is more serious for the particular classification task.

\begin{figure}
  \centering
  \includegraphics[width=\linewidth]{images/confusion_rates.pdf}
  \caption{The relation between classification scores and  rates. The blue curve represents theoretical distribution of the scores of negative samples and the red curve the same for the score of positive samples. Filled areas with light blue or red color represent true-negative and true-positive rates respectively. Similarly the filled areas with dark blue or red color represent false-positive and false-negaives rates.}
  \label{fig: scores and rates}
\end{figure}

In addition to the confusion matrix, there are many other classification metrics, and many of them are derived directly from the confusion matrix. As an example, we can mention accuracy and the balanced accuracy. Accuracy is defined as the ratio of correctly classified samples from all samples~\cite{metz1978basic}
\begin{equation*}
  \accuracy = \frac{\tp + \tn}{n}.
\end{equation*}
However, the accuracy is not suitable for unbalanced datasets, i.e. for dataset where the number of samples in one class is significantly higher then the number of samples in the other class. In such a case, the balanced accuracy is better. The balanced accuracy is defined as an average of true-positive and true-negative rate~\cite{brodersen2010balanced}
\begin{equation*}
  \baccuracy = \frac{1}{2}\Brac{\tpr + \tnr}.
\end{equation*}
The difference can be easily demonstrated on a simple example. Let us suppose that we have 100 samples and 10 of them is negatives and the rest is positive. If we use simple classifier that all samples clasify as positive we will get the following accuracy
\begin{equation*}
  \accuracy = \frac{90 + 0}{100} = 0.9.
\end{equation*}
Even though we know, that the classifier totally ignores negative samples, the accuracy is still 90\%. The reason is, that the used classifier is biased towards the more frequent class. Balanced accuracy solves this problem by using true-positive and true-negative rates instead of counts, which leads to the following results for the given example
\begin{equation*}
  \baccuracy = \Brac{\frac{90}{90} + \frac{0}{10}} = 0.5.
\end{equation*}
In this case the balanced accuracy is only 50\% which is very poor, but is more relevant to the unbalanced dataset. There are many more classification metrics that are based on the confusion matrix~\cite{fawcett2006introduction, metz1978basic, brodersen2010balanced, hossin2015review}. In this work, however, we will use mainly those that we have presented in this section. For simplicity, Table~\ref{tab: classification metrics} provides a summary of binary classification metrics used in this work. 

\begin{table}
  \centering
  \begin{NiceTabular}{ccc}
    \toprule
    \textbf{Name} & \textbf{Aliases} & \textbf{Formula} \\
    \midrule
    true negatives
      & correct rejection
      & $\tn$ \\
    false positives
      & Type I error, false alarm
      & $\fp = \nneg - \tn$ \\
    true positives
      & hity
      & $\tp$ \\
    false negatives
      & Type II error
      & $\fn = \npos - \tp$ \\
    \midrule
    true negative rate
      & specificity, selectivity
      & $\tnr = \frac{\tn}{\nneg}$ \\
    false positive rate
      & fall-out
      & $\fpr = \frac{\fp}{\nneg} = 1 - \tnr$ \\
    true positive rate
      & sensitivity, recall, hit rate
      & $\tpr = \frac{\tp}{\npos}$ \\
    false negative rate
      & miss rate
      & $\fnr = \frac{\fn}{\npos} = 1 - \tpr$ \\
    \midrule
    accuracy
      & ---
      & $\accuracy = \frac{\tp + \tn}{n}$ \\
    balanced accuracy
      & ---
      & $\baccuracy = \frac{\tpr + \tnr}{2}$ \\
    precision
      & positive predictive value
      & $\precision = \frac{\tp}{\tp + \fp}$ \\
    \bottomrule
  \end{NiceTabular}
  \caption{Summary of classification metrics derived from confusion matrix.}
  \label{tab: classification metrics}
\end{table}

\subsection{ROC Analysis}

In the previous section, we defined general binary clasification problem as a minimization task with objective that consists of a weighted sum of the false-positive and false-negative counts~\eqref{eq: Binary classification counts}. For fixed model~$f$ and decision threshold~$t$, the results can be visualized in the Receiver Operating Characteristic space~\cite{egan1975signal}.

\todo[inline]{Finish roc section}

\section{Related Problems}

The aim of classical binary classification is to separate positive and negative samples with the highest possible accuracy. However, in many applications, it is desirable to separate only a certain number of samples. In such a case, the goal is not to maximize the performance on all samples but only the performance on the required samples with the highest relevance. The rest of the samples is irrelevant and therefore the performance on them is not important. Figure~\ref{fig: standard vs. aatp} shows the difference between the standard classifier (classifier 1) that maximizes the accuracy and the classifier that focuses only on the classification at the top (classifier 2). In this particular case, the classifier 2 tries to maximize the number of positive samples that are ranked higher than the worst negative sample, i.e. the negative sample with the highest score. Formally, we can define metric
\begin{equation*}
  \postop(\bm{s}) = \frac{1}{\npos} \sum_{i \in \Ipos} \Iverson{s_i \geq \max_{j \in \Ineg}\{s_j\}}.
\end{equation*}
While classifier 1 has good total~$\accuracy$, its~$\postop$ metric is subpar because of the few negative outliers. On the other hand, classifier 2 has worse total~$\accuracy$, but its~$\postop$ metric is extremely good because more than half of the positive samples are ranked higher than the worst negative sample. While classifier 1 selected different thresholds for the~$\accuracy$ and~$\postop$ metrics, these thresholds coincide for classifier 2. In the rest of the chapter, we will present three main categories of problems that are closely related to the binary classification but do not focus on optimizing overall performance.

\begin{figure}[t]
  \centering
  \includegraphics[width = \linewidth]{images/standard_aatp_comparison.pdf}
  \caption{Difference between standard classifiers (\textbf{Classifier~1}) and classifiers maximizing~$\postop$ metric (\textbf{Classifier~2}). While the former has a good total~$\accuracy$, the latter has a~$\postop$ metric.}
  \label{fig: standard vs. aatp}
\end{figure}

\subsection{Ranking problems}

\todo[inline]{Add proper introduction to ranking problems}

\textbf{Ranking problems:} Ranking problems~\cite{freund2003efficient, agarwal2011infinite, rudin2009pnorm, li2014top} select the most relevant samples and rank them. To each sample, a numerical score is assigned, and the ranking is performed based on this score. Often, only scores above a threshold are considered. As an example, we can mention search engines such as Google, DucDucGo or Yahoo. In such a case, the goal is to provide most relevant results on the first two or three pages. The results on page 50 are usually of no interest to anyone, so it is important to move the most relevant results to the few first pages~\cite{cortes2003auc}.

The first category of problems that is tightly related to binary classification at the top, is the category of ranking problems. Ranking problems have become very important in many different fields
\begin{itemize}
  \item \textbf{Information retrieval systems:} The goal of the information retrieval systems is to rank documents according to relevance to a given query.
  \item \textbf{Recommendation  systems:} The goal is to rank and recommend products based on the user's previous behavior.
  \item 
\end{itemize}
All the examples above can be formulated as bipartite ranking problem~\cite{freund2003efficient, agarwal2005generalization, agarwal2011infinite}, where the goal is to rank the relevant (positive) samples higher than the non-relevant (negative) ones. 

Ranking problems~\cite{freund2003efficient, agarwal2011infinite, rudin2009pnorm, li2014top} select the most relevant samples and rank them. To each sample, a numerical score is assigned, and the ranking is performed based on this score. Often, only scores above a threshold are considered. As an example, we can mention search engines such as Google, DucDucGo or Yahoo. In such a case, the goal is to provide most relevant results on the first two or three pages. The results on page 50 are usually of no interest to anyone, so it is important to move the most relevant results to the few first pages~\cite{cortes2003auc}.

A prototypical example is the RankBoost~\cite{freund2003efficient} maximizing the area under the ROC curve, the Infinite Push~\cite{agarwal2011infinite} or the~$p$-norm push~\cite{rudin2009pnorm} which concentrate on the high-ranked negatives and push them down. Since all these papers include pairwise comparisons of all samples, they can be used only for small datasets. This was alleviated in~\cite{li2014top}, where the authors performed the limit~$p \to \infty$ in~$p$-norm push and obtained the linear complexity in the number of samples. Moreover, since the~$l_{\infty}$-norm is equal to the maximum, this method falls into our framework with the threshold equal to the largest score computed from negative samples.

Many methods, such as \emph{RankBoost}~\cite{freund2003efficient}, \emph{Infinite Push}~\cite{agarwal2011infinite} or \emph{$p$-norm push}~\cite{rudin2009pnorm} employ a pairwise comparison of samples, which makes them infeasible for larger datasets. This was alleviated in \TopPush~\cite{li2014top} where the authors considered the limit~$p \rightarrow \infty$. Since the~$l_{\infty}$ norm from \TopPush is equal to the maximum, the decision threshold from our framework equals to the maximum of scores of negative samples. This was generalized into \TopPushK~\cite{adam2021general} by considering the threshold to be the mean of~$K$ largest scores of negative samples.

\subsection{Accuracy at the Top}

\todo[inline]{Add proper introduction to Accuracy a the Top}

\textbf{Accuracy at the Top:} Accuracy at the Top~\cite{boyd2012accuracy, grill2016learning} is similar to ranking problems. However, instead of ranking the most relevant samples, it only maximizes the number of positive samples (equivalently minimizes the misclassification)  above the top~$\tau$-quantile of scores. The Accuracy at the Top can be very useful for search engines or in applications where identified samples undergo expensive post-processing such as human evaluation. As an example, we can mention cyber security~\cite{grill2016learning}, where a low false-negative rate is crucial as a high number of false alarms would result in the software being uninstalled, or drug development, where potentially useful drugs need to be preselected and manually investigated.

Accuracy at the Top ($\tau$-quantile) was formally defined in~\cite{boyd2012accuracy} and maximizes the number of relevant samples in the top~$\tau$-fraction of ranked samples. When the threshold equals the top~$\tau$-quantile of all scores, this problem falls into our framework. The early approaches aim at solving approximations, for example,~\cite{joachims2005svm} optimizes a convex upper bound on the number of errors among the top samples. Due to the presence of exponentially many constraints, the method is computationally expensive.~\cite{boyd2012accuracy} presented an SVM-like formulation which fixes the index of the quantile and solves~$n$ problems. While this removes the necessity to handle the (difficult) quantile constraint, the algorithm is computationally infeasible for a large number of samples.~\cite{kar2015surrogate} derived upper approximations, their error bounds and solved these approximations.~\cite{grill2016learning} proposed the projected gradient descent method where after each gradient step, the quantile is recomputed.~\cite{eban2017scalable} suggested new formulations for various criteria and argued that they keep desired properties such as convexity.~\cite{tasche2018plug} showed that accuracy at the top is maximized by thresholding the posterior probability of the relevant class. The closest approach to our framework is~\cite{lapin2015top,lapin2018analysis}, where the authors considered multi-class classification problems, and their goal was to optimize the performance on the top few classes and~\cite{mackey2018constrained}, where the authors implicitly removed some variables and derived an efficient algorithm.

\AccatTop~\cite{boyd2012accuracy} focuses on maximizing the number of positive samples above the top~$\tau$-quantile of scores. There are many methods on how to solve accuracy at the top. In~\cite{boyd2012accuracy}, the authors assume that the top quantile is one of the samples, construct~$n$ unconstrained optimization problems with fixed thresholds, solve them and select the best solution. This method is computationally expensive. In~\cite{grill2016learning} the authors propose a fast projected gradient descent method. In our previous paper, we proposed a convex approximation of the accuracy at the top called \PatMat. This method is reasonably fast and guaranteed the existence of global optimum.

\subsection{Hypothesis Testing}

\todo[inline]{Add proper introduction to Hypothesis testing}

\textbf{Hypothesis testing} states a null and an alternative hypothesis. The Neyman-Pearson problem minimizes the Type II error (the null hypothesis is false but it fails to be rejected) while keeping the Type I error (the null hypothesis is true but is rejected) small. If the null hypothesis states that a sample has the positive label, then Type II error happens when a positive sample is below the threshold and thus minimizing the Type II error amounts to minimizing the positives below the threshold.

Hypothesis testing states a null and an alternative hypothesis. The Neyman-Pearson problem minimizes the Type II error (the null hypothesis is false but it fails to be rejected) while keeping the Type I error (the null hypothesis is true but is rejected) small. If the null hypothesis states that a sample has the positive label, then Type II error happens when a positive sample is below the threshold and thus minimizing the Type II error amounts to minimizing the positives below the threshold.
\chapter{Introduction to Classification at the Top}\label{chap: binary classification}

In the previous chapter, we briefly introduced binary classification and what it is suitable for. In this chapter, we introduce binary classification more formally and define the notation used in the rest of the work. Moreover, we discuss several approaches that can be used to measure the performance of binary classifiers. Finally, we present the problem of classification at the top, which is closely related to binary classification. This problem is the main topic of the entire work.

\section{Binary Classification}

As we discussed before, binary classification is the special case of classification in which the total number of classes is two. Although these classes can have arbitrary names, we call the class we are interested in the \emph{positive class} and the other class the \emph{negative class}. Moreover, we use the label~$y=1$ to denote the positive class and~$y=0$ to denote the negative class. The notation used in the rest of the work is summarized in Notation~\ref{not: dataset}.

\begin{notation}[Dataset]\label{not: dataset}
  In this work, we use label~$0$ to encode the negative class and label~$1$ to encode the positive class. By a dataset of size~$n \in \N$ we mean a set of pairs in the following form
  \begin{equation*}
    \mathcal{D} = \Brac[c]{(\bm{x}_i, y_i)}_{i=1}^{n},
  \end{equation*}
  where~$\bm{x}_i \in \R^d$ represents samples and~$y_i \in \{0, 1\}$ corresponding labels. To simplify future notation, we denote a set of all indices of dataset~$\mathcal{D}$ as~$\I = \Ineg \cup \Ipos,$ where
  \begin{equation*}
    \begin{aligned}
      \Ineg & = \Set{i}{i \in \{1, 2, \ldots, n\} \; \land \; y_i = 0}, \\
      \Ipos & = \Set{i}{i \in \{1, 2, \ldots, n\} \; \land \; y_i = 1}.
    \end{aligned}
  \end{equation*}
  We also denote the number of negative samples in~$\mathcal{D}$ as~$\nneg = \Brac[v]{\Ineg}$ and the number of positive samples in~$\mathcal{D}$ as~$\npos = \Brac[v]{\Ipos}.$ The total number of samples is~$n = \nneg + \npos.$ 
\end{notation}

The goal of any classification problem is to classify given samples with the highest possible accuracy or, in other words, with the lowest possible error. In the case of binary classification, there are two types of error: a positive sample is classified as negative, and vice versa. Formally, using the Notation~\ref{not: dataset}, the minimization of these two types of errors can be written as follows
\begin{mini}{\bm{w}, t}{
    C_1 \sum_{i \in \Ineg} \Iverson{s_i \geq t} + C_2 \sum_{i \in \Ipos} \Iverson{s_i < t}
  }{\label{eq: Binary classification}}{}
  \addConstraint{s_i}{= f(\bm{x}_i; \bm{w}), \quad}{i \in \I,}
\end{mini}
where~$C_1, C_2 \in \R,$ the function~$f \colon \R^d \to \R$ is further referenced as a model and~$\Iverson{\cdot{}}$ is the Iverson function which is used to count misclassified samples and is defined as
\begin{equation}\label{eq: iverson}
  \Iverson{x} = \begin{cases}
    0 & \quad \text{if } x \text{ is false}, \\
    1 & \quad \text{if } x \text{ is true}.
  \end{cases}
\end{equation}
Moreover, the vector~$\bm{w} \in \R^d$ represents trainable parameters (weights) of the model~$f$ and~$t \in R$ represents a decision threshold. The parameters~$\bm{w}$ are determined from the training set. Although the decision threshold~$t$ can also be determined from the training data, in many cases, it is fixed. For example, many training algorithms assume that the classification score~$s_i = f(\bm{x}_i; \bm{w})$ given by the model~$f$ represents the probability that the sample~$\bm{x}_i$ belongs to the positive class. Therefore, the decision threshold is set to~$t = 0.5,$ and the sample is classified as positive if its classification score is larger than this threshold. In Notation~\ref{not: classifier}, we summarize the notation used in the rest of the work.

\begin{notation}[Classifier]\label{not: classifier}
  By classifier, we always mean a pair of a model~$f$ and a corresponding decision threshold~$t$. By model, we mean a function $f \colon \R^d \to \R$ which maps samples~$\bm{x}$ to its classification scores~$s$, i.e. for all~$i \in \I$ the classification score is defined as
  \begin{equation*}
    s_i = f(\bm{x}_i; \; \bm{w}),
  \end{equation*}
  where~$\bm{w}$ represents trainable parameters (weights) of the model~$f.$ Predictions are defined for all~$i \in \I$ in the following way
  \begin{equation}\label{eq: prediction}
    \hat{y}_i = \begin{cases}
      1 & \quad \text{if } s_i \geq t, \\
      0 & \quad \text{otherwise.}
    \end{cases}
  \end{equation}
\end{notation}

In the introduction, we briefly described how to find the best classifier. However, this process is more complicated in the real world. The whole process can be split into the four phases:
\begin{itemize}
  \item \textbf{Training:} The first phase is the actual training of the classifier. In this phase, we use some algorithm to find the trainable parameters of the classifier based on the provided training set. Returning to the formulation of binary classification~\ref{eq: Binary classification}, in this phase, we want to find weights~$\bm{w}$ and the threshold~$t.$ However, most formulations have some hyper-parameters, such as~$C_1$ and~$C_2$ in~\ref{eq: Binary classification}. These hyperparameters are fixed for training. Multiple classifiers with different hyperparameter values are typically trained in the training phase.
  \item \textbf{Validation:} The validation phase selects the best hyper-parameter values that lead to the most performant and robust classifier. In this phase, the performance of classifiers from the training phase is evaluated on the so-called validation set. The validation set has the same structure as the training set but contains different samples.
  \item \textbf{Testing:} In the testing phase, the classifier with the best performing values of hyper-parameters is tested against different classifiers. For the testing, the testing set is used. The testing set also has the same structure as the training set (it contains the labels) but contains different samples.
  \item \textbf{Inference:} In the previous phases, we trained multiple classifiers and selected the best one for our specific task. In the inference phase, we apply this classifier to real unlabeled data.
\end{itemize}

\section{Performance Evaluation}\label{sec: performance evaluation}

In the previous section, we defined general binary classification problem~\eqref{eq: Binary classification}. However, we did not discuss how to measure the performance of the resulting classifier. In this section, we introduce basic performance metrics  that are used to measure the performance of binary classifiers.

\subsection{Confusion Matrix}

Based on the prediction~$\hat{y}_i$ from~\eqref{eq: prediction} and the actual label~$y_i$ of the sample~$\bm{x}_i,$ each sample can be assigned to one of the four following categories:
\begin{itemize}
  \item \textbf{True negative:} sample~$\bm{x}_i$ is negative and is classified as negative, i.e.~$y_i = 0 \; \land \; \hat{y}_i = 0.$
  \item \textbf{False positive:} sample~$\bm{x}_i$ is negative and is classified as positive, i.e.~$y_i = 0 \; \land \; \hat{y}_i = 1.$
  \item \textbf{False negative:} sample~$\bm{x}_i$ is positive and is classified as negative, i.e.~$y_i = 1 \; \land \; \hat{y}_i = 0.$
  \item \textbf{True positive:} sample~$\bm{x}_i$ is positive and is classified as positive, i.e.~$y_i = 1 \; \land \; \hat{y}_i = 1.$
\end{itemize}
If we assign each sample from dataset~$\mathcal{D}$ to one of the categories above and count the number of samples in each of these four categories, we get the confusion matrix (sometimes also called contingency table)~\cite{fawcett2006introduction}, see Figure~\ref{fig: confusion matrix}. A confusion matrix consists of four fields that contains number of true-negative (\textbf{tn}), false-positive (\textbf{fp}), false-negative (\textbf{fn}), and true-positive (\textbf{tp}) samples in the whole dataset. More formally, using the prediction rule~\eqref{eq: prediction} we can compute all fields of the confusion matrix as follows
\begin{equation}\label{eq: confusion counts}
  \begin{aligned}
    \tp(\bm{s}, t) & = \sum_{i \in \Ipos}\Iverson{s_i \geq t}, & \quad
    \fn(\bm{s}, t) & = \sum_{i \in \Ipos}\Iverson{s_i < t}, \\
    \tn(\bm{s}, t) & = \sum_{i \in \Ineg}\Iverson{s_i < t}, & \quad
    \fp(\bm{s}, t) & = \sum_{i \in \Ineg}\Iverson{s_i \geq t},
  \end{aligned}
\end{equation}
where~$\bm{s}$ is the vector of classification scores given by model~$f,$ and~$\Iverson{\cdot}$ is the Iverson function~\eqref{eq: iverson}. In the following text, we sometimes use a simplified notation~$\tp = \tp(\bm{s}, t)$ (and similar notation for other counts). In such cases, the vector of classification scores and decision threshold is fixed and is known from the context. Using the simplified notation, we can define true-positive, false-positive, true-negative, and false-negative rates as follows
\begin{equation}\label{eq: confusion rates}
  \begin{aligned}
    \tpr & = \frac{\tp}{\npos}, & \quad
    \fnr & = \frac{\fn}{\npos}, & \quad
    \tnr & = \frac{\tn}{\nneg}, & \quad
    \fpr & = \frac{\fp}{\nneg}.
  \end{aligned}
\end{equation}
Figure~\ref{fig: scores and rates} shows the relation between classification rates and the decision threshold. The blue and red curves represent the theoretical distribution of the scores of negative and positive samples, respectively. If we increase the value of the threshold~$t,$ we decrease the false-positive rate, but at the same time, we also increase the false-negative rate. On the other hand, if we decrease the value of~$t,$ we decrease the false-negative rate, but at the same time, we also increase the false-positive rate. In other words, it is not possible to decrease the false-positive rate only by moving the threshold~$t$ without increasing the false-negative rate and vice versa. Therefore, we always have to find some balance between these two types of errors.

If we look at the general definition of the binary classification problem~\eqref{eq: Binary classification}, the objective function is only the weighted sum of false-positive and false-negative samples. Therefore, we can use the notation~\eqref{eq: confusion rates} and rewrite the problem~\eqref{eq: Binary classification} to
\begin{mini}{\bm{w}, t}{
    C_1 \cdot \fp(\bm{s}, t) + C_2 \cdot \fn(\bm{s}, t)
  }{\label{eq: Binary classification counts}}{}
  \addConstraint{s_i}{= f(\bm{x}_i; \bm{w}), \quad}{i \in \I.}
\end{mini}
Parameters~$C_1, \; C_2 \in \R$ are used to specify which error is more serious for the particular classification task.

\begin{figure}
  \centering
  \begin{NiceTabular}{cccccc}[cell-space-limits = 7pt]
    && \Block[draw=black, line-width=2pt, rounded-corners]{1-2}{
      \textbf{Predicted label}
    } \\
    && $\hat{y} = 0$
    &  $\hat{y} = 1$
    && \Block{1-1}{\textbf{Row total:}} \\
    \Block[draw=black, line-width=2pt, rounded-corners]{2-1}{
      \rotate \textbf{Actual} \\ \textbf{label}
    }
    & $y = 0$
    & \Block[draw=mygreen, fill=mygreen!50, rounded-corners]{1-1}{
      true \\ negatives \\ (\textbf{tn})
    }
    & \Block[draw=myred, fill=myred!50, rounded-corners]{1-1}{
      false \\ positives \\ (\textbf{fp})
    }
    & $\rightarrow$
    & \Block[draw=black, rounded-corners]{1-1}{all \\ negatives \\ ($\nneg$)} \\
    & $y = 1$
    & \Block[draw=myred, fill=myred!50, rounded-corners]{1-1}{
      false \\ negatives \\ (\textbf{fn})
    }
    & \Block[draw=mygreen, fill=mygreen!50, rounded-corners]{1-1}{
      true \\ positives \\ (\textbf{tp})
    }
    & $\rightarrow$
    & \Block[draw=black, rounded-corners]{1-1}{all \\ positives \\ ($\npos$)} \\
    && $\downarrow$
    &  $\downarrow$ \\
    \Block{1-2}{\textbf{Column} \\ \textbf{total:}}
    && \Block[draw=black, rounded-corners]{1-1}{all predicted \\ negatives}
    & \Block[draw=black, rounded-corners]{1-1}{all predicted \\ positives}
  \end{NiceTabular}
  \caption{The confusion matrix for the binary classification problem, where the negative class has the label~$0$ and the positive class has the label~$1.$ The true (target) label is denoted by~$y$ and the predicted label is denoted by~$\hat{y}.$}
  \label{fig: confusion matrix}
\end{figure}

\begin{figure}
  \centering
  \includegraphics{images/confusion_rates.pdf}
  \caption{The relation between classification scores and  rates. The blue/red curve is the theoretical distribution of the scores of negative/positive samples, respectively. The area between the blue line and the x-axis is divided by the decision threshold~$t.$ The left part represents the true-negative rate, while the right part represents the false-positive rate. The area between the red line and the x-axis is also divided by~$t.$ The left part represents a false-negative rate, and a right represents the true-positive rate.}
  \label{fig: scores and rates}
\end{figure}

The confusion matrix is not the only way to measure the performance of binary classifiers. For example, there are many different classification matrices, and many of them are derived directly from the confusion matrix~\cite{fawcett2006introduction, metz1978basic, brodersen2010balanced, hossin2015review}. As an example, we can mention accuracy and balanced accuracy defined as
\begin{align*}
  \accuracy & = \frac{1}{n}\Brac{\tp + \tn}, &
  \baccuracy & = \frac{1}{2}\Brac{\tpr + \tnr}.
\end{align*}
Note that the objective function in~\eqref{eq: Binary classification counts} is accuracy if~$C_1 = C_2 = \frac{1}{\nall}.$ Moreover, for~$C_1 = \frac{1}{2\nneg}$ and~$C_2 = \frac{1}{2\npos},$ the objective function is balanced accuracy. This show the importance of these two performance matrices for standard binary classification. More performance metrics derived from the confusion matrix can be found in Table~\ref{tab: classification metrics}. Moreover, in the following section, we introduce a different approach for the performance evaluation of binary classifiers.

\begin{table}
  \centering
  \begin{NiceTabular}{ccc}
    \CodeBefore
      \rowcolor{\headercol}{1}
      \rowcolors{3}{\rowcol}{}[restart]
    \Body
    \toprule
    \textbf{Name} & \textbf{Aliases} & \textbf{Formula} \\
    \midrule
    true negatives
      & correct rejection
      & $\tn$ \\
    false positives
      & Type I error, false alarm
      & $\fp = \nneg - \tn$ \\
    true positives
      & hit
      & $\tp$ \\
    false negatives
      & Type II error
      & $\fn = \npos - \tp$ \\
    \midrule
    true negative rate
      & specificity, selectivity
      & $\tnr = \frac{\tn}{\nneg}$ \\
    false positive rate
      & fall-out
      & $\fpr = \frac{\fp}{\nneg} = 1 - \tnr$ \\
    true positive rate
      & sensitivity, recall, hit rate
      & $\tpr = \frac{\tp}{\npos}$ \\
    false negative rate
      & miss rate
      & $\fnr = \frac{\fn}{\npos} = 1 - \tpr$ \\
    \midrule
    accuracy
      & ---
      & $\accuracy = \frac{\tp + \tn}{n}$ \\
    balanced accuracy
      & ---
      & $\baccuracy = \frac{\tpr + \tnr}{2}$ \\
    precision
      & positive predictive value
      & $\precision = \frac{\tp}{\tp + \fp}$ \\
    \bottomrule
  \end{NiceTabular}
  \caption{Summary of classification metrics derived from confusion matrix. The first column shows the name used in this work, while the second column shows alternative names that can be found in the literature. The last column shows the formula based on the confusion matrix.}
  \label{tab: classification metrics}
\end{table}

\begin{figure}
  \centering
  \includegraphics{images/roc_space.pdf}
  \caption{A basic representation of the ROC space with five different classifiers. (\textbf{left}) A comparison of ROC curves for two different classifiers. (\textbf{right})}
  \label{fig: roc space}
\end{figure}

\subsection{ROC Analysis}\label{subsec: ROC}

In the previous section, we defined a general binary classification formulation~\eqref{eq: Binary classification counts} that minimizes a weighted sum of false-positive and false-negative counts. Therefore, we always have to find some trade-off between the false-positive and false-negative counts and select the best hyperparameters~$C_1,$ $C_2,$ for given tasks. There is no universal truth which of these two errors is worse. For example, it is probably better to classify a healthy patient as sick and do additional tests than the other way around. On the other hand, in computer security, an antivirus program with a lot of false-positive alerts is useless since it is disruptive to the user. The Receiver Operating Characteristic (ROC) space~\cite{egan1975signal, fawcett2006introduction} is one way to visualize the trade-off between false-positive and false-negative errors.

ROC space is a two-dimensional space with the x-axis equal to the false-positive rate and the y-axis to the true-positive rate. The left-hand side of Figure~\ref{fig: roc space} shows the ROC space with five highlighted points. Each point in the ROC space represents one fixed classifier, i.e., one pair consisting of a model~$f$ and a decision threshold~$t.$ There are several important points in the ROC space. The point~$(0, 0)$ represents a classifier classifying all samples as negative, while~$(1, 1)$ is a classifier classifying all samples as positive. Both these classifiers are useless. On the other hand, the point (0, 1) represents the perfect classifier that classifies all samples correctly since~$\fpr = 0$ and~$\tpr = 1.$

ROC representation allows us to decide whether one classifier is better than another, only in some cases. For example, in Figure~\ref{fig: roc space}, classifier \textbf{B} is better than classifier \textbf{C} since \textbf{B} has a higher true-positive rate and at the same time a lower false-positive rate. On the other hand, it is impossible to say which classifier is better if one has a higher true-positive rate and the other has a lower false-positive rate. We can see this situation for classifier \textbf{B} and \textbf{A}. In such a case, the preference depends on the given problem, as discussed at the beginning of this section.

Another important part of the ROC space is the diagonal line highlighted in red in Figure~\ref{fig: roc space}. Any classifier that appears on this diagonal provides the same performance as a random classifier. For example, classifier \textbf{C} is represented in ROC space by point~$(0.7, 0.7).$ Such classifier randomly classifies 70\% of samples as positive. Therefore, any classifier that appears in ROC space in the lower right triangle is worse than a random classifier. There are usually no classifiers in this area since any classifier from the lower right triangle can be easily improved. If we negate the prediction of such a classifier for every sample, we get its negated version in the upper left triangle. Such a situation is in Figure~\ref{fig: roc space} for classifiers \textbf{E} and \textbf{B}. Since classifier \textbf{E} has a false-negative rate of~$1 - \tpr = 0.8,$ we can deduce that negated classifier will have a true-positive rate of 0.8. Similarly, since classifier \textbf{E} has a true-negative rate of $1 - \fpr = 0.4,$ its negated version will have a false-positive rate of 0.4. Therefore the negated version of classifier \textbf{E} is represented in ROC space by point~$(0.4, 0.8),$ which is classifier~\textbf{B}.

Many classifiers only predict whether samples are positive or negative. As an example, we can mention decision trees. Such classifiers are always represented as a single point in the ROC space. In this text, we consider only classifier from Notation~\ref{not: classifier}, which predict a continuous score instead of a hard prediction. We assume that the classifier consists of the model~$f$ that produces classification scores and the decision threshold~$t.$ Many standard classifiers such as neural networks or logistic regression fall into this setting. Even though the decision threshold is determined during the training process, it is possible to change it and obtain different predictions. This possibility is very often used to produce so-called ROC curves~\cite{fawcett2006introduction}.

ROC curve shows how model~$f$ behaves for different thresholds~$t$ varying from~$-\infty$ to~$+\infty.$ Right-hand side of Figure~\ref{fig: roc space} provides an example of two ROC curves for two different classifiers. \textbf{Classifier 1} provides accuracy~95\% and is represented by the blue dot, while the blue line represents its ROC curve. \textbf{Classifier 2} represented by the green dot provides accuracy~76\%, and the green dashed line represents its ROC curve. A standard method for comparing two classifiers is to compare the corresponding areas under the ROC curves ($\auroc$)~\cite{bradley1997use, hanley1982meaning}. Such an approach is a simple way to reduce the curve to one number. In the case of standard binary classification, the larger the $\auroc$, the better. In Figure~\ref{fig: roc space} we can see that the blue classifier has $\auroc$ 95\% while the green one has only 77\%. Therefore, for most classification problems, the blue classifier is better. Even though we get almost the same values of accuracy and $\auroc$ for both classifiers, the accuracy is not equivalent to $\auroc$. The similarity is only a consequence of the used example.

Since both false-positive and true-positive rates are non-increasing functions of threshold~$t,$ we can efficiently compute the ROC curve from sorted classification scores. Moreover, the $\auroc$ of a classifier is equivalent to the probability that the classifier will rank a randomly chosen positive sample higher than a randomly chosen negative sample~\cite{fawcett2006introduction}. By comparing the classifiers from the right-hand side of Figure~\ref{fig: roc space}, we can deduce that \textbf{Classifier 1} is generally better at a false-positive rate larger than~$0.01.$ Otherwise, \textbf{Classifier 2} is the better one. Therefore,  there is a specific region of the ROC space where \textbf{Classifier 2} outperforms \textbf{Classifier 1}. In the next section, we discuss multiple different problems which focus on the performance only at low false-positive rates.

\section{Classification at the Top}\label{sec: related problems}

As discussed above, \textbf{Classifier 1} focuses on the overall performance, while the  \textbf{Classifier 2} on the performance on low false-positive rates, see Figure~\ref{fig: roc space}. The latter classifier can be handy for search engines such as Google or DuckDuckGo, where the goal is to have all relevant results on the first few pages. The results on page 50 are usually of no interest to anyone, so it is crucial to move the most relevant results to the few first pages~\cite{cortes2003auc, joachims2002optimizing}. Therefore, it is essential to push as many positive samples above some small portion of the worst negative samples (negative samples with the largest classification scores). In this section, we use two different visual representations for the performance of classifiers from the right-hand side of Figure~\ref{fig: roc space} to show the difference and emphasize their advantages.

Figure~\ref{fig: standard vs. aatp} shows the difference between the standard classifier (\textbf{Classifier 1}) that maximizes the accuracy and the classifier that focuses only on the classification at the top (\textbf{Classifier 2}). In this particular case, \textbf{Classifier 2} maximizes the number of positive samples that are ranked higher or equal than the worst negative sample. In other words, \textbf{Classifier 2} maximizes true-positive rate at the smallest possible false-positive rate. If we go back to the example with search engines, the goal of \textbf{Classifier 2} is to push as many relevant results before the first irrelevant. Formally, \textbf{Classifier 2} maximizes the following metric
\begin{equation}\label{eq: metric pos at top}
  \postop(\bm{s}) = \frac{1}{\npos} \sum_{i \in \Ipos} \Iverson{s_i \geq \max_{j \in \Ineg}s_j}.
\end{equation}
For both classifiers, Figure~\ref{fig: standard vs. aatp} shows two different decision thresholds. The black threshold is the one for which the classifier was trained, while the green one represents the worst negative sample. For \textbf{Classifier 2} these two thresholds coincide. We can observe that \textbf{Classifier 1} provides a much better separation of positive and negative samples. Only a few samples above the black threshold ruin perfect separation. On the other hand, the separation provided by \textbf{Classifier 2} is much worse since half of the positive samples are mixed with negative ones. Therefore, the accuracy of \textbf{Classifier 1} is~$95\%$ while the accuracy of \textbf{Classifier 2} is only~$76\%.$ However, in terms of metric~\eqref{eq: metric pos at top} the situation is quite different. Since there are few negative outliers, there is only 19\% of positive samples above the worst negative for \textbf{Classifier 1}, but 53\% for \textbf{Classifier 2}.

The same behavior can also be demonstrated using ROC curves. Figure~\ref{fig: roc space log} shows ROC curves for both classifier with (right) and without (left) logarithmic scaling of x-axis. The blue line represents ROC curve for \textbf{Classifier 1} and the green dashed one for \textbf{Classifier 2}. Moreover, There are two important points for \textbf{Classifier 2}. The blue filled circle corresponds to the black threshold and the blue filled square to the green threshold from Figure~\ref{fig: standard vs. aatp}. Since for \textbf{Classifier 2} both thresholds coincide, there is only one point in Figure~\ref{fig: roc space log} highlighted by a green square. The superiority of \textbf{Classifier 1} in the overall performance is evident from the left-hand side of the figure, since there is only a small region of ROC space, where \textbf{Classifier 2} provides a higher true-positive rate. However, this region is very interesting. The right-hand side of Figure~\ref{fig: roc space log} allows us to concentrate on very low false-positive rates. 
If the false-positive rate is lower than~$7 \cdot 10^{-1},$ then \textbf{Classifier 2} provides better true-positive rate than \textbf{Classifier 1}. Finally, the value of metric~\eqref{eq: metric pos at top} is highlighted using squares for both classifiers and it is clear, that \textbf{Classifier 2} provides higher value of this metric.

\begin{figure}
  \centering
  \includegraphics{images/standard_aatp_comparison.pdf}
  \caption{Difference between standard classifiers (\textbf{Classifier~1}) and classifiers maximizing~$\postop$ metric (\textbf{Classifier~2}). While the former has a good total accuracy, the latter has a good~$\postop$ metric.}
  \label{fig: standard vs. aatp}
\end{figure}

\begin{figure}
  \centering
  \includegraphics{images/roc_space_log.pdf}
  \caption{Difference between standard classifiers (\textbf{Classifier~1}) and classifiers maximizing~$\postop$ metric (\textbf{Classifier~2}). While the former has a good total accuracy, the latter has a good~$\postop$ metric.}
  \label{fig: roc space log}
\end{figure}

The rest of the chapter presents three main categories of problems that focus only on a small number of the most relevant samples. Moreover, in Chapter~\ref{chap: framework}, we show that at least some formulations from these three categories are closely related to binary classification.

\subsection{Ranking Problems}

The first category is the category of ranking problems. The ranking algorithms play a crucial role in many information retrieval problems:
\begin{itemize}
  \item \textbf{Document (Text) retrieval systems} are used for obtaining relevant documents from the collection of documents based on the relevance to the user's query. Such systems are widely used for accessing books, journals, or any other documents. However, the most visible applications are search engines such as Google or DuckDuckGo.
  \item \textbf{Collaborative filtering} is one of the techniques used to predict a user's rating of a new product based on past ratings of users with similar rating patterns. Such systems can be used to generate music or video playlists automatically. Therefore, such systems are widely used in services such as Youtube or Spotify.
\end{itemize}
The two examples above show that ranking problems usually depend on users' feedback or preferences. In binary classification, we only have the labels that represent if the samples are positive or negative. On the other hand, ranking problems use multiple ways to describe the users' feedback. One approach uses the feedback function~$\Phi: \R^d \times \R^d \to \R$ to represent the user's preferences~\cite{freund2003efficient}. In such a case, the feedback function can be defined for all pairs of samples~$(\bm{x}_i, \bm{x}_j)$ in the following way
\begin{equation*}
  \Phi(\bm{x}_i, \bm{x}_j) \; 
  \begin{cases}
    > 0 & \bm{x}_i \text{ is prefered over } \bm{x}_j, \\
    = 0 & \text{no preference,} \\
    < 0 & \bm{x}_j \text{ is prefered over } \bm{x}_i.
  \end{cases}
\end{equation*}
We can see, that the feedback function specify if the user prefers~$\bm{x}_i$ over~$\bm{x}_j$ or not. Moreover, the feedback function also specifies how strong the preference is, i.e., the higher the volume~$\abs{\Phi(\bm{x}_i, \bm{x}_j)},$ the higher the preference. Many ranking algorithms try to find some ordering of all samples that minimizes the number of incorrectly ordered pairs of samples. Consider a ranking function~$r: \R^d \to \R.$ The sample~$\bm{x}_i$ is ranked higher than the sample~$\bm{x}_j$ if~$r(\bm{x}_i) > r(\bm{x}_j).$ Then, the minimization of the number of misordered pairs can be formally written as follows
\begin{mini}{r}{
  \sum_{i \in \I} \sum_{j \in \I} \Iverson{r(\bm{x}_i) \leq r(\bm{x}_j)} \cdot \max\Brac[c]{0, \; \Phi(\bm{x}_i, \bm{x}_j)}.
  }{\label{eq: rankboost}}{}
\end{mini}
This problem is computationally demanding since the objective function contains a pairwise comparison of all samples. Therefore, the problem is not suitable for large data. \emph{RankBoost}~\cite{freund2003efficient} is a boosting algorithm based on the AdaBoost~\cite{freund1997decision} that combines many weak ordering functions to obtain the final ranking. This approach leads to the maximization of the $\auroc$~\cite{rudin2009pnorm}. Therefore, RankBoost focuses on the overall performance. However, as we discussed at the beginning of the section, we want to focus only on the small portion of the most relevant samples in many applications. In such a case, this approach is not ideal.

Consider movie recommendations. In such a case, we only care if the movie is good or not. It is not important if one bad movie is ranked higher than another bad movie. Both movies are still bad and therefore not relevant. Many ranking algorithms~\cite{rudin2009pnorm} use the so-called bipartite ranking to address this situation. In such a situation, each sample is positive (good) or negative (bad), and the goal is to push positive samples above negative ones. The authors of~\cite{rudin2009pnorm} proposed the following formulation
\begin{mini}{r}{
  \Brac{\sum_{j \in \Ineg} \Brac{\sum_{i \in \Ipos} \Iverson{r(\bm{x}_i) \leq r(\bm{x}_j)}}^p}^{\frac{1}{p}}.
  }{\label{eq: p-norm push}}{}
\end{mini}
The authors of~\cite{rudin2009pnorm} also proposed boosting algorithm called \textbf{$p$-Norm Push} to solve the formulation above. Note that for~$p = 1$, the formulation~\eqref{eq: p-norm push} is very similar to the RankBoost~\eqref{eq: rankboost} and the resulting ranking function maximizes the $\auroc$, therefore focusing on optimizing the overall ranking. On the other hand, for~$p \rightarrow +\infty$, the formulation~\eqref{eq: p-norm push} minimizes the largest number of positive samples ranked below any negative sample
\begin{mini*}{r}{
  \max_{j \in \Ineg} \; \sum_{i \in \Ipos} \Iverson{r(\bm{x}_i) \leq r(\bm{x}_j)}.
  }{}{}
\end{mini*}
In such a case, the resulting ranking function focuses only on the absolute top, i.e., it aims to push as many positive samples above the negative sample with the highest rank. Moreover, the formulation above can be equivalently rewritten as follows
\begin{mini}{r}{
  \sum_{i \in \Ipos} \Iverson{r(\bm{x}_i) \leq \max_{j \in \Ineg} \; r(\bm{x}_j)}.
  }{\label{eq: toppush rank}}{}
\end{mini}
Authors of~\cite{agarwal2011infinite} focus on this formulation and introduce a SVM (\emph{Support Vector Machines}~\cite{cortes1995support}) based algorithm called \emph{Infinite Push} to solve it. Finally, authors of~\cite{li2014top} proposed an even more efficient algorithm with the linear complexity in the number of samples called \TopPush. Note that the objective function of the problem above is almost the same as the metric~\eqref{eq: metric pos at top}. Therefore, this \textbf{Classifier 2} from Figures~\ref{fig: standard vs. aatp} and~\ref{fig: roc space log} corresponds to the ranking function given by \TopPush algorithm. It shows a close connection between binary classification and the bipartite ranking problems. 

\subsection{Accuracy at the Top}

In the previous section, we introduced formulation~\eqref{eq: toppush rank}, which focuses on maximizing the number of positive samples above the worst negative sample (the one with the highest rank or highest classification score). This formulation is very useful, as discussed at the beginning of this section. However, such a maximization problem can be unstable since the objective function does not allow false-positive errors. Therefore, if there is one negative outlier with a high score, the number of positive samples above this outlier can be tiny. The authors of~\cite{boyd2012accuracy} focus on a similar problem as \TopPush, but use a different approach. They proposed the following formulation and called it \textbf{Accuracy at the Top}
\begin{mini}{\bm{w}}{
  \frac{1}{\nneg} \fp(\bm{s}, t) + \frac{1}{\npos} \fn(\bm{s}, t)
  }{\label{eq: aatp intro}}{}
  \addConstraint{s_i}{= f(\bm{x}_i; \bm{w}), \quad i \in \I}
  \addConstraint{t}{= \max \Set{t}{\frac{1}{\nall} \sum_{i \in \I} \Iverson{s_i \geq t} \geq \tau},}
\end{mini}
where~$f: \R^d \to \R$ is a model. This formulation focuses on the top $\tau$-fraction of all samples and tries to maximize the number of positive samples and minimize the number of negative samples in it. Even though the goal is to maximize the number of positive samples above the top $\tau$-quantile, the objective function contains false-positive and false-negative rates. It should be sufficient to include only one of them since the definition of the threshold implies the minimization of the other one as well. However, this form of objective function should be more robust~\cite{grill2016learning}. The problem of Accuracy at the Top is useful, for example, in applications where identified samples undergo expensive post-processing, such as human evaluation. For instance, in pharmaceutics, potentially useful drugs must be preselected and manually investigated in drug development. Since the manual investigation is costly, we have to select only a fraction of drugs with the highest potential. However, it is precisely what Accuracy at the Top does.

There are many methods on how to solve Accuracy at the Top, since the formulation is complicated due to the top $\tau$-quantile in the constraint. The early approaches aim at solving approximations. For example, the authors of \cite{joachims2005svm} optimize a convex upper bound on the number of errors among the top samples. Due to exponentially many constraints, the method is computationally expensive. In \cite{boyd2012accuracy} the authors presented a SVM-like formulation. They assume that the top $\tau$-quantile is one of the samples, construct~$n$ unconstrained optimization problems with fixed thresholds, solve them and select the best solution. While this removes the necessity to handle the (difficult) quantile constraint, the algorithm is computationally infeasible for a large number of samples. The authors of~\cite{grill2016learning} proposed the projected gradient descent method, where after each gradient step, the quantile is recomputed. In \cite{eban2017scalable} authors suggested new formulations for various criteria and argued that they keep desired properties such as convexity. Finally, the authors of~\cite{tasche2018plug} showed that Accuracy at the Top is maximized by thresholding the posterior probability of the relevant class.

\subsection{Hypothesis Testing}

The hypothesis testing operates with null~$H_0$ and alternative~$H_1$ hypothesis. The goal is to either reject the null hypothesis in favor of the alternative or not to reject it. Since the problem is binary, two possible errors can occur. Type I occurs when~$H_0$ is true but is rejected, and Type II error happens when~$H_0$ is false but fails to be rejected. The Neyman-Pearson problem minimizes~\cite{neyman1933ontheproblem} Type II error while keeping Type I error smaller than some predefined bound. Using our notation for the Neyman-Pearson problem, the null hypothesis~$H_0$ states that sample~$\bm{x}$ has a negative label. Then Type I error occurs when the sample is false-positive, while Type II error occurs when the sample is false-negative. Therefore, the Neyman-Pearson problem minimizes the false-negative rate with the prescribed level~$\tau$ of the false-positive rate. Such constraint can be written in the form of quantile, i.e., the threshold is the top $\tau$-quantile of scores of all negative samples.
\begin{mini*}{f}{
  \frac{1}{\npos} \fn(\bm{s}, t)
  }{}{}
  \addConstraint{s_i}{= f(\bm{x}_i), \quad i \in \I}
  \addConstraint{t}{= \max \Set{t}{\frac{1}{\nneg} \sum_{i \in \Ineg} \Iverson{s_i \geq t} \geq \tau},}
\end{mini*}
This formulation is very similar to the one for Accuracy at the Top~\eqref{eq: aatp intro}. The main difference is that the quantile in the constraint is not computed from all but only from negative samples. Also, note one key difference in interpretation. The~$\tau$ in Accuracy at the Top represents the total amount of samples we want to process with the smallest possible error. On the other hand, in the Neyman-Pearson problem~$\tau$ represents the maximal acceptable false-positive rate. Therefore, the former approach is useful in situations where we can process only a certain number of samples. The latter is for situations where we have strict constraints on false-positive errors.

\section{Summary}

In this chapter, we introduced the general formulation~\eqref{eq: Binary classification counts} for binary classification and discussed how to measure the performance of binary classifiers. The first approach for performance evaluation is based on the confusion matrix. This approach is very straightforward. Moreover, it is possible to derive many different classification matrices from the confusion matrix. Table~\ref{tab: classification metrics} summarizes classification matrices derived from the confusion matrix used in the upcoming chapters. The second approach introduced in this chapter uses the ROC space to visualize the ability of classifiers to rank positive samples above negative ones. Since standard binary classification focuses on optimizing the overall performance, we discussed that there are many problems closely tied to binary classification that focus on the performance of the most relevant samples. Such problems occur in many applications, from search engines to drug development. We also introduced Ranking problems, the problem of Accuracy at the Top, and the Neyman-Pearson problem and discussed their relation to the binary classification.

\begin{note}
  To improve the readability of the main part of the work, we postpone many results into appendices. Main results are presented in the main part, but all auxiliary results and proofs are located in appendices.
\end{note}

\chapter{Binary Classiciation at the Top}\label{chap: framework}

In the previous chapter, we introduced the general formulation~\eqref{eq: Binary classification counts} and fundamental evaluation matrices for the binary classification problems. Furthermore, in Section~\ref{sec: related problems}, we introduced three problems closely related to binary classification but focused on specific performance criteria. Namely: \emph{Accuracy at the top} problem, \emph{Ranking problems}, and the problem of \emph{Hypothesis testing}. Even though these problems are usually considered separately, they have one crucial thing in common. All three problems aim to minimize the number of misclassified samples below (or above) a certain threshold. In the rest of the chapter, we focus on this common property. We show that all these problems fall into the following unified framework for binary classification at the top
\begin{mini}{\bm{w}}{
  \lambda_1 \cdot \fp(\bm{s}, t) + \lambda_2 \cdot \fn(\bm{s}, t)
}{\label{eq: aatp counts}}{}
  \addConstraint{s_i}{= f(\bm{x}_i; \bm{w}), \quad}{i \in \I}
  \addConstraint{t}{= G\Brac{\bm{s}, \bm{y}},}
\end{mini}
where function~$G \colon \R^n \times \{0, 1\}^n \to \R$ takes the scores and labels of all samples and computes the decision threshold. The concrete form of the function~$G$ that defines the decision threshold depends on the used problem. As we show later in the chapter, all problems mentioned above differ only in the definition of the function~$G.$  Note the important distinction from the standard binary classification~\eqref{eq: Binary classification counts}: the decision threshold is no longer fixed (as in the case of neural networks) or trained independently (as in SVM) but is a function of scores of all samples. Therefore, the minimization in problem~\eqref{eq: aatp counts} is performed only concerning the one variable~$\bm{w}.$

\section{Surrogate formulation}\label{sec: surrogate formulation}

The objective function of problem~\eqref{eq: aatp counts} is a weighted sum of false-positive and false-negative counts. Since these counts are discontinuous due to the presence of the Iverson function (see~\eqref{eq: confusion counts}), the whole objective function is discontinuous too. Therefore, problem~\eqref{eq: aatp counts} is difficult to solve. One way how to simplify the problem is to derive its continuous approximation. Since the only discontinuous part of the objective function is the Iverson function, the usual approach is to employ a surrogate function to replace it~\cite{li2014top, grill2016learning}.

\begin{notation}[Surrogate function]\label{not: surrogates}
  In the text below, the symbol~$l$ denotes any convex non-negative non-decreasing function with~$l(0) = 1$. As examples of such function we can mention the hinge loss function or the quadratic hinge loss functions defined as follows
  \begin{align*}
    l_{\text{hinge}}(s) & = \max\Brac[c]{0, 1 + s}, &
    l_{\text{quadratic}}(s) & = \Brac{\max\Brac[c]{0, 1 + s}}^2.
  \end{align*}
  Moreover, parameter~$\vartheta > 0$ is used to scale inputs to any surrogate function.
\end{notation}

Notation~\ref{not: surrogates} summarizes all assumptions that a proper surrogate function must fulfill and introduces the two most often used surrogate functions: hinge and quadratic hinge loss functions. Moreover, Figure~\ref{fig: surrogates} compares these two surrogate functions with the Iverson function. It is clear that the surrogate function always provides  an upper approximation of the Iverson function. In other words, if a surrogate function~$l$ satisfies assumptions from Notation~\ref{not: surrogates}, then~$l(s) \geq \Iverson{s \geq 0}$ holds for any~$s \in \R.$ Besides that, Figure~\ref{fig: surrogates} shows how the scaling parameter~$\vartheta$ affects the approximation quality of the surrogate function. If the scaling parameter is greater than 1, the surrogate function approximates the Iverson function better on interval~$(-\infty, 0)$. In the opposite case, the approximation is better on interval~$(0, \infty)$. The usual choice of scaling parameter is~$\vartheta = 1,$ and we used this choice for all surrogate functions used in the objective functions. However, we also use surrogate functions for approximation of the decision threshold. In such a case, the scaling parameter plays a crucial role for some theoretical guaranties, as shown in upcoming chapters.

\begin{figure}[t]
  \centering
  \includegraphics[width = \linewidth]{images/surrogates.pdf}
  \caption{Comparison of the approximation quality of the Iverson function using different surrogate functions and scaling parameters.}
  \label{fig: surrogates}
\end{figure}

With a properly defined surrogate function, we can define the surrogate approximation of the objective function of problem~\eqref{eq: aatp counts}. To follow the notation from the previous chapter, we first replace the Iverson function in~\eqref{eq: aatp counts}. Using any surrogate function~$l$ that satisfies assumptions from Notation~\ref{not: surrogates}, the true counts~\eqref{eq: aatp counts} may be approximated by their surrogate counterparts defined by
\begin{equation}\label{eq: confusion counts surrogate}
  \begin{aligned}
    \tps(\bm{s}, t) & = \sum_{i \in \Ipos}l(s_i - t), & \qquad
    \fns(\bm{s}, t) & = \sum_{i \in \Ipos}l(t - s_i), \\
    \tns(\bm{s}, t) & = \sum_{i \in \Ineg}l(t - s_i), &
    \fps(\bm{s}, t) & = \sum_{i \in \Ineg}l(s_i - t).
  \end{aligned}
\end{equation}
Since the surrogate function provides upper approximation of the Iverson function, the surrogate counts~\eqref{eq: confusion counts surrogate} provide upper approximations of the true counts~\eqref{eq: confusion counts}. By replacing the true counts in the objective function of~\eqref{eq: aatp counts} with their surrogate counterparts and adding a regularization for better numerical stability, we get
\begin{mini}{\bm{w}}{
  \frac{1}{2} \norm{\bm{w}}^2 + \lambda_1 \cdot \fps(\bm{s}, t) + \lambda_2 \cdot \fns(\bm{s}, t)
  }{\label{eq: aatp surrogate}}{}
  \addConstraint{s_i}{= f(\bm{x}_i; \bm{w}), \quad i \in \I}
  \addConstraint{t}{= G\Brac{\bm{s}, \bm{y}}.}
\end{mini}
The resulting objective function is continuous, and therefore the problem is easier to solve than the original problem~\eqref{eq: aatp counts}. No additional theoretical properties can be derived without knowing the concrete form of model~$f$ and function~$G.$ Therefore, the rest of the chapter is dedicated to problems that fall into the general framework~\eqref{eq: aatp surrogate} and concrete form of~$G$ for such problems. More precisely, we focus on the three problems introduced in Section~\ref{sec: related problems} and show how to rewrite them to our general formulation~\eqref{eq: aatp surrogate}. Most of these problems are defined originally only for the linear model since this choice allows to derive nice theoretical properties and efficient solving algorithms. However, this chapter focuses on the problem formulation itself rather than on how to solve it. Therefore for all problems, we derive their version with general model~$f.$ The discussion of the theoretical properties for specific forms of~$f$ is provided in Chapter~\ref{chap: linear},~\ref{chap: dual}, and~\ref{chap: deep}.

\begin{notation}[Classification scores]\label{not: scores}
  In Notation~\ref{not: classifier}, we defined vector~$\bm{s} \in \R^n$ of scores of all samples with components defined for any~$i \in \I$ as
  \begin{equation*}
    s_i = f(\bm{x}_i; \bm{w}), \quad i \in \I,
  \end{equation*}
  where~$f \colon \R^d \to \R$ represents an arbitrary model. To simplify the upcoming sections, we define a sorted version of vector~$\bm{s}$ with decreasing components and denote it as~$\bm{s}_{[\cdot]}.$ It means that components of~$\bm{s}_{[\cdot]}$ fulfill
  \begin{equation*}
    s_{[1]}   \geq s_{[2]} \geq \dots \geq s_{[n - 1]} \geq s_{[n]}.
  \end{equation*}
  Moreover, we denote negative samples as~$\bm{x}^-$ and positive samples as~$\bm{x}^+.$ Finally, we define vectors~$\bm{s}^- \in \R^{\nneg},$ $\bm{s}^+ \in \R^{\npos}$ of scores of all positive and negative samples with components defined as
  \begin{equation*}
    \begin{aligned}
      s^-_j & = f(\bm{x}^-_j; \bm{w}), \quad j = 1, \; 2, \ldots, \; \nneg, \\
      s^+_i & = f(\bm{x}^+_i; \bm{w}), \quad i = 1, \; 2, \ldots, \; \npos,
    \end{aligned}
  \end{equation*}
  and their sorted versions~$\bm{s}^-_{[\cdot]}$, $\bm{s}^+_{[\cdot]}$ with decreasing components.
\end{notation}

\section{Ranking problems}\label{sec: ranking}

The first category of problems from Section~\ref{sec: related problems} is a category of ranking problems. The general goal of problems from this category is to rank positive (relevant) samples higher than negative ones. That can be achieved in many different ways, but we focus only on the problems that concentrate on the high-ranked negative samples and try to push as many positive samples as possible above them. The simplest case is when the goal is to maximize the number of positive samples above the worst negative. Since the worst negative sample is the negative sample with the highest classification score, the decision threshold for such a case is the highest score corresponding to the negative sample. Then the aim is to maximize the number of true-positive samples above this threshold or, equivalently, minimize the number of false-negative negative below it, which may be written as
\begin{mini}{\bm{w}}{
  \frac{1}{\npos} \fn(\bm{s}, t)
  }{\label{eq: toppush}}{}
  \addConstraint{s_i}{= f(\bm{x}_i; \bm{w}), \quad i \in \I}
  \addConstraint{t}{= s_{[1]}^-}
\end{mini}
Note that the decision threshold~$t$ in the previous formulation is a function of classification scores. Therefore, the formulation is just a special case of the general formulation~\eqref{eq: aatp counts} for $\lambda_1 = 0$ and~$\lambda_2 = \nicefrac{1}{\npos}$. The authors in~\cite{li2014top} proposed an efficient method to solve formulation~\eqref{eq: toppush} and called it \TopPush. They replaced the true counts in the objective function of~\eqref{eq: toppush} with its surrogate counterpart in the same way as we did in Section~\ref{sec: surrogate formulation}. The resulting formulation has the following form
\begin{mini}{\bm{w}}{
  \frac{1}{2} \norm{\bm{w}}^2 + \frac{1}{\npos} \fns(\bm{s}, t)
  }{\label{eq: toppush surrogate}}{}
  \addConstraint{s_i}{= f(\bm{x}_i; \bm{w}), \quad i \in \I}
  \addConstraint{t}{= s_{[1]}^-,}
\end{mini}
which again falls into our framework~\eqref{eq: aatp surrogate}. To stress the origin of this formulation, we denote it as \TopPush. Unfortunately, \TopPush formulation can be very sensitive to outliers, especially when the linear model is used, as shown in~\ref{sec: stability}. To robustify the formulation, we follow the idea presented in~\cite{lapin2015top} and replace the highest negative score by the mean of~$K$ highest negative scores. The resulting formulation is as follows
\begin{mini}{\bm{w}}{
  \frac{1}{2} \norm{\bm{w}}^2 + \frac{1}{\npos} \fns(\bm{s}, t)
  }{\label{eq: toppushK surrogate}}{}
  \addConstraint{s_i}{= f(\bm{x}_i; \bm{w}), \quad i \in \I}
  \addConstraint{t}{= \sum_{i = 1}^{K} s_{[i]}^-},
\end{mini}
To emphasize the similarity with the \TopPush, we call this formulation \TopPushK. It is also possible to use the value of $K$-th highest negative score as the threshold. Such a choice may be advantageous in some cases, and we will discuss it in Chapter~\ref{chap: deep}. For now, we will stick to the formulation that uses the mean since it will allow us to derive some crucial theoretical properties, as shown in Section~\ref{sec: convexity}. 

\section{Accuracy at the Top}\label{sec: aatp}

The second problem from Section~\ref{sec: related problems} is the problem of Accuracy at the Top~\cite{boyd2012accuracy}. This problem aims to find an ordering of samples so that samples whose scores are among the top $\tau$-quantile are as relevant as possible. The top~$\tau$-quantile of all scores is defined by
\begin{equation}\label{eq: aatp quantile} 
  t_1(\bm{s})
    = \max \Set{t}{\frac{1}{n} \sum_{i \in \I} \Iverson{s_i \geq t} \geq \tau}.
\end{equation}
All relevant samples should be ranked above the quantile~$t_1$ and all irrelevant samples below the quantile~$t_1$ in an ideal case. Thus, the main difference to the ranking problems is that the problem of Accuracy at the Top considers both classification errors and does not focus only on false-negative samples. The original formulation~\cite{boyd2012accuracy} considers a balanced dataset with the same number of positive and negative samples. Paper~\cite{grill2016learning} reformulated the problem for the unbalanced dataset and derived the following formulation
\begin{mini}{\bm{w}}{
  \frac{1}{\nneg} \fp(\bm{s}, t) + \frac{1}{\npos} \fn(\bm{s}, t)
  }{\label{eq: aatp}}{}
  \addConstraint{s_i}{= f(\bm{x}_i; \bm{w}), \quad i \in \I}
  \addConstraint{t}{= \max \Set{t}{\frac{1}{n} \sum_{i \in \I} \Iverson{s_i \geq t} \geq \tau}.}
\end{mini}
This formulation already falls into our framework~\eqref{eq: aatp counts} for~$\lambda_1 = \nicefrac{1}{\nneg}$ and~$\lambda_2 = \nicefrac{1}{\npos}$. Moreover, the authors of~\cite{boyd2012accuracy, grill2016learning} used the same surrogate trick to get rid of the discontinuous objective function, as we used in Section~\ref{sec: surrogate formulation}. Thus, by replacing replaces false-positive and false-negative counts in the objective function with their surrogate counterparts we get
\begin{mini}{\bm{w}}{
  \frac{1}{2} \norm{\bm{w}}^2 + \frac{1}{\nneg}\fps(\bm{s}, t) + \frac{1}{\npos} \fns(\bm{s}, t)
  }{\label{eq: grill}}{}
  \addConstraint{s_i}{= f(\bm{x}_i; \bm{w}), \quad i \in \I}
  \addConstraint{t}{= \max \Set{t}{\frac{1}{n} \sum_{i \in \I} \Iverson{s_i \geq t} \geq \tau}.}{~}
\end{mini}
This formulation falls into our framework~\eqref{eq: aatp surrogate} for~$\lambda_1 = \nicefrac{1}{\nneg}$ and~$\lambda_2 = \nicefrac{1}{\npos}$. Even though the original formulation is presented in~\cite{boyd2012accuracy}, we denote the previous formulation as \Grill based on the name of the first author of~\cite{grill2016learning}. There are two reasons for that. The first one is that we used an unbalanced dataset as in~\cite{grill2016learning}. The second one is that we use an algorithm proposed in~\cite{grill2016learning} for numerical experiments since the one from~\cite{boyd2012accuracy} is suitable only for a small dataset.

The \Grill formulation~\eqref{eq: grill} is still challenging to solve due to the form of the decision threshold~\eqref{eq: aatp quantile}. The authors of~\cite{boyd2012accuracy} removed the necessity to handle the difficult quantile constraint by setting quantile as one of the samples and solving~$\nall$ independent problems. However, such an approach is infeasible for a large number of samples. The authors of~\eqref{eq: grill} proposed the projected gradient descent method, where after each gradient step, the quantile is recomputed. This approach is suitable for large data but lacks theoretical guarantees. In the following text, we propose two approximations of the true quantile~\eqref{eq: aatp quantile} that can be used to simplify formulation~\eqref{eq: grill}. The first one is a simple approximation by the mean of~$n\tau$ highest scores
\begin{equation}\label{eq: aatp quantile mean} 
  t_2(\bm{s}) = \frac{1}{n\tau} \sum_{i=1}^{n\tau} s_{[i]}.
\end{equation}
where for simplicity we assume, that~$n\tau$ is an integer. The main purpose of~\eqref{eq: aatp quantile mean} is to provide a convex approximation of the non-convex quantile~\eqref{eq: aatp quantile}. In fact, it is known is that it is the tightest convex approximation of~\eqref{eq: aatp quantile}. Putting~\eqref{eq: aatp quantile mean} into the constraint results in the following problem, which we call \TopMeanK
\begin{mini}{\bm{w}}{
  \frac{1}{2} \norm{\bm{w}}^2 + \frac{1}{\npos} \fns(\bm{s}, t)
  }{\label{eq: topmeank}}{}
  \addConstraint{s_i}{= f(\bm{x}_i; \bm{w}), \quad i \in \I}
  \addConstraint{t}{= \frac{1}{K} \sum_{i=1}^{K} s_{[i]},}
\end{mini}
where~$K = n\tau.$ Besides changing the form of the decision threshold, we also simplified the objective function. This change allows preserving the convexity of the formulation for the linear model as shown in Section~\ref{sec: convexity}. The resulting formulation is very similar to the \TopPushK formulation from the previous section. The only difference is that the threshold for \TopMeanK is computed from scores of all samples and not only from the negative ones.

The second option how to approximate the true quantile is to use surrogate counterparts to replace true counts in~\eqref{eq: aatp quantile} and solve the following equality
\begin{equation}\label{eq: aatp quantile surrogate}
  t_3(\bm{s}) \quad \text{solves} \quad \frac{1}{n}\sum_{i \in \I} l\Brac{\vartheta(s_i - t)} = \tau, 
\end{equation}
where~$\vartheta > 0$ is scaling parameter. Since this threshold uses the surrogate approximation, we denote it as surrogate top $\tau$-quantile. We get the following formulation by replacing the true quantile in the constrain and simplifying the objective function
\begin{mini}{\bm{w}}{
  \frac{1}{2} \norm{\bm{w}}^2 + \frac{1}{\npos} \fns(\bm{s}, t)
  }{\label{eq: patmat}}{}
  \addConstraint{s_i}{= f(\bm{x}_i; \bm{w}), \quad i \in \I}
  \addConstraint{t}{\quad \text{solves} \quad \frac{1}{n}\sum_{i \in \I} l\Brac{\vartheta(s_i - t)} = \tau.}
\end{mini}
This formulation also used only false negatives in the objective to preserve the convexity for the linear model. In such a case, the formulation is easily solvable due to the convexity and requires almost no tuning. Together with the fact that formulation~\eqref{eq: patmat} provides a good approximation to the Accuracy at the Top problem, we named it \PatMat (Precision At the Top \& Mostly Automated Tuning).

\section{Neyman-Pearson problem}\label{sec: Neyman-Pearson}

The last problem that we introduce in Section~\ref{sec: related problems} is the Neyman-Pearson problem, which is closely related to hypothesis testing. The hypothesis testing operates with null~$H_0$ and alternative~$H_1$ hypotheses. The goal is to decide to either reject the null hypotheses in favor of the alternative or not reject it. Since this problem is binary, two possible errors can occur. Type I occurs when~$H_0$ is true but is rejected, and Type II error happens when~$H_0$ is false but fails to be rejected. The Neyman-Pearson problem minimizes Type II error while keeping Type I error smaller than some predefined bound. Using our notation for the Neyman-Pearson problem, the null hypothesis~$H_0$ states that sample~$\bm{x}$ has a negative label. Then Type I error occurs when the sample is false-positive, while Type II error when the sample is false-negative, see Table~\ref{tab: classification metrics}. In other words, Type II corresponds to the false-negative rate, and Type I error false-positive rate. Therefore, if the bound on the Type I error is~$\tau$, we may write this as
\begin{equation}\label{eq: NP quantile}
  t_1^{\rm NP}(\bm{s})
    = \max \Set{t}{\frac{1}{\nneg} \sum_{i \in \Ineg} \Iverson{s_i \geq t} \geq \tau}.
\end{equation}
Note that we only count the false-positive samples in~\eqref{eq: NP quantile} instead of counting all positives in~\eqref{eq: aatp quantile}. Then, we may write the Neyman-Pearson problem as
\begin{mini}{\bm{w}}{
  \frac{1}{\npos} \fn(\bm{s}, t)
  }{\label{eq: NP problem}}{}
  \addConstraint{s_i}{= f(\bm{x}_i; \bm{w}), \quad i \in \I}
  \addConstraint{t}{= \max \Set{t}{\frac{1}{\nneg} \sum_{i \in \Ineg} \Iverson{s_i \geq t} \geq \tau}.}{~}
\end{mini}
This problem falls within our framework for~\eqref{eq: aatp counts} for~$\lambda_1 = 0$ and~$\lambda_2 = \nicefrac{1}{\npos}$. Moreover, formulation~\eqref{eq: NP problem} differs from~\eqref{eq: aatp} by two things. The first one is the absence of a false-positive rate in the objective function. The second one is that the threshold is computed from negative samples only. Therefore, we can use the same techniques to approximate both objective function and the decision threshold.

To follow the previous section, we first derive the Neyman-Pearson alternative to the \Grill formulation. We need to add false-positive counts in the objective function to do that. Moreover, we also need to replace true counts with their surrogate counterparts and add the regularization. The resulting formulation is as follows
\begin{mini}{\bm{w}}{
  \frac{1}{2} \norm{\bm{w}}^2 + \frac{1}{\nneg}\fps(\bm{s}, t) + \frac{1}{\npos} \fns(\bm{s}, t)
  }{\label{eq: grill np}}{}
  \addConstraint{s_i}{= f(\bm{x}_i; \bm{w}), \quad i \in \I}
  \addConstraint{t}{= \max \Set{t}{\frac{1}{\nneg} \sum_{i \in \Ineg} \Iverson{s_i \geq t} \geq \tau}.}{~}
\end{mini}
We denote this formulation as \GrillNP to emphasize the relation with the original \Grill formulation and the Neyman-Pearson problem.

The second formulation~\eqref{eq: topmeank} from the previous section, uses mean of~$\nall \tau$ highest scores to approximate true quantile~\eqref{eq: aatp quantile}. In the same way, we can approximate true quantile~\eqref{eq: NP quantile} by the mean of~$\nneg\tau$ highest of scores corresponding to the negative samples
\begin{equation}\label{eq: np quantile mean} 
  t_2^{\rm NP}(\bm{s}) = \frac{1}{\nneg\tau} \sum_{i=1}^{\nneg\tau} s^-_{[i]}.
\end{equation}
For simplicity, we again assume that~$\nneg\tau$ is an integer. Putting~\eqref{eq: np quantile mean} into the constraint results in the Neyman-Pearson alternative to \TopMeanK defined as
\begin{mini}{\bm{w}}{
  \frac{1}{2} \norm{\bm{w}}^2 + \frac{1}{\npos} \fns(\bm{s}, t)
  }{\label{eq: tau-fpl}}{}
  \addConstraint{s_i}{= f(\bm{x}_i; \bm{w}), \quad i \in \I}
  \addConstraint{t}{= \frac{1}{\nneg\tau} \sum_{i=1}^{\nneg\tau} s^-_{[i]}.}
\end{mini}
This problem already appeared in~\cite{zhang2018tau} under the name \tauFPL. Formulation~\eqref{eq: tau-fpl} has almost the same form as formulation~\eqref{eq: topmeank}. The only difference is that for \tauFPL we have~$K = \nneg \tau$ while for \TopPushK, the value of~$K$ is small. Thus, even though we started from two different problems, we arrived at two approximations that differ only in the value of one parameter. 
This slight difference shows a close relationship between the ranking problems and the Neyman-Pearson problem and the need for a unified theory to handle these problems.

The last formulation~\eqref{eq: patmat} from the previous sections uses the surrogate approximation of the true quantile~\eqref{eq: aatp quantile}. The surrogate approximation of the true quantile~\eqref{eq: NP quantile} reads
\begin{equation}\label{eq: np quantile surrogate}
  t_3^{\rm NP}(\bm{s}) \quad \text{solves} \quad \frac{1}{\nneg}\sum_{i \in \Ineg} l\Brac{\vartheta(s_i - t)} = \tau. 
\end{equation}
Putting~\eqref{eq: np quantile surrogate} into the constraint results in the Neyman-Pearson alternative to \PatMat in the following form
\begin{mini}{\bm{w}}{
  \frac{1}{2} \norm{\bm{w}}^2 + \frac{1}{\npos} \fns(\bm{s}, t)
  }{\label{eq: patmat np}}{}
  \addConstraint{s_i}{= f(\bm{x}_i; \bm{w}), \quad i \in \I}
  \addConstraint{t}{\text{solves} \quad \frac{1}{\nneg}\sum_{i \in \Ineg} l\Brac{\vartheta(s_i - t)} = \tau,}
\end{mini}
We call this formulation \PatMatNP to stress the similarity with \PatMat. The only difference between these two formulations is that only negative samples are involved in computing the decision threshold for \PatMatNP, while \PatMat uses all samples.

\section{Summary}

\todo[inline]{Add summary of the framework and formulations that fall into the framework}

\begin{table}
  \centering
  \begin{NiceTabular}{lcccccc}
    \CodeBefore
      \rowcolor{\headercol}{1}
      \rowcolors{3}{\rowcol}{}[restart]
    \Body
    \toprule
    \Block[c]{1-1}{\textbf{Formulation}}
      & \textbf{Label}
      & \textbf{Source}
      & \textbf{Ours}
      & $\lambda_1$
      & $\lambda_2$
      & \textbf{Threshold} \\
    \midrule
    \TopPush
      & \eqref{eq: toppush surrogate}
      & \cite{li2014top}
      & \nomark
      & 0
      & $\frac{1}{\npos}$
      & $s_{[1]}^-$ \\
    \TopPushK
      & \eqref{eq: toppushK surrogate}
      & \cite{adam2021general}
      & \yesmark
      & 0
      & $\frac{1}{\npos}$
      & $\sum_{i = 1}^{K} s_{[i]}^-$ \\
    \midrule
    \Grill
      & \eqref{eq: grill}
      & \cite{grill2016learning}
      & \nomark
      & $\frac{1}{\nneg}$
      & $\frac{1}{\npos}$
      & $\max \Set{t}{\frac{1}{n} \sum_{i \in \I} \Iverson{s_i \geq t} \geq \tau}$ \\
    \TopMeanK
      & \eqref{eq: topmeank}
      & ---
      & \nomark
      & 0
      & $\frac{1}{\npos}$
      & $\frac{1}{K} \sum_{i=1}^{K} s_{[i]}$ \\
    \PatMat
      & \eqref{eq: patmat}
      & \cite{adam2021general}
      & \yesmark
      & 0
      & $\frac{1}{\npos}$
      & $\frac{1}{n} \sum_{i \in \I} l\Brac{\vartheta(s_i - t)} = \tau$ \\
    \midrule
    \GrillNP
      & \eqref{eq: grill np}
      & ---
      & \nomark
      & $\frac{1}{\nneg}$ 
      & $\frac{1}{\npos}$
      & $\max \Set{t}{ \frac{1}{\nneg} \sum_{i \in \Ineg} \Iverson{s_i \geq t} \geq \tau}$ \\
    \tauFPL
      & \eqref{eq: tau-fpl}
      & \cite{zhang2018tau}
      & \nomark
      & 0
      & $\frac{1}{\npos}$
      & $\frac{1}{\nneg\tau} \sum_{i=1}^{\nneg\tau} s^-_{[i]}$ \\
    \PatMatNP
      & \eqref{eq: patmat np}
      & \cite{adam2021general}
      & \yesmark
      & 0
      & $\frac{1}{\npos}$
      & $\frac{1}{\nneg} \sum_{i \in \Ineg} l\Brac{\vartheta(s_i - t)} = \tau$ \\
    \bottomrule
  \end{NiceTabular}
  \caption{Summary of problem fomrulations that fall in the framework~\eqref{eq: aatp surrogate}. Column \textbf{Formulation} shows the name of the formulation that we use in this work. Column \textbf{Label} represents the label of the formulation in this text. Column \textbf{Source} is the citation of the work where the formulation was introduced. Column \textbf{Ours} shows whether the formulation was introduced in any of our previous papers. The last three columns show the values of parameters~$\lambda_1,$~$\lambda_2$ and the form of the decision threshold for framework~\eqref{eq: aatp surrogate}.}
  \label{tab: summary formulations}
\end{table}
\chapter{Primal Formulation: Linear Model}\label{chap: linear}

In the previous chapter, we introduced the general framework for binary classification at the top. Table~\ref{tab: summary formulations} summarizes all formulations that fall into this framework. In this chapter, we focus on the particular case when the model~$f$ is linear, i.e., the model is in the following form
\begin{equation*}
  f(\bm{x}; \bm{w}) = \bm{w}^{\top} \bm{x},
\end{equation*}
where~$\bm{w} \in \R^d$ is the normal vector to the separating hyperplane. In such a case, framework~\eqref{eq: aatp surrogate} simplifies into the form below
\begin{mini*}{\bm{w}}{
  \frac{1}{2} \norm{\bm{w}}^2 + \lambda_1 \cdot \fps(\bm{s}, t) + \lambda_2 \cdot \fns(\bm{s}, t)
  }{}{}
  \addConstraint{s_i}{= \bm{w}^{\top} \bm{x}_i, \quad i \in \I}
  \addConstraint{t}{= G\Brac{\bm{s}, \bm{y}}.}
\end{mini*}
In the upcoming sections, we provide a theoretical analysis of this unified framework using linear model. We consider the problem formulations from Chapter~\ref{chap: framework} and not individual algorithms which specify how to solve these formulations. The theoretical properties we mainly focus on are as follows:
\begin{itemize}
  \item \textit{Convexity} implies a guaranteed convergence for many optimization algorithms or their better convergence rates~\cite{boyd2004convex}.
  \item \textit{Differentiability} increases the speed of convergence.
  \item \textit{Stability} is a general term, by which we mean that the global minimum is not at~$\bm{w} = \bm{0}$. This actually is the case for many formulations from Table~\ref{tab: summary formulations} and results in the situation where the separating hyperplane is degenerate and does not actually exist.
\end{itemize}
We show the results only for formulations from Section~\ref{sec: ranking} and~\ref{sec: aatp} for better readability. Formulations in Section~\ref{sec: aatp} are mostly identical to the ones from Section~\ref{sec: Neyman-Pearson}. The only difference is that all formulations in Section~\ref{sec: aatp} compute the decision threshold from all samples, while formulations in Section~\ref{sec: Neyman-Pearson} use only negative samples. Therefore, the results for both sections are identical, and we show only the ones for Section~\ref{sec: aatp}. 

\section{Convexity}\label{sec: convexity}

Convexity is one of the most important properties in numerical optimization. It ensures that the optimization problem has neither stationary points nor local minima. All points of interest are global minima. Moreover, it allows for faster convergence rates. This section shows that some of the formulations from Table~\ref{tab: summary formulations} are convex and, therefore, easier to solve. The first result is summarized in the following proposition. Note that we denote the thresholds as functions of weights~$\bm{w}.$ This dependence holds since the thresholds are defined in Section~\ref{sec: aatp} as functions of scores~$\bm{s}.$

\begin{restatable}{proposition}{propconvex}\label{prop: convexity}
  Consider fixed vector of scores~$\bm{s}$ with elements defined as~$s_i = \bm{w}^{\top} \bm{x}_i$ for all~$i \in \I.$ Moreover, consider thresholds for \TopPush, \Grill, \TopMeanK and \PatMat from Section~\ref{sec: ranking} and~\ref{sec: aatp} defined as
  \begin{align*}
    t_0(\bm{w}) &
      = s_{[1]}^-, &
    t_1(\bm{w}) &
      = \max \Set{t}{\frac{1}{\nall} \sum_{i \in \I} \Iverson{s_i \geq t} \geq \tau}, \\
    t_2(\bm{w}) &
      = \frac{1}{K} \sum_{i=1}^{K} s_{[i]}, &
    t_3(\bm{w}) &
      \;\; \text{solves} \;\; \frac{1}{\nall} \sum_{i \in \I} l\Brac{\vartheta(s_i - t)} = \tau,
  \end{align*}
  Then thresholds~$t_0$, ~$t_2$ and~$t_3$ are convex functions of weights~$\bm{w},$ while the threshold~$t_1$ is non-convex.
\end{restatable}

The proposition says that \Grill formulation uses non-convex threshold while \TopPush, \TopMeanK, and \PatMat use the convex ones. Moreover, the thresholds for \tauFPL and \TopPushK are convex since both formulations use almost the same threshold as \TopMeanK. The same holds for the thresholds of \PatMat and \PatMatNP formulations. Notice that all formulations that have a convex threshold use the same objective function.

\begin{restatable}{theorem}{thmconvex}\label{thm: convexity}
  If the threshold~$t = t(\bm{w})$ is a convex function of weights~$\bm{w}$, then function
  \begin{equation*}
    L(\bm{w}) = \fns(\bm{s}, t) = \sum_{i \in \Ipos}l(t - \bm{w}^{\top} \bm{x}_i)
  \end{equation*}
  is convex.
\end{restatable}

While the proof of Theorem~\ref{thm: convexity} is simple, it points to the necessity of considering only false-negatives in the objective function. Due to this theorem, almost all formulations from Table~\ref{tab: summary formulations} are convex optimization problems. There are only two exceptions: \Grill and \GrillNP are not convex problems.

\section{Differentiability}

Similar to convexity, differentiability is crucial for improving the convergence rate. Moreover, differentiability can often be used to derive better termination criteria for numerical algorithms. The next theorem shows which formulations from Table~\ref{tab: summary formulations} are differentiable.

\begin{restatable}{theorem}{derivative}\label{thm: differentiability}
  Consider thresholds from Proposition~\ref{prop: convexity}.  Threshold~$t_0,$~$t_1$ and~$t_2$ are non-differentiable functions of weights~$\bm{w}.$ Moreover, if the surrogate function~$l$ is differentiable, threshold~$t_3$ is a differentiable function of weights~$\bm{w},$ and its derivative equals
  \begin{equation*}
    \nabla t_3(\bm{w}) = \frac{
      \sum_{i \in \I} l'\Brac{\vartheta (\bm{w}^{\top} \bm{x}_i - t_3(\bm{w}))}\bm{x}_i
      }{
        \sum_{j \in \I} l'\Brac{\vartheta (\bm{w}^{\top} \bm{x}_j - t_3(\bm{w}))}
      }.
  \end{equation*}
\end{restatable}

Due to the previous theorem and Theorem~\ref{thm: convexity}, only \PatMat, and \PatMatNP are convex and differentiable optimization problems. These properties allow us to prove the convergence of the stochastic gradient descent for these two formulations, as shown in Section~\ref{sec:convergence}.

\section{Stability}\label{sec: stability}

We first provide a simple example and show that many formulations from Table~\ref{tab: summary formulations} are degenerate for it. Then we analyze general conditions under which this degenerate behavior happens.

\begin{restatable}[Degenerate Behaviour]{example}{degeneratebehavior}\label{ex: degenerate behaviour}
  Consider~$n$ negative samples uniformly distributed in~$[-1,0]\times[-1,1]$,~$n$ positive samples uniformly distributed in~$[0,1]\times[-1,1]$ and one negative sample at~$(2,0).$ An illustration of such settings is provided in Figure~\ref{fig: degenerate behaviour} (\textbf{left}). If~$n$ is large enough, the point at~$(2,0)$ is an outlier and the problem is (almost) perfectly separable using the separating hyperplane with normal vector~$\bm{w}_1 = (1, 0)$. 
\end{restatable}

There are two important solutions for Example~\ref{ex: degenerate behaviour}. The first is the optimal solution~$\bm{w}_1=(1,0),$ which generates the optimal separating hyperplane. The second is~$\bm{w}_0=(0,0),$ a degenerate solution that does not generate any separating hyperplane. Since the dataset is perfectly separable by $\bm{w}_1$, we expect that~$\bm{w}_1$ provides a lower value of the objective function than~$\bm{w}_0$ for all formulations from Table~\ref{tab: summary formulations}. However, it is not happening. Table~\ref{tab: example} shows the threshold~$t$ and the value of the objective function~$L$ for~$\bm{w}_0$ and~$\bm{w}_1.$ For the precise computation of the results, see Appendix~\ref{app: stability}. By highlighting the better objective in Table~\ref{tab: example} by green, we see that \TopPush and \TopMeanK has a better objective in~$\bm{w}_0.$ It can be shown that~$\bm{w}_0$ is even the global minimum for these two formulations. This situation raises the question whether some tricks, such as early stopping or excluding a small ball around zero, cannot overcome this difficulty. The answer is negative, as shown in Figure~\ref{fig: degenerate behaviour} (\textbf{right}). Here, we run \TopPush with hinge loss as a surrogate and no regularization from several starting points. In all cases, \TopPush converges to zero from one of the three possible directions, and all these directions are far from the normal vector to the separating hyperplane.

\begin{figure}
  \centering
  \includegraphics[width=\linewidth]{images/toppush_convergence.pdf}
  \caption{Distribution of positive (red circles) and negative samples (blue circles) for the example from Example~\ref{ex: degenerate behaviour}. (\textbf{left}) Contour plot of the objective function value for \TopPush with hinge loss as a surrogate and no regularization and its convergence (orange lines) to the zero vector from~$12$ different initial points. (\textbf{right})}
  \label{fig: degenerate behaviour}
\end{figure}

\begin{table}
  \centering
  \begin{NiceTabular}{lccccc}
    \CodeBefore
      \rowcolor{\headercol}{1-2}
      \rowcolors{4}{\rowcol}{}[restart]
    \Body
    \toprule
    \Block[c]{2-1}{\textbf{Formulation}}
      & \Block{2-1}{\textbf{Label}}
      & \Block{1-2}{$\bm{w}_0=(0,0)$}
      && \Block{1-2}{$\bm{w}_1=(1,0)$} \\
    \cline{3-6}
    & & $t$
      & $L$
      & $t$
      & $L$ \\
    \midrule
    \TopPush
      & \eqref{eq: toppush surrogate}
      & $0$
      & \Block[fill=mygreen!50]{1-1}{$1$}
      & $2$
      & $2.5$ \\
    \TopPushK
      & \eqref{eq: toppushK surrogate}
      & $0$
      & $1$
      & $\frac{2}{K}$
      & \Block[fill=mygreen!50]{1-1}{$0.5 + \frac{2}{K}$} \\
    \midrule
    \Grill
      & \eqref{eq: grill}
      & $0$
      & $2$
      & $1-2\tau$
      & \Block[fill=mygreen!50]{1-1}{$1.5+2\tau(1-\tau)$} \\
    \TopMeanK
      & \eqref{eq: topmeank}
      & $0$
      & \Block[fill=mygreen!50]{1-1}{$1$}
      & $1-\tau$
      & $1.5-\tau$ \\
    \PatMat
      & \eqref{eq: patmat}
      & $\frac{1}{\vartheta}(1-\tau)$
      & $1+\frac{1}{\vartheta}(1-\tau)$
      & $\frac{1}{\vartheta}(1-\tau)$
      & \Block[fill=mygreen!50]{1-1}{$0.5 + \frac{1}{\vartheta}(1-\tau)$} \\
      \midrule
      \GrillNP
        & \eqref{eq: grill np}
        & $0$
        & $2$
        & $-\tau$
        & $???$ \\
      \tauFPL
        & \eqref{eq: tau-fpl}
        & $0$
        & $1$
        & $\frac{2}{\tau\nneg}$
        & \Block[fill=mygreen!50]{1-1}{$0.5 + \frac{2}{\tau\nneg}$} \\
      \PatMatNP
        & \eqref{eq: patmat np}
        & $\frac{1}{\vartheta}(1-\tau)$
        & $1+\frac{1}{\vartheta}(1-\tau)$
        & $\frac{1}{\vartheta}(1-\tau) - 0.5$
        & \Block[fill=mygreen!50]{1-1}{$\frac{1}{\vartheta}(1-\tau)$} \\
    \bottomrule
  \end{NiceTabular}
  \caption{Comparison of formulations from Table~\ref{tab: summary formulations} on the problem from Example~\ref{ex: degenerate behaviour}. The table shows the threshold and the objective function value for two solutions: the optimal solution~$\bm{w}_1=(1,0)$ and degenerate solution~$\bm{w}_0=(0,0).$ Three formulations have the global minimum (denoted by green color) at~$\bm{w}_0,$ which does not generate any separating hyperplane.}
  \label{tab: example}
\end{table}

The convexity derived in the previous section guarantees that there are no local minima. However, as we showed in the example above, the global minimum may be at~$\bm{w} = \bm{0}$. Such a situation is highly undesirable since~$\bm{w}$ is the normal vector to the separating hyperplane, and the zero vector provides no information. In the rest of the section, we analyze when this situation happens. Theorem~\ref{thm:large_t} states that if the decision threshold~$t = t(\bm{w})$ is above a certain value, then~$\bm{0}$ has a better (lower) objective than~$\bm{w}$. If this happens for all~$\bm{w}$, then~$\bm{0}$ is the global minimum.

\begin{restatable}{theorem}{larget}\label{thm:large_t}
  Consider any of these formulations: \TopPush, \TopPushK, \TopMeanK or \tauFPL. Fix any~$\bm{w}$ and denote the corresponding objective function~$L(\bm{w})$ and threshold~$t(\bm{w})$. If we have
  \begin{equation}\label{eq:w_zero_nn}
    t(\bm{w})\geq \frac{1}{\npos} \sum_{i \in \Ipos} \bm{w}^{\top} \bm{x}_i,
  \end{equation}
  then~$L(\bm{0}) \leq L(\bm{w})$. Specifically, using Notation~\ref{not: scores} we get the following implications
  \begin{align*}\label{eq:w_zero}
    s_{[1]}^- \geq \frac{1}{\npos} \sum_{i=1}^{\npos} s_{i}^+ \quad
      & \implies \quad L(\bm{0}) \leq L(\bm{w}) \text{ for } \TopPush, \\
    \frac{1}{K}\sum_{i=1}^K s_{[i]}^- \geq \frac{1}{\npos} \sum_{i=1}^{\npos} s_{i}^+  \quad
      & \implies \quad L(\bm{0}) \leq L(\bm{w}) \text{ for } \TopPushK, \\
    \frac{1}{K} \sum_{i=1}^{K} s_{[i]} \geq \frac{1}{\npos} \sum_{i=1}^{\npos} s_{i}^+ \quad
      & \implies \quad L(\bm{0}) \leq L(\bm{w}) \text{ for } \TopMeanK, \\
    \frac{1}{\nneg\tau} \sum_{i=1}^{\nneg\tau} s_{[i]}^- \geq \frac{1}{\npos}\sum_{i=1}^{\npos} s_{i}^+ \quad
      & \implies \quad L(\bm{0}) \leq L(\bm{w}) \text{ for } \tauFPL. \\
  \end{align*}
\end{restatable}

The proof of Theorem~\ref{thm:large_t} employs the fact that all formulations in the theorem statement have only false-negatives in the objective. If we use the zero solution~$\bm{w}_0=\bm{0},$ all classification scores~$s_i$ are equal to zero, the threshold~$t$ equals zero, and the objective function~$L$ equals one. On the other hand, if the threshold~$t$ is large, many positive samples have scores below the threshold, and the false-negatives samples have the average surrogate value larger than one. In such a case,~$\bm{w}_0 = \bm{0}$ becomes the global minimum for some formulations. More specifically, \TopPush fails if there are outliers, and \TopMeanK fails whenever there are many positive samples.

\begin{corollary}\label{cor:toppush}
  Consider the \TopPush formulation. If positive samples lie in the convex hull of negative samples, then~$\bm{w}=\bm{0}$ is the global minimum.
\end{corollary}

\begin{corollary}\label{cor:topmean}
  Consider the \TopMeanK formulation. If~$\npos\geq n\tau$, then~$\bm{w}=\bm{0}$ is the global minimum.
\end{corollary}

There are two fixes to the situation described above:
\begin{itemize}
  \item Include false-positives to the objective. This approach is taken by \Grill and \GrillNP and necessarily results in the loss of convexity as shown in Section~\ref{sec: convexity}.
  \item Move the threshold away from zero even when all scores~$\bm{s}$ are zero. This approach is taken by our formulations \PatMat and \PatMatNP and keeps convexity.
\end{itemize}
The following theorem shows the advantage of the second approach.

\begin{restatable}{theorem}{patmatzero}\label{thm:patmat_zero}
  Consider the \PatMat or \PatMatNP formulation with the hinge loss as a surrogate and no regularization. Assume that for some~$\bm{w}$ we have
  \begin{equation}\label{eq:patmat_zero}
    \frac{1}{\npos}\sum_{i \in \Ipos}\bm{w}^{\top} \bm{x}_i > \frac{1}{\nneg}\sum_{j \in \Ineg}\bm{w}^{\top} \bm{x}_j.
  \end{equation}
  Then there exists a scaling parameter~$\vartheta_0$ for the surrogate top $\tau$-quantile~\eqref{eq: aatp quantile surrogate} or~\eqref{eq: np quantile surrogate} such that~$L(\bm{w}) < L(\bm{0})$ for all~$\vartheta \in (0, \vartheta_0)$.
\end{restatable}

This theorem shed some light on the behavior of the formulations. Theorem~\ref{thm:large_t} states that the stability of \tauFPL requires
\begin{equation}\label{eq:stability1}
  \frac{1}{\nneg\tau}\sum_{i=1}^{\nneg\tau}s_{[i]}^- < \frac{1}{\npos}\sum_{i=1}^{\npos} s_{i}^+,
\end{equation}
while Theorem~\ref{thm:patmat_zero} states that the stability of \PatMatNP is ensured by
\begin{equation}\label{eq:stability2}
  \frac{1}{\nneg}\sum_{i=1}^{\nneg}s_{[i]}^- < \frac{1}{\npos}\sum_{i=1}^{\npos} s_{i}^+.
\end{equation}

Consequently, if \tauFPL is stable, then~\eqref{eq:stability1} is satisfied. The right-hand sides of~\eqref{eq:stability1} and \eqref{eq:stability2} are the same, while the left-hand side of~\eqref{eq:stability2} is always smaller than the left-hand side of~\eqref{eq:stability1}. This means that whenever~\eqref{eq:stability1} is satisfied, \eqref{eq:stability2} is also satisfied. Thus, if \tauFPL is stable, \PatMatNP is stable as well. At the same time, there may be a considerable difference in the stability of both formulations. Since the scores of positive samples should be above the scores of negative samples, the scores~$\bm{s}$ may be interpreted as performance. Then formula~\eqref{eq:stability1} states that if the mean performance of a small number of the worst negative samples is larger than the average performance of all positive samples, then \tauFPL fails. On the other hand, formula~\eqref{eq:stability2} states that if the average performance of all positive samples is better than the average performance of all negative samples, then \PatMatNP is stable. The former may well happen as accuracy at the top is interested in a good performance of only a few positive samples.

In the same way, it can be shown that the stability of \TopMeanK implies the stability of \PatMat.

\section{Stochastic Gradient Descent}\label{sec:convergence}

In the previous section, we analyzed the formulations from Table~\ref{tab: summary formulations}, but we did not consider any optimization algorithms. In this section we show a basic version of the stochastic gradient descent and then its convergent version. Due to considering the threshold, the gradient computed on a minibatch is a biased estimate of the true gradient. Therefore we need to use variance reduction techniques similar to SAG~\cite{schmidt2017minimizing}, and the proof is rather complex.

Many optimization algorithms for solving the formulations from Table~\ref{tab: summary formulations} use primal-dual or purely dual formulations. Authors of~\cite{eban2017scalable} introduced dual variables and used alternating optimization to the resulting min-max problem. In~\cite{li2014top, zhang2018tau}, authors dualized the problem and solved it with the steepest gradient ascent. Authors of~\cite{macha2020nonlinear} followed the same path but added kernels to handle non-linearity. We follow the ideas of~\cite{mackey2018constrained} and~\cite{adam2019machine} and solve the problems directly in their primal formulations. Therefore, even though we use the same formulation for \TopPush as~\cite{li2014top} or for \tauFPL as~\cite{zhang2018tau}, our solution process is different. However, both algorithms should converge to the same point due to convexity.

For the convergence proof of stochastic gradient descent, we need differentiability. Due to Theorem~\ref{thm: differentiability}, we have only two formulations that are differentiable: \PatMat and \PatMatNP. Therefore, in the rest of the section, we consider only these two formulations. For simplicity, we show the proof only for \PatMat. The proof for \PatMatNP is almost the same.

The decision in variable \PatMat formulation~\eqref{eq: patmat} is the normal vector of the separating hyperplane~$\bm{w}.$ Therefore, the gradient descent algorithm uses the following rule
\begin{equation*}
  \bm{w}^{k+1} \gets \bm{w}^k - \alpha^k \cdot \nabla L(\bm{w}^k),
\end{equation*}
where~$k \in \N$ denotes iteration,~$\alpha^k$ is a step size, and~$\nabla L$ is a gradient of the objective function. Since the decision threshold~$t$ depends on~$\bm{w},$ we need to use the chain rule to compute the gradient of the objective function. For each~$\bm{w}$, the threshold~$t$ can be computed uniquely as discussed in Section~\ref{sec: patmat threshold alg}. We stress this dependence by writing~$t(\bm{w})$ instead of~$t$. Note that the convexity is preserved. Then we can compute the derivative via the chain rule
\begin{equation}\label{eq:derivatives}
  \begin{aligned}
  L(\bm{w})
    & = \frac{1}{2}\norm{\bm{w}}^2 + \frac{1}{\npos} \sum_{i \in \Ipos} l(t(\bm{w}) - \bm{w}^{\top} \bm{x}_i) , \\
  \nabla L(\bm{w})
    & = \bm{w} + \frac{1}{\npos} \sum_{i \in \Ipos} l'(t(\bm{w}) - \bm{w}^{\top} \bm{x}_i)(\nabla t(\bm{w}) - \bm{x}_i).
  \end{aligned}
\end{equation}
The only unknown part is the computation of~$\nabla t(\bm{w})$. Theorem~\ref{thm: differentiability} shows the computation for \PatMat with efficient computation method presented in Section~\ref{sec: patmat threshold alg}. Since we derive the gradient for the objective function in~\eqref{eq:derivatives}, it is easy to derive how to apply the stochastic gradient descent. We only have to partition the dataset into minibatches and provide an update of the weights~$\bm{w}$ based only on a minibatch, namely by replacing the mean over the whole dataset in~\eqref{eq:derivatives} by a mean over the minibatch. 

Even though we focus only on the \PatMat formulation, the relations~\eqref{eq:derivatives} can be used for almost all formulations from Table~\ref{tab: summary formulations}. The only two exceptions are \Grill and \GrillNP, which use a slightly different objective function. However, the form on the gradient of the objective function for these two formulations is very similar. Nevertheless, the rest of the section that shows the convergence of the stochastic gradient is applicable only for \PatMat.

Consider piecewise disjoint minibatches~$\Imb^1, \; \Imb^2, \ldots, \; \Imb^m$ which cycle periodically, i.e., for all~$k$ we have~$\Imb^{k+m} = \Imb^k$. At iteration~$k$ we have the decision variable~$\bm{w}^k$ and the active minibatch~$\Imb^k$. First, we update the vector of scores~$\bm{s}^k$ only on the active minibatch by setting
\begin{equation}\label{eq:defin_z}
  s_i^k = \begin{cases}
    \bm{x}_i^\top \bm{w}^k & \text{for all } i \in \Imb^k, \\
    s_i^{k-1} & \text{otherwise.}
  \end{cases} 
\end{equation}
We keep scores from previous minibatches intact. We use Section~\ref{sec: patmat threshold alg} to compute the surrogate quantile~$t^k$ as the unique solution of
\begin{equation}\label{eq:update_t}
  \sum_{i \in \I}l\Brac{\vartheta\Brac{s_i^k - t^k}} = n\tau.
\end{equation}
This is an approximation of the surrogate quantile~$t(\bm{w}^k)$ from \eqref{eq: aatp quantile surrogate}. The only difference from the true quantile is that we use delayed scores. Then we approximate the derivative~$\nabla L(\bm{w}^k)$ from \eqref{eq:derivatives} by
\begin{equation}\label{eq:update_g}
  g(\bm{w}^k)
    = \bm{w}^k + \frac{1}{\nmbpos^k} \sum_{i \in \Imbpos^k} l'(t^k - s_i^k)(\nabla t^k - \bm{x}_i),
\end{equation}
where~$\nabla t^k$ is an approximation of~$\nabla t(\bm{w}^k)$ from Theorem~\ref{thm: differentiability}. To define the approximation~$\nabla t^k,$ we first need to define and artificial variable
\begin{equation}\label{eq:update_a}
  \bm{a}^k = \sum_{i \in \Imb^k} l'\Brac{\vartheta\Brac{s_i^k - t^k}}\bm{x}_i.
\end{equation}
Note that~$\bm{a}^k$ is an approximation of the numerator of~$\nabla t(\bm{w}^k)$ from Theorem~\ref{thm: differentiability}. However, this approximation uses only the current minibatch. Since our minibatches cycle periodically, we can sum the last $m$ variables~$\bm{a}^k$ and get the approximation computed from all samples and all delayed scores. Then, the approximation of~$\nabla t(\bm{w}^k)$ can be defined as follows
\begin{equation}\label{eq:update_nablat}
  \nabla t^k
    = \frac{\bm{a}^k + \bm{a}^{k-1} + \dots + \bm{a}^{k - m + 1}}{\sum_{i \in \I} l'\Brac{\vartheta\Brac{s_i^k - t^k}}}.
\end{equation}
A perhaps more straightforward possibility would be to consider only~$\bm{a}^k$ in the numerator of~\eqref{eq:update_nablat}. However, presented choice enables us to prove the convergence, and it adds stability to the algorithm for small minibatches.

\begin{restatable}{theorem}{sgd}\label{thm:sgd}
  Consider the \PatMat formulation, stepsizes~$\alpha^k = \frac{1}{k+1}\alpha^0,$ and piecewise disjoint minibatches~$\Imb^1, \; \Imb^2, \ldots, \; \Imb^m$ which cycle periodically~$\Imb^{k+m} = \Imb^k$. If~$l$ is the smoothened hinge function defined by
  \begin{equation}\label{eq: smooth hinge}
    l(s) = \begin{cases}
      0 & \text{for } s < -1 - \varepsilon, \\
      \frac{1}{4\varepsilon}(1 + s + \varepsilon)^2 & \text{for } - 1 - \varepsilon \leq s < - 1 + \varepsilon, \\
      1 + s & \text{otherwise,}
    \end{cases}
  \end{equation}
  where~$\varepsilon > 0,$ then Algorithm~\ref{alg:sgd} converges to the global minimum of \eqref{eq: patmat}.
\end{restatable}

The whole procedure of the stochastic gradient descent for \PatMat formulation is summarized in Algorithm~\ref{alg:sgd}. Note that there are no vector operations outside of the current minibatch~$\Imb^k$. Moreover, note that a proper initialization for the first~$m$ iterations is needed.

\begin{algorithm}
  \begin{algorithmic}[1]
    \Require Dataset~$\mathcal{D}$, minibatches~$\Imb^1,\; \Imb^2, \ldots, \; \Imb^m$, and stepsize~$\alpha^k$
    \State Initialize weights~$\bm{w}^0$
    \For{$k = 0, \; 1, \ldots $}
    \State Select a minibatch~$\Imb^k$
    \State Compute~$s_i^k$ for all~$i \in \Imb^k$ according to \eqref{eq:defin_z}
    \State Compute~$t^k$ according to \eqref{eq:update_t}
    \State Compute~$\bm{a}^k$ according to \eqref{eq:update_a}
    \State Compute~$\nabla t^k$ according to \eqref{eq:update_nablat}
    \State Compute~$g(\bm{w}^k)$ according to \eqref{eq:update_g}
    \State Set~$\bm{w}^{k+1} \gets \bm{w}^k - \alpha^k g(\bm{w}^k)$
    \EndFor
  \end{algorithmic}
  \caption{Stochastic gradient descent for \PatMat formulation}
  \label{alg:sgd}
\end{algorithm}

\section{Summary}

In this chapter, we derived theoretical properties for formulations from Table~\ref{tab: summary formulations} with the linear model. We focused on the convexity, differentiability, and stability of formulations since these three properties are crucial for fast and proper convergence. All results are summarized in Table~\ref{tab: summary formulations properties linear}. We showed that \TopPush, \TopPushK, \TopMeanK, and \tauFPL are convex, but all these formulations are vulnerable to having the global minimum at~$\bm{w}=0$. On the other hand, \Grill and \GrillNP are stable, but they are not convex or differentiable. Finally, our formulations \PatMat and \PatMatNP satisfy all three theoretical properties. 

A similar comparison is performed in Figure~\ref{fig:thresholds}. Methods in green and yellow are convex, while formulations in red are non-convex. Based on Theorem~\ref{thm:large_t}, four formulations in yellow are vulnerable to having the global minimum at~$\bm{w}=0$. This theorem states that the higher the threshold, the more vulnerable the formulation is. The full arrows depict this dependence. If it points from one formulation to another, the latter one has a smaller threshold and thus is less vulnerable to this undesired global minima. The dotted arrows indicate that this usually holds but not always. The precise formulation is provided in Appendix~\ref{sec: threshold comparison aatp}. This complies with Corollaries~\ref{cor:toppush} and~\ref{cor:topmean} which state that \TopPush and \TopMeanK are most vulnerable. At the same time, it says that \tauFPL is the best one from the yellow formulations. Finally, even though \PatMatNP has a worse approximation of the true threshold than \tauFPL due to Lemma~\ref{lemma: threshold comparison NP}, it is more stable due to the discussion after Theorem~\ref{thm:patmat_zero}. Similarly, \PatMat has a worse approximation of the true threshold than \TopMeanK due to Lemma~\ref{lemma: threshold comparison}, but is more stable.

\begin{table}
  \centering
  \begin{NiceTabular}{lccccc}
    \CodeBefore
      \rowcolor{\headercol}{1}
      \rowcolors{3}{\rowcol}{}[restart]
    \Body
    \toprule
    \Block[c]{1-1}{\textbf{Formulation}}
      & \textbf{Label}
      & \textbf{Hyperparameters}
      & \textbf{Convex}
      & \textbf{Differentiable}
      & \textbf{Stable} \\
    \midrule
    \TopPush
      & \eqref{eq: toppush surrogate}
      & ---
      & \yesmark
      & \nomark
      & \nomark \\
    \TopPushK
      & \eqref{eq: toppushK surrogate}
      & $K$
      & \yesmark
      & \nomark
      & \nomark \\
    \midrule
    \Grill
      & \eqref{eq: grill}
      & $\tau$
      & \nomark
      & \nomark
      & \yesmark \\
    \TopMeanK
      & \eqref{eq: topmeank}
      & $\tau$
      & \yesmark
      & \nomark
      & \nomark \\
    \PatMat
      & \eqref{eq: patmat}
      & $\tau$, $\vartheta$
      & \yesmark
      & \yesmark
      & \yesmark \\ 
    \midrule
    \GrillNP
      & \eqref{eq: grill np}
      & $\tau$
      & \nomark
      & \nomark
      & \yesmark \\
    \tauFPL
      & \eqref{eq: tau-fpl}
      & $\tau$
      & \yesmark
      & \nomark
      & \nomark \\
    \PatMatNP
      & \eqref{eq: patmat np}
      & $\tau$, $\vartheta$
      & \yesmark
      & \yesmark
      & \yesmark \\
    \bottomrule
  \end{NiceTabular}
  \caption{Summary of the formulations from Table~\ref{tab: summary formulations}. The second column shows the hyperparameters available for each formulation. The last three columns show whether the formulation is differentiable, convex, and stable (in the sense of having global minimum in~$\bm{w}=\bm{0}$).}
  \label{tab: summary formulations properties linear}
\end{table}

\begin{figure}
  \centering
  \includegraphics[width = \linewidth]{images/methods_relation.pdf}
  \caption{Summary of the formulations from Table~\ref{tab: summary formulations}. Methods in green and yellow are convex, while formulations in red are non-convex. Moreover, methods in yellow are vulnerable to having the global minimum at~$\bm{w}=0$. A full (dotted) arrow pointing from one formulation to another shows that the latter formulation has (usually) a smaller threshold.}
  \label{fig:thresholds}
\end{figure}

\chapter{Dual Formulation: Linear Model}\label{chap: dual}

In Chapter~\ref{chap: framework}, we introduced a general framework for binary classification at the top. Moreover, we showed that several problem classes, considered separate problems so far, fit into this framework. The summary of all formulations is provided in Table~\ref{tab: summary formulations}. In Chapter~\ref{chap: linear} we discussed a special case when the linear model is used. Then formulation~\eqref{eq: aatp surrogate} reads
\begin{mini}{\bm{w}}{
  \frac{1}{2} \norm{\bm{w}}^2 + \lambda_1 \cdot \fps(\bm{s}, t) + \lambda_2 \cdot \fns(\bm{s}, t)
  }{\label{eq: aatp surrogate liner primal}}{}
  \addConstraint{s_i}{= \bm{w}^{\top} \bm{x}_i, \quad i \in \I}
  \addConstraint{t}{= G\Brac{\bm{s}, \bm{y}}.}
\end{mini}
Many formulations have nice theoretical properties such as convexity or differentiability in this specific case. However, many real-world problems are not linearly separable, and in such cases, the approach from the previous section is not sufficient. In this chapter, we use the similarity of~\eqref{eq: aatp surrogate liner primal} to primal formulation of SVM~\cite{cortes1995support} and derive dual forms for all formulations from Table~\ref{tab: summary formulations}. Then we use the kernel method~\cite{scholkopf2001learning} to introduce nonlinearity into the dual formulations. Moreover, as dual problems are generally computationally expensive, we propose an efficient method to solve them.

\section{Derivation of Dual Problems}\label{sec:Derivation of dual problems}

As discussed in the introduction, this section is dedicated to deriving dual forms for all formulations from Table~\ref{tab: summary formulations}. We do not discuss \Grill and \GrillNP formulations in the following text since both formulations are not convex, and therefore their primal and dual formulations are not equivalent. Since many of the remaining formulations are very similar, we divide them into two families:
\begin{itemize}
  \item \textbf{\TopPushK family:} \TopPush, \TopPushK, \TopMeanK and \tauFPL.
  \item \textbf{\PatMat family:} \PatMat and \PatMatNP.
\end{itemize}
Both families use surrogate false-negative rate as an objective function. Moreover, all formulations from \TopPushK family use the mean of~$K$ highest scores of all or negative samples as a threshold and differ only in the definition of~$K.$ Finally, both formulations from \PatMat family use a surrogate approximation of the top $\tau$-quantile of scores of all or negative samples. In other words, we have two families of formulations that share the same objective function and the same form of the decision threshold. Therefore, we derive all results for the general form of these two families. Before we start, we need to introduce the concept of conjugate functions.

\pagebreak

\begin{definition}[Conjugate function~\cite{boyd2004convex}]\label{def: conjugate}
  Let~$l \colon \R^n \to \R.$ The function~$l^{\star} \colon \R^n \to \R,$ defined as
  \begin{equation*}
    l^{\star} (\bm{y})
      =  \sup_{\bm{x} \in \domain l} \{\bm{y}^{\top}\bm{x} - l(\bm{x})\}.
  \end{equation*}
  is called the conjugate function of~$l.$ The domain of the conjugate function
  consists of $\bm{y} \in \R^n$ for which the supremum is finite. 
\end{definition}

These functions will play a crucial role in the resulting form of dual problems. Recall the hinge loss and quadratic hinge loss function defined in Notation~\ref{not: surrogates}
\begin{align*}
    l_{\text{hinge}}(s) & = \max\Brac[c]{0, 1 + s}, &
    l_{\text{quadratic}}(s) & = \Brac{\max\Brac[c]{0, 1 + s}}^2.
\end{align*}
The conjugate function for the hinge loss can be found in~\cite{shnlev2014accelerated} and has the following form
\begin{equation}\label{eq: conjugate hinge}
  l_{\text{hinge}}^{\star}(y) =
  \begin{cases}
    -y & \text{if } y \in [0, 1], \\
    \infty & \text{otherwise.}
  \end{cases}  
\end{equation}
Similarly, the conjugate function for the quadratic hinge was computed in~\cite{kanamori2013conjugate} as
\begin{equation}\label{eq: conjugate quadratic hinge}
  l_{\text{quadratic}}^{\star}(y) =
  \begin{cases}
    \frac{y^2}{4} - y & \text{if } y \geq 0, \\
    \infty & \text{otherwise.}
  \end{cases}
\end{equation}

\begin{notation}[Kernel Matrix]\label{not: kernel matrix}
  To simplify the future notation, we introduce matrix~$\X$ of all samples. Each row of~$\X$ represents one sample and is defined for all~$i \in \I$ as
  \begin{equation*}
    \X_{i, \bullet} = \bm{x}_i^{\top}.
  \end{equation*}
  In the same way, we defined matrices~$\X^+,$~$\X^-$ of all negative and positive samples with rows defined as
  \begin{align*}
    \X^{-}_{i, \bullet} & = \bm{x}_i^{\top} \quad i = 1, \;, 2, \ldots, \; n^-, \\
    \X^{+}_{i, \bullet} & = \bm{x}_i^{\top} \quad i = 1, \;, 2, \ldots, \; n^+.
  \end{align*}
  Moreover, for all formulations that use only negative samples to compute the threshold~$t$, we define kernel matrix~$\Kneg$ as
  \begin{equation*}
    \Kneg = \Matrix{\X^+ \\ - \X^-} \Matrix{\X^+ \\ - \X^-}^\top = \Matrix{\X^+ \X^{+\top} & -\X^+ \X^{-\top} \\ -\X^- \X^{+\top} & \X^- \X^{-\top} }.
  \end{equation*}
  and for all formulations that use only all samples to compute the threshold~$t$, we define kernel matrix~$\Kall$ as
  \begin{equation*}
    \Kall = \Matrix{\X^+ \\ - \X} \Matrix{\X^+ \\ - \X}^\top = \Matrix{\X^+ \X^{+\top} & -\X^+ \X^{\top} \\ -\X \X^{+\top} & \X \X^{\top} }.
  \end{equation*}
  In the rest of the text, matrix~$\K$ always refers to one of the kernel matrices defined above. 
\end{notation}

\subsection{Family of \TopPushK Formulations}

In this section, we focus on the family of \TopPushK formulations. The general optimization problem that covers all formulations from this family can be written in the following way
\begin{mini!}{\bm{w}}{
  \frac{1}{2} \norm{\bm{w}}^2 + C \sum_{i \in \Ipos} l(t - \bm{w}^{\top} \bm{x}_i)
  }{\label{eq: toppushk family}}{\label{eq: toppushk family L}}
  \addConstraint{s_j}{= \bm{w}^{\top} \bm{x}_j, \quad j \in \Itil \label{eq: toppushk family c1}}
  \addConstraint{t}{= \frac{1}{K} \sum_{j = 1}^{K} s_{[j]}, \label{eq: toppushk family c2}}
\end{mini!}
where~$C \in \R.$ The set of indices~$\Itil$ equals~$\I$ for \TopMeanK and~$\Ineg$ for other formulations. The parameter~$K$ equals~$1$ for \TopPush, $K$ for \TopPushK, $\nall \tau$ for \TopMeanK, and $\nneg \tau$ for \tauFPL. Note that we use an alternative formulation with constant~$C,$ since it is more similar to the standard SVM, and we wanted to stress this similarity. For~$C = \nicefrac{1}{\npos}$ the new formulation is identical to the original one.

The following theorem shows the dual form of formulation~\eqref{eq: toppushk family}. The dual formulation for \TopPush was originally derived in~\cite{li2014top}. We only show, that our general dual formulation also covers this special case. To keep the readability as simple as possible, we postpone all proofs to Appendix~\ref{app: dual}.

\begin{restatable}[Dual formulation for \TopPushK family]{theorem}{topdual}\label{thm: toppushk family dual}
  Consider Notation~\ref{not: kernel matrix}, surrogate function~$l,$ and formulation~\eqref{eq: toppushk family}. Then the corresponding dual problem has the following form
  \begin{maxi!}{\bm{\alpha}, \bm{\beta}}{
    - \frac{1}{2} \vecab^\top \K \vecab
    - C \sum_{i = 1}^{\npos} l^{\star}\Brac{\frac{\alpha_i}{C}}
    }{\label{eq: toppushk family dual}}{\label{eq: toppushk family dual L}}
    \addConstraint{\sum_{i = 1}^{\npos} \alpha_i}{= \sum_{j = 1}^{\ntil} \beta_j \label{eq: toppushk family dual c1}}
    \addConstraint{0 \leq \beta_j}{\leq \frac{1}{K} \sum_{i = 1}^{\npos} \alpha_i, \quad j = 1, 2, \ldots, \ntil, \label{eq: toppushk family dual c2}}
  \end{maxi!}
  where~$l^{\star}$ is conjugate function of~$l$ and
  \begin{center}
    \renewcommand*{\arraystretch}{1}
    \begin{NiceTabular}{lcccc}
        & $K$
        & $\K$
        & $\ntil$
        & $\tilde{\bm{x}}_j$ \\
      \midrule
      \TopPush
        & $1$
        & $\Kneg$
        & $\nneg$
        & $\bm{x}^-_j$ \\
      \TopPushK
        & $K$
        & $\Kneg$
        & $\nneg$
        & $\bm{x}^-_j$ \\
      \TopMeanK
        & $\nall \tau$
        & $\Kall$
        & $\nall$
        & $\bm{x}_j$ \\
      \tauFPL
        & $\nneg \tau$
        & $\Kneg$
        & $\nneg$
        & $\bm{x}^-_j$ \\
    \end{NiceTabular}
  \end{center}
  If~$K = 1,$ the upper bound in the second constraint~\eqref{eq: toppushk family dual c2} vanishes due to the first constraint. Finally, the primal variables~$\bm{w}$ can be computed from dual variables as follows
  \begin{equation}\label{eq: toppushk family dual to primal}
    \bm{w} = \sum_{i = 1}^{\npos} \alpha_i \bm{x}^+_i - \sum_{j = 1}^{\ntil} \beta_j \tilde{\bm{x}}_j.
  \end{equation}
\end{restatable}

\subsection{Family of \PatMat Formulations}

In the same way, as for \TopPushK family, we introduce a general optimization problem that covers all formulations from \PatMat family and reads
\begin{mini}{\bm{w}}{
  \frac{1}{2} \norm{\bm{w}}^2 + C \sum_{i \in \Ipos} l(t - \bm{w}^{\top} \bm{x}_i)
  }{\label{eq: patmat family}}{}
  \addConstraint{t}{\;\; \text{solves} \;\; \frac{1}{\ntil}\sum_{i \in \Itil} l\Brac{\vartheta(\bm{w}^{\top} \bm{x}_j - t)} = \tau,}
\end{mini}
where~$C \in \R.$ For \PatMat we have~$\Itil = \I$ and~$\ntil = \nall.$ For \PatMatNP we have~$\Itil = \Ineg$ and~$\ntil = \nneg.$ Again, we use the alternative formulation with constant~$C.$ The following theorem shows the dual form of the formulation~\eqref{eq: patmat family}.

\begin{restatable}[Dual formulation for \PatMat family]{theorem}{patdual}\label{thm: patmat family dual}
  Consider Notation~\ref{not: kernel matrix}, surrogate function~$l,$ and formulation~\eqref{eq: patmat family}. Then the corresponding dual problem has the following form
  \begin{maxi!}{\bm{\alpha}, \bm{\beta}, \delta}{
    - \frac{1}{2} \vecab^\top \K \vecab
    - C \sum_{i = 1}^{\npos} l^{\star}\Brac{\frac{\alpha_i}{C}}
    - \delta \sum_{j = 1}^{\ntil} l^{\star} \Brac{\frac{\beta_j}{\delta\vartheta }}
    - \delta \ntil \tau
    }{\label{eq: patmat family dual}}{\label{eq: patmat family dual L}}
    \addConstraint{\sum_{i = 1}^{\npos} \alpha_i}{= \sum_{j = 1}^{\ntil} \beta_j \label{eq: patmat family dual c1}}
    \addConstraint{\delta }{\geq 0, \label{eq: patmat family dual c2}}
  \end{maxi!}
  where~$l^{\star}$ is conjugate function of~$l,$~$\vartheta > 0$ is a scaling parameter and
  \begin{center}
    \renewcommand*{\arraystretch}{1}
    \begin{NiceTabular}{lccc}
        & $\K$
        & $\ntil$
        & $\tilde{\bm{x}}_j$ \\
      \midrule
      \PatMat
        & $\Kall$
        & $\nall$
        & $\bm{x}_j$ \\
      \PatMatNP
        & $\Kneg$
        & $\nneg$
        & $\bm{x}^-_j$ \\
    \end{NiceTabular}
  \end{center}
  Finally, the primal variables~$\bm{w}$ can be computed from dual variables as follows
  \begin{equation}\label{eq: patmat family dual to primal}
    \bm{w} = \sum_{i = 1}^{\npos} \alpha_i \bm{x}^+_i - \sum_{j = 1}^{\ntil} \beta_j \tilde{\bm{x}}_j.
  \end{equation}
\end{restatable}

\begin{note}
  For simplicity, the rest of the chapter covers only the \TopPushK formulation with hinge loss. We use this formulation since it is the prototypical example for the \TopPushK family of formulations. The results for the rest of the formulations from this family can be derived almost identically. Moreover, results for the \PatMat family of formulations can be derived similarly. Therefore, derivations for the \TopPushK family with quadratic hinge loss and the \PatMat family with hinge and quadratic hinge loss are postponed to Appendix~\ref{app: dual}.
\end{note}

As we mentioned at the beginning of the chapter, our goal is to extend our framework to be usable for linearly inseparable problems. We derived dual formulations for \TopPushK and \PatMat families in two previous sections. In this section, we show how to employ the kernels method~\cite{scholkopf2001learning} to introduce nonlinearity into these dual formulations. For simplicity, we focus only on formulations from \TopPushK family that compute the decision threshold only from negative samples. As mentioned in Notation~\ref{not: kernel matrix}, all such formulations use kernel matrix~$\Kneg.$ The derivation is the same for all other formulations.

To add kernels, we first realize that the classification score~$s_j$ for any sample~$\bm{x}_j \in \R^d$ reads 
\begin{equation}\label{eq:pred_linear}
  s_j
    = \bm{w}^{\top} \bm{x}_j
    = \sum_{i = 1}^{\npos} \alpha_i \bm{x}_j^{\top} \bm{x}_i^+ - \sum_{i = 1}^{\nneg} \beta_i \bm{x}_j^{\top} \bm{x}_i^-,
\end{equation}
where~$\bm{\alpha} \in \R^{\npos},$~$\bm{\beta} \in \R^{\nneg}$ are dual variables. This relation follows from the proof of Theorem~\ref{thm: toppushk family dual}. Consider now any kernel function~$k: \R^d \times \R^d \to \R.$ Then the first part of the objective function~\eqref{eq: toppushk family dual L} amounts to
\begin{equation*}
  \vecab^\top \Kneg \vecab
    = \vecab^\top \Matrix{\X^+ \X^{+\top} & -\X^+ \X^{-\top} \\ -\X^- \X^{+\top} & \X^- \X^{-\top} } \vecab.
\end{equation*}
Using the standard trick, we can replace the kernel matrix~$\Kneg$ with matrix in the following form
\begin{equation}\label{eq:kernel_nonlinear}
  \Kneg = \Matrix{k\Brac{\X^+, \X^{+}} & -k\Brac{\X^+, \X^{-}} \\ -k\Brac{\X^-, \X^{+}} & k\Brac{\X^-, \X^{-}}},
\end{equation}
where~$k(\cdot,\; \cdot)$ is applied to all rows of both arguments. Then for any sample~$\bm{x}_j$, the classification score~\eqref{eq:pred_linear} is replaced by
\begin{equation}\label{eq: scores dual kernel}
  s_j = \sum_{i = 1}^{\npos} \alpha_i k\Brac{\bm{x}_j, \bm{x}^+_i} - \sum_{i = 1}^{\nneg} \beta_i k\Brac{\bm{x}_j, \bm{x}^-_i}.
\end{equation}

\section{Coordinate Descent Algorithm}\label{sec: coordinate descent}

In the previous sections, we derived dual formulations for \TopPushK and \PatMat families of formulations. Moreover, we showed how to incorporate non-linear kernels into these formulations. As a result, we can use all presented formulations even for linearly non-separable problems. However, the dimension of the dual problems is at least equal to the number of all samples~$n,$ and therefore, it is computationally expensive to use standard techniques such as gradient descent. To handle this issue, the standard coordinate descent algorithm~\cite{chang2008coordinate, hsieh2008dual} has been proposed in the context of SVMs. In this section, we derive a coordinate descent algorithm suitable for our dual problems~(\ref{eq: toppushk family dual},~\ref{eq: patmat family dual}). We also show that we can reduce the whole optimization problem to a one-dimensional quadratic optimization problem with a closed-form solution in every iteration. Therefore, every iteration of our algorithm is cheap. For a review of other approaches see~\cite{batmaz2019review,werner2019review}.

Before we start, recall that the classification scores can be computed directly from dual variables as shown in~\eqref{sec: coordinate descent}. Using the definition of kernel matrix~$\K$, we can define a vector of scores~$\bm{s}$ by
\begin{equation}\label{eq: dual scores}
  \bm{s} = \K \vecab.
\end{equation}
Note that dual scores are not identical to the primal ones~\eqref{eq:pred_linear} (even though we use the same notation). The main difference is that dual scores use kernel function~$k.$ Therefore, they are equivalent only if the kernel function is defined as a dot product, i.e., if~$k(\bm{x}, \bm{z}) = \bm{x}^{\top} \bm{z}.$ Moreover, the vector of dual scores~\eqref{eq: dual scores} is not identical to the vector of primal scores from Notation~\ref{not: scores}, even with this choice of the kernel function. The reason is the construction of the kernel matrix~$\K$ from Notation~\ref{not: kernel matrix}. For example, for~$\Kneg$, the resulting vector of dual scores is a permutation of the vector of primal scores. Similarly, for~$\Kall$, the vector of primal scores corresponds to the elements of~\eqref{eq: dual scores} with indices~$\npos + 1, \ldots, \npos + \nall.$ Therefore for both definitions of kernel matrices from Notation~\ref{not: scores}, we can quickly get the vector of primal scores by permuting vector~\eqref{eq: dual scores} or selecting only some elements from vector~\eqref{eq: dual scores}. To simplify the indexing of the vector of scores~\eqref{eq: dual scores} and kernel matrix~$\K$, we introduce a new notation in Notation~\ref{not: dual update rules}.

\pagebreak

\begin{notation}\label{not: dual update rules}
  Consider any index~$l$ that satisfies~$1 \leq l \leq \npos + \ntil.$ Note that the length of dual variable~$\bm{\alpha}$ is~$\npos$ for both formulations~\eqref{eq: toppushk family dual} and~\eqref{eq: patmat family dual}. Therefore, we can define auxiliary index~$\hat{l}$ as 
  \begin{equation*}
    \hat{l} = \begin{cases}
      l & \text{if } l \leq \npos, \\
      l - \npos & \text{otherwise}.
    \end{cases}
  \end{equation*}
  Then the index~$l$ can be safely used for kernel matrix~$\K$ or vector of scores~$\bm{s},$ while its corresponding version~$\hat{l}$ can be used for dual variables~$\bm{\alpha}$ or~$\bm{\beta}.$
\end{notation}

\subsection{Family of \TopPushK Formulations}\label{sec: Top coordinate descent}

Consider dual formulation~\eqref{eq: toppushk family dual} from Theorem~\ref{thm: toppushk family dual} and fixed feasible dual variables~$\bm{\alpha},$~$\bm{\beta}.$ Our goal in this section is to derive an efficient iterative procedure for solving this problem. We follow the ideas presented in~\cite{chang2008coordinate, hsieh2008dual} for solving SVMs using a coordinate descent algorithm. However, we must modify the approach since we have an additional constraint~\eqref{eq: toppushk family dual c1}.  Due to this constraint, we always have to update (at least) two components of dual variables~$\bm{\alpha},$~$\bm{\beta}.$ There are only three update rules which modify two components of~$\bm{\alpha},$~$\bm{\beta},$ and satisfy constraints~\eqref{eq: toppushk family dual c1} and~\eqref{eq: dual scores}. The first one updates two components of~$\bm{\alpha}$
\begin{subequations}\label{eq: update rules}
\begin{align}\label{eq: update rule a,a}
  \alphak & \to \alphak + \Delta, & \quad
  \alphal & \to \alphal - \Delta, & \quad
  \bm{s} & \to \bm{s} + \Brac{\K_{\bullet, k} - \K_{\bullet, l}}\Delta,
\end{align}
where~$\K_{\bullet, i}$ denotes $i$-th column of~$\K$ and indices~$\hat{k},$~$\hat{l}$ are defined in Notation~\ref{not: dual update rules}. Note that the update rule for~$\bm{s}$ does not use matrix multiplication but only vector addition. The second rule updates one component of~$\bm{\alpha}$ and one component of~$\bm{\beta}$ 
\begin{align}\label{eq: update rule a,b}
  \alphak & \to \alphak + \Delta, & \quad
  \betal  & \to \betal  + \Delta, & \quad
  \bm{s} & \to \bm{s} + \Brac{\K_{\bullet, k} + \K_{\bullet, l}}\Delta,
\end{align}
and the last one updates two components of~$\bm{\beta}$
\begin{align}\label{eq: update rule b,b}
  \betak & \to \betak + \Delta, & \quad
  \betal & \to \betal - \Delta, & \quad
  \bm{s}  & \to \bm{s} + \Brac{\K_{\bullet, k} - \K_{\bullet, l}}\Delta.
\end{align}
\end{subequations}
Using any of the update rules above, the problem~\eqref{eq: toppushk family dual} can be written as a one-dimensional quadratic problem in the following form
\begin{maxi*}{\Delta}{
  -\frac{1}{2} a(\bm{\alpha}, \bm{\beta}) \Delta^2
  - b(\bm{\alpha}, \bm{\beta}) \Delta
  - c(\bm{\alpha}, \bm{\beta})
  }{}{}
  \addConstraint{\Delta_{lb}(\bm{\alpha}, \bm{\beta})}{\leq \Delta \leq \Delta_{ub}(\bm{\alpha}, \bm{\beta})}
\end{maxi*}
where~$a,$~$b,$~$c,$~$\Delta_{lb},$~$\Delta_{ub}$ are constants with respect to~$\Delta.$ The optimal solution to this problem is
\begin{equation}\label{eq: Delta optimal}
  \Delta^{\star} = \clip{\Delta_{lb}}{\Delta_{ub}}{\gamma},
\end{equation}
where~$\gamma = -\frac{b}{a}$ and~$\clip{a}{b}{x}$ amounts to clipping (projecting)~$x$ to interval~$[a, b].$ Since we assume one of the update rules~\eqref{eq: update rules}, the constraint~\eqref{eq: toppushk family dual c1} is always satisfied after the update. Even though all three update rules hold for any surrogate, the calculation of the optimal~$\Delta^{\star}$ depends on the concrete form of surrogate function. In the following text, we show the closed-form formula for~$\Delta^{\star},$ when the hinge loss or quadratic hinge loss is used as a surrogate function.

\subsubsection{Hinge loss}

We start with the hinge loss function from Notation~\ref{not: surrogates}. Plugging the conjugate~\eqref{eq: conjugate hinge} of the hinge loss into the dual formulation~\eqref{eq: toppushk family dual} yields
\begin{maxi!}{\bm{\alpha}, \bm{\beta}}{
  - \frac{1}{2} \vecab^\top \K \vecab
  + \sum_{i = 1}^{\npos} \alpha_i
  }{\label{eq: Top dual hinge}}{\label{eq: Top dual hinge L}}
  \addConstraint{\sum_{i = 1}^{\npos} \alpha_i}{= \sum_{j = 1}^{\ntil} \beta_j
  \label{eq: Top dual hinge c1}}
  \addConstraint{0 \leq \alpha_i}{\leq C,}{i = 1, 2, \ldots, \npos
  \label{eq: Top dual hinge c2}}
  \addConstraint{0 \leq \beta_j}{\leq \frac{1}{K} \sum_{i = 1}^{\npos} \alpha_i, \quad}{j = 1, 2, \ldots, \ntil.
  \label{eq: Top dual hinge c3}}
\end{maxi!}
The form of~$\K$ and~$\ntil$ depends on the used formulation as discussed in Theorem~\ref{thm: toppushk family dual}. Moreover, the upper bound in~\eqref{eq: Top dual hinge c3} can be omitted for~$K = 1.$ Since we know the form of the optimal solution~\eqref{eq: Delta optimal}, we only need to show how to compute~$\Delta_{lb},$~$\Delta_{ub}$ and~$\gamma$ for all update rules~\eqref{eq: update rules}. The following three propositions provide closed-form formulas for all three update rules. To keep the presentation as simple as possible, we postpone all proofs to Appendix~\ref{sec: toppushk family coordinate proofs}.

\begin{restatable}[Update rule~\eqref{eq: update rule a,a} for problem~\eqref{eq: Top dual hinge}]{proposition}{topruleaa}\label{prop: toppushk family hinge update a,a}
  Consider problem~\eqref{eq: Top dual hinge}, update rule~\eqref{eq: update rule a,a}, indices~$1 \leq k \leq \npos$ and~$1 \leq l \leq \npos$ and Notation~\ref{not: dual update rules}. Then the optimal solution~$\Delta^{\star}$ is given by~\eqref{eq: Delta optimal} where
  \begin{align*}
    \Delta_{lb} & = \max\{- \alphak,\; \alphal - C\}, \\
    \Delta_{ub} & = \min\{C - \alphak,\; \alphal \}, \\
    \gamma & = -\frac{s_k - s_l}{\K_{kk} + \K_{ll} - \K_{kl} - \K_{lk}}.
  \end{align*}
\end{restatable}

\begin{restatable}[Update rule~\eqref{eq: update rule a,b} for problem~\eqref{eq: Top dual hinge}]{proposition}{topruleab}\label{prop: toppushk family hinge update a,b}
  Consider problem~\eqref{eq: Top dual hinge}, update rule~\eqref{eq: update rule a,b}, indices~$1 \leq k \leq \npos$ and~$\npos + 1 \leq l \leq \ntil$ and Notation~\ref{not: dual update rules}. Let us define
  \begin{equation*}
    \beta_{\max} = \max_{j \in \{1, 2, \ldots, \ntil \} \setminus \{\hat{l}\}} \beta_j.
  \end{equation*}
  Then the optimal solution~$\Delta^{\star}$ is given by~\eqref{eq: Delta optimal} where
  \begin{align*}
    \Delta_{lb} & = 
      \begin{cases*}
        \max \Brac[c]{- \alphak, \;  -\betal} & K = 1, \\
        \max \Brac[c]{- \alphak, \;  -\betal, \; K\beta_{\max} - \sum_{i = 1}^{\npos} \alpha_i} & \textrm{otherwise},
      \end{cases*} \\
    \Delta_{ub} & = 
      \begin{cases*}
          C - \alphak & K = 1, \\
          \min \Brac[c]{C - \alphak, \; \frac{1}{K-1}\Brac{\sum_{i = 1}^{\npos} \alpha_i - K \betal}}  & \textrm{otherwise}.
      \end{cases*} \\
    \gamma & = - \frac{s_k + s_l - 1}{\K_{kk} + \K_{ll} + \K_{kl} + \K_{lk}}.
  \end{align*}
\end{restatable}

\pagebreak

\begin{restatable}[Update rule~\eqref{eq: update rule b,b} for problem~\eqref{eq: Top dual hinge}]{proposition}{toprulebb}\label{prop: toppushk family hinge update b,b}
  Consider problem~\eqref{eq: Top dual hinge}, update rule~\eqref{eq: update rule b,b}, indices~$\npos + 1 \leq k \leq \ntil$ and~$\npos + 1 \leq l \leq \ntil$ and Notation~\ref{not: dual update rules}. Then the optimal solution~$\Delta^{\star}$ is given by~\eqref{eq: Delta optimal} where
  \begin{align*}
    \Delta_{lb} & = 
      \begin{cases*}
        - \betak & K = 1, \\
        \max \Brac[c]{- \betak,\; \betal - \frac{1}{K} \sum_{i = 1}^{\npos} \alpha_i} & \textrm{otherwise},
      \end{cases*} \\
    \Delta_{ub} & = 
      \begin{cases*}
        \betal & K = 1, \\
        \min \Brac[c]{\frac{1}{K} \sum_{i = 1}^{\npos} \alpha_i - \betak,\; \betal} & \textrm{otherwise}.
      \end{cases*} \\
    \gamma & = -\frac{s_k - s_l}{\K_{kk} + \K_{ll} - \K_{kl} - \K_{lk}}.
  \end{align*}
\end{restatable}

\subsubsection{Initialization}

For all update rules~\eqref{eq: update rules} we assumed that the current solution~$\bm{\alpha},$~$\bm{\beta}$ is feasible. So to create an iterative algorithm that solves problem~\eqref{eq: Top dual hinge} or~\eqref{eq: Top dual quadratic}, we need to have a way how to obtain an initial feasible solution. Such a task can be formally written as a projection of random variables~$\bm{\alpha}^0,$~$\bm{\beta}^0$ to the feasible set of solutions
\begin{mini}{\bm{\alpha}, \bm{\beta}}{
  \frac{1}{2} \norm{\bm{\alpha} - \bm{\alpha}^0}^2
  + \frac{1}{2} \norm{\bm{\beta} - \bm{\beta}^0}^2
  }{\label{eq: toppushk family initialization}}{}
  \addConstraint{\sum_{i = 1}^{\npos} \alpha_i}{= \sum_{j = 1}^{\ntil} \beta_j}
  \addConstraint{0 \leq \alpha_i}{\leq C_1, \quad i = 1, 2, \ldots, \npos,}
  \addConstraint{0 \leq \beta_j}{\leq \frac{1}{K} \sum_{i = 1}^{\npos} \alpha_i, \quad j = 1, 2, \ldots, \ntil,}
\end{mini}
where the upper bound in the second constraint depends on the used surrogate function. For hinge loss function, the upper bound is~$C_1 = C,$ while for the quadratic hinge loss it is~$C_1 = + \infty.$ To solve problem~\eqref{eq: toppushk family initialization}, we follow the same approach as in~\cite{adam2020projections}. In the following theorem, we show that problem~\eqref{eq: toppushk family initialization} can be written as a system of two equations of two variables~$\lambda$ and~$\mu.$ Moreover, the theorem shows the concrete form of feasible solution~$\bm{\alpha},$~$\bm{\beta}$ that depends only on~$\lambda$ and~$\mu.$

\begin{restatable}{theorem}{topinit}\label{thm: toppushk family initialization}
  Consider problem~\eqref{eq: toppushk family initialization}, some initial solution~$\bm{\alpha}^0,$~$\bm{\beta}^0$ and denote the sorted version (in non-decreasing order) of~$\bm{\beta}^0$ as~$\bm{\beta}_{[\cdot]}^0.$ Then if the following condition holds
  \begin{equation}\label{eq:problem3_cond}
    \sum_{j = 1}^{K} \Brac{\beta_{[\ntil - K + j]}^0 + \max_{i = 1, \ldots, \npos} \alpha_i^0} \le 0,
  \end{equation}
  the optimal solution of~\eqref{eq: toppushk family initialization} amounts to~$\bm{\alpha} = \bm{\beta} = \bm{0}.$ In the opposite case, the following system of two equations
  \begin{subequations}\label{eq: toppushk family init alg}
    \begin{align}
      \sum_{i=1}^{\npos} \clip{0}{C_1}{ \alpha_i^0 - \lambda + \frac{1}{K} \sum_{j=1}^{\ntil} \clip[u]{0}{+\infty}{\beta_j^0 + \lambda - \mu}} - K \mu
      & = 0, \label{eq: toppushk family init alg 1} \\
      \sum_{j=1}^{\ntil} \clip{0}{\mu}{\beta_j^0 + \lambda} - K\mu
      & = 0, \label{eq: toppushk family init alg 2}
    \end{align}
  \end{subequations}
  has a solution~$(\lambda, \mu)$ with $\mu > 0,$ and the optimal solution of~\eqref{eq: toppushk family initialization} is equal to
  \begin{align*}
    \alpha_i
      & = \clip{0}{C_1}{\alpha_i^0 - \lambda + \frac{1}{K} \sum_{j=1}^{\ntil} \clip[u]{0}{+\infty}{\beta_j^0 + \lambda - \mu}}, \\
    \beta_j & = \clip{0}{\mu}{\beta_j^0 + \lambda}.
  \end{align*}
\end{restatable}

Theorem~\ref{thm: toppushk family initialization} shows the optimal solution of~\eqref{eq: toppushk family initialization} that depends only on~$(\lambda, \mu)$ but does not provide any way to find such a solution. In the following text, we show that the number of variables in the system of equations~\eqref{eq: toppushk family init alg} can be reduced to one. For any fixed $\mu$, we denote the function on the left-hand side of~\eqref{eq: toppushk family init alg 2} by 
\begin{equation*}
  g(\lambda; \mu) := \sum_{j=1}^{\ntil} \clip{0}{\mu}{\beta_j^0 + \lambda} - K\mu.
\end{equation*}
Then~$g$ is non-decreasing in~$\lambda$ but not necessarily strictly increasing. We denote by~$\lambda(\mu)$ any such~$\lambda$ solving~\eqref{eq: toppushk family init alg 2} for a fixed~$\mu$. Denote~$\bm{z}$ the sorted version of~$-\bm{\beta}^0$. Then we have
\begin{equation*}
  g(\lambda; \mu)
    = \sum_{\Set{j}{\lambda - z_j \in [0, \mu)}}(\lambda - z_j)
    + \sum_{\Set{j}{\lambda - z_j \ge \mu}}\mu - K\mu.
\end{equation*}
Now we can easily compute~$\lambda(\mu)$ by solving~$g(\lambda(\mu); \mu) = 0$ for fixed~$\mu.$ To get the solution efficiently, we derive Algorithm~\ref{alg: toppushk family lambda}, which can  be described as follows: Index~$i$ will run over~$\bm{z}$ while index~$j$ will run over~$\bm{z} + \mu$. At every iteration, we know the values of~$g(z_{i-1}; \mu)$ and~$g(z_{j-1}+\mu; \mu)$ and we want to evaluate~$g$ at the next point. We denote the number of indices~$j$ such that $\lambda - z_j \in[0, \mu)$ by~$d$. If~$z_i \le z_j + \mu$, then we consider~$\lambda = z_i$ and since one index enters the set~$\Set{j}{\lambda - z_j \in [0, \mu)}$, we increase~$d$ by one. On the other hand, if~$z_i > z_j + \mu$, then we consider $\lambda = z_j + \mu$ and since one index leaves the set~$\Set{j}{\lambda - z_j \in [0, \mu)}$, we decrease~$d$ by one. In both cases,~$g$ is increased by~$d$ times the difference between the new~$\lambda$ and old~$\lambda$. Once~$g$ exceeds~$0$, we stop the algorithm and linearly interpolate between the last two values. To prevent an overflow, we set~$z_{m+1} = + \infty$. Concerning the initial values, since~$z_1 \le z_1 + \mu$, we set $i=2$, $j=1$ and $d=1$. 

\begin{algorithm}
  \centering
  \begin{algorithmic}[1]
    \Require vector $-\bm{\beta}^0$ sorted into $\bm{z}$
    \State $i \gets 2$, $j \gets 1$, $d \gets 1$
    \State $\lambda \gets z_1$, $g \gets - K\mu$
    \While{$g < 0$}
      \If {$z_i \le z_j + \mu$}
        \State $g \gets g + d(z_i - \lambda)$
        \State $\lambda\gets z_i$, $d \gets d+1$, $i \gets i+1$
      \Else
        \State $g \gets g + d(z_j + \mu - \lambda)$
        \State $\lambda \gets z_j + \mu$, $d \gets d - 1$, $j \gets j + 1$
      \EndIf
    \EndWhile
    \State \textbf{return} linear interpolation of the last two values of $\lambda$
  \end{algorithmic}
  \caption{An efficient algorithm for computing~$\lambda(\mu)$ from~\eqref{eq: toppushk family initialization} for fixed~$\mu.$.}
  \label{alg: toppushk family lambda}
\end{algorithm}

Since~$\lambda(\mu)$ can be computed for fixed~$\mu$ using Algorithm~\ref{alg: toppushk family lambda}, we can define auxiliary function~$h$ in the following form
\begin{equation}\label{eq: toppushk family h}
  h(\mu)
    = \sum_{i=1}^{\npos} \clip{0}{C_1}{\alpha_i^0 - \lambda(\mu) + \frac{1}{K} \sum_{j=1}^{\ntil} \clip[u]{0}{+\infty}{\beta_j^0+\lambda(\mu) - \mu}} - K \mu.
\end{equation}
Then the system of equations~\eqref{eq: toppushk family init alg} is equivalent to~$h(\mu) = 0.$ The following lemma describes properties of~$h.$ Since~$h$ is decreasing in~$\mu$ on~$(0, \infty)$, any root-finding algorithm such as bisection can be used to find the optimal solution.

\begin{restatable}{lemma}{topinith}\label{lemma: toppushk family h}
  Even though~$\lambda(\mu)$ is not unique, function~$h$ from~\eqref{eq: toppushk family h} is well-defined in the sense that it gives the same value for every choice of~$\lambda(\mu)$. Moreover,~$h$ is decreasing in~$\mu$ on~$(0, + \infty)$.
\end{restatable}

\section{Summary}

In this chapter, we derived dual formulation for \TopPushK and \PatMat family of formulations. Moreover, we derived simple update rules that can be used to improve the current feasible solution. We also showed that these update rules have closed-form formulas, and therefore they are simple to compute. Finally, we showed how to find an initial feasible solution. This section combines all these intermediate results into Algorithm~\ref{alg:Coordinate descent} and discusses its computational complexity.

\begin{algorithm*}
  \begin{minipage}{0.48\textwidth}
    \centering
    \begin{algorithmic}[1]
      \State Set~$\bm{\alpha},$~$\bm{\beta}$ using Theorem~\ref{thm: toppushk family initialization}
      \State Set~$\bm{s}$ based on~\eqref{eq: dual scores} \label{alg: line 1}
      \Repeat \label{alg: line 2}
        \State Pick random~$k$ from~$\{1, \ldots, \npos + \ntil\}$ \label{alg: line 3}
        \For{$l \in \{1, \ldots, \npos + \ntil  \}$} \label{alg: line 4}
            \State Compute~$\Delta_{l}$  \label{alg: line 5}
        \EndFor
        \State Select the best~$\Delta_{l}$ \label{alg: line 7}
        \State Update~$\bm{\alpha}$,~$\bm{\beta},$~$\bm{s}$ according to~\eqref{eq: update rules} \label{alg: line 8}
        \State \label{alg: line 9}
      \Until{stopping criterion is satisfied}
    \end{algorithmic}
  \end{minipage}%
  \hfill
  \begin{minipage}{0.48\textwidth}
    \centering
    \begin{algorithmic}[1]
      \State Set~$\bm{\alpha},$~$\bm{\beta},$~$\delta$ using Theorem~\ref{thm: patmat family initialization}
      \State Set~$\bm{s}$ based on~\eqref{eq: dual scores}
      \Repeat
        \State Pick random~$k$ from~$\{1, \ldots, \npos + \ntil \}$ 
        \For{$l \in \{1, \ldots, \npos + \ntil \}$}
            \State Compute~$\Delta_{l}$ and~$\delta_{l}$
        \EndFor
        \State Select the best~$\Delta_{l}$ and~$\delta_{l}$
        \State Update~$\bm{\alpha}$,~$\bm{\beta},$~$\bm{s}$ according to~\eqref{eq: update rules}
        \State set~$\delta \leftarrow \delta_{l}$
      \Until{stopping criterion is satisfied}
    \end{algorithmic}
  \end{minipage}
  \caption{Coordinate descent algorithm for \TopPushK family of formulations (\textbf{left}) and \PatMat  family of formulations (\textbf{right}).}
  \label{alg:Coordinate descent}
\end{algorithm*}

The left column in Algorithm~\ref{alg:Coordinate descent} describe the algorithm for \TopPushK family while the right column for \PatMat family. In step~\ref{alg: line 1} we initialize~$\bm{\alpha}$,~$\bm{\beta}$ and~$\delta$ to some feasible value using Theorem~\ref{thm: toppushk family initialization} or  Theorem~\ref{thm: patmat family initialization}. Then, based on~\eqref{eq: dual scores} we compute scores~$\bm{s}$. Each \repeatloop loop in step~\ref{alg: line 2} updates two coordinates as shown in~\eqref{eq: update rules}. In step~\ref{alg: line 3} we select a random index~$k$ and in the \forloop loop in step~\ref{alg: line 4} we compute the optimal~$(\Delta_l,\delta_l)$ for all possible combinations~$(k,l)$ as in~\eqref{eq: update rules}. In step~\ref{alg: line 7} we select the best pair~$(\Delta_l,\delta_l)$ which maximizes the coresponding objective function. Finally, based on the selected update rule we update~$\bm{\alpha}$,~$\bm{\beta}$,~$\bm{s}$ and~$\delta$ in steps~\ref{alg: line 8} and~\ref{alg: line 9}.

Now we derive the computational complexity of each \repeatloop loop from step~\ref{alg: line 2}. The computation of~$(\Delta_l,\delta_l)$ amounts to solving a quadratic optimization problem in one variable. As we showed in Sections~\ref{sec: Top coordinate descent} and~\ref{sec: Pat coordinate descent}, there is a closed-form solution and step~\ref{alg: line 5} can be performed in~$O(1)$. Since this is embedded in a \forloop loop in step~\ref{alg: line 4}, the whole complexity of this loop is~$O(\npos + \ntil)$. Step~\ref{alg: line 8} requires~$O(1)$ for the update of~$\bm{\alpha}$ and~$\bm{\beta}$ while~$O(\npos + \ntil)$ for the update of~$\bm{s}$. Since the other steps are~$O(1)$, the total complexity of the \repeatloop loop is~$O(\npos + \ntil)$. This holds only if the kernel matrix~$\K$ is precomputed. In the opposite case, all complexities must be multiplied by the cost of computation of components of~$\K$, which is~$O(d)$. This complexity analysis is summarized in Table~\ref{tab:Computational complexity}.

\begin{table}[h]
  \centering
  \begin{NiceTabular}{lcc}
    \CodeBefore
      \rowcolor{\headercol}{1}
      \rowcolors{3}{\rowcol}{}[restart]
    \Body
    \toprule
    \Block[c]{1-1}{Operation}
      & $\K$ precomputed
      & $\K$ not precomputed \\
    \midrule
    Evaluation of~$\Delta_l$
      & $O(1)$
      & $O(d)$ \\
    Update of~$\bm{\alpha}$ and~$\bm{\beta}$
      & $O(1)$
      & $O(1)$ \\
    Update of~$\bm{s}$
      & $O\Brac{\npos + \ntil}$
      & $O\Brac{(\npos + \ntil)d}$ \\
    \midrule
    Total per iteration
    & $O\Brac{\npos + \ntil}$
    & $O\Brac{(\npos + \ntil)d}$ \\
    \bottomrule
  \end{NiceTabular}
  \caption{Computational complexity of one \repeatloop loop (which updates two coordinates of~$\bm{\alpha}$ or~$\bm{\beta}$) from Algorithm~\ref{alg:Coordinate descent}.}
  \label{tab:Computational complexity}
\end{table}

\chapter{Deep}

Since this task considers only scores above the threshold,~\cite{boyd2012accuracy} named it \AccatTop. The important distinction from standard classifiers is that this threshold is no longer fixed, as in the case of~$0.5$, but depends on all samples. Therefore, the objective is non-additive and non-decomposable. This brings both theoretical and numerical issues. Standard machine learning algorithms use minibatch sampling. However, when the threshold is computed on a minibatch, it provides a lower estimate of the true threshold. Therefore, the sampled threshold is a biased estimate of the true threshold. Figure~\ref{fig:thresholds1} illustrates this phenomenon. The bias between the true and sampled thresholds is large even for medium-sized minibatches. Backpropagation then propagates this sampling error through the whole gradient, and consequently, the minibatch gradient is a biased estimate of the true gradient. This brings numerical issues~\cite{bottou2018optimization}.

\begin{figure}[!ht]
  \centering
  \includegraphics[width = \linewidth]{images/deep_threshold_bias.pdf}
  \caption{The bias between the sampled and true thresholds computed from scores following the standard normal distribution. The threshold separates the top~$1\%$ of samples with the highest scores.}
  \label{fig:thresholds1}
\end{figure}

\begin{figure*}[!ht]
  \centering
  \includegraphics[width = \linewidth]{images/standard_aatp_comparison.pdf}
  \caption{Difference between standard classifiers (top row) and classifiers maximizing accuracy at the top (bottom row). While the former has a good total accuracy, the latter has a good top acccuracy.}
  \label{fig:difference}
\end{figure*}

Our method mitigates this bias. It is based on several results.~\cite{li2014top} proposed the TopPush formulation of the accuracy at the top and solved it in its dual formulation.~\cite{adam2019patmat} solved the TopPush formulation directly in its primal form for linear classifiers. Since we generalize the linear TopPush into non-linear classifiers, we name our method \DeepTopPush. We stay in the primal form to be able to employ stochastic gradient descent. Due to non-decomposability, we need to propose a way of computing the gradient and reduce the bias mentioned above. Since the threshold always equals to one of the scores~\cite{boyd2012accuracy}, its computation has a simple local formula. We implicitly remove some variables and apply the chain rule (backpropagation) to compute the gradient in an end-to-end manner. To reduce the bias, we need to improve the approximation quality of the sampled threshold. We employ again the fact that the true threshold corresponds to one sample. Since this sample changes slowly during optimization, we modify the idea of~\cite{adam2019machine} and enhance the current minibatch by the sample, which equalled the sampled threshold on the previous minibatch. As this added sample usually propagates across multiple minibatches, it tracks the threshold, and this trick mitigates the sampled threshold bias. The main contributions of the paper are as follows:
\begin{itemize}
  \item We propose \DeepTopPush, which is a simple and scalable method for accuracy at the top.
  \item We show that \DeepTopPush increases the computational time only slightly, yet it achieves better performance than prior art methods.
  \item We show both theoretically and numerically that enhancing the minibatch by one sample reduces the bias of the sampled gradient.
\end{itemize}
The paper is organized as follows: Section~\ref{sec:deeptheory} introduces a general formulation of accuracy at the top. Section~\ref{sec:solving} derives formulas for the bias of the sampled threshold and proposes \DeepTopPush to minimize it. Section~\ref{sec:numerics} shows the good performance of \DeepTopPush on multiple images recognition datasets, a real-world medical application, and a malware detection dataset, where we detected 46\% malware at an extremely low false alarm rate of~$10^{-5}$. To promote reproducibility, our codes are available online.

\section{Accuracy at the top}\label{sec:deeptheory}

This section introduces the accuracy at the top. A standard deep network~$f$ with weights~$\bm{w}$ takes inputs~$\bm{x}_i$, transforms them into scores~$z_i$, and computes the total loss based on these scores and labels~$y_i$. On the other hand, accuracy at the top solves
\begin{equation}\label{eq:problem}
  \begin{aligned}
    \minimize{w,z,t}
    & \lambda_1 \sum_{i\in I^-}\II_{z_i \ge t} + \lambda_2\sum_{i\in I^+}\II_{z_i < t} \\
    \st
    & z_i = f(\bm{w};\bm{x}_i), \\
    & t = G(\{(z_i, y_i)\}_{i\in I}).
  \end{aligned}
\end{equation}
Similarly to the standard network, the classifier~$f$ computes the score~$z_i$ for each sample~$\bm{x}_i$. Then a general function~$G$ takes the scores and labels of \textbf{all} samples and computes the threshold~$t$. This makes the problem non-decomposable. The objective function equals the weighted sum of false-positives (negative samples above the threshold) and false-negatives (positive samples below the threshold). Here,~$I$,~$I^+$ and~$I^-$ are the sets of all, positive and negative labels, respectively, and~$\II$ is the characteristic ($0/1$) function counting how many times the argument is satisfied. Setting \eqref{eq:problem} includes TopPush~\cite{li2014top} which minimizes the number of positive samples below the highest-ranked negative sample. This fits into \eqref{eq:problem} with~$\lambda_1=0$,~$\lambda_2=1$ and~$t=\max_{i\in I^-} z_i$.

Figure~\ref{fig:difference} shows the difference between the standard approach with cross-entropy and accuracy at the top. While classifier 1 has good total accuracy, its top accuracy is subpar because of the few negative outliers. On the other hand, classifier 2 has worse total accuracy, but its top accuracy is extremely good because more than half of the positive samples are on the top. While classifier 1 selected different thresholds for the accuracy and top metrics, these thresholds coincide for classifier 2.

Table~\ref{table:summary} shows other special cases of \eqref{eq:problem} including maximizing precision at a given level of recall~\cite{mackey2018constrained} or recall at~$K$. The threshold~$t$ always equals to the sample with the~$j^*$-th highest score on all, positive, or negative samples. The problems differ only in~$j^*$ and from which samples the threshold is computed. For example, Pat\&Mat-NP~\cite{adam2019patmat} minimizes the false negative rate (equivalently maximizes the true positive rate) under the constraint that the false positive rate is at most~$\tau$.

\begin{table}[!ht]
  \centering
  \begin{tabular}{@{}llllll@{}}
    \toprule
    Name &~$\lambda_1$ &~$\lambda_2$ & $t$ computed from & $j^*$ \\
    \midrule
    Prec@Rec    & $1$ & $0$ & positive samples & $n^+\tau$ \\
    Rec@K       & $0$ & $1$ & all samples      & $K$ \\
    TopPush     & $0$ & $1$ & negative samples & $1$ \\
    Pat\&Mat-NP & $0$ & $1$ & negative samples & $n^-\tau$ \\
    \bottomrule
  \end{tabular}
  \caption{Selected problems of setting \eqref{eq:problem}. False-positives and false-negatives have weights~$\lambda_1$ and~$\lambda_2$, the threshold~$t$ equals to the sample with the~$j^*$-th highest score on all, positive, or negative samples.}
  \label{table:summary}
\end{table}

\subsection{Related works}

There is a close connection between accuracy at the top and ranking problems~\cite{batmaz2019review,werner2019review}. This was, together with similarities to the Neyman-Pearson problem, showed in~\cite{adam2019patmat}. A special case of the ranking problems attempts to rank positive samples above negative samples. Several approaches, such as RankBoost~\cite{freund2003efficient}, Infinite Push~\cite{agarwal2011infinite} or~$p$-norm push~\cite{rudin2009pnorm} employ a positive-negative pairwise comparison of scores, which can handle only small datasets. TopPush~\cite{li2014top} converts the pairwise sum into a single sum and minimizes the false-negatives below a threshold given by the maximum score corresponding to negative samples. Thus, it converts ranking into accuracy at the top problems.

Two approaches for solving \eqref{eq:problem} exist. The first approach considers the threshold constraint as it is, while the second approach uses heuristics to approximate it. In the first approach, Acc@Top~\cite{boyd2012accuracy} argues that the threshold equals one of the scores. They fix the index of a sample and solve as many optimization problems as there are samples.~\cite{eban2017scalable,adam2019patmat,kumar2021implicit} write the threshold as a constraint and replace both the objective and the constraint via surrogates.~\cite{eban2017scalable} uses Lagrange multipliers to obtain a minimax problem,~\cite{mackey2018constrained} implicitly removes the threshold as an optimization variable and uses the chain rule to compute the gradient while~\cite{macha2020nonlinear} solves an SVM-like dual formulation with kernels.~\cite{grill2016learning} uses the same formulation but applies surrogates only to the objective and recomputes the threshold after each gradient step. TFCO~\cite{cotter2019optimization} solves a general class of constrained problems via a minimax reformulation. In the second approach, SoDeep~\cite{engilberge2019sodeep} or SmoothI~\cite{thonet2021smoothi} use the fact that the threshold may be easily computed from sorted scores. They approximate the sorting operator by a network trained on artificial data. AP-Perf~\cite{fathony2019ap} considers a general metric and hedges against the worst-case perturbation of scores. The authors argue that the problem is bilinear in scores and use duality arguments. However, the bilinearity is lost when optimizing with respect to the weights of the original network. 

\section{DeepTopPush as a method for maximizing accuracy at the top}\label{sec:solving}

This section first shows a basic algorithm to solve \eqref{eq:problem}. We then argue that the stochastic gradient descent produces a biased estimate of the true gradient, and we mention two strategies for mitigating this bias. Based on one strategy, we propose the \DeepTopPush algorithm. The whole section assumes that the classifier~$f$ is differentiable.

\subsection{Basic algorithm for solving accuracy at the top}

Even though the presented technique can be applied to any formulation from Table~\ref{table:summary}, for simplicity, we derive it only for the TopPush formulation, where~$\lambda_1=0$ and~$\lambda_2=1$. This amounts to minimizing the false-negatives in \eqref{eq:problem}. Since the function~$\II$ in the formulation \eqref{eq:problem} is discontinuous, it is usually replaced by a general surrogate function~$l$ which is continuous and non-decreasing. This leads to
\begin{equation}\label{eq:problem_surr1}
  \begin{aligned}
    \minimize{w,z,t}
    & \frac{1}{n^+}\sum_{i\in I^+}l(t-z_i) \\
    \st
    & z_i = f(\bm{w};\bm{x}_i), \\
    & t   = G(\{(z_i, y_i)\}_{i\in I}).
\end{aligned}
\end{equation}
To apply the stochastic gradient descent, we need to compute the gradient. The core idea follows~\cite{mackey2018constrained} which was proposed in a more general context in~\cite{adam2019machine}. It rewrites problem \eqref{eq:problem_surr1} into its equivalent form
\begin{equation}\label{eq:problem_surr2}
  \minimize{w}
  \frac{1}{n^+}\sum_{i\in I^+}l\Brac{G\Brac{\{(f(\bm{w};\bm{x}_j), y_j)\}_{j\in I}} - f(\bm{w};\bm{x}_i)}.
\end{equation}
This form removed the constraints and it has the advantage that the only optimization variable is~$\bm{w}$ instead of~$(\bm{w}, \bm{z}, t)$ in \eqref{eq:problem_surr1}. In all cases from Table~\ref{table:summary}, the threshold~$t$ always equals to one of the scores, let it have index~$j^*$ and then~$t=z_{j^*}$. Denoting the objective of \eqref{eq:problem_surr2} by~$L(\bm{w})$, the chain rule implies that the gradient of the objective from \eqref{eq:problem_surr2} equals to
\begin{equation}\label{eq:grad1}
  \nabla L(\bm{w}) = \frac{1}{n^+} \sum_{i\in I^+}l'(t-z_i)\big(\nabla_w f(\bm{w};\bm{x}_{j^*}) - \nabla_w f(\bm{w};\bm{x}_i)\big).
\end{equation}
The stochastic gradient descent replaces the sum over all positive samples~$I^+$ with a sum over all positive samples in a minibatch~$\Imin^+$. However, as both the threshold~$t$ and the index~$j^*$ depend on all scores~$z_i$, they need to be approximated on the minibatch as well. We denote these approximations by~$\hat t$ and~$\hat j$, respectively. Denoting the number of positive samples in the minibatch by~$\nmin^+$, we replace the true gradient \eqref{eq:grad1} by the \textit{sampled gradient}
\begin{equation}\label{eq:grad2}
  \nabla \hat L = \frac{1}{\nmin^+}\sum_{i\in \Imin^+}l'(\hat t-z_i)\big(\nabla_w f(\bm{w};\bm{x}_{\hat j}) - \nabla_w f(\bm{w};\bm{x}_i) \big),
\end{equation}
The most straightforward way is to choose the sampled threshold~$\hat t$ by the same rule as the true threshold~$t$. As an example, if~$t$ is the~$100^{\rm th}$ largest score on the whole dataset and~$\frac{n}{\nmin}=20$ is the ratio of sizes of the whole dataset and of the minibatch, we select the sampled threshold~$\hat t$ as the~$5^{\rm th}$ largest score on the minibatch. We summarize this procedure in Algorithm~\ref{alg1}.

\begin{figure*}
  \begin{minipage}{0.48\textwidth}
    \begin{algorithm}[H]
      \centering
      \begin{algorithmic}[1]
        \State Initialize weights~$\bm{w}$
        \Repeat
        \State Select minibatch~$\Imin$
        \State \phantom{$\Iminh$}
        \State Compute~$z_i\gets f(\bm{w};\bm{x}_i)$ for~$i\in\Imin$
        \State Set~$\hat t \gets G(\{(z_i,y_i)\}_{i\in\Imin})$
        \State 
        \State Compute~$\nabla \hat L$ based on~$\Imin$\phantom{$\Iminh$}
        \State Make a gradient step
        \Until{stopping criterion is satisfied}
      \end{algorithmic}
      \caption{Basic algorithm for solving \eqref{eq:problem} \\}
      \label{alg1}
    \end{algorithm}
  \end{minipage}
  \hfill
  \begin{minipage}{0.48\textwidth}
    \begin{algorithm}[H]
      \centering
      \begin{algorithmic}[1]
        \State Initialize weights~$\bm{w}$, random index~$j^*$
        \Repeat
        \State Select minibatch~$\Imin$
        \State Enhance minibatch~$\Iminh = \Imin \cup \{j^*\}$
        \State Compute~$z_i\gets f(\bm{w};\bm{x}_i)$ for~$i\in\Iminh$
        \State Set~$\hat t \gets \{\max z_i \mid i\in \Iminh \cap I^-\}$
        \State Find index~$j^*$ such that~$t = z_{j^*}$
        \State Compute~$\nabla \hat L$ based on~$\Iminh\cap I^+$
        \State Make a gradient step
        \Until{stopping criterion is satisfied}
      \end{algorithmic}
      \caption{DeepTopPush as an efficient method for maximizing accuracy at the top.}
      \label{alg2}
    \end{algorithm}
  \end{minipage}
\end{figure*}

\subsection{Bias of the sampled gradient}

Convergence proofs of the stochastic gradient descent require that the sampled gradient is an unbiased estimate of the true gradient~\cite{bottou2018optimization}. This means that
\begin{equation}\label{eq:defin_bias}
  \bias(\bm{w}) := \nabla L(\bm{w}) - \EE \nabla \hat L(\bm{w})
\end{equation}
equals to~$0$ for all~$\bm{w}$. A comparison of \eqref{eq:grad1} and \eqref{eq:grad2} shows that a necessary condition is that the sampled threshold~$\hat t$ is an unbiased estimate of the true threshold~$t$. However, the sampled version underestimates the true value, which is evident for the maximum where the sampled maximum is always smaller or equal to the true maximum. The next result quantifies the difference between the sampled and true thresholds.

\begin{proposition}[\cite{glynn1996importance}]\label{proposition:bound}
  Let~$X$ be an absolutely continuous random variable with distribution function~$F$, let~$X_1,\dots,X_n$ be iid samples from~$X$ and let~$\tau\in(0,1)$. Denote the true threshold~$t=F^{-1}(1-\tau)$ and the sampled threshold~$\hat t=X_{[\lceil n\tau\rceil]}$. If~$F$ is differentiable with a positive gradient at~$t$, then
  \begin{equation*}
    \sqrt{n}(t - \hat t) \rightarrow N\left(0, \frac{\tau(1-\tau)}{F'(t)^2}\right),
  \end{equation*}
  where the convergence is in distribution and~$N$ denotes the normal distribution.
\end{proposition}

This proposition states that when the minibatch size increases to infinity, the variance of the sampled threshold is approximately~$\frac{\tau(1-\tau)}{nF'(t)^2}$. Figure~\ref{fig:thresholds1} in the introduction shows this empirically for the case where the scores follow the standard normal distribution and~$\tau=0.01$ is the desired top fraction. The approximation is poor with both large bias and standard deviation. Even though this result gives us insight into the bias of the sampled threshold, we are ultimately interested in the bias of the sampled gradient~$\nabla \hat L(\bm{w})$. To do so, recall that~$j^*$ is the threshold index on the whole dataset ($t=z_{j^*}$) while~$\hat j$ is the threshold index on the minibatch ($\hat t=z_{\hat j}$). We split the computation based on whether these two indices are identical.

\begin{lemma}\label{lemma:convergence}
  Let~$j^*$ be unique. Assume that the selection of positive and negative samples into the minibatch is independent and that the threshold is computed from negative samples while the objective is computed from positive samples. Then the conditional expectation of the sampled gradient satisfies
  \begin{equation*}
    \EE\Brac{\nabla \hat L(\bm{w}) \mid \hat j=j^*} =  \nabla L(\bm{w}).
  \end{equation*}
\end{lemma}
\begin{proof}
  If~$j^*$ is unique, then the true threshold~$t$ is a differentiable function. The differentiability of~$L$ and~$\hat L$ follows from the chain rule. If~$\hat j=j^*$ holds, then the sampled gradient equals to
  \begin{equation}\label{eq:grad_min_aux}
    \nabla \hat L(\bm{w})= \frac{1}{\nmin^+}\sum_{i\in \Imin^+}l'(t-z_i)\big(\nabla_w f(\bm{w};\bm{x}_{j^*}) - \nabla_w f(\bm{w};\bm{x}_i) \big).
  \end{equation}
  The summands are identical to the ones in \eqref{eq:grad1}. Since the sum is performed with respect to positive samples, the threshold is computed from negative samples, the lemma statement follows.
\end{proof}

Now we present the main result about the bias.

\begin{theorem}\label{theorem:convergence}
  Under the assumptions of Lemma~\ref{lemma:convergence}, the bias of the sampled gradient from \eqref{eq:defin_bias} satisfies
  \begin{equation}\label{eq:comp_bias}
    \bias(\bm{w}) = \PP(\hat j\neq j^*) \Brac{\nabla L(\bm{w}) - \EE\Brac{\nabla \hat L(\bm{w}) \mid \hat j\neq j^*}}.
  \end{equation}
\end{theorem}
\begin{proof}
  The law of total expectation implies
  \begin{equation*}
    \begin{aligned}
      \EE \nabla \hat L(\bm{w})
      & = \PP(\hat j=j^*)\EE(\nabla \hat L(\bm{w}) \mid \hat j=j^*) \\
      & \qquad + \PP(\hat j\neq j^*)\EE(\nabla \hat L(\bm{w}) \mid \hat j\neq j^*),
    \end{aligned}
  \end{equation*}
  from where the statement follows due to definiton \eqref{eq:defin_bias} and Lemma~\ref{lemma:convergence}.
\end{proof}

The assumptions of Theorem~\ref{theorem:convergence} holds for all methods from Table~\ref{table:summary} with the exception of Rec@K. For this method, the bias contains an additional term, as we show in the appendix.

The bias \eqref{eq:comp_bias} consists of a multiplication of two terms. We propose two strategies for reducing the bias. The first strategy reduces both terms, while the second strategy reduces only the first term.

\subsection{Bias reduction: Increasing minibatches size}\label{sec:bias1}

The natural choice to mitigate the bias is to work with large minibatches. Even though this is not a standard way, some works suggest this route~\cite{you2019large}. When the minibatch is large, it contains more samples and the chance that~$\hat j$ differs from~$j^*$ decreases. This reduces the first term in~\eqref{eq:comp_bias}. Moreover, Proposition~\ref{proposition:bound} ensures that the difference between the sampled threshold~$\hat t$ and the true threshold~$t$ is small. Then the difference between the true gradient \eqref{eq:grad1} and the sampled gradient \eqref{eq:grad2} decreases as well. This reduces the second term in \eqref{eq:comp_bias}. This approach is applicable to any method from Table~\ref{table:summary}.

\subsection{Bias reduction: Incorporating delayed values}\label{sec:bias2}

Various reasons may enforce the use of small minibatches. Then Algorithm~\ref{alg1} is not suitable for a small fraction of top samples. For example, a minibatch of size~$32$ with~$16$ negative samples must have thresholds~$\tau\ge \frac{100}{16}=6.25\%$. However, we need to aim for much smaller thresholds.

We propose a simple fix based on the reasoning that when the weights~$\bm{w}$ of a neural network are updated, the scores~$\bm{z}$ usually do not change much, especially for a small learning rate. This means that if a sample has the largest score, it will likely have the largest score even after the gradient step. Since the threshold~$t$ for TopPush equals the largest score corresponding to negative samples, we can easily track it. We enhance the current minibatch by the negative sample from the previous minibatch with the highest score. This significantly increases the chance that the sampled threshold is the true threshold and, due to the first term in \eqref{eq:comp_bias}, reduces the bias of the sampled gradient.

We summarize the procedure in Algorithm~\ref{alg2} and show it next to Algorithm~\ref{alg1} to highlight the differences. In every iteration, it stores the index~$j^*$ of the sample, which equals the threshold (step~7). We add it to the enhanced minibatch (step 4). Since we can track only the maximum, we set the threshold as the maximum of scores from negative samples (step~6) and minimize false-positives. Since Algorithm~\ref{alg2} uses the same formulation as \textit{TopPush}~\cite{li2014top} but can handle an arbitrary classifier, we name it \DeepTopPush. We provide empirical evidence of why our technique works later in Section~\ref{sec:delay}.

\section{Numerical experiments}\label{sec:numerics}

This section presents numerical results for \DeepTopPush. Table~\ref{table:summary} shows that it is similar to \PatMatNP. While the former maximizes the number of positives above the largest negative, while the latter maximizes the number of positives above the~$n^-\tau$-largest negative. The former may be understood as requiring no false-positives, while the latter allows for false positive rate~$\tau$.

Section~\ref{sec:bias1} showed that we can use large minibatches to obtain good results for \PatMatNP for small fractions of top samples~$\tau$. Section~\ref{sec:bias2} showed that \DeepTopPush works well even with small minibatches if we track the threshold by enhancing the minibatch by one sample. We present numerical comparisons in several sections, each with a different purpose. Comparison with the prior art \TFCO and \APPerf is performed on several visual recognition datasets and shows that \DeepTopPush outperforms other methods. Then we present two real-world applications. The first one shows that \DeepTopPush can handle ranking problems. The second one presents results on a complex malware detection problem. Finally, we show similarities between \DeepTopPush and \PatMatNP and explain why enhancing the minibatch in Algorithm~\ref{alg2} works.

\subsection{Dataset description and Computational setting}\label{sec:set}

We consider the following image recognition datasets: FashionMNIST~\cite{xiao2017fashionmnist}, CIFAR100~\cite{krizhevsky2009learning}, SVHN2~\cite{netzer2011reading} and ImageNet~\cite{russakovsky2015imagenet}. These datasets were converted to binary classification tasks by selecting one class as the positive class and the rest as the negative class. ImageNet merged turtles and non-turtles. We also consider the 3A4 dataset~\cite{ma2015deep} with molecules and their activity levels. Finally, malware analysis reports of executable files were provided by a cybersecurity company. This is an extremely tough dataset as individual samples are JSON files whose size ranges from 1kB to 2.5MB. Moreover, they contain different features, and their features may have variable lengths. All datasets are summarized in the appendix.

We use truncated quadratic loss~$l(z) = (\max\{0, 1 + z\})^2$ as the surrogate function and~$\tau=\frac{1}{n^-}$ and~$\tau=0.01$. This first one computes the true positive rate above the second highest-ranked negative, while the latter allows for the false positive rate of~$1\%$. All algorithms were run for~$200$ epochs on an NVIDIA P100 GPU card with balanced minibatches of 32 samples. The only exception was Malware Detection, which was run on a cluster in a distributed manner, and where the minibatch size was~$20000$. For the evaluation of numerical experiments, we use the standard receiver operating characteristic (ROC) curve. All results are computed from the test set. All codes were implemented in the Julia language~\cite{bezanson2017julia}. The network structure was the same for all methods; we describe them in the online appendix.

\subsection{Comparison with prior art}\label{sec:comparison}

We compare our methods with \BaseLine, which uses the weighted cross-entropy. Moreover, we use two prior art methods which have codes available online, namely \TFCO~\cite{cotter2019optimization,narasimhan2019optimizing} and \APPerf~\cite{fathony2019ap}. We did not implement the original TopPush because its duality arguments restrict the classifiers to only linear ones. Table~\ref{table:time} shows the time requirement per epoch. All methods besides \APPerf have similar time requirements, while \APPerf is much slower. This difference increases drastically when the minibatch size increases, as noted in~\cite{fathony2019ap}. We do not present the results for SVHN for \APPerf because it was too slow and for \TFCO because we encountered a TensorFlow memory error. All these methods are designed to maximize true-positives when the false positive rate is at most~$\tau$. This is the same as for \PatMatNP.

\begin{table}[!ht]
  \centering
  \begin{tabular}{@{}llll@{}}
      \toprule      
       & FashionMNIST & CIFAR100 & SVHN \\
      \midrule
      BaseLine
        & 4.4s & 5.1s & 62.8s \\
      DeepTopPush
        & 4.8s & 5.6s & 66.6s \\
      Pat\&Mat-NP
        & 4.8s & 5.6s & 66.6s \\
      TFCO
        & 7.2s & 6.5s & - \\
      AP-Perf
        & 95.3s & 81.2s & - \\
      \bottomrule
  \end{tabular}
  \caption{Time requirements per epoch for investigated methods for minibatches of size~$\nmin=32$.}
  \label{table:time}
\end{table}

\begin{table}[ht]
  \centering
  \footnotesize
  \begin{tabular}{@{}c|llllll@{}}
    \toprule
    & \thead{Dataset}
    & \thead{BaseLine}
    & \thead{DeepTopPush}
    & \thead{Pat\&Mat-NP}
    & \thead{TFCO}
    & \thead{AP-Perf} \\
    \midrule
    \multirow{4}{*}{\rotatebox[origin=c]{90}{\parbox[c]{1.5cm}{\centering tpr@fpr $\tau=\nicefrac{1}{n^-}$}}}
    & FashionMNIST
      & $5.06 \pm 1.41$
      & \best $27.30 \pm 5.91$
      & $22.21 \pm 5.62$
      & $11.30 \pm 3.44$
      & $9.90$ \\
    & CIFAR100
      & $1.70 \pm 0.46$
      & \best $14.40 \pm 5.44$
      & $8.10 \pm 3.45$
      & $7.70 \pm 2.28$
      & $5.00$ \\
    & 3A4
      & $2.58 \pm 0.61$ 
      & \best $5.61 \pm 1.70$
      & $3.79 \pm 0.90$
      & $3.03 \pm 1.52$
      & $3.03$ \\
    & SVHN
      & $6.51 \pm 1.37$
      & \best $12.21 \pm 5.39$
      & $12.07 \pm 4.41$ 
      & - & -\\
    \midrule
    \multirow{4}{*}{\rotatebox[origin=c]{90}{\parbox[c]{1.5cm}{\centering tpr@fpr $\tau=0.01$}}}
    & FashionMNIST
      & $63.14 \pm 1.39$
      & \best $75.37 \pm 1.18$
      & $74.11 \pm 1.00$
      & $73.27 \pm 2.92$
      & $64.60$ \\
    & CIFAR100
      & $49.40 \pm 4.90$
      & \best $70.20 \pm 2.14$
      & $66.30 \pm 2.33$
      & $67.30 \pm 1.79$
      & $65.00$ \\
    & 3A4
      & $57.80 \pm 0.35$ 
      & $60.08 \pm 3.35$
      & \best $65.91 \pm 0.59$
      & $54.55 \pm 10.22$
      & $63.64$ \\
    & SVHN
      & $84.72 \pm 0.84$
      & $91.05 \pm 1.45$
      & \best $91.07 \pm 0.30$
      & - & - \\
    \bottomrule
  \end{tabular}
  \caption{The true positive rates (in \%) at two levels of false positive rates averaged across ten indepenedent runs with standard deviation. The best methods are highlighted.}
  \label{tab:Overall comparison}
\end{table}

Table~\ref{tab:Overall comparison} shows the true positive rate (tpr) above the second-largest negative and at the prescribed false positive rate (fpr)~$\tau=0.01$. Using the second-largest negative, which corresponds to~$\tau=\frac{1}{n^-}$, allows for one outlier. The results are averaged over ten independent runs except for AP-Perf, which is too slow. The best result for each metric (in columns) is highlighted. All methods are better than \BaseLine. This is not surprising as all these methods are designed to work well for low false positive rates. \DeepTopPush outperforms all other methods at the top, while it performs well at the low fpr of~$\tau=0.01$. There \PatMatNP, which also falls into our framework, performs well. Both these methods outperform the state of the art methods. 

Figure~\ref{fig: roc curves} \textbf{A)} shows the ROC curves on CIFAR100 averaged over ten independent runs. We use the logarithmic~$x$ axis to highlight low fpr modes. \DeepTopPush performs significantly the best again whenever the false positive rate is smaller than~$0.01$.

As a further test, we performed a simple experiment on ImageNet. We modified the pre-trained EfficientNet B0~\cite{tan2019efficientnet} by removing the last dense layer and adding another dense layer with one output. Then we retrained the newly added layer to perform well at the top. The original EfficientNet achieved~$68.0\%$ at the top, while \DeepTopPush achieved~$70.0\%$ for the same metric. This shows that \DeepTopPush can provide better accuracy at the top than pre-trained networks.

\subsection{Application to ranking}

The 3A4 dataset contains information about activity levels of approximately~$50000$ molecules, each with about~$10000$ descriptors. The activity level corresponds to the usefulness of the molecule for creating new drugs. Since medical scientists can focus on properly investigating only a small number of molecules, it is important to select a small number of molecules with high activity.

We converted the continuous activity level into binary by considering a threshold on the activity. Since the input is large-dimensional, and there is no spatial structure to use convolutional neural networks, we used PCA to reduce the dimension to~$100$. Then we created a network with two hidden layers and applied \DeepTopPush to it. The test activity was evaluated at the continuous (and not binary level). Table~\ref{tab:Overall comparison} shows again the results at the top. \DeepTopPush outperforms other methods. Figure~\ref{fig:molecules} shows that high scores (output of the network) indeed correspond to high activity. Thus, even though the problem was ``binarized'' and its dimension reduced, our algorithm was able to select a small number of molecules with high activity levels. These molecules can be used for further manual (expensive) investigation.

\begin{figure}[!ht]
  \centering
  \includegraphics[width = \linewidth]{images/deep_molecules.pdf}
  \caption{Results for the 3A4 dataset. The goal was to assign large scores to a few molecules with high activity (scores on top-right are preferred).}
  \label{fig:molecules}
\end{figure}

\begin{figure*}[!ht]
  \centering
  \includegraphics[width = \linewidth]{images/deep_results1.pdf}
  \caption{\textbf{A)} ROC curves averaged over ten runs on the CIFAR100 dataset. \textbf{B)} ROC curve for Malware Detection dataset. The circles show the thresholds the methods were optimized for.}
  \label{fig: roc curves}
\end{figure*}

\subsection{Real-world application}

This section shows a real-world application of the accuracy at the top. A renowned cybersecurity company provided malware analysis reports of executable files. Its structure is highly complicated because each sample has a different number of features, and features may have a complicated structure, such as a list of ports to which the file connects. This is in sharp contrast with standard datasets, where each sample has the same number of features, and each feature is a real number. We processed the data by a public implementation of hierarchical multi-instance learning (HMIL)~\cite{pevny2017using}. Then we applied \DeepTopPush and \PatMatNP at~$\tau=10^{-3}$ and~$\tau=10^{-2}$. The latter maximizes the true positives rate when the false positive rate is at most~$\tau$. The minibatch size was~$20000$, which allowed us to obtain precise threshold estimates and unbiased sampled gradients due to Section~\ref{sec:bias1}.

Figure~\ref{fig: roc curves} \textbf{B)} shows the performance on the test set. \DeepTopPush is again the best at low false positive rates. This is extremely important in cybersecurity as it prevents false alarms for malware. Even at the extremely low false positive rate~$\tau=10^{-5}$, our algorithm correctly identified~$46\%$ of malware. The circles denote the thresholds for which the methods were optimized. \DeepTopPush should have the best performance at the leftmost point, \PatMatNP ($\tau=10^{-3}$) at~$\tau=10^{-3}$ and similarly \PatMatNP($\tau=10^{-2}$).

\subsection{Impact of enhancing the minibatch}\label{sec:delay}

The crucial aspect of \DeepTopPush is enhancing the minibatch by one sample. In all presented results with the exception of the Malware Detection, the minibatch contained only 32 samples. Then the discussion in Section~\ref{sec:bias2} implies that \PatMatNP equals to \DeepTopPush without enhancing the minibatch. In other words, \PatMatNP uses Algorithm~\ref{alg1} while \DeepTopPush uses Algorithm~\ref{alg2}. As Table~\ref{tab:Overall comparison} clearly shows that \DeepTopPush ourperforms \PatMatNP, this implies that using the delayed values is beneficial.

Figure~\ref{fig:thresholds2} shows explanation for this behaviour. The full blue line shows the behaviour of \DeepTopPush while the dotted grey line shows \PatMatNP. As explained in the previous paragraph, their difference demonstrates the effect of enhancing the minibatch by one delayed value. The top subfigure compares thresholds with the true threshold (dashed black). While the threshold for \PatMatNP jumps wildly, it is smooth for \DeepTopPush, and it often equals the true threshold. Theorem~\ref{theorem:convergence} then implies that our sampled gradient is an unbiased estimate of the true gradient. This is even more pronounced in the bottom subfigure, which shows the angle between the true gradient and the computed gradient. This angle is important because~\cite{nocedal2006numerical} showed that if this angle is uniformly in the interval~$[0,90)$, then gradient descent schemes converge. This is precisely what happened for \DeepTopPush. When the threshold is correct, the true and estimated gradients are parallel to each other, and the gradient descent moves in the correct direction.

\begin{figure}[!ht]
  \centering
  \includegraphics[width = \linewidth]{images/deep_thresholds.pdf}
  \caption{The thresholds (top) and angle between true and sampled gradients (bottom) for Algorithm~\ref{alg1} (full blue) and Algorithm~\ref{alg2} (dotted gray).}
  \label{fig:thresholds2}
\end{figure}

\section{Conclusions}

We proposed \DeepTopPush as an efficient method for solving the constrained non-decomposable problem of accuracy at the top, which focuses on the performance only above a threshold. We implicitly removed some optimization variables, created an unconstrained end-to-end network and used the stochastic gradient descent to train it. We modified the minibatch so that the sampled threshold (computed on a minibatch) is a good estimate of the true threshold (computed on all samples). We showed both theoretically and numerically that this procedure reduces the bias of the sampled gradient. The time increase over the standard method with no threshold is small. We demonstrated the usefulness of \DeepTopPush both on visual recognition datasets, a ranking problem and on a real-world application of malware detection.

\chapter{Numerical Experiments}\label{chap: experiments}

In the previous sections, we derived a general framework for classification at the top and showed that multiple well-known formulations fall into it. The summary of all formulations presented in this work is in Table~\ref{tab: summary formulations}. The goal of this chapter is to verify the properties of these formulations experimentally.

\section{Settings}\label{sec: settings}

In this section, we describe in detail all settings used for experiments. The section consists of five subsections. The first one discusses which formulations from Table~\ref{tab: summary formulations} we use for the experimental evaluation. In this subsection, we also introduce baseline formulations used for the comparison. In the second one, we introduce datasets used in experiments and describe their structure. A detailed description of the datasets is then provided in separate sections with the results of the experiments. The third and fourth subsections contain a detailed description of performance metrics. The last subsection contains a description of tools used for implementation. All codes used for experiments, as well as all configurations of all experiments, are publicly available on GitHub:
\begin{center}
  \url{https://github.com/VaclavMacha/ClassificationAtTopExperiments.jl}
\end{center}

\subsection{Formulations}

Formulations from Table~\ref{tab: summary formulations} can be divided into three categories:
\begin{itemize}
  \item The first category contains \TopPush and \TopPushK formulations. These formulations minimize the surrogate approximation of the false-negative rate and use the mean of a small fraction of the negative samples with the highest scores as a threshold.
  \item The second category consists of \Grill, \TopMeanK, and \PatMat formulation. Similarly to \TopPush and \TopPushK, these formulations use the surrogate approximation of the false-negative rate as an objective function. The \Grill formulation also adds the surrogate approximation of the false-positive rate into the objective function. All three formulations use some approximation of the top $\tau$-quantile of all scores as a threshold.
  \item The last category consists of \GrillNP, \tauFPL, and \PatMatNP. These formulations use the same objectives as their corresponding formulations from the previous category. However, they differ in the definition of the decision threshold. All three formulations use some kind of approximation of the top $\tau$-quantile of negative scores as a threshold.
\end{itemize}
To simplify the setup of all experiments, we decided to focus on formulations that use only negative samples for the threshold computation, i.e., formulations from the first and third categories. Moreover, we decided to omit the \GrillNP formulation in the final experiments because of its poor results in preliminary experiments. The performance of selected formulations can be compared by basic performance metrics, as shown later in Section~\ref{sec: performance criteria}.

In total, we use four different formulations from Table~\ref{tab: summary formulations}, namely \TopPush, \TopPushK, \tauFPL, and \PatMatNP. Moreover, for \TopPushK, we use two different values of~$K = \{5, 10\}$ and consider the resulting formulations as separate formulations, i.e., we have \TopPushK(5) and \TopPushK(10). Similarly, for \tauFPL and \PatMat we use two different values of~$\tau = \{0.01, 0.05\}.$ For all formulations, we use the hinge loss defined in Notation~\ref{not: surrogates} as a surrogate function.

The final number of unique formulations is seven. To show that they bring advantages, we must compare them to standard methods. In previous chapters, we showed how to solve presented formulations in their primal (Chapters~\ref{chap: linear} and~\ref{chap: deep}) and dual form (Chapter~\ref{chap: dual}). Whenever we use the primal form in the experiments, we use binary cross-entropy defined in the following way as a baseline formulation
\begin{mini}{\bm{w}}{
  \frac{1}{\nall} \sum_{i \in \I} \Brac{- y_i \log(s_i) - (1 - y_i) \log (1 - s_i)}
  }{\label{eq: crossentropy}}{}
  \addConstraint{s_i}{= f(\bm{x}_i; \bm{w}), \quad i \in \I.}
\end{mini}
We decided to use binary cross-entropy since it is one of the most used objective functions for binary classification in machine learning applications. We will denote binary cross-entropy in the following text as \BaseLine. In experiments with dual forms of our formulations, we use C-SVC variant of SVM~\cite{boser1992training, cortes1995support,chang2011libsvm} defined by
\begin{mini}{\bm{w}, b, \bm{\xi}}{
  \frac{1}{2} \norm{\bm{w}}^2 + C \sum_{i \in \I} \xi_i
  }{\label{eq: SVM}}{}
  \addConstraint{y_i}{\Brac{\bm{w}^{\top} \phi(\bm{x}_i) + b} \geq 1 - \xi_i, \quad i \in \I}
  \addConstraint{\xi_i}{\geq 0, \quad i \in \I,}
\end{mini}
where~$y_i \in \{-1, 1\}$ for all~$i \in \I$ and~$\phi(\bm{x}_i)$ maps~$\bm{x}_i$ into a higher-dimensional space (see Section~\ref{sec: kernels}). The corresponding dual form is as follows
\begin{maxi}{\bm{\alpha}}{
  - \frac{1}{2} \bm{\alpha}^{\top} \K \bm{\alpha} - \sum_{i = 1}^{\nall} \alpha_i
  }{\label{eq: SVM dual}}{}
  \addConstraint{\sum_{i = 1}^{\nall} y_i \alpha_i}{= 0}
  \addConstraint{0 \leq \alpha_i }{\leq C, \quad i = 1, 2, \ldots, \nall,}
\end{maxi}
where the kernel matrix~$\K$ is defined for all~$i, j = 1, 2, \ldots, \nall$ as
\begin{equation*}
  \K_{i,j} = y_i y_j k(\bm{x}_i, \bm{x}_j) = \phi(\bm{x}_i)^{\top} \phi(\bm{x}_j).
\end{equation*}
Note that the dual form of C-SVC is very similar to the dual forms of our formulations derived in Chapter~\ref{chap: dual}. We will denote C-SVC as \SVM.

In total, we have nine different formulations for experiments, as seen in Table~\ref{tab: formulations experiments summary}. \BaseLine formulation is used only for experiments with primal forms of our formulations, while \SVM is used only when dual forms are used. The following section discusses which hyper-parameters are used for each formulation. We also show how we convert these parameters for dual forms of the formulations.

\pagebreak

\subsection{Hyperparameters}

The selected formulations differ in the number of available hyper-parameters. Therefore, we decided to use a fixed value for all but one of the hyper-parameters for each formulation. We then use six different values for the remaining non-fixed hyper-parameter to fine-tune the formulation. For most of the considered formulations, the only hyper-parameter is the regularization constant~$\lambda$. The only exceptions are the formulations derived from \PatMatNP since they also have the scaling parameter~$\vartheta.$ Therefore, we use the following six values of this hyper-parameter
\begin{equation*}
  \lambda \in \Brac[c]{10^{-5}, 10^{-4}, 10^{-3}, 10^{-2}, 10^{-1}, 1}
\end{equation*}
for all formulations except \PatMatNP. For formulations derived from \PatMatNP, we fixed~$\lambda$ to~$10^{-3}$ and use the following six different values of the scaling parameter
\begin{equation*}
  \vartheta \in \Brac[c]{10^{-5}, 10^{-4}, 10^{-3}, 10^{-2}, 10^{-1}, 1}.
\end{equation*}
Since we used a slightly different (but equivalent) primal formulation for the derivation of the dual forms, we use~$\lambda$ to compute the hyper-parameter~$C$ used in these dual forms
\begin{equation*}
  C = \frac{1}{\lambda \ntil},
\end{equation*}
where~$\ntil = \nall$ for \SVM and~$\ntil = \npos$ otherwise. In all experiments, the best hyperparameter is selected based on the validation data and the appropriate performance metric. A summary of all used formulations and their hyper-parameters is in Table~\ref{tab: formulations experiments summary}.

\begin{table}[!ht]
  \centering
  \begin{NiceTabular}{lcccc}
    \CodeBefore
      \rowcolor{\headercol}{1}
      \rowcolors{3}{\rowcol}{}[restart]
    \Body
    \toprule
    \textbf{Formulation}
      & \textbf{Fixed parameters}
      & \textbf{Hyper-parameter}
      & \textbf{Primal Form}
      & \textbf{Dual Form} \\
    \midrule
    \BaseLine
      & ---
      & $\lambda$
      & \yesmark
      & \nomark \\
    \SVM
      & ---
      & $\lambda$
      & \nomark 
      & \yesmark \\
    \midrule
    \TopPush
      & ---
      & $\lambda$
      & \yesmark
      & \yesmark \\
    \TopPushK(5)
      & $K = 5$
      & $\lambda$
      & \yesmark
      & \yesmark \\
    \TopPushK(10)
      & $K = 10$
      & $\lambda$
      & \yesmark
      & \yesmark \\
    \tauFPL(0.01)
      & $\tau = 0.01$
      & $\lambda$
      & \yesmark
      & \yesmark \\
    \tauFPL(0.05)
      & $\tau = 0.05$
      & $\lambda$
      & \yesmark
      & \yesmark \\
    \PatMatNP(0.01)
      & $\tau = 0.01,$ $\lambda = 0.001$
      & $\vartheta$
      & \yesmark
      & \yesmark \\
    \PatMatNP(0.05)
      & $\tau = 0.05,$ $\lambda = 0.001$
      & $\vartheta$
      & \yesmark
      & \yesmark \\
    \bottomrule
  \end{NiceTabular}
  \caption{Summary of all formulations used for experiments. The first column shows the aliases used for the formulations when describing the experiment results. The second column shows fixed hyperparameters used for each formulation, while the third column shows which hyper-parameters are tuned using validation data. The last two columns indicate whether the formulation is used in experiments with primal forms, dual forms, or both.}
  \label{tab: formulations experiments summary}
\end{table}

\pagebreak

\subsection{Datasets}

We consider various datasets summarized in Table~\ref{tab: datasets summary} for the numerical experiments. All these datasets can be divided into three categories:
\begin{enumerate}
  \item \textbf{Image Recognition:} In this category, we test formulations from Table~\ref{tab: formulations experiments summary} on datasets from the domain of image recognition. We use this domain since it is one of the most popular with plenty of publicly available datasets.
  \item \textbf{Steganalysis:} In this category, we use selected formulations in the domain of steganalysis. In this domain, the problem of maximizing the true-positive rate at the specific level of the false-positive rate is well-known and essential, as we show at the beginning of Section~\ref{sec: steganalysis}.
  \item \textbf{Malware Detection:} In this category, we use selected formulations for malware detection. Like in steganalysis, maximizing the true-positive rate at the specific level of the false-positive rate is crucial for malware detection, as discussed at the beginning of Section~\ref{sec: malware detection}.
\end{enumerate}
Each category has a separate section later in the text. It is worth mentioning that not all datasets used in experiments are primarily designed for the classification at the top. For example, all datasets from the first category are general-purpose image classification datasets. We use these datasets since they are publicly available and well-known. Since all these datasets are multi-class, we need to adjust the labels to get binary classification problems. Therefore, for each data set, we select one class as the positive class and consider the rest as the negative class. 

\begin{table}[!ht]
  \centering
  \resizebox{\columnwidth}{!}{%
    \begin{NiceTabular}{lccrrrrrr}
      \CodeBefore
      \rowcolor{\headercol}{1-2}
      \rowcolors{4}{\rowcol}{}[restart]
      \Body
      \toprule
      \Block[c]{2-1}{\textbf{Dataset}}
      & \Block[c]{2-1}{$y^+$}
      & \Block[c]{2-1}{$d$}
      & \Block[c]{1-2}{\textbf{Train}}
      && \Block[c]{1-2}{\textbf{Validation}}
      && \Block[c]{1-2}{\textbf{Test}} \\
      \cline{4-9}
      &&& \Block[c]{1-1}{$n$}
      & \Block[c]{1-1}{$\frac{\npos}{n}$}
      & \Block[c]{1-1}{$n$}
      & \Block[c]{1-1}{$\frac{\npos}{n}$}
      & \Block[c]{1-1}{$n$}
      & \Block[c]{1-1}{$\frac{\npos}{n}$} \\
      \midrule
      MNIST
      & 1
      & $28 \times 28 \times 1$
      & 45 000
      & 11.3\%
      & 15 000
      & 11.2\%
      & 10 000
      & 11.4\% \\
      FashionMNIST
      & 1
      & $28 \times 28\times 1$
      & 45 000
      & 10.0\%
      & 15 000
      & 9.9\%
      & 10 000
      & 10.0\% \\
      CIFAR10
      & 1
      & $32 \times 32 \times 3$
      & 37 500
      & 10.0\%
      & 12 500
      & 9.9\%
      & 10 000
      & 10.0\% \\
      CIFAR20
      & 1
      & $32 \times 32 \times 3$
      & 37 500
      & 5.0\%
      & 12 500
      & 5.1\%
      & 10 000
      & 5.0\% \\
      CIFAR100
      & 1
      & $32 \times 32 \times 3$
      & 37 500
      & 1.0\%
      & 12 500
      & 1.0\%
      & 10 000
      & 1.0\% \\
      SVHN2
      & 1
      & $32 \times 32\times 3$
      & 54 944
      & 18.9\%
      & 18 313
      & 18.9\%
      & 26 032
      & 19.6\% \\
      SVHN2-Extra
      & 1
      & $32 \times 32\times 3$
      & 453 291
      & 17.3\%
      & 151 097
      & 17.1\%
      & 26 032
      & 19.6\% \\
      \midrule
      \bad{Nsf5}
      & ---
      & $22 510 \times 1$
      & 186 583
      & 9.1\%
      & 62 194
      & 9.1\%
      & 248 776
      & 9.1\% \\
      \bad{JMiPOD}
      & ---
      & $256 \times 256\times 3$
      & 186 515
      & 9.1\%
      & 62 172
      & 9.1\%
      & 248 686
      & 9.1\% \\
      \midrule
      \bad{Malware}
      & ---
      & variable
      & 6 580 166
      & 87.22\%
      & ---
      & ---
      & 800 346
      & 91.8\% \\
      \bottomrule
    \end{NiceTabular}
  }
  \caption{Structure of the used datasets: The training, validation and testing sets show the positive label~$y^+,$ the number of features~$d$, samples~$n$ and the fraction of positive samples~$\frac{\npos}{n}$. Datasets depicted in red are not publicly available.}
  \label{tab: datasets summary}
\end{table}

\subsection{Performance Criteria}\label{sec: performance criteria}

In this section, we describe which performance criteria are used for evaluation and how these criteria are related to the tested formulations.

As we discussed at the beginning of Section~\ref{sec: settings}, we decided to test only formulations that minimize the false-negative rate (or a combination of false-negative and false-positive rate) and use only negative samples for the threshold computation. This choice allows us to use simple metrics to compare used formulations. The first metric that we use in experiments is~$\tpratk$ defined as follows
\begin{equation*}
  \tpratk = \frac{1}{\npos} \sum_{i \in \Ipos} \Iverson{s_i \geq t} \quad \text{where} \quad t = \frac{1}{K} \sum_{j = 1}^{K} s^{-}_{[j]}.
\end{equation*}
This metric computes the true-positive rate at threshold~$t$ defined as the mean of $K$-largest negative scores. For~$K = 1$, the threshold corresponds to the threshold used by \TopPush formulation. Otherwise, threshold~$t$ corresponds to the threshold used by \TopPushK. Moreover, since minimizing the false-negative rate is equivalent to maximizing the true-positive rate, both \TopPush and \TopPushK should optimize the $\tpratk$ metric. In the upcoming experiments, we use this metric with three different values of~$K \in \{1, 5, 10\}.$

The second metric is defined in a similar way
\begin{equation*}
  \tpratfpr = \frac{1}{\npos} \sum_{i \in \Ipos} \Iverson{s_i \geq t} \quad \text{where} \quad t
  = \max \Set{t}{\frac{1}{\nneg} \sum_{i \in \Ineg} \Iverson{s_i \geq t} \geq \tau}.
\end{equation*}
This metric computes the true-positive rate at a specific top $\tau$-quantile of negative scores. This metric is ideal for testing the performance of \tauFPL and \PatMatNP formulations since both maximize the true-positive rate and use some approximation of the true top $\tau$-quantile of negative scores as a threshold. In experiments, we use this metric with two different values of~$\tau \in \{0.01, 0.05\}.$

The two previous metrics are specific to the formulations from our framework. However, we should also test if the baseline formulations work correctly. Since the baseline methods are designed to optimize overall performance, we use the area under the ROC curve to measure the overall performance. The summary of all used metrics is in Table~\ref{tab: metrics summary}.

\begin{table}[!ht]
  \centering
  \begin{NiceTabular}{lcccccc}
    \CodeBefore
    \rowcolor{\headercol}{1-2}
    \rowcolors{4}{\rowcol}{}[restart]
    \Body
    \toprule
    \Block[c]{2-1}{\textbf{Formulation}}
      & \Block[c]{2-1}{$\auroc$}
      & \Block[c]{1-3}{$\tpratk$}
      &&& \Block[c]{1-2}{$\tpratfpr$} \\
    \cline{3-7}
      && $1$  
      & $5$
      & $10$
      & $0.01$
      & $0.05$ \\
    \midrule
    \BaseLine
      & \yesmark
      & \nomark
      & \nomark
      & \nomark
      & \nomark
      & \nomark \\
    \SVM
      & \yesmark
      & \nomark
      & \nomark
      & \nomark
      & \nomark
      & \nomark \\
    \midrule
    \TopPush
      & \nomark
      & \yesmark
      & \nomark
      & \nomark
      & \nomark
      & \nomark \\
    \TopPushK(5)
      & \nomark
      & \nomark
      & \yesmark
      & \nomark
      & \nomark
      & \nomark \\
    \TopPushK(10)
      & \nomark
      & \nomark
      & \nomark
      & \yesmark
      & \nomark
      & \nomark \\
    \tauFPL(0.01) and \PatMatNP(0.01)
      & \nomark
      & \nomark
      & \nomark
      & \nomark
      & \yesmark
      & \nomark \\
    \tauFPL(0.05) and \PatMatNP(0.05)
      & \nomark
      & \nomark
      & \nomark
      & \nomark
      & \nomark
      & \yesmark \\
    \bottomrule
  \end{NiceTabular}
  \caption{The summary of all used performance metrics used for evaluation. In total, we use six different metrics and nine different formulations. For each formulation~\yesmark denotes the metric in which the formulation should be the best.}
  \label{tab: metrics summary}
\end{table}

\subsection{Critical Difference Diagrams}\label{sec: cd evaluation}

All metrics from Section~\ref{sec: performance criteria} can be used to compare different formulations on a single dataset. However, these metrics are unsuitable for comparing multiple formulations on multiple datasets. To address this issue, we follow the suggestion from~\cite{demvsar2006statistical} and use the Friedman test~\cite{friedman1940comparison}.

Consider that we have~$m,$ datasets, and~$k$ formulations. Then for each dataset~$i$, each formulation~$j$ is ranked by rank~$r^i_j$ according to some performance criterium. Any performance metric from the previous section can be used. The formulation that provides the best result gets ranked 1; the second best gets ranked 2, and so on. If two formulations provide the same results, the average ranks are assigned. The average rank overall dataset for formulation~$j$ is computed as
\begin{equation*}
  R_j = \frac{1}{m} \sum_{i = 1}^{m} r^{i}_{j}.
\end{equation*}
The Friedman test compares the average ranks of formulations under the null hypothesis, which states that all formulations are equivalent. Therefore, their average ranks should be equal. If the null hypothesis is rejected, we proceed with the post hoc Nemenyi test~\cite{nemenyi1963distribution} that compares all formulations to each other. The performance of the two formulations is significantly different if the corresponding average
ranks differ by at least the critical difference
\begin{equation*}
  CD = q_{\alpha} \sqrt{\frac{k(k + 1)}{6D}},
\end{equation*}
where critical values~$q_{\alpha}$ are based on the Studentized range statistic divided by~$\sqrt{2},$ see Table 5(a) in~\cite{demvsar2006statistical}. The results of this post hoc test can be easily visualized using critical difference diagrams proposed in~\cite{demvsar2006statistical}. The $x$-axis of such a diagram shows the average rank over all datasets for each formulation. Formulations that are not significantly different according to the Nemenyi test are connected using a green horizontal line. As an example, see Figure~\ref{fig: primal linear CD}.

\subsection{Implementation}

For the implementation of all experiments, we use the Julia programming language~\cite{bezanson2017julia}. The dual formulations are implemented from scratch, while the primal formulations are implemented using Flux.jl~\cite{innes:2018, Flux.jl-2018} library. This library provides all the necessary tools for building neural networks. Moreover, the library allows the implementation of a custom gradient for any function, which allows us to implement all formulations from Table~\ref{tab: summary formulations}. For experiments with SVM, we use the Julia wrapper for the LIBSVM library~\cite{chang2011libsvm}.

\section{Image Recognition}

In this section, we present experiments with six well-known image recognition datasets. All these datasets are publicly available. MNIST~\cite{deng2012mnist} and FashionMNIST~\cite{xiao2017fashionmnist} are grayscale datasets of digits and fashion items, respectively. CIFAR100~\cite{krizhevsky2009learning} is a dataset of colored images of different items grouped into 100 classes. CIFAR10 and CIFAR20 merge these classes into 10 and 20 superclasses, respectively. Finally, SVHN2~\cite{netzer2011reading} contains colored images of house numbers. All these datasets are originally divided only into training and test sets. We select 25\% samples from the training set to obtain the validation set. For a more detailed description of the structure of datasets, see Table~\ref{tab: datasets summary}.

None of these datasets is primarily designed for the problem of classification at the top. We presented these datasets to show that introduced formulations can be helpful even for general-purpose datasets and may lead to improve performance on specific metrics. However, there are some drawbacks to using these datasets. For example, many state-of-the-art neural network architectures achieve almost the perfect classification on some of these datasets. Therefore, there is no room for improvement, and we have to use much simpler architectures to show the behavior of formulations presented in this work (Section~\ref{sec: results primal nonlinear}).


\subsection{Primal Formulation: Linear Model}\label{sec: results primal linear}

In this section, we present results for a primal form of formulations from Table~\ref{tab: formulations experiments summary} with a linear model. For training, we use stochastic gradient descent with balanced mini-batches of size 512. As an optimizer, we use the ADAM~\cite{kingma2014adam} with default settings and initial step length~$\alpha = 0.01$, which we discount every five epochs by the factor of~$0.8$ using an exponential decay scheme. We also use a fixed number of epochs to 100, and repeat each experiment ten times with different random seeds.

For comparison of all formulations, we use two different approaches. The first one compares concrete performance metrics on each dataset separately. Since we have six hyperparameters for each formulation, we always select the best result for each formulation on the validation set based on the criterion for which the specific formulation is optimized. Then for each formulation, we select the median of the best results from ten independent runs. The second approach uses critical difference diagrams introduced in Section~\ref{sec: cd evaluation}. One of the basic assumptions of the critical difference diagrams to work appropriately is a large number of used datasets. Since we performed all experiments for each formulation and each dataset ten times with different random seeds for train/valid/test split, we decided to consider each of these runs as a separate dataset. It is important to say that we use this setting only for the critical difference diagrams. From Table~\ref{tab: primal linear medians} and Figure~\ref{fig: primal linear CD}, we make several observations:
\begin{itemize}
  \item \TopPushK(5) and \TopPushK(10) provides a slight (not statistically significant) improvement over \TopPush in most of the experiments, as shown in Figure~\ref{fig: primal linear CD}, where a green line connects all three formulations for almost all metrics. The only exception is the $\tpratfpr=0.01$ metric, for which \TopPush is significantly worse than other formulations.
  \item \BaseLine formulation works the best for $\auroc$ metric and is not suitable for any other metric; see Figure~\ref{fig: primal linear CD} or Table~\ref{tab: primal linear medians}. \BaseLine provides consistently the best results for $\auroc$ for all datasets. On the other hand, \BaseLine works poorly for metrics that operate at the absolute top, such as~$\tpratk = 1,$~$\tpratk = 5,$ or $\tpratk = 10.$
  \item \PatMatNP formulations provide very good results for all metrics. Moreover, \PatMatNP(0.01) is the best formulation for $\tpratfpr = 0.01,$ and \PatMatNP(0.05) for $\tpratfpr = 0.05.$ It means that both methods are the best for the criterion for which they are optimized. The same behavior can also be seen in Table~\ref{tab: primal linear medians}, where \PatMatNP(0.05) is the best formulation for $\tpratfpr = 0.01$ almost for all datasets.
  \item \tauFPL(0.05) works very well for both $\tpratfpr = 0.01$ and $\tpratfpr = 0.05.$. The formulation achieves almost as good results for both metrics as \PatMatNP formulations.
  \item All formulations provide very poor results for CIFAR and SVHN2 datasets, as shown in Table~\ref{tab: primal linear medians}. The use of a simple linear model causes this. However, the obtained results are still relevant since we compare the relative performance of the formulations to each other and not the absolute performance.
\end{itemize}

\begin{figure}[!p]
  \centering
  \documentclass{standalone}
\usepackage[ddmmyyyy]{datetime}

% ------------------------------------------------------------------------------
% Packages
% ------------------------------------------------------------------------------
% Page setting
\usepackage[explicit]{titlesec}
\usepackage{sectsty}
\usepackage{fancyhdr}

% Text options
\usepackage{lmodern}
\usepackage[T1]{fontenc}
\usepackage[utf8]{inputenc}
\usepackage{xspace}

\usepackage{amsfonts}
\usepackage{dsfont}
\usepackage{pifont}

\usepackage[color=myred!50]{todonotes}

% Graphics and colors
\usepackage{graphicx}
\usepackage{import}
\usepackage{graphics}
\usepackage{xcolor}

\definecolor{myred}{RGB}{150,0,0}  
\definecolor{mygreen}{RGB}{0,150,0}
\definecolor{myblue}{RGB}{0, 101, 189}
\definecolor{myyellow}{RGB}{220, 206, 0}
\definecolor{myorange}{RGB}{255, 153, 51}
\definecolor{mycyan}{RGB}{51, 204, 204}
\definecolor{mypurple}{RGB}{204, 0, 153}

\newcommand{\doccol}{\color{myblue}}

% Hyperrefs
\usepackage{hyperref}
\hypersetup{
  pdfusetitle,
  unicode = true,
  bookmarks = true,
  bookmarksnumbered = false,
  bookmarksopen = true,
  breaklinks = false,
  pdfborderstyle = {},
  backref = false,
  colorlinks = true,
  linkcolor = myblue,
  urlcolor = myred,
  citecolor = mygreen,
}

% enumerate and itemize
\usepackage{enumitem}

% Appendix
\usepackage[title, titletoc]{appendix}

% Captions
\usepackage{caption}
\usepackage{subcaption}

\captionsetup[figure]{position = bottom}
\captionsetup[table]{position = bottom}

% Figures

% Tables
\usepackage{booktabs}
\usepackage{nicematrix}

\renewcommand{\arraystretch}{1.5}

% Algorithms
\usepackage{algorithm}
\usepackage{algorithmicx}
\usepackage{algpseudocode}

% Math
\usepackage{amsmath}
\usepackage{amsthm}
\usepackage{amssymb}
\usepackage{mathtools}
\usepackage{nicefrac}
\usepackage{bm}
\usepackage{thmtools}
\usepackage{thm-restate}
\usepackage{optidef}

% Theorems
\usepackage[framemethod=TikZ]{mdframed}
\usepackage{xifthen}

% Tikz and pfgplots
\usepackage{tikz}
\usepackage{pgfplots}
\usepackage{pgfplotstable}

\usetikzlibrary{shapes}
\usetikzlibrary{arrows}
\usetikzlibrary{automata}
\usetikzlibrary{positioning}
\usetikzlibrary{calc}
\usetikzlibrary{intersections}

\pgfplotsset{compat=newest}
\usepgfplotslibrary{groupplots}
\usepgfplotslibrary{fillbetween}

% ------------------------------------------------------------------------------
% Math declarations
% ------------------------------------------------------------------------------
\newcommand{\Brac}[2][r]{%
  \ifx r#1 \left(       #2 \right)       \else
  \ifx c#1 \left\{      #2 \right\}      \else
  \ifx s#1 \left[       #2 \right]       \else
  \ifx v#1 \left\vert   #2 \right\vert   \else
  \ifx a#1 \left\langle #2 \right\rangle \else
  \ifx t#1 \left\lceil  #2 \right\rceil  \else
  \ifx b#1 \left\lfloor #2 \right\rfloor \else
  \ifx n#1 \left\|      #2 \right\|      \else
  \mathrm{Illegal~option}%
  \fi\fi\fi\fi\fi\fi\fi\fi
}

\newcommand{\clip}[4][s]{
  \ifx s#1 \mathrm{clip}_{\Brac[s]{#2,\; #3}}\Brac{#4} \else
  \ifx u#1 \mathrm{clip}_{\left[#2,\; #3\right)}\Brac{#4} \else
  \ifx l#1 \mathrm{clip}_{\left(#2,\; #3\right]}\Brac{#4} \else
  \mathrm{Illegal~option}%
  \fi\fi\fi
}

\newcommand{\yesmark}{\textcolor{mygreen}{\ding{51}}}%
\newcommand{\nomark}{\textcolor{myred}{\ding{55}}}
\newcommand{\good}[1]{\textcolor{mygreen}{#1}}
\newcommand{\bad}[1]{\textcolor{myred}{#1}}

\newcommand{\R}{\mathbb{R}}
\newcommand{\N}{\mathbb{N}}
\newcommand{\X}{\mathbb{X}}
\newcommand{\Xc}{\mathcal{X}}
\newcommand{\D}{\mathcal{D}}

\newcommand{\I}{\mathcal{I}}
\newcommand{\Itil}{\tilde{\mathcal{I}}}
\newcommand{\Ineg}{\I_{-}}
\newcommand{\Ipos}{\I_{+}}

\newcommand{\Imb}{\I_{\text{mb}}}
\newcommand{\Imbneg}{\I_{\text{mb},-}}
\newcommand{\Imbpos}{\I_{\text{mb},+}}

\newcommand{\nall}{n}
\newcommand{\nneg}{n_{-}}
\newcommand{\npos}{n_{+}}
\newcommand{\ntil}{\tilde{n}}

\newcommand{\nmb}{n_{\text{mb}}}
\newcommand{\nmbneg}{n_{\text{mb},-}}
\newcommand{\nmbpos}{n_{\text{mb},+}}

\newcommand{\K}{\mathbb{K}}
\newcommand{\Kall}{\K^{\pm}}
\newcommand{\Kneg}{\K^{-}}

\newcommand{\updateaa}{\Delta_{\alpha,\alpha}}
\newcommand{\updateab}{\Delta_{\alpha,\beta}}
\newcommand{\updatebb}{\Delta_{\beta,\beta}}

\newcommand{\alphak}{\alpha_{\hat{k}}}
\newcommand{\alphal}{\alpha_{\hat{l}}}
\newcommand{\betak}{\beta_{\hat{k}}}
\newcommand{\betal}{\beta_{\hat{l}}}

\newcommand{\norm}[1]{\Brac[n]{#1}}
\newcommand{\abs}[1]{|#1|}
\newcommand{\inner}[2]{\Brac[a]{#1, \; #2}}
\newcommand{\dd}[1]{\mathop{}\!\mathrm{d}#1}
\newcommand{\minimize}[1]{\ifthenelse{\isempty{#1}}{\operatorname*{minimize}\quad}{\operatorname*{minimize}_{#1}\quad}}
\newcommand{\maximize}[1]{\ifthenelse{\isempty{#1}}{\operatorname*{maximize}\quad}{\operatorname*{maximize}_{#1}\quad}}
\newcommand{\st}{\operatorname{subject\ to}}
\newcommand{\argmin}{\operatorname*{argmin}}
\newcommand{\eps}{{\varepsilon}}

\newcommand{\Imin}{I_{\rm mb}}
\newcommand{\Iminh}{I_{\rm mb}^{\rm enh}}
\newcommand{\nmin}{n_{\rm mb}}
\newcommand{\II}{\mathds{1}}
\newcommand{\Iverson}[1]{\mathds{1}_{\Brac[s]{#1}}}

\newcommand{\EE}{\mathbb{E}}
\newcommand{\PP}{\mathbb{P}}
\newcommand{\bias}{\operatorname{bias}}

\newcommand{\Matrix}[1]{\begin{pmatrix} #1 \end{pmatrix}}
\newcommand{\Set}[2]{\Brac[c]{#1 \; \middle\vert \; #2}}
\newcommand{\domain}{\operatorname*{dom}}

\newcommand{\repeatloop}{\texttt{repeat}\xspace}
\newcommand{\forloop}{\texttt{for}\xspace}

\newcommand{\vecab}{\Matrix{\bm{\alpha} \\ \bm{\beta}}}

% models
\newcommand{\AccatTop}{\emph{Accuracy at the Top}\xspace}
\newcommand{\TopPush}{\emph{TopPush}\xspace}
\newcommand{\TopPushK}{\emph{TopPushK}\xspace}
\newcommand{\tauFPL}{{\emph{$\tau$-FPL}}\xspace}
\newcommand{\TopMeanK}{\emph{TopMeanK}\xspace}
\newcommand{\PatMat}{\emph{Pat}\&\emph{Mat}\xspace}
\newcommand{\PatMatNP}{{\emph{Pat}\&\emph{Mat-NP}}\xspace}
\newcommand{\Grill}{\emph{Grill}\xspace}
\newcommand{\GrillNP}{\emph{Grill-NP}\xspace}
\newcommand{\DeepTopPush}{\emph{DeepTopPush}\xspace}
\newcommand{\TFCO}{\emph{TFCO}\xspace}
\newcommand{\APPerf}{\emph{Ap-Perf}\xspace}
\newcommand{\BaseLine}{\emph{BaseLine}\xspace}

% counts and rates
\DeclareMathOperator{\tp}{tp}
\DeclareMathOperator{\tn}{tn}
\DeclareMathOperator{\fp}{fp}
\DeclareMathOperator{\fn}{fn}
\DeclareMathOperator{\tpr}{tpr}
\DeclareMathOperator{\tnr}{tnr}
\DeclareMathOperator{\fpr}{fpr}
\DeclareMathOperator{\fnr}{fnr}

\DeclareMathOperator{\tps}{\overline{tp}}
\DeclareMathOperator{\tns}{\overline{tn}}
\DeclareMathOperator{\fps}{\overline{fp}}
\DeclareMathOperator{\fns}{\overline{fn}}

\DeclareMathOperator{\accuracy}{acc}
\DeclareMathOperator{\baccuracy}{bacc}
\DeclareMathOperator{\precision}{precision}
\DeclareMathOperator{\recall}{recall}
\DeclareMathOperator{\pratrec}{Precision@Recall}
\DeclareMathOperator{\postop}{pos@top}

% ------------------------------------------------------------------------------
% Document
% ------------------------------------------------------------------------------
\tikzstyle{line_node} = [line width=1pt, rounded corners, color=black, ->]
\tikzstyle{line_cv} = [line width=3pt, color=mygreen, line cap=round]

\begin{document}
\begin{tikzpicture}
  \node at (5.5,2.4) {True-positive rate at False-positive rate $0.05$}; 
  \draw (1,0) -- (10,0); 
  \foreach \x in {1,...,10} \draw (\x,0.1) -- (\x,-0.1) node[anchor=north]{$\x$}; 
  \draw[line_node] (2.43,0) -- (2.43,0.4) -- (0.9, 0.4) node[anchor=east] {\PatMatNP(0.05)}; 
  \draw[line_node] (3.72,0) -- (3.72,0.8) -- (0.9, 0.8) node[anchor=east] {\tauFPL(0.05)}; 
  \draw[line_node] (3.77,0) -- (3.77,1.2) -- (0.9, 1.2) node[anchor=east] {\PatMatNP(0.01)}; 
  \draw[line_node] (3.86,0) -- (3.86,1.6) -- (0.9, 1.6) node[anchor=east] {\BaseLine}; 
  \draw[line_node] (4.32,0) -- (4.32,2.0) -- (0.9, 2.0) node[anchor=east] {\TopPushK(10)}; 
  \draw[line_node] (5.08,0) -- (5.08,2.0) -- (10.1, 2.0) node[anchor=west] {\TopPushK(5)}; 
  \draw[line_node] (6.13,0) -- (6.13,1.6) -- (10.1, 1.6) node[anchor=west] {\tauFPL(0.01)}; 
  \draw[line_node] (6.79,0) -- (6.79,1.2) -- (10.1, 1.2) node[anchor=west] {\TopPush}; 
  \draw[line_node] (9.37,0) -- (9.37,0.8) -- (10.1, 0.8) node[anchor=west] {\GrillNP(0.05)}; 
  \draw[line_node] (9.53,0) -- (9.53,0.4) -- (10.1, 0.4) node[anchor=west] {\GrillNP(0.01)}; 
  \draw[line_cv] (2.43,0.2) -- (3.86, 0.2); 
  \draw[line_cv] (3.72,0.4) -- (5.08, 0.4); 
  \draw[line_cv] (5.08,0.2) -- (6.13, 0.2); 
  \draw[line_cv] (6.13,0.2) -- (6.79, 0.2); 
  \draw[line_cv] (9.37,0.2) -- (9.53, 0.2); 

  \node at (5.5,5.9) {True-positive rate at False-positive rate $0.01$}; 
  \draw (1,3.5) -- (10,3.5); 
  \foreach \x in {1,...,10} \draw (\x,3.6) -- (\x,3.4) node[anchor=north]{$\x$}; 
  \draw[line_node] (2.89,3.5) -- (2.89,3.9) -- (0.9, 3.9) node[anchor=east] {\PatMatNP(0.05)}; 
  \draw[line_node] (3.09,3.5) -- (3.09,4.3) -- (0.9, 4.3) node[anchor=east] {\PatMatNP(0.01)}; 
  \draw[line_node] (3.68,3.5) -- (3.68,4.7) -- (0.9, 4.7) node[anchor=east] {\tauFPL(0.05)}; 
  \draw[line_node] (4.13,3.5) -- (4.13,5.1) -- (0.9, 5.1) node[anchor=east] {\TopPushK(10)}; 
  \draw[line_node] (4.67,3.5) -- (4.67,5.5) -- (0.9, 5.5) node[anchor=east] {\TopPushK(5)}; 
  \draw[line_node] (5.41,3.5) -- (5.41,5.5) -- (10.1, 5.5) node[anchor=west] {\BaseLine}; 
  \draw[line_node] (5.54,3.5) -- (5.54,5.1) -- (10.1, 5.1) node[anchor=west] {\tauFPL(0.01)}; 
  \draw[line_node] (6.79,3.5) -- (6.79,4.7) -- (10.1, 4.7) node[anchor=west] {\TopPush}; 
  \draw[line_node] (9.31,3.5) -- (9.31,4.3) -- (10.1, 4.3) node[anchor=west] {\GrillNP(0.05)}; 
  \draw[line_node] (9.5,3.5) -- (9.5,3.9) -- (10.1, 3.9) node[anchor=west] {\GrillNP(0.01)}; 
  \draw[line_cv] (2.89,3.7) -- (4.13, 3.7); 
  \draw[line_cv] (3.68,3.9) -- (4.67, 3.9); 
  \draw[line_cv] (4.13,4.1) -- (5.54, 4.1); 
  \draw[line_cv] (5.41,3.7) -- (6.79, 3.7); 
  \draw[line_cv] (9.31,3.7) -- (9.5, 3.7); 

  \node at (5.5,9.4) {True-positive rate at $K = 10$}; 
  \draw (1,7.0) -- (10,7.0); 
  \foreach \x in {1,...,10} \draw (\x,7.1) -- (\x,6.9) node[anchor=north]{$\x$}; 
  \draw[line_node] (3.19,7.0) -- (3.19,7.4) -- (0.9, 7.4) node[anchor=east] {\PatMatNP(0.01)}; 
  \draw[line_node] (4.17,7.0) -- (4.17,7.8) -- (0.9, 7.8) node[anchor=east] {\TopPushK(5)}; 
  \draw[line_node] (4.28,7.0) -- (4.28,8.2) -- (0.9, 8.2) node[anchor=east] {\tauFPL(0.05)}; 
  \draw[line_node] (4.56,7.0) -- (4.56,8.6) -- (0.9, 8.6) node[anchor=east] {\TopPushK(10)}; 
  \draw[line_node] (4.62,7.0) -- (4.62,9.0) -- (0.9, 9.0) node[anchor=east] {\tauFPL(0.01)}; 
  \draw[line_node] (5.12,7.0) -- (5.12,9.0) -- (10.1, 9.0) node[anchor=west] {\PatMatNP(0.05)}; 
  \draw[line_node] (5.13,7.0) -- (5.13,8.6) -- (10.1, 8.6) node[anchor=west] {\TopPush}; 
  \draw[line_node] (6.94,7.0) -- (6.94,8.2) -- (10.1, 8.2) node[anchor=west] {\BaseLine}; 
  \draw[line_node] (8.36,7.0) -- (8.36,7.8) -- (10.1, 7.8) node[anchor=west] {\GrillNP(0.05)}; 
  \draw[line_node] (8.63,7.0) -- (8.63,7.4) -- (10.1, 7.4) node[anchor=west] {\GrillNP(0.01)}; 
  \draw[line_cv] (3.19,7.2) -- (4.62, 7.2); 
  \draw[line_cv] (4.17,7.4) -- (5.13, 7.4); 
  \draw[line_cv] (6.94,7.2) -- (8.36, 7.2); 
  \draw[line_cv] (8.36,7.2) -- (8.63, 7.2); 

  \node at (5.5,12.9) {True-positive rate at $K = 5$}; 
  \draw (1,10.5) -- (10,10.5); 
  \foreach \x in {1,...,10} \draw (\x,10.6) -- (\x,10.4) node[anchor=north]{$\x$}; 
  \draw[line_node] (3.59,10.5) -- (3.59,10.9) -- (0.9, 10.9) node[anchor=east] {\PatMatNP(0.01)}; 
  \draw[line_node] (4.18,10.5) -- (4.18,11.3) -- (0.9, 11.3) node[anchor=east] {\TopPushK(5)}; 
  \draw[line_node] (4.37,10.5) -- (4.37,11.7) -- (0.9, 11.7) node[anchor=east] {\tauFPL(0.05)}; 
  \draw[line_node] (4.78,10.5) -- (4.78,12.1) -- (0.9, 12.1) node[anchor=east] {\TopPushK(10)}; 
  \draw[line_node] (4.79,10.5) -- (4.79,12.5) -- (0.9, 12.5) node[anchor=east] {\TopPush}; 
  \draw[line_node] (4.82,10.5) -- (4.82,12.5) -- (10.1, 12.5) node[anchor=west] {\tauFPL(0.01)}; 
  \draw[line_node] (5.11,10.5) -- (5.11,12.1) -- (10.1, 12.1) node[anchor=west] {\PatMatNP(0.05)}; 
  \draw[line_node] (7.0,10.5) -- (7.0,11.7) -- (10.1, 11.7) node[anchor=west] {\BaseLine}; 
  \draw[line_node] (8.09,10.5) -- (8.09,11.3) -- (10.1, 11.3) node[anchor=west] {\GrillNP(0.05)}; 
  \draw[line_node] (8.28,10.5) -- (8.28,10.9) -- (10.1, 10.9) node[anchor=west] {\GrillNP(0.01)}; 
  \draw[line_cv] (3.59,10.7) -- (5.11, 10.7); 
  \draw[line_cv] (7.0,10.7) -- (8.28, 10.7); 

  \node at (5.5,16.4) {True-positive rate at $K = 1$}; 
  \draw (1,14.0) -- (10,14.0); 
  \foreach \x in {1,...,10} \draw (\x,14.1) -- (\x,13.9) node[anchor=north]{$\x$}; 
  \draw[line_node] (4.44,14.0) -- (4.44,14.4) -- (0.9, 14.4) node[anchor=east] {\PatMatNP(0.01)}; 
  \draw[line_node] (4.69,14.0) -- (4.69,14.8) -- (0.9, 14.8) node[anchor=east] {\tauFPL(0.05)}; 
  \draw[line_node] (4.98,14.0) -- (4.98,15.2) -- (0.9, 15.2) node[anchor=east] {\TopPushK(10)}; 
  \draw[line_node] (5.0,14.0) -- (5.0,15.6) -- (0.9, 15.6) node[anchor=east] {\PatMatNP(0.05)}; 
  \draw[line_node] (5.04,14.0) -- (5.04,16.0) -- (0.9, 16.0) node[anchor=east] {\tauFPL(0.01)}; 
  \draw[line_node] (5.18,14.0) -- (5.18,16.0) -- (10.1, 16.0) node[anchor=west] {\TopPush}; 
  \draw[line_node] (5.19,14.0) -- (5.19,15.6) -- (10.1, 15.6) node[anchor=west] {\TopPushK(5)}; 
  \draw[line_node] (6.38,14.0) -- (6.38,15.2) -- (10.1, 15.2) node[anchor=west] {\BaseLine}; 
  \draw[line_node] (6.98,14.0) -- (6.98,14.8) -- (10.1, 14.8) node[anchor=west] {\GrillNP(0.05)}; 
  \draw[line_node] (7.14,14.0) -- (7.14,14.4) -- (10.1, 14.4) node[anchor=west] {\GrillNP(0.01)}; 
  \draw[line_cv] (4.44,14.2) -- (5.19, 14.2); 
  \draw[line_cv] (4.98,14.4) -- (6.38, 14.4); 
  \draw[line_cv] (6.38,14.2) -- (7.14, 14.2); 
\end{tikzpicture}
\end{document}

  \caption{\textbf{Primal formulation with linear model:} Critical difference diagrams (level of importance 0.05) of the Nemenyi post hoc test for the Friedman test. Each diagram shows the mean rank of each method, with rank one being the best. The green horizontal lines group methods with mean ranks that are not significantly different. The critical difference diagrams were computed for mean rank averages over all datasets.}
  \label{fig: primal linear CD}
\end{figure}


\begin{table}[!p]
  \centering
  \underline{$\tpratk =10$}
  \vspace{0.25cm}\\
  \resizebox{\columnwidth}{!}{% 
    \begin{NiceTabular}{lccccccc}
      \CodeBefore
        \rowcolor{\headercol}{1}
        \rowcolors{3}{\rowcol}{}[restart]
      \Body
      \toprule
      \textbf{Formulation}
        & \textbf{MNIST}
        & \textbf{FashionMNIST}
        & \textbf{CIFAR10}
        & \textbf{CIFAR20}
        & \textbf{CIFAR100}
        & \textbf{SVHN2}
        & \textbf{SVHN2Extra}\\
      \midrule
      \BaseLine
        & 68.54
        & 85.65
        & \worst{1.25}
        & \best{1.40}
        & \worst{2.00}
        & 0.04
        & \worst{0.02}\\
      \TopPush
        & 89.38
        & \best{93.70}
        & 2.25
        & \worst{0.40}
        & 4.00
        & \worst{0.02}
        & \worst{0.02}\\
      \TopPushK(5)
        & 89.60
        & 93.30
        & 3.30
        & 1.10
        & 3.00
        & 0.04
        & 0.06\\
      \TopPushK(10)
        & \best{89.64}
        & 92.75
        & 3.90
        & 0.70
        & 4.50
        & 0.04
        & \best{0.09}\\
      \tauFPL(0.01)
        & 83.35
        & 92.40
        & 2.90
        & 0.80
        & 2.50
        & 0.03
        & \worst{0.02}\\
      \tauFPL(0.05)
        & \worst{40.26}
        & \worst{79.65}
        & 3.60
        & 1.00
        & \best{5.50}
        & 0.04
        & \best{0.09}\\
      \PatMatNP(0.01)
        & 87.88
        & 92.75
        & \best{5.00}
        & 1.10
        & 4.00
        & 0.12
        & \best{0.09}\\
      \PatMatNP(0.05)
        & 51.72
        & 81.60
        & 3.80
        & 1.20
        & 3.00
        & \best{0.14}
        & 0.06\\
      \bottomrule
    \end{NiceTabular}
  }
  \vspace{0.25cm}\\
  \underline{$\tpratfpr = 0.05$}
  \vspace{0.25cm}\\
  \resizebox{\columnwidth}{!}{% 
    \begin{NiceTabular}{lccccccc}
      \CodeBefore
        \rowcolor{\headercol}{1}
        \rowcolors{3}{\rowcol}{}[restart]
      \Body
      \toprule
      \textbf{Formulation}
        & \textbf{MNIST}
        & \textbf{FashionMNIST}
        & \textbf{CIFAR10}
        & \textbf{CIFAR20}
        & \textbf{CIFAR100}
        & \textbf{SVHN2}
        & \textbf{SVHN2Extra}\\
      \midrule
      \BaseLine
        & \best{99.65}
        & 99.30
        & 43.40
        & 32.90
        & 54.50
        & 5.53
        & \worst{5.91}\\
      \TopPush
        & \worst{99.12}
        & \worst{98.30}
        & \worst{31.10}
        & \worst{16.30}
        & 43.50
        & \worst{5.22}
        & 6.40\\
      \TopPushK(5)
        & 99.21
        & \worst{98.30}
        & 35.50
        & 20.20
        & \worst{42.50}
        & 6.78
        & 7.42\\
      \TopPushK(10)
        & 99.30
        & \worst{98.30}
        & 37.90
        & 20.80
        & 46.50
        & 6.15
        & 8.07\\
      \tauFPL(0.01)
        & 99.47
        & 98.75
        & 35.85
        & 24.00
        & 44.00
        & 6.90
        & 7.76\\
      \tauFPL(0.05)
        & 99.56
        & 99.20
        & 39.50
        & 25.70
        & 50.50
        & 8.16
        & 9.69\\
      \PatMatNP(0.01)
        & 99.52
        & 98.85
        & 45.35
        & 33.70
        & 56.00
        & 9.28
        & \best{12.47}\\
      \PatMatNP(0.05)
        & \best{99.65}
        & \best{99.40}
        & \best{46.80}
        & \best{34.80}
        & \best{58.50}
        & \best{9.34}
        & 12.25\\
      \bottomrule
    \end{NiceTabular}
  }
  \vspace{0.25cm}\\
  \underline{$\auroc$}
  \vspace{0.25cm}\\
  \centering
  \resizebox{\columnwidth}{!}{% 
    \begin{NiceTabular}{lccccccc}
      \CodeBefore
        \rowcolor{\headercol}{1}
        \rowcolors{3}{\rowcol}{}[restart]
      \Body
      \toprule
      \textbf{Formulation}
        & \textbf{MNIST}
        & \textbf{FashionMNIST}
        & \textbf{CIFAR10}
        & \textbf{CIFAR20}
        & \textbf{CIFAR100}
        & \textbf{SVHN2}
        & \textbf{SVHN2Extra}\\
      \midrule
      \BaseLine
        & \best{99.86}
        & \best{99.83}
        & \best{84.00}
        & \best{76.26}
        & \best{88.48}
        & \best{57.82}
        & \best{56.10}\\
      \TopPush
        & \worst{99.78}
        & \worst{99.42}
        & \worst{73.84}
        & 65.76
        & 82.10
        & 51.08
        & 51.30 \\
      \TopPushK(5)
        & 99.80
        & \worst{99.42}
        & 76.67
        & \worst{65.70}
        & \worst{81.52}
        & \worst{50.98}
        & 50.69\\
      \TopPushK(10)
        & 99.82
        & 99.48
        & 77.74
        & 66.87
        & 81.91
        & 51.90
        & \worst{50.58}\\
      \tauFPL(0.01)
        & 99.84
        & 99.72
        & 77.34
        & 69.24
        & 82.96
        & 51.04
        & 50.62\\
      \tauFPL(0.05)
        & 99.81
        & 99.80
        & 79.41
        & 70.86
        & 84.56
        & 51.78
        & 50.76\\
      \PatMatNP(0.01)
        & 99.85
        & 99.68
        & 82.34
        & 74.56
        & 86.13
        & 56.38
        & 51.93\\
      \PatMatNP(0.05)
        & 99.84
        & 99.81
        & 83.35
        & 75.44
        & 87.22
        & 56.40
        & 52.50\\
      \bottomrule
    \end{NiceTabular}
  }
  \caption{\textbf{Primal formulation with linear model:} Each table corresponds to one performance metric, and all presented results are medians of ten independent runs for each pair of datasets and formulation. The best result for each dataset is highlighted in green, while the worst result is highlighted in red.}
  \label{tab: primal linear medians}
\end{table}

\pagebreak

\subsection{Dual Formulation: Linear Model}\label{sec: results dual}

In this section, we present results for a dual form of formulations from Table~\ref{tab: formulations experiments summary} with a Gaussian kernel model. For training, we use the coordinate descent algorithm introduced in Section~\ref{sec: coordinate descent}. We set a number of steps to 20 epochs. For all experiments, we use precomputed kernel matrix with a Gaussian kernel function defined as 
\begin{equation*}
  k(\bm{x}_i, \bm{x}_j) = \exp\Brac[c]{- \frac{\norm{\bm{x}_i - \bm{x}_j}^2}{d}},
\end{equation*}
where~$d$ is the dimension of the primal problem. We used this since it is the default setting for radial basis kernel type in LIBSVM~\cite{chang2011libsvm}.

In Figure~\ref{fig: dual convergence}, we investigate the convergence of the coordinate descend algorithm introduced in Section~\ref{sec: coordinate descent} for three formulations, namely \TopPush, \TopPushK, and \PatMatNP. In each column, we show the primal and dual objective function convergence for one formulation. To solve the primal problem, we used full gradient descent. For this experiment, we use the Ionosphere dataset~\cite{sigillito1989classification}. This dataset is used only for this experiment. We can see that \TopPush and \TopPushK converge to the same objective for primal and dual problems. It means that both problems were solved to optimality. However, there is a little gap between the optimal primal and dual form solution for \PatMatNP. In other words, \PatMatNP may suffer from convergence issues when solving the proposed coordinate descent algorithm.  

For comparison of all formulations, we use the same two approaches as in Section~\ref{sec: results primal linear}. From Table~\ref{tab: dual gauss medians} and Figure~\ref{fig: dual gauss CD}, we make several observations:
\begin{itemize}
  \item We observe that some formulations have problems with convergence and, in some cases, even diverge for some datasets. The improper choice of the kernel function can cause it. As a result, CD diagrams may provide unreliable results. If the formulation diverges in a few experiments, it immediately obtains very high ranks for these experiments that skew the final diagram. It is especially evident for \PatMatNP and \SVM formulations.
  \item Figure~\ref{fig: dual gauss CD} shows that \PatMatNP formulations provide the worst results for all metrics. It can be caused by the bad convergence of the coordinate descent algorithm, as shown in Figure~\ref{fig: dual convergence}. However, it is important to say that Figure~\ref{fig: dual gauss CD} shows only relative results. From Table~\ref{tab: dual gauss medians} is clear that even though \PatMatNP usually provides worse results than other formulations, the results are, in many cases, only slightly worse.
  \item Similarly to \PatMatNP, the \SVM formulation does not perform well for most metrics. However, as shown in Table~\ref{tab: dual gauss medians}, the results are usually only slightly worse than those of other formulations.
  \item \tauFPL formulations work very well for $\tpratfpr = 0.01,$ $\tpratfpr = 0.05$ and $\auroc$ metric.
  \item \TopPush, \TopPushK(5) and \TopPushK(10) provides very good results for $\tpratk = 1,$ $\tpratk = 5$ and $\tpratk = 10.$
\end{itemize}

\pagebreak

\begin{figure}
  \centering
  \includegraphics{images/convergence_dual.pdf}
  \caption{Convergence of the objectives for the primal (red line) and dual (blue dashed line) forms with linear kernel.}
  \label{fig: dual convergence}
\end{figure}

\subsection{Primal Formulation: Non-Linear Model}\label{sec: results primal nonlinear}

In this section, we present results for a primal form of formulations from Table~\ref{tab: formulations experiments summary} with a non-linear model. For training, we use the same setting as in Section~\ref{sec: results primal linear}. For MNIST and FashionMNIST datasets, we use a neural network consisting of two convolution layers, followed by a max pooling layer and one fully connected layer of the proper size for binary classification. The rest of the datasets use a similar architecture but with three convolutional layers instead of just two. We purposely do not use state-of-the-art architectures since they often lead to a perfect separation of used datasets. Our goal is to show that formulations from Table~\ref{tab: formulations experiments summary} can improve specific metrics such as $\tpratfpr = 0.05$ (when compared to \BaseLine) even with these subpar architectures.

For comparison of all formulations, we use the same two approaches as in Section~\ref{sec: results primal linear}. From Table~\ref{tab: primal nonlinear medians} and Figure~\ref{fig: primal nonlinear  CD}, we make several observations:
\begin{itemize}
  \item Most formulations perform well on the criteria for which they are optimized.
  \item Most formulations provide almost perfect separation on MNIST and FashionMNIST datasets.
  \item \DeepTopPush does not provide good results. In fact, the formulation is the worst in five out of six metrics in Figure~\ref{fig: primal nonlinear  CD}. However, it can be caused by using relatively large mini-batches concerning the size of the datasets. The true power of the \DeepTopPush is shown in Section~\ref{sec: steganalysis} and~\ref{sec: malware detection}.
  \item \BaseLine formulation performs consistently very well for all metrics. Nevertheless, the \BaseLine is the best for neither of the metrics.
  \item \TopPushK(10) fails many times. It is especially evident from Table~\ref{tab: primal nonlinear medians}. \TopPushK(10) achieves 100\% for $\tpratk = 10$ metric for most datasets, which seems as a very good result. However, if we take a look at the~$\auroc$, we can see that the formulation achieves 0\% for the same datasets. The reason for that is simple, \TopPushK(10) assigns the same score to all samples and therefore achieves 100\% true-positive rate but also 100\% false-positive rate.
  \item \PatMatNP formulations provide very good results for all metrics. Moreover, \PatMatNP(0.01) is the best formulation for $\tpratfpr = 0.01,$ and \PatMatNP(0.05) for $\tpratfpr = 0.05.$ It means that both methods are the best for the criterion for which they are optimized. This behavior can also be seen in Table~\ref{tab: primal linear medians}, where \PatMatNP(0.05) is the best formulation for $\tpratfpr = 0.01$ for all datasets.
  \item \tauFPL formulations do not work well.
\end{itemize}

\begin{figure}[!p]
  \centering
  \documentclass{standalone}
\usepackage[ddmmyyyy]{datetime}

% ------------------------------------------------------------------------------
% Packages
% ------------------------------------------------------------------------------
% Page setting
\usepackage[explicit]{titlesec}
\usepackage{sectsty}
\usepackage{fancyhdr}

% Text options
\usepackage{lmodern}
\usepackage[T1]{fontenc}
\usepackage[utf8]{inputenc}
\usepackage{xspace}

\usepackage{amsfonts}
\usepackage{dsfont}
\usepackage{pifont}

\usepackage[color=myred!50]{todonotes}

% Graphics and colors
\usepackage{graphicx}
\usepackage{import}
\usepackage{graphics}
\usepackage{xcolor}

\definecolor{myred}{RGB}{150,0,0}  
\definecolor{mygreen}{RGB}{0,150,0}
\definecolor{myblue}{RGB}{0, 101, 189}
\definecolor{myyellow}{RGB}{220, 206, 0}
\definecolor{myorange}{RGB}{255, 153, 51}
\definecolor{mycyan}{RGB}{51, 204, 204}
\definecolor{mypurple}{RGB}{204, 0, 153}

\newcommand{\doccol}{\color{myblue}}

% Hyperrefs
\usepackage{hyperref}
\hypersetup{
  pdfusetitle,
  unicode = true,
  bookmarks = true,
  bookmarksnumbered = false,
  bookmarksopen = true,
  breaklinks = false,
  pdfborderstyle = {},
  backref = false,
  colorlinks = true,
  linkcolor = myblue,
  urlcolor = myred,
  citecolor = mygreen,
}

% enumerate and itemize
\usepackage{enumitem}

% Appendix
\usepackage[title, titletoc]{appendix}

% Captions
\usepackage{caption}
\usepackage{subcaption}

\captionsetup[figure]{position = bottom}
\captionsetup[table]{position = bottom}

% Figures

% Tables
\usepackage{booktabs}
\usepackage{nicematrix}

\renewcommand{\arraystretch}{1.5}

% Algorithms
\usepackage{algorithm}
\usepackage{algorithmicx}
\usepackage{algpseudocode}

% Math
\usepackage{amsmath}
\usepackage{amsthm}
\usepackage{amssymb}
\usepackage{mathtools}
\usepackage{nicefrac}
\usepackage{bm}
\usepackage{thmtools}
\usepackage{thm-restate}
\usepackage{optidef}

% Theorems
\usepackage[framemethod=TikZ]{mdframed}
\usepackage{xifthen}

% Tikz and pfgplots
\usepackage{tikz}
\usepackage{pgfplots}
\usepackage{pgfplotstable}

\usetikzlibrary{shapes}
\usetikzlibrary{arrows}
\usetikzlibrary{automata}
\usetikzlibrary{positioning}
\usetikzlibrary{calc}
\usetikzlibrary{intersections}

\pgfplotsset{compat=newest}
\usepgfplotslibrary{groupplots}
\usepgfplotslibrary{fillbetween}

% ------------------------------------------------------------------------------
% Math declarations
% ------------------------------------------------------------------------------
\newcommand{\Brac}[2][r]{%
  \ifx r#1 \left(       #2 \right)       \else
  \ifx c#1 \left\{      #2 \right\}      \else
  \ifx s#1 \left[       #2 \right]       \else
  \ifx v#1 \left\vert   #2 \right\vert   \else
  \ifx a#1 \left\langle #2 \right\rangle \else
  \ifx t#1 \left\lceil  #2 \right\rceil  \else
  \ifx b#1 \left\lfloor #2 \right\rfloor \else
  \ifx n#1 \left\|      #2 \right\|      \else
  \mathrm{Illegal~option}%
  \fi\fi\fi\fi\fi\fi\fi\fi
}

\newcommand{\clip}[4][s]{
  \ifx s#1 \mathrm{clip}_{\Brac[s]{#2,\; #3}}\Brac{#4} \else
  \ifx u#1 \mathrm{clip}_{\left[#2,\; #3\right)}\Brac{#4} \else
  \ifx l#1 \mathrm{clip}_{\left(#2,\; #3\right]}\Brac{#4} \else
  \mathrm{Illegal~option}%
  \fi\fi\fi
}

\newcommand{\yesmark}{\textcolor{mygreen}{\ding{51}}}%
\newcommand{\nomark}{\textcolor{myred}{\ding{55}}}
\newcommand{\good}[1]{\textcolor{mygreen}{#1}}
\newcommand{\bad}[1]{\textcolor{myred}{#1}}

\newcommand{\R}{\mathbb{R}}
\newcommand{\N}{\mathbb{N}}
\newcommand{\X}{\mathbb{X}}
\newcommand{\Xc}{\mathcal{X}}
\newcommand{\D}{\mathcal{D}}

\newcommand{\I}{\mathcal{I}}
\newcommand{\Itil}{\tilde{\mathcal{I}}}
\newcommand{\Ineg}{\I_{-}}
\newcommand{\Ipos}{\I_{+}}

\newcommand{\Imb}{\I_{\text{mb}}}
\newcommand{\Imbneg}{\I_{\text{mb},-}}
\newcommand{\Imbpos}{\I_{\text{mb},+}}

\newcommand{\nall}{n}
\newcommand{\nneg}{n_{-}}
\newcommand{\npos}{n_{+}}
\newcommand{\ntil}{\tilde{n}}

\newcommand{\nmb}{n_{\text{mb}}}
\newcommand{\nmbneg}{n_{\text{mb},-}}
\newcommand{\nmbpos}{n_{\text{mb},+}}

\newcommand{\K}{\mathbb{K}}
\newcommand{\Kall}{\K^{\pm}}
\newcommand{\Kneg}{\K^{-}}

\newcommand{\updateaa}{\Delta_{\alpha,\alpha}}
\newcommand{\updateab}{\Delta_{\alpha,\beta}}
\newcommand{\updatebb}{\Delta_{\beta,\beta}}

\newcommand{\alphak}{\alpha_{\hat{k}}}
\newcommand{\alphal}{\alpha_{\hat{l}}}
\newcommand{\betak}{\beta_{\hat{k}}}
\newcommand{\betal}{\beta_{\hat{l}}}

\newcommand{\norm}[1]{\Brac[n]{#1}}
\newcommand{\abs}[1]{|#1|}
\newcommand{\inner}[2]{\Brac[a]{#1, \; #2}}
\newcommand{\dd}[1]{\mathop{}\!\mathrm{d}#1}
\newcommand{\minimize}[1]{\ifthenelse{\isempty{#1}}{\operatorname*{minimize}\quad}{\operatorname*{minimize}_{#1}\quad}}
\newcommand{\maximize}[1]{\ifthenelse{\isempty{#1}}{\operatorname*{maximize}\quad}{\operatorname*{maximize}_{#1}\quad}}
\newcommand{\st}{\operatorname{subject\ to}}
\newcommand{\argmin}{\operatorname*{argmin}}
\newcommand{\eps}{{\varepsilon}}

\newcommand{\Imin}{I_{\rm mb}}
\newcommand{\Iminh}{I_{\rm mb}^{\rm enh}}
\newcommand{\nmin}{n_{\rm mb}}
\newcommand{\II}{\mathds{1}}
\newcommand{\Iverson}[1]{\mathds{1}_{\Brac[s]{#1}}}

\newcommand{\EE}{\mathbb{E}}
\newcommand{\PP}{\mathbb{P}}
\newcommand{\bias}{\operatorname{bias}}

\newcommand{\Matrix}[1]{\begin{pmatrix} #1 \end{pmatrix}}
\newcommand{\Set}[2]{\Brac[c]{#1 \; \middle\vert \; #2}}
\newcommand{\domain}{\operatorname*{dom}}

\newcommand{\repeatloop}{\texttt{repeat}\xspace}
\newcommand{\forloop}{\texttt{for}\xspace}

\newcommand{\vecab}{\Matrix{\bm{\alpha} \\ \bm{\beta}}}

% models
\newcommand{\AccatTop}{\emph{Accuracy at the Top}\xspace}
\newcommand{\TopPush}{\emph{TopPush}\xspace}
\newcommand{\TopPushK}{\emph{TopPushK}\xspace}
\newcommand{\tauFPL}{{\emph{$\tau$-FPL}}\xspace}
\newcommand{\TopMeanK}{\emph{TopMeanK}\xspace}
\newcommand{\PatMat}{\emph{Pat}\&\emph{Mat}\xspace}
\newcommand{\PatMatNP}{{\emph{Pat}\&\emph{Mat-NP}}\xspace}
\newcommand{\Grill}{\emph{Grill}\xspace}
\newcommand{\GrillNP}{\emph{Grill-NP}\xspace}
\newcommand{\DeepTopPush}{\emph{DeepTopPush}\xspace}
\newcommand{\TFCO}{\emph{TFCO}\xspace}
\newcommand{\APPerf}{\emph{Ap-Perf}\xspace}
\newcommand{\BaseLine}{\emph{BaseLine}\xspace}

% counts and rates
\DeclareMathOperator{\tp}{tp}
\DeclareMathOperator{\tn}{tn}
\DeclareMathOperator{\fp}{fp}
\DeclareMathOperator{\fn}{fn}
\DeclareMathOperator{\tpr}{tpr}
\DeclareMathOperator{\tnr}{tnr}
\DeclareMathOperator{\fpr}{fpr}
\DeclareMathOperator{\fnr}{fnr}

\DeclareMathOperator{\tps}{\overline{tp}}
\DeclareMathOperator{\tns}{\overline{tn}}
\DeclareMathOperator{\fps}{\overline{fp}}
\DeclareMathOperator{\fns}{\overline{fn}}

\DeclareMathOperator{\accuracy}{acc}
\DeclareMathOperator{\baccuracy}{bacc}
\DeclareMathOperator{\precision}{precision}
\DeclareMathOperator{\recall}{recall}
\DeclareMathOperator{\pratrec}{Precision@Recall}
\DeclareMathOperator{\postop}{pos@top}

% ------------------------------------------------------------------------------
% Document
% ------------------------------------------------------------------------------
\tikzstyle{line_node} = [line width=1pt, rounded corners, color=black, ->]
\tikzstyle{line_cv} = [line width=3pt, color=mygreen, line cap=round]

\begin{document}
\begin{tikzpicture}
  \node at (4.5,2.0) {ROC AUC}; 
  \draw (1,0) -- (8,0); 
  \foreach \x in {1,...,8} \draw (\x,0.1) -- (\x,-0.1) node[anchor=north]{$\x$}; 
  \draw[line_node] (2.08,0) -- (2.08,0.4) -- (0.9, 0.4) node[anchor=east] {\tauFPL(0.05)}; 
  \draw[line_node] (3.56,0) -- (3.56,0.8) -- (0.9, 0.8) node[anchor=east] {\tauFPL(0.01)}; 
  \draw[line_node] (4.22,0) -- (4.22,1.2) -- (0.9, 1.2) node[anchor=east] {\TopPushK(10)}; 
  \draw[line_node] (4.38,0) -- (4.38,1.6) -- (0.9, 1.6) node[anchor=east] {\TopPushK(5)}; 
  \draw[line_node] (4.64,0) -- (4.64,1.6) -- (8.1, 1.6) node[anchor=west] {\TopPush}; 
  \draw[line_node] (5.08,0) -- (5.08,1.2) -- (8.1, 1.2) node[anchor=west] {\SVM}; 
  \draw[line_node] (5.99,0) -- (5.99,0.8) -- (8.1, 0.8) node[anchor=west] {\PatMatNP(0.01)}; 
  \draw[line_node] (6.04,0) -- (6.04,0.4) -- (8.1, 0.4) node[anchor=west] {\PatMatNP(0.05)}; 
  \draw[line_cv] (3.56,0.2) -- (4.64, 0.2); 
  \draw[line_cv] (4.22,0.4) -- (5.08, 0.4); 
  \draw[line_cv] (4.64,0.6) -- (5.99, 0.6); 
  \draw[line_cv] (5.08,0.2) -- (6.04, 0.2); 

  \node at (4.5,5.5) {True-positive rate at False-positive rate $0.05$}; 
  \draw (1,3.5) -- (8,3.5); 
  \foreach \x in {1,...,8} \draw (\x,3.6) -- (\x,3.4) node[anchor=north]{$\x$}; 
  \draw[line_node] (2.18,3.5) -- (2.18,3.9) -- (0.9, 3.9) node[anchor=east] {\tauFPL(0.05)}; 
  \draw[line_node] (4.31,3.5) -- (4.31,4.3) -- (0.9, 4.3) node[anchor=east] {\TopPushK(10)}; 
  \draw[line_node] (4.34,3.5) -- (4.34,4.7) -- (0.9, 4.7) node[anchor=east] {\tauFPL(0.01)}; 
  \draw[line_node] (4.47,3.5) -- (4.47,5.1) -- (0.9, 5.1) node[anchor=east] {\TopPushK(5)}; 
  \draw[line_node] (4.54,3.5) -- (4.54,5.1) -- (8.1, 5.1) node[anchor=west] {\TopPush}; 
  \draw[line_node] (4.98,3.5) -- (4.98,4.7) -- (8.1, 4.7) node[anchor=west] {\SVM}; 
  \draw[line_node] (5.52,3.5) -- (5.52,4.3) -- (8.1, 4.3) node[anchor=west] {\PatMatNP(0.01)}; 
  \draw[line_node] (5.65,3.5) -- (5.65,3.9) -- (8.1, 3.9) node[anchor=west] {\PatMatNP(0.05)}; 
  \draw[line_cv] (4.31,3.7) -- (5.65, 3.7); 

  \node at (4.5,9.0) {True-positive rate at False-positive rate $0.01$}; 
  \draw (1,7.0) -- (8,7.0); 
  \foreach \x in {1,...,8} \draw (\x,7.1) -- (\x,6.9) node[anchor=north]{$\x$}; 
  \draw[line_node] (2.78,7.0) -- (2.78,7.4) -- (0.9, 7.4) node[anchor=east] {\tauFPL(0.05)}; 
  \draw[line_node] (3.08,7.0) -- (3.08,7.8) -- (0.9, 7.8) node[anchor=east] {\tauFPL(0.01)}; 
  \draw[line_node] (4.36,7.0) -- (4.36,8.2) -- (0.9, 8.2) node[anchor=east] {\SVM}; 
  \draw[line_node] (4.44,7.0) -- (4.44,8.6) -- (0.9, 8.6) node[anchor=east] {\TopPush}; 
  \draw[line_node] (4.64,7.0) -- (4.64,8.6) -- (8.1, 8.6) node[anchor=west] {\TopPushK(5)}; 
  \draw[line_node] (4.71,7.0) -- (4.71,8.2) -- (8.1, 8.2) node[anchor=west] {\TopPushK(10)}; 
  \draw[line_node] (5.98,7.0) -- (5.98,7.8) -- (8.1, 7.8) node[anchor=west] {\PatMatNP(0.05)}; 
  \draw[line_node] (6.02,7.0) -- (6.02,7.4) -- (8.1, 7.4) node[anchor=west] {\PatMatNP(0.01)}; 
  \draw[line_cv] (2.78,7.2) -- (3.08, 7.2); 
  \draw[line_cv] (3.08,7.4) -- (4.36, 7.4); 
  \draw[line_cv] (4.36,7.2) -- (4.71, 7.2); 
  \draw[line_cv] (4.64,7.2) -- (5.98, 7.2); 
  \draw[line_cv] (4.71,7.2) -- (6.02, 7.2); 

  \node at (4.5,12.5) {True-positive rate at $K = 10$}; 
  \draw (1,10.5) -- (8,10.5); 
  \foreach \x in {1,...,8} \draw (\x,10.6) -- (\x,10.4) node[anchor=north]{$\x$}; 
  \draw[line_node] (2.93,10.5) -- (2.93,10.9) -- (0.9, 10.9) node[anchor=east] {\TopPush}; 
  \draw[line_node] (3.11,10.5) -- (3.11,11.3) -- (0.9, 11.3) node[anchor=east] {\tauFPL(0.01)}; 
  \draw[line_node] (3.36,10.5) -- (3.36,11.7) -- (0.9, 11.7) node[anchor=east] {\TopPushK(5)}; 
  \draw[line_node] (3.47,10.5) -- (3.47,12.1) -- (0.9, 12.1) node[anchor=east] {\TopPushK(10)}; 
  \draw[line_node] (4.11,10.5) -- (4.11,12.1) -- (8.1, 12.1) node[anchor=west] {\tauFPL(0.05)}; 
  \draw[line_node] (5.14,10.5) -- (5.14,11.7) -- (8.1, 11.7) node[anchor=west] {\SVM}; 
  \draw[line_node] (6.91,10.5) -- (6.91,11.3) -- (8.1, 11.3) node[anchor=west] {\PatMatNP(0.05)}; 
  \draw[line_node] (6.98,10.5) -- (6.98,10.9) -- (8.1, 10.9) node[anchor=west] {\PatMatNP(0.01)}; 
  \draw[line_cv] (2.93,10.7) -- (4.11, 10.7); 
  \draw[line_cv] (4.11,10.9) -- (5.14, 10.9); 
  \draw[line_cv] (6.91,10.7) -- (6.98, 10.7); 

  \node at (4.5,16.0) {True-positive rate at $K = 5$}; 
  \draw (1,14.0) -- (8,14.0); 
  \foreach \x in {1,...,8} \draw (\x,14.1) -- (\x,13.9) node[anchor=north]{$\x$}; 
  \draw[line_node] (2.68,14.0) -- (2.68,14.4) -- (0.9, 14.4) node[anchor=east] {\tauFPL(0.01)}; 
  \draw[line_node] (3.01,14.0) -- (3.01,14.8) -- (0.9, 14.8) node[anchor=east] {\TopPush}; 
  \draw[line_node] (3.23,14.0) -- (3.23,15.2) -- (0.9, 15.2) node[anchor=east] {\TopPushK(10)}; 
  \draw[line_node] (3.39,14.0) -- (3.39,15.6) -- (0.9, 15.6) node[anchor=east] {\TopPushK(5)}; 
  \draw[line_node] (4.68,14.0) -- (4.68,15.6) -- (8.1, 15.6) node[anchor=west] {\tauFPL(0.05)}; 
  \draw[line_node] (5.18,14.0) -- (5.18,15.2) -- (8.1, 15.2) node[anchor=west] {\SVM}; 
  \draw[line_node] (6.89,14.0) -- (6.89,14.8) -- (8.1, 14.8) node[anchor=west] {\PatMatNP(0.05)}; 
  \draw[line_node] (6.94,14.0) -- (6.94,14.4) -- (8.1, 14.4) node[anchor=west] {\PatMatNP(0.01)}; 
  \draw[line_cv] (2.68,14.2) -- (3.39, 14.2); 
  \draw[line_cv] (3.39,14.4) -- (4.68, 14.4); 
  \draw[line_cv] (4.68,14.2) -- (5.18, 14.2); 
  \draw[line_cv] (6.89,14.2) -- (6.94, 14.2); 

  \node at (4.5,19.5) {True-positive rate at $K = 1$}; 
  \draw (1,17.5) -- (8,17.5); 
  \foreach \x in {1,...,8} \draw (\x,17.61) -- (\x,17.39) node[anchor=north]{$\x$}; 
  \draw[line_node] (2.79,17.5) -- (2.79,17.9) -- (0.9, 17.9) node[anchor=east] {\TopPush}; 
  \draw[line_node] (3.0,17.5) -- (3.0,18.3) -- (0.9, 18.3) node[anchor=east] {\TopPushK(10)}; 
  \draw[line_node] (3.05,17.5) -- (3.05,18.7) -- (0.9, 18.7) node[anchor=east] {\TopPushK(5)}; 
  \draw[line_node] (3.6,17.5) -- (3.6,19.1) -- (0.9, 19.1) node[anchor=east] {\tauFPL(0.01)}; 
  \draw[line_node] (5.18,17.5) -- (5.18,19.1) -- (8.1, 19.1) node[anchor=west] {\SVM}; 
  \draw[line_node] (5.43,17.5) -- (5.43,18.7) -- (8.1, 18.7) node[anchor=west] {\tauFPL(0.05)}; 
  \draw[line_node] (6.43,17.5) -- (6.43,18.3) -- (8.1, 18.3) node[anchor=west] {\PatMatNP(0.01)}; 
  \draw[line_node] (6.52,17.5) -- (6.52,17.9) -- (8.1, 17.9) node[anchor=west] {\PatMatNP(0.05)}; 
  \draw[line_cv] (2.79,17.7) -- (3.6, 17.7); 
  \draw[line_cv] (5.18,17.7) -- (6.52, 17.7); 
\end{tikzpicture}
\end{document}

  \caption{\textbf{Dual formulations with gaussian kernel:} Critical difference diagrams (level of importance 0.05) of the Nemenyi post hoc test for the Friedman test. Each diagram shows the mean rank of each method, with rank one being the best. The green horizontal lines group methods with mean ranks that are not significantly different. The critical difference diagrams were computed for mean rank averages over all datasets.}
  \label{fig: dual gauss CD}
\end{figure}

\begin{table}[!p]
  \centering
  \underline{$\tpratk =10$}
  \vspace{0.25cm}\\
  \resizebox{\columnwidth}{!}{% 
    \begin{NiceTabular}{lcccccc}
      \CodeBefore
        \rowcolor{\headercol}{1}
        \rowcolors{3}{\rowcol}{}[restart]
      \Body
      \toprule
      \textbf{Formulation}
        & \textbf{MNIST}
        & \textbf{FashionMNIST}
        & \textbf{CIFAR10}
        & \textbf{CIFAR20}
        & \textbf{CIFAR100}
        & \textbf{SVHN2}\\
      \midrule
      \SVM
        & 97.89
        & \best{95.40}
        & 9.10
        & 4.90
        & \best{11.50}
        & 4.52 \\
      \TopPush
        & 97.62
        & 94.80
        & 10.45
        & \best{6.10}
        & 11.00
        & 5.23 \\
      \TopPushK(5)
        & 97.97
        & 94.90
        & 10.05
        & 6.00
        & 11.0
        & 5.07 \\
      \TopPushK(10)
        & 97.97
        & 94.90
        & 9.85
        & \best{6.10}
        & 11.00
        & 5.18 \\
      \tauFPL(0.01)
        & \best{98.02}
        & 95.05
        & \best{10.70}
        & 5.90
        & 10.5
        & \best{5.25} \\
      \tauFPL(0.05)
        & 92.56
        & \worst{92.20}
        & 10.15
        & 5.10
        & 10.0
        & 5.24 \\
      \PatMatNP(0.01)
        & 88.37
        & 92.50
        & \worst{7.45}
        & 1.40
        & \worst{5.00}
        & \worst{4.02} \\
      \PatMatNP(0.05)
        & \worst{52.60}
        & 92.50
        & \worst{7.45}
        & \worst{1.30}
        & \worst{5.00}
        & 4.05 \\
      \bottomrule
    \end{NiceTabular}
  }
  \vspace{0.25cm}\\
  \underline{$\tpratfpr = 0.05$}
  \vspace{0.25cm}\\
  \resizebox{\columnwidth}{!}{% 
    \begin{NiceTabular}{lcccccc}
      \CodeBefore
        \rowcolor{\headercol}{1}
        \rowcolors{3}{\rowcol}{}[restart]
      \Body
      \toprule
      \textbf{Formulation}
        & \textbf{MNIST}
        & \textbf{FashionMNIST}
        & \textbf{CIFAR10}
        & \textbf{CIFAR20}
        & \textbf{CIFAR100}
        & \textbf{SVHN2}\\
      \midrule
      \SVM
        & 99.74
        & 98.90
        & 60.00
        & \best{44.80}
        & 59.00
        & \worst{59.72} \\
      \TopPush
        & 99.74
        & 98.80
        & 57.10
        & \worst{37.70}
        & 59.50
        & 72.54 \\
      \TopPushK(5)
        & \best{99.82}
        & 98.90
        & 56.25
        & 38.80
        & \worst{57.50}
        & 71.40 \\
      \TopPushK(10)
        & \best{99.82}
        & 98.90
        & 56.90
        & 38.70
        & 58.00
        & 71.61 \\
      \tauFPL(0.01)
        & \best{99.82}
        & 98.90
        & 58.10
        & 39.10
        & 59.00
        & 73.52 \\
      \tauFPL(0.05)
        & 99.74
        & \best{99.10}
        & \best{60.80}
        & 44.40
        & 61.00
        & \best{74.26} \\
      \PatMatNP(0.01)
        & \worst{99.30}
        & \worst{98.10}
        & \worst{54.70}
        & 44.60
        & 62.50
        & 63.47 \\
      \PatMatNP(0.05)
        & 99.38
        & \worst{98.10}
        & \worst{54.70}
        & 44.50
        & \best{63.50}
        & 63.48 \\
      \bottomrule
    \end{NiceTabular}
  }
  \vspace{0.25cm}\\
  \underline{$\auroc$}
  \vspace{0.25cm}\\
  \resizebox{\columnwidth}{!}{% 
    \begin{NiceTabular}{lcccccc}
      \CodeBefore
        \rowcolor{\headercol}{1}
        \rowcolors{3}{\rowcol}{}[restart]
      \Body
      \toprule
      \textbf{Formulation}
        & \textbf{MNIST}
        & \textbf{FashionMNIST}
        & \textbf{CIFAR10}
        & \textbf{CIFAR20}
        & \textbf{CIFAR100}
        & \textbf{SVHN2}\\
      \midrule
      \SVM
        & 99.94
        & 99.66
        & 90.02
        & 79.75
        & 87.80
        & \worst{90.14} \\
      \TopPush
        & 99.94
        & 99.56
        & 89.35
        & 79.06
        & \worst{87.03}
        & 92.77 \\
      \TopPushK(5)
        & 99.95
        & 99.64
        & 89.05
        & 79.13
        & 87.21
        & 92.60 \\
      \TopPushK(10)
        & 99.95
        & 99.67
        & 89.16
        & 79.27
        & 87.78
        & 92.67 \\
      \tauFPL(0.01)
        & \best{99.97}
        & 99.68
        & 89.83
        & 79.07
        & 87.64
        & 92.98 \\
      \tauFPL(0.05)
        & 99.93
        & \best{99.80}
        & \best{90.34}
        & \best{80.17}
        & 88.56
        & \best{93.16} \\
      \PatMatNP(0.01)
        & \worst{99.78}
        & \worst{99.40}
        & 87.62
        & 78.82
        & \best{89.78}
        & 90.80 \\
      \PatMatNP(0.05)
        & \worst{99.78}
        & \worst{99.40}
        & \worst{87.61}
        & \worst{78.76}
        & 89.52
        & 90.82 \\
      \bottomrule
    \end{NiceTabular}
  }
  \caption{\textbf{Dual formulations with gaussian kernel:} Each table corresponds to one performance metric, and all presented results are medians of ten independent runs for each pair of datasets and formulation. The best result for each dataset is highlighted in green, while the worst result is highlighted in red.}
  \label{tab: dual gauss medians}
\end{table}

\begin{figure}[!p]
  \centering
  \documentclass{standalone}
\usepackage[ddmmyyyy]{datetime}

% ------------------------------------------------------------------------------
% Packages
% ------------------------------------------------------------------------------
% Page setting
\usepackage[explicit]{titlesec}
\usepackage{sectsty}
\usepackage{fancyhdr}

% Text options
\usepackage{lmodern}
\usepackage[T1]{fontenc}
\usepackage[utf8]{inputenc}
\usepackage{xspace}

\usepackage{amsfonts}
\usepackage{dsfont}
\usepackage{pifont}

\usepackage[color=myred!50]{todonotes}

% Graphics and colors
\usepackage{graphicx}
\usepackage{import}
\usepackage{graphics}
\usepackage{xcolor}

\definecolor{myred}{RGB}{150,0,0}  
\definecolor{mygreen}{RGB}{0,150,0}
\definecolor{myblue}{RGB}{0, 101, 189}
\definecolor{myyellow}{RGB}{220, 206, 0}
\definecolor{myorange}{RGB}{255, 153, 51}
\definecolor{mycyan}{RGB}{51, 204, 204}
\definecolor{mypurple}{RGB}{204, 0, 153}

\newcommand{\doccol}{\color{myblue}}

% Hyperrefs
\usepackage{hyperref}
\hypersetup{
  pdfusetitle,
  unicode = true,
  bookmarks = true,
  bookmarksnumbered = false,
  bookmarksopen = true,
  breaklinks = false,
  pdfborderstyle = {},
  backref = false,
  colorlinks = true,
  linkcolor = myblue,
  urlcolor = myred,
  citecolor = mygreen,
}

% enumerate and itemize
\usepackage{enumitem}

% Appendix
\usepackage[title, titletoc]{appendix}

% Captions
\usepackage{caption}
\usepackage{subcaption}

\captionsetup[figure]{position = bottom}
\captionsetup[table]{position = bottom}

% Figures

% Tables
\usepackage{booktabs}
\usepackage{nicematrix}

\renewcommand{\arraystretch}{1.5}

% Algorithms
\usepackage{algorithm}
\usepackage{algorithmicx}
\usepackage{algpseudocode}

% Math
\usepackage{amsmath}
\usepackage{amsthm}
\usepackage{amssymb}
\usepackage{mathtools}
\usepackage{nicefrac}
\usepackage{bm}
\usepackage{thmtools}
\usepackage{thm-restate}
\usepackage{optidef}

% Theorems
\usepackage[framemethod=TikZ]{mdframed}
\usepackage{xifthen}

% Tikz and pfgplots
\usepackage{tikz}
\usepackage{pgfplots}
\usepackage{pgfplotstable}

\usetikzlibrary{shapes}
\usetikzlibrary{arrows}
\usetikzlibrary{automata}
\usetikzlibrary{positioning}
\usetikzlibrary{calc}
\usetikzlibrary{intersections}

\pgfplotsset{compat=newest}
\usepgfplotslibrary{groupplots}
\usepgfplotslibrary{fillbetween}

% ------------------------------------------------------------------------------
% Math declarations
% ------------------------------------------------------------------------------
\newcommand{\Brac}[2][r]{%
  \ifx r#1 \left(       #2 \right)       \else
  \ifx c#1 \left\{      #2 \right\}      \else
  \ifx s#1 \left[       #2 \right]       \else
  \ifx v#1 \left\vert   #2 \right\vert   \else
  \ifx a#1 \left\langle #2 \right\rangle \else
  \ifx t#1 \left\lceil  #2 \right\rceil  \else
  \ifx b#1 \left\lfloor #2 \right\rfloor \else
  \ifx n#1 \left\|      #2 \right\|      \else
  \mathrm{Illegal~option}%
  \fi\fi\fi\fi\fi\fi\fi\fi
}

\newcommand{\clip}[4][s]{
  \ifx s#1 \mathrm{clip}_{\Brac[s]{#2,\; #3}}\Brac{#4} \else
  \ifx u#1 \mathrm{clip}_{\left[#2,\; #3\right)}\Brac{#4} \else
  \ifx l#1 \mathrm{clip}_{\left(#2,\; #3\right]}\Brac{#4} \else
  \mathrm{Illegal~option}%
  \fi\fi\fi
}

\newcommand{\yesmark}{\textcolor{mygreen}{\ding{51}}}%
\newcommand{\nomark}{\textcolor{myred}{\ding{55}}}
\newcommand{\good}[1]{\textcolor{mygreen}{#1}}
\newcommand{\bad}[1]{\textcolor{myred}{#1}}

\newcommand{\R}{\mathbb{R}}
\newcommand{\N}{\mathbb{N}}
\newcommand{\X}{\mathbb{X}}
\newcommand{\Xc}{\mathcal{X}}
\newcommand{\D}{\mathcal{D}}

\newcommand{\I}{\mathcal{I}}
\newcommand{\Itil}{\tilde{\mathcal{I}}}
\newcommand{\Ineg}{\I_{-}}
\newcommand{\Ipos}{\I_{+}}

\newcommand{\Imb}{\I_{\text{mb}}}
\newcommand{\Imbneg}{\I_{\text{mb},-}}
\newcommand{\Imbpos}{\I_{\text{mb},+}}

\newcommand{\nall}{n}
\newcommand{\nneg}{n_{-}}
\newcommand{\npos}{n_{+}}
\newcommand{\ntil}{\tilde{n}}

\newcommand{\nmb}{n_{\text{mb}}}
\newcommand{\nmbneg}{n_{\text{mb},-}}
\newcommand{\nmbpos}{n_{\text{mb},+}}

\newcommand{\K}{\mathbb{K}}
\newcommand{\Kall}{\K^{\pm}}
\newcommand{\Kneg}{\K^{-}}

\newcommand{\updateaa}{\Delta_{\alpha,\alpha}}
\newcommand{\updateab}{\Delta_{\alpha,\beta}}
\newcommand{\updatebb}{\Delta_{\beta,\beta}}

\newcommand{\alphak}{\alpha_{\hat{k}}}
\newcommand{\alphal}{\alpha_{\hat{l}}}
\newcommand{\betak}{\beta_{\hat{k}}}
\newcommand{\betal}{\beta_{\hat{l}}}

\newcommand{\norm}[1]{\Brac[n]{#1}}
\newcommand{\abs}[1]{|#1|}
\newcommand{\inner}[2]{\Brac[a]{#1, \; #2}}
\newcommand{\dd}[1]{\mathop{}\!\mathrm{d}#1}
\newcommand{\minimize}[1]{\ifthenelse{\isempty{#1}}{\operatorname*{minimize}\quad}{\operatorname*{minimize}_{#1}\quad}}
\newcommand{\maximize}[1]{\ifthenelse{\isempty{#1}}{\operatorname*{maximize}\quad}{\operatorname*{maximize}_{#1}\quad}}
\newcommand{\st}{\operatorname{subject\ to}}
\newcommand{\argmin}{\operatorname*{argmin}}
\newcommand{\eps}{{\varepsilon}}

\newcommand{\Imin}{I_{\rm mb}}
\newcommand{\Iminh}{I_{\rm mb}^{\rm enh}}
\newcommand{\nmin}{n_{\rm mb}}
\newcommand{\II}{\mathds{1}}
\newcommand{\Iverson}[1]{\mathds{1}_{\Brac[s]{#1}}}

\newcommand{\EE}{\mathbb{E}}
\newcommand{\PP}{\mathbb{P}}
\newcommand{\bias}{\operatorname{bias}}

\newcommand{\Matrix}[1]{\begin{pmatrix} #1 \end{pmatrix}}
\newcommand{\Set}[2]{\Brac[c]{#1 \; \middle\vert \; #2}}
\newcommand{\domain}{\operatorname*{dom}}

\newcommand{\repeatloop}{\texttt{repeat}\xspace}
\newcommand{\forloop}{\texttt{for}\xspace}

\newcommand{\vecab}{\Matrix{\bm{\alpha} \\ \bm{\beta}}}

% models
\newcommand{\AccatTop}{\emph{Accuracy at the Top}\xspace}
\newcommand{\TopPush}{\emph{TopPush}\xspace}
\newcommand{\TopPushK}{\emph{TopPushK}\xspace}
\newcommand{\tauFPL}{{\emph{$\tau$-FPL}}\xspace}
\newcommand{\TopMeanK}{\emph{TopMeanK}\xspace}
\newcommand{\PatMat}{\emph{Pat}\&\emph{Mat}\xspace}
\newcommand{\PatMatNP}{{\emph{Pat}\&\emph{Mat-NP}}\xspace}
\newcommand{\Grill}{\emph{Grill}\xspace}
\newcommand{\GrillNP}{\emph{Grill-NP}\xspace}
\newcommand{\DeepTopPush}{\emph{DeepTopPush}\xspace}
\newcommand{\TFCO}{\emph{TFCO}\xspace}
\newcommand{\APPerf}{\emph{Ap-Perf}\xspace}
\newcommand{\BaseLine}{\emph{BaseLine}\xspace}

% counts and rates
\DeclareMathOperator{\tp}{tp}
\DeclareMathOperator{\tn}{tn}
\DeclareMathOperator{\fp}{fp}
\DeclareMathOperator{\fn}{fn}
\DeclareMathOperator{\tpr}{tpr}
\DeclareMathOperator{\tnr}{tnr}
\DeclareMathOperator{\fpr}{fpr}
\DeclareMathOperator{\fnr}{fnr}

\DeclareMathOperator{\tps}{\overline{tp}}
\DeclareMathOperator{\tns}{\overline{tn}}
\DeclareMathOperator{\fps}{\overline{fp}}
\DeclareMathOperator{\fns}{\overline{fn}}

\DeclareMathOperator{\accuracy}{acc}
\DeclareMathOperator{\baccuracy}{bacc}
\DeclareMathOperator{\precision}{precision}
\DeclareMathOperator{\recall}{recall}
\DeclareMathOperator{\pratrec}{Precision@Recall}
\DeclareMathOperator{\postop}{pos@top}

% ------------------------------------------------------------------------------
% Document
% ------------------------------------------------------------------------------
\tikzstyle{line_node} = [line width=1pt, rounded corners, color=black, ->]
\tikzstyle{line_cv} = [line width=3pt, color=mygreen, line cap=round]

\begin{document}
\begin{tikzpicture}
  \node at (6.0,2.8) {ROC AUC}; 
  \draw (1,0) -- (11,0); 
  \foreach \x in {1,...,11} \draw (\x,0.1) -- (\x,-0.1) node[anchor=north]{$\x$}; 
  \draw[line_node] (2.79,0) -- (2.79,0.4) -- (0.9, 0.4) node[anchor=east] {\PatMatNP(0.01)}; 
  \draw[line_node] (3.04,0) -- (3.04,0.8) -- (0.9, 0.8) node[anchor=east] {\PatMatNP(0.05)}; 
  \draw[line_node] (4.41,0) -- (4.41,1.2) -- (0.9, 1.2) node[anchor=east] {\BaseLine}; 
  \draw[line_node] (4.45,0) -- (4.45,1.6) -- (0.9, 1.6) node[anchor=east] {\tauFPL(0.05)}; 
  \draw[line_node] (4.93,0) -- (4.93,2.0) -- (0.9, 2.0) node[anchor=east] {\TopPushK(10)}; 
  \draw[line_node] (5.66,0) -- (5.66,2.4) -- (0.9, 2.4) node[anchor=east] {\TopPushK(5)}; 
  \draw[line_node] (6.27,0) -- (6.27,2.0) -- (11.1, 2.0) node[anchor=west] {\tauFPL(0.01)}; 
  \draw[line_node] (6.89,0) -- (6.89,1.6) -- (11.1, 1.6) node[anchor=west] {\DeepTopPush}; 
  \draw[line_node] (7.54,0) -- (7.54,1.2) -- (11.1, 1.2) node[anchor=west] {\TopPush}; 
  \draw[line_node] (9.94,0) -- (9.94,0.8) -- (11.1, 0.8) node[anchor=west] {\GrillNP(0.05)}; 
  \draw[line_node] (10.09,0) -- (10.09,0.4) -- (11.1, 0.4) node[anchor=west] {\GrillNP(0.01)}; 
  \draw[line_cv] (2.79,0.2) -- (4.45, 0.2); 
  \draw[line_cv] (4.41,0.4) -- (5.66, 0.4); 
  \draw[line_cv] (4.93,0.2) -- (6.27, 0.2); 
  \draw[line_cv] (5.66,0.2) -- (6.89, 0.2); 
  \draw[line_cv] (6.27,0.2) -- (7.54, 0.2); 
  \draw[line_cv] (9.94,0.2) -- (10.09, 0.2); 

  \node at (6.0,6.3) {True-positive rate at False-positive rate $0.05$}; 
  \draw (1,3.5) -- (11,3.5); 
  \foreach \x in {1,...,11} \draw (\x,3.6) -- (\x,3.4) node[anchor=north]{$\x$}; 
  \draw[line_node] (2.4,3.5) -- (2.4,3.9) -- (0.9, 3.9) node[anchor=east] {\PatMatNP(0.05)}; 
  \draw[line_node] (2.75,3.5) -- (2.75,4.3) -- (0.9, 4.3) node[anchor=east] {\PatMatNP(0.01)}; 
  \draw[line_node] (4.23,3.5) -- (4.23,4.7) -- (0.9, 4.7) node[anchor=east] {\tauFPL(0.05)}; 
  \draw[line_node] (4.66,3.5) -- (4.66,5.1) -- (0.9, 5.1) node[anchor=east] {\BaseLine}; 
  \draw[line_node] (4.76,3.5) -- (4.76,5.5) -- (0.9, 5.5) node[anchor=east] {\TopPushK(10)}; 
  \draw[line_node] (5.79,3.5) -- (5.79,5.9) -- (0.9, 5.9) node[anchor=east] {\TopPushK(5)}; 
  \draw[line_node] (6.54,3.5) -- (6.54,5.5) -- (11.1, 5.5) node[anchor=west] {\tauFPL(0.01)}; 
  \draw[line_node] (7.24,3.5) -- (7.24,5.1) -- (11.1, 5.1) node[anchor=west] {\DeepTopPush}; 
  \draw[line_node] (7.66,3.5) -- (7.66,4.7) -- (11.1, 4.7) node[anchor=west] {\TopPush}; 
  \draw[line_node] (9.92,3.5) -- (9.92,4.3) -- (11.1, 4.3) node[anchor=west] {\GrillNP(0.05)}; 
  \draw[line_node] (10.05,3.5) -- (10.05,3.9) -- (11.1, 3.9) node[anchor=west] {\GrillNP(0.01)}; 
  \draw[line_cv] (2.4,3.7) -- (2.75, 3.7); 
  \draw[line_cv] (2.75,3.9) -- (4.23, 3.9); 
  \draw[line_cv] (4.23,3.7) -- (5.79, 3.7); 
  \draw[line_cv] (4.76,3.7) -- (6.54, 3.7); 
  \draw[line_cv] (5.79,3.7) -- (7.24, 3.7); 
  \draw[line_cv] (6.54,3.7) -- (7.66, 3.7); 
  \draw[line_cv] (9.92,3.7) -- (10.05, 3.7); 

  \node at (6.0,9.8) {True-positive rate at False-positive rate $0.01$}; 
  \draw (1,7.0) -- (11,7.0); 
  \foreach \x in {1,...,11} \draw (\x,7.1) -- (\x,6.9) node[anchor=north]{$\x$}; 
  \draw[line_node] (2.29,7.0) -- (2.29,7.4) -- (0.9, 7.4) node[anchor=east] {\PatMatNP(0.01)}; 
  \draw[line_node] (2.88,7.0) -- (2.88,7.8) -- (0.9, 7.8) node[anchor=east] {\PatMatNP(0.05)}; 
  \draw[line_node] (4.79,7.0) -- (4.79,8.2) -- (0.9, 8.2) node[anchor=east] {\tauFPL(0.05)}; 
  \draw[line_node] (4.87,7.0) -- (4.87,8.6) -- (0.9, 8.6) node[anchor=east] {\BaseLine}; 
  \draw[line_node] (5.15,7.0) -- (5.15,9.0) -- (0.9, 9.0) node[anchor=east] {\TopPushK(10)}; 
  \draw[line_node] (5.61,7.0) -- (5.61,9.4) -- (0.9, 9.4) node[anchor=east] {\TopPushK(5)}; 
  \draw[line_node] (6.16,7.0) -- (6.16,9.0) -- (11.1, 9.0) node[anchor=west] {\tauFPL(0.01)}; 
  \draw[line_node] (6.84,7.0) -- (6.84,8.6) -- (11.1, 8.6) node[anchor=west] {\DeepTopPush}; 
  \draw[line_node] (7.45,7.0) -- (7.45,8.2) -- (11.1, 8.2) node[anchor=west] {\TopPush}; 
  \draw[line_node] (9.91,7.0) -- (9.91,7.8) -- (11.1, 7.8) node[anchor=west] {\GrillNP(0.05)}; 
  \draw[line_node] (10.06,7.0) -- (10.06,7.4) -- (11.1, 7.4) node[anchor=west] {\GrillNP(0.01)}; 
  \draw[line_cv] (2.29,7.2) -- (2.88, 7.2); 
  \draw[line_cv] (4.79,7.2) -- (6.16, 7.2); 
  \draw[line_cv] (5.15,7.2) -- (6.84, 7.2); 
  \draw[line_cv] (6.16,7.2) -- (7.45, 7.2); 
  \draw[line_cv] (9.91,7.2) -- (10.06, 7.2); 

  \node at (6.0,13.3) {True-positive rate at $K = 10$}; 
  \draw (1,10.5) -- (11,10.5); 
  \foreach \x in {1,...,11} \draw (\x,10.6) -- (\x,10.4) node[anchor=north]{$\x$}; 
  \draw[line_node] (4.66,10.5) -- (4.66,10.9) -- (0.9, 10.9) node[anchor=east] {\BaseLine}; 
  \draw[line_node] (4.91,10.5) -- (4.91,11.3) -- (0.9, 11.3) node[anchor=east] {\TopPush}; 
  \draw[line_node] (5.14,10.5) -- (5.14,11.7) -- (0.9, 11.7) node[anchor=east] {\TopPushK(5)}; 
  \draw[line_node] (5.18,10.5) -- (5.18,12.1) -- (0.9, 12.1) node[anchor=east] {\tauFPL(0.01)}; 
  \draw[line_node] (5.31,10.5) -- (5.31,12.5) -- (0.9, 12.5) node[anchor=east] {\DeepTopPush}; 
  \draw[line_node] (5.6,10.5) -- (5.6,12.9) -- (0.9, 12.9) node[anchor=east] {\TopPushK(10)}; 
  \draw[line_node] (5.89,10.5) -- (5.89,12.5) -- (11.1, 12.5) node[anchor=west] {\GrillNP(0.01)}; 
  \draw[line_node] (5.96,10.5) -- (5.96,12.1) -- (11.1, 12.1) node[anchor=west] {\GrillNP(0.05)}; 
  \draw[line_node] (6.01,10.5) -- (6.01,11.7) -- (11.1, 11.7) node[anchor=west] {\tauFPL(0.05)}; 
  \draw[line_node] (7.9,10.5) -- (7.9,11.3) -- (11.1, 11.3) node[anchor=west] {\PatMatNP(0.01)}; 
  \draw[line_node] (9.44,10.5) -- (9.44,10.9) -- (11.1, 10.9) node[anchor=west] {\PatMatNP(0.05)}; 
  \draw[line_cv] (4.66,10.7) -- (6.01, 10.7); 
  \draw[line_cv] (7.9,10.7) -- (9.44, 10.7); 

  \node at (6.0,16.8) {True-positive rate at $K = 5$}; 
  \draw (1,14.0) -- (11,14.0); 
  \foreach \x in {1,...,11} \draw (\x,14.1) -- (\x,13.9) node[anchor=north]{$\x$}; 
  \draw[line_node] (4.66,14.0) -- (4.66,14.4) -- (0.9, 14.4) node[anchor=east] {\BaseLine}; 
  \draw[line_node] (4.89,14.0) -- (4.89,14.8) -- (0.9, 14.8) node[anchor=east] {\TopPush}; 
  \draw[line_node] (5.14,14.0) -- (5.14,15.2) -- (0.9, 15.2) node[anchor=east] {\TopPushK(5)}; 
  \draw[line_node] (5.18,14.0) -- (5.18,15.6) -- (0.9, 15.6) node[anchor=east] {\tauFPL(0.01)}; 
  \draw[line_node] (5.31,14.0) -- (5.31,16.0) -- (0.9, 16.0) node[anchor=east] {\DeepTopPush}; 
  \draw[line_node] (5.59,14.0) -- (5.59,16.4) -- (0.9, 16.4) node[anchor=east] {\TopPushK(10)}; 
  \draw[line_node] (5.89,14.0) -- (5.89,16.0) -- (11.1, 16.0) node[anchor=west] {\GrillNP(0.01)}; 
  \draw[line_node] (5.96,14.0) -- (5.96,15.6) -- (11.1, 15.6) node[anchor=west] {\GrillNP(0.05)}; 
  \draw[line_node] (6.06,14.0) -- (6.06,15.2) -- (11.1, 15.2) node[anchor=west] {\tauFPL(0.05)}; 
  \draw[line_node] (7.89,14.0) -- (7.89,14.8) -- (11.1, 14.8) node[anchor=west] {\PatMatNP(0.01)}; 
  \draw[line_node] (9.44,14.0) -- (9.44,14.4) -- (11.1, 14.4) node[anchor=west] {\PatMatNP(0.05)}; 
  \draw[line_cv] (4.66,14.2) -- (6.06, 14.2); 
  \draw[line_cv] (7.89,14.2) -- (9.44, 14.2); 

  \node at (6.0,20.3) {True-positive rate at $K = 1$}; 
  \draw (1,17.5) -- (11,17.5); 
  \foreach \x in {1,...,11} \draw (\x,17.61) -- (\x,17.39) node[anchor=north]{$\x$}; 
  \draw[line_node] (4.64,17.5) -- (4.64,17.9) -- (0.9, 17.9) node[anchor=east] {\BaseLine}; 
  \draw[line_node] (4.82,17.5) -- (4.82,18.3) -- (0.9, 18.3) node[anchor=east] {\TopPush}; 
  \draw[line_node] (5.09,17.5) -- (5.09,18.7) -- (0.9, 18.7) node[anchor=east] {\tauFPL(0.01)}; 
  \draw[line_node] (5.09,17.5) -- (5.09,19.1) -- (0.9, 19.1) node[anchor=east] {\TopPushK(5)}; 
  \draw[line_node] (5.29,17.5) -- (5.29,19.5) -- (0.9, 19.5) node[anchor=east] {\DeepTopPush}; 
  \draw[line_node] (5.54,17.5) -- (5.54,19.9) -- (0.9, 19.9) node[anchor=east] {\TopPushK(10)}; 
  \draw[line_node] (5.79,17.5) -- (5.79,19.5) -- (11.1, 19.5) node[anchor=west] {\GrillNP(0.01)}; 
  \draw[line_node] (5.92,17.5) -- (5.92,19.1) -- (11.1, 19.1) node[anchor=west] {\GrillNP(0.05)}; 
  \draw[line_node] (5.99,17.5) -- (5.99,18.7) -- (11.1, 18.7) node[anchor=west] {\tauFPL(0.05)}; 
  \draw[line_node] (8.36,17.5) -- (8.36,18.3) -- (11.1, 18.3) node[anchor=west] {\PatMatNP(0.01)}; 
  \draw[line_node] (9.46,17.5) -- (9.46,17.9) -- (11.1, 17.9) node[anchor=west] {\PatMatNP(0.05)}; 
  \draw[line_cv] (4.64,17.7) -- (5.99, 17.7); 
  \draw[line_cv] (8.36,17.7) -- (9.46, 17.7); 
\end{tikzpicture}
\end{document}

  \caption{\textbf{Primal formulations with non-linear model:} Critical difference diagrams (level of importance 0.05) of the Nemenyi post hoc test for the Friedman test. Each diagram shows the mean rank of each method, with rank one being the best. The green horizontal lines group methods with mean ranks that are not significantly different. The critical difference diagrams were computed for mean rank averages over all datasets.}
  \label{fig: primal nonlinear CD}
\end{figure}

\begin{table}[!p]
  \centering
  \underline{$\tpratk =10$}
  \vspace{0.25cm}\\
  \resizebox{\columnwidth}{!}{% 
    \begin{NiceTabular}{lccccccc}
      \CodeBefore
        \rowcolor{\headercol}{1}
        \rowcolors{3}{\rowcol}{}[restart]
      \Body
      \toprule
      \textbf{Formulation}
        & \textbf{MNIST}
        & \textbf{FashionMNIST}
        & \textbf{CIFAR10}
        & \textbf{CIFAR20}
        & \textbf{CIFAR100}
        & \textbf{SVHN2}
        & \textbf{SVHN2Extra}\\
      \midrule
      \BaseLine
        & \best{99.26}
        & \best{98.10}
        & 11.40
        & 3.50
        & 5.00
        & 11.34
        & 15.95 \\
      \DeepTopPush
        & 98.42
        & 97.60
        & \worst{0.20}
        & 0.20
        & \worst{0.00}
        & \worst{0.17}
        & \worst{0.00} \\
      \TopPushK(5)
        & 98.54
        & 97.50
        & 2.00
        & \worst{0.00}
        & 8.50
        & 10.58
        & 16.12 \\
      \TopPushK(10)
        & 98.24
        & 96.90
        & 12.55
        & \best{100.00}
        & \best{100.00}
        & \best{100.00}
        & \best{100.00} \\
      \tauFPL(0.01)
        & 98.72
        & 97.50
        & 1.00
        & 0.20
        & 9.00
        & 9.52
        & \worst{0.00} \\
      \tauFPL(0.05)
        & 96.78
        & 96.50
        & 14.80
        & 0.30
        & 8.50
        & 13.05
        & 12.48 \\
      \PatMatNP(0.01)
        & 98.54
        & 97.45
        & \best{32.45}
        & 4.80
        & 20.00
        & 13.92
        & 19.33 \\
      \PatMatNP(0.05)
        & \worst{82.86}
        & \worst{94.30}
        & 26.55
        & 5.90
        & 11.50
        & 11.36
        & 15.98 \\
      \bottomrule
    \end{NiceTabular}
  }
  \vspace{0.25cm}\\
  \underline{$\tpratfpr = 0.05$}
  \vspace{0.25cm}\\
  \resizebox{\columnwidth}{!}{% 
    \begin{NiceTabular}{lccccccc}
      \CodeBefore
        \rowcolor{\headercol}{1}
        \rowcolors{3}{\rowcol}{}[restart]
      \Body
      \toprule
      \textbf{Formulation}
        & \textbf{MNIST}
        & \textbf{FashionMNIST}
        & \textbf{CIFAR10}
        & \textbf{CIFAR20}
        & \textbf{CIFAR100}
        & \textbf{SVHN2}
        & \textbf{SVHN2Extra}\\
      \midrule
      \BaseLine
        & \best{100.00}
        & \best{99.90}
        & 83.35
        & 48.00
        & 82.00
        & 94.66
        & 97.71 \\
      \DeepTopPush
        & \worst{99.82}
        & \worst{99.70}
        & \worst{5.85}
        & 8.90
        & 9.00
        & 40.14
        & \worst{0.00} \\
      \TopPushK(5)
        & \best{100.00}
        & 99.85
        & 34.30
        & 7.70
        & 53.50
        & 86.54
        & 93.96 \\
      \TopPushK(10)
        & \best{100.00}
        & \best{99.90}
        & 27.95
        & \worst{0.00}
        & \worst{0.00}
        & \worst{0.00}
        & \worst{0.00} \\
      \tauFPL(0.01)
        & \best{100.00}
        & \best{99.90}
        & 24.40
        & 11.50
        & 65.50
        & 87.60
        & \worst{0.00} \\
      \tauFPL(0.05)
        & \best{100.00}
        & \best{99.90}
        & 82.75
        & 18.00
        & 66.50
        & 94.51
        & 97.52 \\
      \PatMatNP(0.01)
        & \best{100.00}
        & \best{99.90}
        & 91.55
        & 52.20
        & 79.00
        & 95.49
        & 98.43 \\
      \PatMatNP(0.05)
        & \best{100.00}
        & \best{99.90}
        & \best{91.75}
        & \best{57.70}
        & \best{85.00}
        & \best{95.53}
        & \best{98.50} \\
      \bottomrule
    \end{NiceTabular}
  }
  \vspace{0.25cm}\\
  \underline{$\auroc$}
  \vspace{0.25cm}\\
  \resizebox{\columnwidth}{!}{% 
    \begin{NiceTabular}{lccccccc}
      \CodeBefore
        \rowcolor{\headercol}{1}
        \rowcolors{3}{\rowcol}{}[restart]
      \Body
      \toprule
      \textbf{Formulation}
        & \textbf{MNIST}
        & \textbf{FashionMNIST}
        & \textbf{CIFAR10}
        & \textbf{CIFAR20}
        & \textbf{CIFAR100}
        & \textbf{SVHN2}
        & \textbf{SVHN2Extra}\\
      \midrule
      \BaseLine
        & \best{100.00}
        & \best{99.98}
        & 96.85
        & 84.67
        & 95.94
        & 98.54
        & 99.20 \\
      \DeepTopPush
        & 99.98
        & \worst{99.95}
        & \worst{49.51}
        & 59.40
        & 56.68
        & 83.12
        & 1.61 \\
      \TopPushK(5)
        & \best{100.0}
        & 99.97
        & 77.10
        & 55.50
        & 84.24
        & 96.57
        & 98.32 \\
      \TopPushK(10)
        & \best{100.00}
        & \best{99.98}
        & 74.26
        & \worst{0.00}
        & \worst{0.00}
        & \worst{0.00}
        & \worst{0.00} \\
      \tauFPL(0.01)
        & \best{100.0}
        & \best{99.98}
        & 70.96
        & 60.03
        & 90.16
        & 96.68
        & 25.84 \\
      \tauFPL(0.05)
        & 99.99
        & 99.97
        & 95.86
        & 68.18
        & 90.76
        & 98.50
        & 99.12 \\
      \PatMatNP(0.01)
        & 99.99
        & \best{99.98}
        & 97.90
        & 84.38
        & 93.84
        & 98.74
        & \best{99.38} \\
      \PatMatNP(0.05)
        & \worst{99.96}
        & 99.96
        & \best{98.24}
        & \best{88.39}
        & \best{96.56}
        & \best{98.76}
        & 99.32 \\
      \bottomrule
    \end{NiceTabular}
  }
  \caption{\textbf{Primal formulations with non-linear model:} Each table corresponds to one performance metric, and all presented results are medians of ten independent runs for each pair of datasets and formulation. The best result for each dataset is highlighted in green, while the worst result is highlighted in red.}
  \label{tab: primal nonlinear medians}
\end{table}

\pagebreak

\section{Steganalysis}\label{sec: steganalysis}

In the previous section, we presented results on standard image recognition datasets. Even though the results are quite good on these datasets, they did not fully show the importance of the problem of classification at the top. To show the importance of this problem properly,  we need to find the field in which the maximizing true-positive rate at the low false-positive rate is an important task. Such a field can be, for example, steganalysis.

The standard way to share secret information these days is through encryption. However, in such a case, the presence of a secret message (even though encrypted) is obvious. Steganography aims to hide the fact that communication is taking place by hiding the secret message within an ordinary file (usually called a cover file) to avoid detection. The secret message is then extracted at its destination. The secret data can be hidden in almost any type of digital content. However, the most popular are images. There are two reasons for this. The first of them is the ubiquity of images on the Internet and, therefore, the ease of using them as cover files for secret messages. The second reason is their large potential payload, i.e., it is possible to hide a lot of information in high-resolution images. With an appropriate cover image and steganography tools, it is possible to create a stego-image (image with a hidden message) that can not be recognized from the cover image by human perception. However, each tool leaves a fingerprint or signature in the image that can be used to detect stego images. The field that tries to detect stego images and possibly decrypt messages from them is called steganalysis. In steganalysis, the goal is to achieve the best true-positive rate with the lowest possible false-positive rate. Therefore steganalysis is the domain suitable for the problem of classification at the top.~\cite{morkel2005overview, silman2001steganography} 

For the experiments, we have a large private dataset of cover images comprising approximately 450 000 images from Flickr. All these images are in JPEG format with a quality factor of 80. Since the dataset does not contain any stego images, we use two different ways to generate them. For this work, we named the resulting datasets as \textbf{Nsf5} and \textbf{JMiPOD} based on the algorithms used for generating stego images.

\subsection{Nsf5}

In this case, we generate stego images using simulated F5 with matrix embedding turned off, and we use payload 0.2. Since we are interested in low false-positive rates, we need a lot of negative (cover) samples to estimate it. It is evident in~\eqref{eq: patmat np} where the threshold~$t$ is a surrogate approximation of false-positive rate, i.e., the threshold is computed only from negative samples. Positive (stego) samples occur only in the objective function. Since generating stego images is expensive and we do not need them to estimate the false-positive rate, we decided to use 10\% of all cover images to generate their stego counterparts. All images are then described using 22 500 features and split into train/validation/test sets in ratio 0.45/0.05/0.5. The resulting sizes of the splits, as well as the number of stego images in them, are summarized in Table~\ref{tab: datasets summary}. 

Since the resulting classification task is relatively easier to solve, we decided to use a simple linear model. The number of training samples and their size are not too large. Therefore we can load the whole dataset into memory. It allows us to use full gradient descent instead of its stochastic version. As an optimizer, we use the ADAM~\cite{kingma2014adam} with default settings and fixed step length~$\alpha = 0.01.$ We also use a fixed number of epochs to 1000 for all formulations. Finally, we repeat each experiment ten times with ten different random seeds.

Figure~\ref{fig: steganalysis nsf5} shows ROC curves for the test set of \textbf{Nsf5} dataset. For simplicity, we show ROC curves only for one experiment run. Moreover, Table~\ref{tab: steganalysis nsf5} shows seven different performance metrics computed for each formulation. Each shown result in this table is a median of ten independent runs. \BaseLine provides inferior results for all metrics. Surprisingly, \BaseLine is the worst even for the $\auroc.$ On the other hand, \DeepTopPush excels at very low false-positive rates, as seen from the table and the figure. In fact, \DeepTopPush provides the best results for four out of seven performance metrics (the best results are highlighted in green). Note that all these four metrics operate at extremely low false-positive rates. We can also see that \PatMatNP($10^{-5}$) is the best at false-positive rate~$10^{-4}$. This is probably caused by the approximation of the true top $\tau$-quantile of all scores of negative samples used in \PatMatNP formulation. Therefore, \PatMatNP($10^{-5}$) is optimized for a false-positive rate slightly higher than~$10^{-5}$ and as a consequence outperforms \PatMatNP($10^{-5}$) at false-positive rate~$10^{-4}$. Similar behavior can be seen for \PatMatNP($10^{-4}$) and \PatMatNP($10^{-3}$) at false-positive rate~$10^{-4}$.

\begin{figure}
  \centering
  \includegraphics{images/stego_nsft5.pdf}
  \caption{\textbf{Nsf5 dataset:} ROC curves with logarithmic $x$-axis.}
  \label{fig: steganalysis nsf5}
\end{figure}

\begin{table}[!t]
  \centering
  \begin{NiceTabular}{lccccccc}
    \CodeBefore
    \rowcolor{\headercol}{1-2}
    \rowcolors{4}{\rowcol}{}[restart]
    \Body
    \toprule
    \Block[c]{2-1}{\textbf{Formulation}}
    & \Block[c]{2-1}{$\auroc$}
    & \Block[c]{1-3}{$\tpratk$}
    &&& \Block[c]{1-3}{$\tpratfpr$} \\
    \cline{3-8}
    && $1$
    & $10$
    & $5$
    & $10^{-5}$
    & $10^{-4}$
    & $10^{-3}$ \\
    \midrule
    \BaseLine
    & \worst{95.84}
    & \worst{0.0}
    & \worst{0.0}
    & \worst{0.0}
    & \worst{0.0}
    & \worst{0.02}
    & \worst{0.7} \\
    \DeepTopPush
    & 98.29
    & \best{5.07}
    & \best{35.48}
    & \best{57.66}
    & \best{48.65}
    & 89.56
    & 93.67 \\
    \PatMatNP($10^{-5}$)
    & 98.81
    & 2.55
    & 23.02
    & 47.24
    & 35.28
    & \best{91.9}
    & 95.84 \\
    \PatMatNP($10^{-4}$)
    & 98.98
    & \worst{0.0}
    & 0.05
    & 4.34
    & 1.78
    & 79.76
    & \best{96.18} \\
    \PatMatNP($10^{-3}$)
    & \best{99.26}
    & \worst{0.0}
    & \worst{0.0}
    & 0.01
    & \worst{0.0}
    & 0.29
    & 91.98 \\
    \bottomrule
  \end{NiceTabular}
  \caption{\textbf{NSF5 dataset:} All presented results are medians of ten independent runs with different random seeds. Each column of the table corresponds to one performance metric, and every row to one formulation. The best result for each metric is highlighted in green, while the worst is highlighted in red.}
  \label{tab: steganalysis nsf5}
\end{table}

\subsection{JMiPOD}

In this case, we first select all images that can be cropped to size $256 \times 256 \times 3$ and then cropped them losslessly using \emph{jpegtran} library. Then, we use JMiPOD~\cite{cogranne2020steganography} algorithm to generate stego images with payload 0.1. We use the same approach as in the case of Nsf5 dataset and use only 10\% of cover images to generate their stego counterparts. We split the data into train/validation/test sets in a ratio of 0.375/0.125/0.5. The resulting sizes of the splits and the number of stego images in them are summarized in Table~\ref{tab: datasets summary}. 

In this case, the resulting classification task is quite complicated. Therefore we decided to use pre-trained EfficientNet-B0~\cite{tan2019efficientnet} as a model. Originally the model was trained for 1000 classes. Therefore, we removed the last fully-connected layer and replaced it with a randomly initialized fully-connected layer of appropriate size for binary classification. The resulting model is large, and it is not possible to use a full gradient. For this reason, we use stochastic gradient descent with balanced mini-batches of size 256. As an optimizer, we use the ADAM~\cite{kingma2014adam} with default settings and fixed step length~$\alpha = 0.01.$ Finally, we use a fixed number of epochs to 30 for all formulations, and we repeat each experiment ten times with different random seeds.

\begin{figure}[!t]
  \centering
  \includegraphics{images/stego_jmipod.pdf}
  \caption{\textbf{JMiPOD dataset:} ROC curves with logarithmic $x$-axis.}
  \label{fig: steganalysis jmipod}
\end{figure}

\begin{table}[!t]
  \centering
  \begin{NiceTabular}{lccccccc}
    \CodeBefore
      \rowcolor{\headercol}{1-2}
      \rowcolors{4}{\rowcol}{}[restart]
    \Body
    \toprule
    \Block[c]{2-1}{\textbf{Formulation}}
      & \Block[c]{2-1}{$\auroc$}
      & \Block[c]{1-3}{$\tpratk$}
      &&& \Block[c]{1-3}{$\tpratfpr$} \\
    \cline{3-8}
      && $1$
      & $10$
      & $5$
      & $10^{-5}$
      & $10^{-4}$
      & $10^{-3}$ \\
    \midrule
    \BaseLine
      & 97.5
      & \worst{13.52}
      & \worst{24.65}
      & \worst{29.54}
      & \worst{27.28}
      & \worst{44.58}
      & \worst{63.84} \\
    \DeepTopPush
      & \worst{97.26}
      & \best{34.25}
      & \best{42.57}
      & \best{47.3}
      & \best{43.59}
      & 60.42
      & 73.67 \\
    \PatMatNP($10^{-5}$)
      & 97.66
      & 21.24
      & 31.6
      & 39.38
      & 33.54
      & 60.35
      & 78.04 \\
    \PatMatNP($10^{-4}$)
      & 97.49
      & 25.66
      & 36.76
      & 45.19
      & 38.28
      & 63.49
      & 77.43 \\
    \PatMatNP($10^{-3}$)
      & \best{98.0}
      & 26.99
      & 38.76
      & 44.83
      & 42.17
      & \best{64.5}
      & \best{78.11} \\
    \bottomrule
  \end{NiceTabular}
  \caption{\textbf{JMiPOD dataset:} All presented results are medians of ten independent runs with different random seeds. Each column of the table corresponds to one performance metric, and every row to one formulation. The best result for each metric is highlighted in green, while the worst is highlighted in red.}
  \label{tab: steganalysis jmipod}
\end{table}

Figure~\ref{fig: steganalysis jmipod} shows ROC curves for the test set of \textbf{JMiPOD} dataset. Moreover, Table~\ref{tab: steganalysis jmipod} shows seven performance metrics. Each shown result in this table is a median of ten independent runs. Since trained models use stochastic gradient descent, the results are not as evident as for the Nsf5 dataset. \BaseLine still provides the worst results for most metrics, but the differences are much smaller than for the Nsf5 dataset. We can see that \DeepTopPush again provides the best performance for four of seven metrics. It shows that the enhanced minibatch used in \DeepTopPush Algorithm~\ref{alg: deep toppush} improves the approximation quality of the true threshold and reduces the bias of sampled gradient (as we already showed in Figure~\ref{fig:thresholds2}). Even though \PatMatNP($10^{-3}$), \PatMatNP($10^{-4}$), and \PatMatNP($10^{-5}$) were trained for different levels of false-positive rate, they all perform similarly. As we said before, the decision threshold~$t$ of \PatMatNP model approximates the true top $\tau$-quantile of all scores of negative samples. Since we use mini-batches with 128 negative samples, the smallest quantile that can be found on this minibatch is~$\tau = \frac{1}{128}=0.0078125.$ If we try to approximate smaller quantiles, we always get the same results. Therefore, \PatMatNP($10^{-3}$), \PatMatNP($10^{-4}$), \PatMatNP($10^{-5}$) should work almost identically, and we can see from both the figure and the table, that these three formulations provide similar results.

\section{Malware Detection}\label{sec: malware detection}

In the previous section, we presented results from the domain of steganalysis. Another domain in which formulations from the presented framework can be very useful is the domain of malware detection. As an example, consider standard antivirus software on a personal computer. Every user wants to be protected, so the goal of antivirus software is to detect as much malware as possible. However, if the antivirus is too restrictive, it can easily happen that clean software is marked as malware, i.e., the antivirus can easily produce false alarms. If the antivirus produces false alarms too often, it can be very annoying to the user and may lead to uninstalling the antivirus completely. Therefore, the goal of every antivirus is to maximize a true-positive rate at a very low false-positive rate, which is precisely what the formulations from the framework do.

In this section, we present results on a real-world dataset provided by a renowned cybersecurity company. The dataset consists of malware analysis reports of executable files. The dataset is extremely tough as individual samples are JSON files whose size ranges from 1kB to 2.5MB. The sample structure is highly complicated because each sample has a different number of features, and features may have a complicated structure, such as a list of ports to which the file connects. This contrasts sharply with standard datasets, where each sample has the same number of features, and each feature is a real number. The usual approach to processing such complicated data is to manually create feature vectors and use them for training instead of the original data. However, such an approach is extremely time-demanding and requires expert knowledge of the original data. For this reason, we decided to use a different approach called Hierarchical Multiple Instance Learning (HMIL)~\cite{pevny2017using}. For the training, we use a publicly available implementation of HMIL~\cite{mandlik2021mill}, which allows training models directly from JSON files without requiring complicated feature extraction.

Since the dataset is huge (see Table~\ref{tab: datasets summary}), we train each formulation only one time. Moreover, we use only the formulations that worked the best in the previous experiments, i.e., we use only the \BaseLine, \PatMatNP($10^{-2}$), \PatMatNP($10^{-3}$) and \DeepTopPush. As an optimizer, we use the ADAM~\cite{kingma2014adam} with default settings and fixed step length~$\alpha = 0.01.$ We also use balanced mini-batches of size 2000, which allows us to obtain a very good estimate of the true thresholds as discussed in Section~\ref{sec: biased threshold estimate}. Finally, we use a fixed number of epochs to 100 for all formulations.

Figure~\ref{fig: malware detection} shows the ROC curves for all formulations. For use filled circles to highlight the thresholds for which the formulations were optimized. It is clear that \DeepTopPush is the best at low false-positive rates. Even at the extremely low false positive rate $\tau=10^{-5}$, \DeepTopPush correctly identified $46\%$ of malware. We can also see that \PatMatNP($10^{-3}$) is the best at false-positive rate~$10^{-3}$, which is exactly the point for which the formulation should be optimized. However, at this false-positive rate, \DeepTopPush performs almost as well as \PatMatNP($10^{-3}$). Finally, all formulations perform equally well at the false-positive rate~$10^{-2}$.

\begin{figure}
  \centering
  \includegraphics{images/malware_detection.pdf}
  \caption{\textbf{Malware detection:} ROC curves with logarithmic $x$-axis. The circles show the thresholds the formulations were optimized for.}
  \label{fig: malware detection}
\end{figure}
\hfill

\chapter*{Conclusion}
\addcontentsline{toc}{chapter}{Conclusion}

In this work, we studied the problem of classification at the top that is closely related to the binary classification. We showed that many of well known categories of problems such as ranking, accuracy at the top or hypothesis testing are closely related classification at the top. In Chapter~\ref{chap: framework}, we study these three categories in detail and showed, that they can be all formulated in a similar way. This lead us to introduce unified framework for classification at the top~\eqref{eq: aatp surrogate}. We showed that several known formulation (\TopPush, \Grill, \tauFPL) fall into our framework and derived some completely new formulations (\PatMat, \PatMatNP). The summary of all presented formulations is in Table~\ref{tab: summary formulations}.

In Chapter~\ref{chap: linear}, we performed a theoretical analysis of the presented formulations when the linear model is used. We showed that known formulations suffer from certain disadvantages. While \TopPush and \tauFPL are sensitive to outliers, \Grill is non-convex. On the other hand, we shoed that newly introduce \PatMat and \PatMatNP formulations are robust and convex. We also  proved the global convergence of the stochastic gradient descent for \PatMat and \PatMatNP.

In Chapter~\ref{chap: dual}, we extended the framework~\eqref{eq: aatp surrogate} to nonlinear problems. We showed, that all presented formulations (with the exception of \Grill and \GrillNP) can be divided into two families based on the form of the constrains, namely \TopPushK and \PatMat family of formulations. We derived dual formulation for \TopPushK and \PatMat family of formulations. Moreover, we proposed a new coordinate descent algorithm for solving the resulting dual problems. For selected surrogate functions we also derived the closed-form formulas needed in the coordinate descent algorithm. Since the coordinate decent algorithm has to be initialized wit the feasible solution, we also showed how to find an initial feasible solution.

In Chapter~\ref{chap: deep}, we study the primal formulations with non-linear models. We showed, that when we use non-linear model, the resulting formulations are non-decomposable. This property is caused by the special threshold constraint in~\eqref{eq: aatp surrogate}, and prevents us from using of stochastic gradient descent in standard way. We introduce modified stochastic gradient descend for our formulations. Unfortunatelly, we showed that using of stochastic gradient descent peads to the biased eastimate of the true gradient. We suggested that this can migitiate by using large minibatche, however, such an approach is often not possible. For such cases, we proposed \DeepTopPush as an efficient alternative to \TopPush formulation, that does not suffer from this issue. For \DeepTopPush, we implicitly removed some optimization variables, created an unconstrained end-to-end network and used the stochastic gradient descent to train it. We modified the minibatch so that the sampled threshold (computed on a minibatch) is a good estimate of the true threshold (computed on all samples). We showed both theoretically and numerically that this procedure reduces the bias of the sampled gradient.

In Chapter~\ref{chap: experiments}, we performed a numerical comparison of presented formulations. We showed a good performance of our newly introduce formulation \PatMatNP, when used in its primal form. We also showed, that \DeepTopPush formulation can be very useful, especially for very large real-world dataset. On steganalysis datasets and malware detection dataset, we demonstrated, that standard formulations provide poor results at very low false-positive rates, while formulations proposed in this work, work very well on them.


% ------------------------------------------------------------------------------
% Appendix
% ------------------------------------------------------------------------------
\part*{Apendices}
\appendix

\chapter{Appendix for Chapter~\ref{chap: framework}}\label{app: framework}

\lemmattcomparison*
\begin{proof}
  Since~$\bm{s}^+$ and~$\bm{s}^-$ are computed on disjunctive indices, we have
  \begin{equation*}
    s_{[n\tau]} \geq \min\{s_{[\npos\tau]}^+, \; s_{[\nneg\tau]}^-\}.
  \end{equation*}
  Since~$s_{[n\tau]}$ is the threshold for \Grill and~$s_{[\nneg\tau]}^-$ is the threshold for \GrillNP, the first statement follows. The second part can be shown in a similar way.
\end{proof}

\lemmapatmatalg*
\begin{proof}
  Observe that
  \begin{align*}
    h(t_{j})
      = h\Brac{s_{[j]}+\frac{1}{\vartheta}}
      & = \sum_{i \in \I} l\Brac{\vartheta\Brac{s_i - \Brac{s_{[j]} + \frac{1}{\vartheta}}}} - n\tau \\
      & = \sum_{i \in \I} \max\Brac[c]{0,\; 1 + \vartheta\Brac{s_i - s_{[j]} - \frac{1}{\vartheta}}} - n\tau \\
      & = \sum_{i \in \I} \max\Brac[c]{0,\; \vartheta \Brac{s_i - s_{[j]}}} - n\tau \\
      & = \sum_{i = 1}^{j - 1} \vartheta(s_{[i]} - s_{[j]}) - n\tau,
  \end{align*}
  where the last equality holds since~$\vartheta > 0$ and~$s_{[i]} - s_{[j]} \leq 0$ for all~$i \geq j.$ From here, we obtain~$h(t_{1}) = -n\tau$. Moreover, we have
  \begin{align*}
    h(t_{j})
    & = \sum_{i = 1}^{j - 1} \vartheta(s_{[i]} - s_{[j]}) - n\tau \\
    & = \sum_{i = 1}^{j - 2} \vartheta(s_{[i]} - s_{[j]}) + \vartheta(s_{[j-1]} - s_{[j]}) - n\tau \\
    & = \sum_{i = 1}^{j - 2} \vartheta(s_{[i]} - s_{[j]} + s_{[j - 1]} - s_{[j - 1]}) + \vartheta(s_{[j-1]} - s_{[j]}) - n\tau \\
    & = \sum_{i = 1}^{j - 2} \vartheta(s_{[i]} - s_{[j - 1]}) + \sum_{i = 1}^{j - 2} \vartheta(s_{[j - 1]} - s_{[j]}) + \vartheta(s_{[j - 1]} - s_{[j]}) - n\tau \\
    & = h(t_{j - 1}) + (j - 1) \vartheta(s_{[j - 1]} - s_{[j]}) \\
    & = h(t_{j - 1}) + (j - 1) \vartheta(t_{j - 1} - t_{j}),
  \end{align*}
  which finishes the proof.
\end{proof}

\chapter{Appendix for Chapter~\ref{chap: linear}}

Firstly we recall definitions of the thresholds defined in Section~\ref{sec: aatp}
\begin{equation*}
  \begin{aligned}
    t_1(\bm{w}) & = \max \Set{t}{\frac{1}{n} \sum_{i \in \I} \Iverson{s_i \ge t} \ge \tau} \\
    t_2(\bm{w}) & = \frac{1}{K} \sum_{i=1}^{K} s_{[i]} \\
    t_3(\bm{w}) & \quad \text{solves} \quad \frac{1}{n} \sum_{i \in \I} l\Brac{\vartheta(s_i - t)} = \tau, \\
  \end{aligned}
\end{equation*}
and also recall, that we assume linear classifier, i.e. scores are defined for all~$i \in \I$ as~$s_i = \bm{w}^{\top} \bm{x}_i.$

\section{Convexity}

\propconvex*
\begin{proof}[Proof Proposition~\ref{prop:convex} on page~\pageref{prop:convex}]
  It is easy to show that the quantile~$t_1$ is not convex. Due to~\cite{lapin2015top}, the mean of the~$K$ highest values of a vector is a convex function and therefore,~$t_2$ is a convex function. It remains to analyze~$t_3$. Let us dfefine function~$g$ as follows
  \begin{equation*}
    g(\bm{w},t) := \frac{1}{n} \sum_{i \in \I} l(\bm{w}^\top \bm{x}_i - t) - \tau.
  \end{equation*}
  where we for simplicity set~$\vartheta = 1.$ Then~$t_3$ is defined via an implicit equation~$g(\bm{w},t) = 0.$ Moreover, since~$l$ is convex, we immediately obtain that~$g$ is jointly convex in both variables. To show the convexity, consider~$\bm{w}, \; \hat{\bm{w}} \in \R^d$ and the corresponding thresholds~$t = t_3(\bm{w})$,~$\hat{t} = t_3(\hat{\bm{w}})$. Then for any~$\lambda\in[0,1]$ we have 
  \begin{equation}\label{eq:proof_conv1}
    g\Brac{\lambda \bm{w} + (1 - \lambda)\hat{\bm{w}}, \;\lambda t + (1 - \lambda)\hat{t}}
    \le \lambda g(\bm{w}, t) + (1 - \lambda) g(\hat{\bm{w}}, \hat{t}) = 0,
  \end{equation}
  where the inequality follows from the convexity of~$g$ and the equality from
  \begin{equation*}
    g(\bm{w}, t) = g(\hat{\bm{w}}, \hat{t}) = 0,
  \end{equation*}
  which holds true since both~$t$ and $\hat{t}$ solves~\eqref{eq: aatp quantile surrogate}. From the definition of the surrogate quantile function~$t_3$ we have
  \begin{equation}\label{eq:proof_conv2}
    g(\lambda\bm{w} + (1-\lambda)\hat{\bm{w}}, t_3(\lambda\bm{w} + (1-\lambda)\hat{\bm{w}})) = 0.
  \end{equation}
  Since~$g$ is non-increasing in the second variable, from~\eqref{eq:proof_conv1} and~\eqref{eq:proof_conv2} we deduce
  \begin{equation*}
    t_3(\lambda\bm{w} + (1-\lambda)\hat{\bm{w}})
    \le \lambda t + (1-\lambda)\hat{t}
    =   \lambda t_3(\bm{w})+(1-\lambda) t_3(\hat{\bm{w}}),
  \end{equation*}
  which implies that function~$\bm{w}\mapsto t_3(\bm{w})$ is convex.
\end{proof}

\thmconvex*
\begin{proof}[Proof of Theorem~\ref{thm:convex} on page~\pageref{thm:convex}]
  Due to the definition of the surrogate counts~\eqref{eq: confusion counts surrogate}, the function~$L$ equals to
  \begin{equation*}
    L(\bm{w}) = \fns(\bm{s}, t(\bm{w})) = \sum_{i \in \Ipos} l \Brac{t(\bm{w}) - \bm{w}^\top \bm{x}_i}.
  \end{equation*}
  Here we write~$t(\bm{w})$ to stress the dependence of~$t$ on~$\bm{w}$. Since~$\bm{w}\mapsto t(\bm{w})$ is a convex function, we also have that~$\bm{w} \mapsto t(\bm{w}) - \bm{w}^\top \bm{x}$ is a convex function. From its definition, the surrogate function~$l$ is convex and non-decreasing. Since a composition of a convex function with a non-decreasing convex function is a convex function, this finishes the proof.
\end{proof}

\section{Differentiability}

\derivative* 
\begin{proof}[Proof of Theorem~\ref{thm:derivative} on page~\pageref{thm:derivative}]
  The result for~$t_3$ follows directly from the implicit function theorem. The non-differentiability of~$t_1$ and~$t_2$ happens whenever the threshold value is achieved at two different scores.
\end{proof}

\pagebreak

\section{Stability}\label{app: stability}

In this section, we derive the results presented from Section~\ref{sec: stability} more properly. 
\degeneratebehavior*
\noindent Moreover, we assume that~$n$ is large and the outlier may be ignored for the computation of thresholds which require a large number of points. Since the computation is simple for other formulations, we show it only for \PatMat. For~$\bm{w}_0 = (0,0)$, we get
\begin{equation*}
  \tau
  = \frac{1}{n}\sum_{i \in \I} l\Brac{\vartheta(\bm{w}_0^\top \bm{x}_i - t)}
  = l(0 - \vartheta t) = 1 - \vartheta t,
\end{equation*}
which implies
\begin{equation*}\label{eq: PatMat threshold 0}
  t = \nicefrac{1-\tau}{\vartheta}
\end{equation*}
and consequently the value of the objective function is
\begin{equation}\label{eq: PatMat objective 0}
  L(\bm{w}_0)
    = \frac{1}{\npos} \sum_{i \in \Ipos} l(t - \bm{w}_0^\top \bm{x}_i)
    = l(t - 0)
    = 1 + t,
\end{equation}
where the last equality follows from definition of the hinge loss function and the fact that~$t \ge 0.$ This finishes the computation for~$\bm{w}_0$. For~$\bm{w}_1 = (1,0)$ the computation goes similar. Since~$\bm{w}_1^\top \bm{x}$  for~$i \in \I$ has the uniform distribution on~$[-1,1],$ we have
\begin{equation*}
  \tau
    = \frac{1}{n} \sum_{i \in \I} l\Brac{\vartheta (\bm{w}_1^\top \bm{x}_i - t)}
    \approx \frac{1}{2}\int_{-1}^{1} l\Brac{\vartheta(s-t)} \dd{s}
    = \frac{1}{2} \int_{-1}^{1} \max\{0, 1 + \vartheta(s - t)\}\dd{s}
\end{equation*}
If~$\vartheta \le \tau$, then
\begin{equation*}
  1 + \vartheta(s - t)
    \ge 1 + \vartheta(-1 - t)
    = 1 - \vartheta - 1 + \tau
    = \tau - \beta
    \ge 0.
\end{equation*}
Using this inequality, we can ignore the max operator in the relation for the~$\tau$ above and get
\begin{equation}\label{eq:example1}
  \tau
    =\frac{1}{2} \int_{-1}^{1} (1+\vartheta(s - t))\dd{s}
    = 1 - \vartheta t + \frac{\vartheta}{2}\int_{-1}^{1}s\dd{s}
    = 1 - \vartheta t,
\end{equation}
and thus again~$t = \nicefrac{1-\tau}{\vartheta}$. Finally, since~$\bm{w}_1^\top \bm{x}$  for~$i \in \Ipos$ has the uniform distribution on~$[0,1],$ we have
\begin{equation*}
  L(\bm{w}_1)
    = \frac{1}{\npos} \sum_{i \in \Ipos} l(t - \bm{w}_1^\top \bm{x}_i)
    \approx \int_{0}^{1} l(t-s)\dd{s}
    = \int_{0}^{1} (1 + t - s)\dd{s}
    = 0.5 + t.
\end{equation*}
Results for \PatMatNP can be obtained in a similar way. In the rest of the section we provide proofs for all the theorems from Section~\ref{sec: stability}.

\larget*
\begin{proof}[Proof of Theorem~\ref{thm:large_t} on page~\pageref{thm:large_t}]
  All mentioned formulations use surogate approximation of the false-negative rate as the objective function~$L.$ For the linear classifier, the objective function has the following form
  \begin{equation*}
    L(\bm{w})
      = \frac{1}{\npos}\sum_{i \in \Ipos}l(t - \bm{w}^\top \bm{x}_i)
  \end{equation*}
  Due to~$l(0) = 1$ and the convexity of~$l$ we have~$l(s) \ge 1 + cs$, where~$c$ equals to the derivative of~$l$ at~$0$. Then we have
  \begin{equation*}
    L(\bm{w}) 
      \ge \frac{1}{\npos} \sum_{i \in \Ipos}(1 + c(t-\bm{w}^\top \bm{x}_i))
      = 1 + c\Brac{t - \frac{1}{\npos}\sum_{i \in \Ipos}\bm{w}^\top \bm{x}}
      \ge 1,
  \end{equation*}
  where the last inequality follows from assumption~\eqref{eq:w_zero_nn}. Now we realize that for any formulation from the statement, the corresponding threshold for~$\bm{w}=0$ equals to~$t=0$, and thus~$L(\bm{0})=1$. But then~$L(\bm{0}) \le L(\bm{w})$. The second part of the result follows from the form of thresholds~$t(\bm{w})$.
\end{proof}

\patmatzero*
\begin{proof}[Proof of Theorem~\ref{thm:patmat_zero} on page~\pageref{thm:patmat_zero}]
  Firstly recall thewe use linear model and Notation~\ref{not: scores} and define the following auxilliary variables
  \begin{equation*}
    \begin{aligned}
      s_{\min} & = \min \Set{s_i}{i \in \I}, \qquad
      s_{\max} & = \max \Set{s_i}{i \in \I}, \qquad
      \bar{s} & = \frac{1}{n} \sum_{i \in \I} s_i. \\
    \end{aligned}
  \end{equation*}
  Using the definition of~$\bar{s}$ we get the following relation
  \begin{equation}\label{eq:patmat_zero_aux0}
    \bar{s}
      = \frac{1}{n} \sum_{i \in \I} s_i
      = \frac{1}{n}\sum_{i \in \Ipos} s_i + \frac{1}{n} \sum_{i \in \Ineg} s_i
      < \frac{1}{n} \sum_{i \in \Ipos} s_i + \frac{\nneg}{n\npos} \sum_{i \in \Ipos} s_i
      = \frac{1}{\npos} \sum_{i \in \Ipos} s_i,
  \end{equation}
  where the inequality follows from~\eqref{eq:patmat_zero} and the last equality follows from
  \begin{equation*}
    \frac{1}{n} + \frac{\nneg}{n\npos}
      = \frac{1}{n} \Brac{1 + \frac{\nneg}{\npos}} 
      = \frac{1}{n} \frac{\npos + \nneg}{\npos}
      = \frac{1}{n} \frac{n}{\npos}
      = \frac{1}{\npos}.
  \end{equation*}
  Moreover, since the average of elements of the  vector is smaller or equal to the maximum of elements of the same vector, we get the following relation
  \begin{equation*}
    \bar{s}
      < \frac{1}{\npos} \sum_{i \in \Ipos} s_i
      \le \max \Set{s_i}{i \in \Ipos}
      \le \max \Set{s_i}{i \in \I}
      = s_{\max}
  \end{equation*}
  where the first inequality follows from~\eqref{eq:patmat_zero_aux0}. The lower bound for~$\bar{s}$ can be computed in a similar way. Altogether, we have~$s_{\min} < \bar{s} < s_{\max}$. Then we can define
  \begin{equation*}
    \vartheta_0 = \min\Brac[c]{\frac{\tau}{\bar{s} - s_{\min}}, \; \frac{1-\tau}{s_{\max}-\bar{s}}, \; \tau},
  \end{equation*}
  observe that~$\vartheta_0 > 0$, fix any~$\vartheta \in (0, \vartheta_0)$ and define
  \begin{equation*}
    t = \frac{1 - \tau}{\vartheta} + \bar{s}.
  \end{equation*}
  Then we obtain for any~$i \in \I$
  \begin{equation*}
    1 + \vartheta(s_i - t)
      \ge 1 + \vartheta(s_{\min} - t)
      = 1 + \vartheta s_{\min} - 1 + \tau - \vartheta\bar{s}
      = \tau - \vartheta (\bar{s} - s_{\min}),
  \end{equation*}
  where the first equality follows from the definition of~$t.$ From the definition~$\vartheta_0$ we known the following
  \begin{equation*}
    0 < \vartheta \le \vartheta_0 \le \frac{\tau}{\bar{s} - s_{\min}}.
  \end{equation*}
  Since~$\bar{s} - s_{\min} > 0,$ we get the following inequality
  \begin{equation}\label{eq:patmat_zero_aux1}
    1 + \vartheta(s_i - t)
      = \tau - \vartheta (\bar{s} - s_{\min})
      \ge \tau - \frac{\tau}{\bar{s} - s_{\min}} (\bar{s} - s_{\min})
      = 0
  \end{equation}
  Moreover, combining the definition of the hinge loss function in Notation~\ref{not: surrogates} and the inequality above, we have
  \begin{equation*}
    l\Brac{\vartheta (s_i - t)} = \max\{0, 1 + \vartheta (s_i - t), 0\} = \Brac{1 + \vartheta(s_i - t)}.
  \end{equation*}
  Finally, replacing the hinge loss in the left hand side of~\eqref{eq: aatp quantile surrogate} leads to
  \begin{equation*}
    \begin{aligned}
      \frac{1}{n} \sum_{i \in \I} l\Brac{\vartheta (s_i - t)}
      & = \frac{1}{n}\sum_{i \in \I}\Brac{1 + \vartheta(s_i - t)} \\
      & = 1 - \vartheta t + \frac{\vartheta}{n} \sum_{i \in \I} s_i \\
      & = 1 - \vartheta \Brac{\frac{1 - \tau}{\vartheta} + \bar{s}} + \vartheta \bar{s} \\
      & = \tau,
    \end{aligned}
  \end{equation*}
  where the third equality employs the definition of~$\bar{s}$ and~$t$. But this means that~$t$ is the threshold corresponding to~$\bm{w}$, i.e. it solves~\eqref{eq: aatp quantile surrogate}. Similarly to~\eqref{eq:patmat_zero_aux1} we get
  \begin{equation}\label{eq:patmat_zero_aux2}
    1 + t - s_i
    \ge 1 + t-s_{\max}
    =   1 + \frac{1-\tau}{\vartheta} + \bar{s} - s_{\max}
    \ge \frac{1-\tau}{\vartheta} + \bar{s} - s_{\max}
    \ge 0,
  \end{equation}
  where the last inequality follows from the definition of~$\vartheta_0$. Then for the objective we have
  \begin{equation*}
    \begin{aligned}
      L(\bm{w}) = \frac{1}{\npos}\sum_{i \in \Ipos}l(t-s_i)
      & = \frac{1}{\npos}\sum_{i \in \Ipos}\Brac{1+t-s_i} \\
      & = 1 + t - \frac{1}{\npos}\sum_{i \in \Ipos} s_i \\
      & < 1 + \Brac{\frac{1 - \tau}{\vartheta} + \bar{s}} - \bar{s} \\
      & = 1 + \frac{1-\tau}{\vartheta} \\
      & = L(\bm{0}),\\
    \end{aligned}
  \end{equation*}
  where the second equality follows from~\eqref{eq:patmat_zero_aux2}, the only inequality from~\eqref{eq:patmat_zero_aux0} and the last equality from~\eqref{eq: PatMat threshold 0} and~\eqref{eq: PatMat objective 0}. Thus, we finished the proof for \PatMat. The proof for \PatMatNP can be performed in an identical way by replacing in the definition of~$\bar{s}$ the mean with respect to all samples by the mean with respect to all negative samples.
\end{proof}

\section{Threshold comparison}\label{app:relations}

Whenever the objective contains only false-negatives, a lower threshold~$t$ means a lower objective function. Therefore, a lower threshold is preferred. The two following lemmas compares thresholds defined in Chapter~\ref{chap: framework} in terms of approximation quality.
\begin{lemma}[Thresholds relation~\cite{zhang2018tau}]\label{prop: threholds}
  We always have
  \begin{equation*}
    t_1(\bm{s}) \le t_2(\bm{s}) \le t_3(\bm{s}).
  \end{equation*}
\end{lemma}

\pagebreak

\begin{lemma}\label{lemma:thresholds2}
  Consider the \Grill, \GrillNP, \TopMeanK and \tauFPL formulations and the notation from Notation~\ref{not: scores}. Then we have the following statements:
  \begin{equation*}
    \begin{aligned}
      s_{[\npos\tau]}^+ > s_{[\nneg\tau]}^-
        & \implies \Grill \text{ has larger threshold than }\GrillNP, \\
      \frac{1}{\npos\tau}\sum_{i=1}^{\npos\tau} s_{[i]}^+
      > \frac{1}{\nneg\tau}\sum_{i=1}^{\nneg\tau} s_{[i]}^-
        & \implies \TopMeanK \text{ has larger threshold than }\tauFPL. \\
    \end{aligned}
  \end{equation*}
\end{lemma}

\begin{proof}
  Since~$\bm{s}^+$ and~$\bm{s}^-$ are computed on disjunctive indices, we have
  \begin{equation*}
    s_{[n\tau]} \ge \min\{s_{[\npos\tau]}^+, \; s_{[\nneg\tau]}^-\}.
  \end{equation*}
  Since~$s_{[n\tau]}$ is the threshold for \Grill and~$s_{[\nneg\tau]}^-$ is the threshold for \GrillNP, the first statement follows. The second part can be shown in a similar way.
\end{proof}
  
\noindent Since the goal of the presented formulations is to push~$s^+$ above~$s^-$, we may expect that the conditions in Lemma~\ref{lemma:thresholds2} hold true. 

\section{Computing the threshold for \PatMat}\label{app:threshold}

We show how to efficiently compute the threshold~\eqref{eq: aatp quantile surrogate} for \PatMat with linear model and the hinge surrogate from Notation~\ref{not: surrogates}. Consider function
\begin{equation}\label{eq:defin_h}
  h(t) = \sum_{i \in \I} l\Brac{\vartheta(s_i - t)} - n\tau.
\end{equation}
Then solving~\eqref{eq:update_t} is equivalent to looking for~$\hat{t}$ such that~$h(\hat{t}) = 0$. Function~$h$ is continuous and strictly decreasing (until it hits the global minimum) with~$h(t) \to \infty$ as~$t \to -\infty$ and~$h(t) \to -n\tau$ as~$t \to \infty$. Thus, there is a unique solution to the equation~$h(t) = 0$. For sorted data, the following lemma gives advice on how to solve equation~$h(t) = 0$. 

\begin{lemma}
  Consider vector of scores~$\bm{s}$ and its sorted version~$\bm{s}_{[\cdot]}$ with decreasing elements as defined in Notation~\ref{not: scores}. Define~$\gamma = \nicefrac{1}{\vartheta}$. Then 
  \begin{equation}\label{eq:update_h}
    h(s_{[j]} + \gamma) = h(s_{[j - 1]} + \gamma) + (j - 1) \vartheta(s_{[j - 1]} - s_{[j]})
  \end{equation}
  for all~$i = 2, \; 3, \ldots, n$ with the initial condition~$h(s_{[1]} + \gamma) = -n\tau$.
\end{lemma}
\begin{proof}
Observe first that
\begin{equation*}
  \begin{aligned}
    h(s_{[j]}+\gamma)
      & = \sum_{i \in \I} l\Brac{\vartheta(s_i - (s_{[j]} + \gamma))} - n\tau \\
      & = \sum_{i \in \I} \max\Brac[c]{0,\; 1 + \vartheta\Brac{s_i - s_{[j]} - \frac{1}{\gamma}}} - n\tau \\
      & = \sum_{i = 1}^{j - 1} \vartheta(s_{[i]} - s_{[j]}) - n\tau,
  \end{aligned}
\end{equation*}
where the last equality holds since~$\vartheta > 0$ and~$s_{[i]} - s_{[j]} \le 0$ for all~$i \geq j$
From here, we obtain~$h(s_{[1]} + \gamma) = -n\tau$. Moreover, we have
\begin{equation*}
  \begin{aligned}
    h(s_{[j]} + \gamma)
    & = \sum_{i = 1}^{j - 1} \vartheta(s_{[i]} - s_{[j]}) - n\tau \\
    & = \sum_{i = 1}^{j - 2} \vartheta(s_{[i]} - s_{[j]}) + \vartheta(s_{[j-1]} - s_{[j]}) - n\tau \\
    & = \sum_{i = 1}^{j - 2} \vartheta(s_{[i]} - s_{[j]} \pm s_{[j - 1]}) + \vartheta(s_{[j-1]} - s_{[j]}) - n\tau \\
    & = \sum_{i = 1}^{j - 2} \vartheta(s_{[i]} - s_{[j - 1]}) + \sum_{i = 1}^{j - 2} \vartheta(s_{[j - 1]} - s_{[j]}) + \vartheta(s_{[j - 1]} - s_{[j]}) - n\tau \\
    & = h(s_{[j - 1]} + \gamma) + (j - 1) \vartheta(s_{[j - 1]} - s_{[j]}),
  \end{aligned}
\end{equation*}
which finishes the proof.
\end{proof}

\noindent Thus, to solve~$h(t) = 0$ with the hinge surrogate, we start with~$t_1 = s_{[1]}  + \gamma$ and~$h(t_1) = -n\tau$. Then we start decreasing~$t$ according to~\eqref{eq:update_h} until we find some~$t_i = s_{[i]} + \gamma$ such that~$h(t_i) > 0$. The desired~$t$ then lies between~$t_i$ and~$t_{i-1}$. Since~$h$ is a piecewise linear function with
\begin{equation*}
  h(t) = h(t_{i-1}) + \frac{t - t_{i-1}}{t_{i} - t_{i-1}}\Brac{h(t_{i}) - h(t_{i-1})}
\end{equation*}
for~$t \in [t_{i-1}, \; t_{i}]$, the precise value of~$\hat{t}$ can be computed by a simple interpolation
\begin{equation*}
  \hat{t}
    = t_{i-1} - h(t_{i-1})\frac{t_{i} - t_{i-1}}{h(t_{i}) - h(t_{i-1})}
    = t_{i-1} - h(t_{i-1})\frac{t_{i} - t_{i-1}}{-(i-1)\vartheta(t_{i} - t_{i-1})}
    = t_{i-1} + \frac{h(t_{i-1})}{\vartheta(i-1)}.
\end{equation*}

\section{Convergence of stochastic gradient descent}

The proof is divided into three parts. In Section~\ref{app:sgd1}, we prove a general statement for convergence of stochastic gradient descent with a convex objective. In Section~\ref{app:sgd2} we apply it to Theorem~\ref{thm:sgd}. The proof is based on auxiliary results from Section~\ref{app:sgd3}.

\subsection{General result}\label{app:sgd1}

Consider a differentiable objective function~$L$ and the optimization method
\begin{equation}\label{eq:update}
  \bm{w}^{k+1} = \bm{w}^k - \alpha^k g(\bm{w}^k),
\end{equation}
where~$\alpha^k > 0$ is a stepsize and~$g(\bm{w}^k)$ is an approximation of the gradient~$\nabla L(\bm{w}^k)$. Assume the following:
\begin{enumerate}[label={(A\arabic*)}]
  \item \label{ass_convex}~$L$ is differentiable, convex and attains a global minimum;
  \item \label{ass_gbound}~$\norm{g(\bm{w}^k)}\le B$ for all~$k$;
  \item \label{ass_alpha1} the stepsize is non-increasing and satisfies~$\sum_{k=0}^\infty \alpha^k = \infty$;
  \item \label{ass_alpha2} the stepsize satisfies~$\sum_{k=0}^\infty (\alpha^k)^2<\infty$;
  \item \label{ass_alpha3} the stepsize satisfies~$\sum_{k=0}^\infty \norm{\alpha^{k+1}-  \alpha^k}<\infty$.
\end{enumerate}
Assumptions~\ref{ass_alpha1}-\ref{ass_alpha3} are satisfied for example for~$\alpha^k = \nicefrac{\alpha^0}{k+1}$. We start with the general result.

\begin{theorem}\label{thm:convergence}
  Assume that~\ref{ass_convex}-\ref{ass_alpha2} is satisfied. If there exists some~$C$ such that for some global minimum of~$\bm{w}^*$ of~$L$ we have
  \begin{equation}\label{eq:nec_cond}
    \sum_{k=0}^\infty \alpha^k \inner{g(\bm{w}^k) - \nabla L(\bm{w}^k)}{\bm{w}^* - \bm{w}^k} \le C,
  \end{equation}
  then the sequence~$\{\bm{w}^k\}$ generated by~\eqref{eq:update} is bounded and~$L(\bm{w}^k) \to L(\bm{w}^*)$. Thus, all its convergent subsequences converge to some global minimum of~$L$.
\end{theorem}
\begin{proof}
  Note first that the convexity of~$L$ from~\ref{ass_convex} implies
  \begin{equation}\label{eq:convex_estimate}
    \inner{\nabla L(\bm{w}^k)}{\bm{w}^* - \bm{w}^k} \le L(\bm{w}^*) - L(\bm{w}^k).
  \end{equation}
  Then we have
  \begin{equation*}
    \begin{aligned}
      \norm{\bm{w}^{k+1} - \bm{w}^*}^2
        = \; & \norm{\bm{w}^k - \alpha^k g(\bm{w}^k) - \bm{w}^*}^2 \\
        = \; &\norm{\bm{w}^k - \bm{w}^*}^2 + 2\alpha^k\inner{g(\bm{w}^k)}{\bm{w}^* - \bm{w}^k} + (\alpha^k)^2 \norm{g(\bm{w}^k)}^2 \\
        \le \; &\norm{\bm{w}^k - \bm{w}^*}^2 + 2\alpha^k\inner{g(\bm{w}^k) \pm \nabla L(\bm{w}^k)}{\bm{w}^* - \bm{w}^k} + (\alpha^k)^2 B^2\\
        \le \; & \norm{\bm{w}^k - \bm{w}^*}^2 + 2 \alpha^k \inner{g(\bm{w}^k) - \nabla L(\bm{w}^k)}{\bm{w}^* - \bm{w}^k} \\
        & + 2 \alpha^k \Brac{L(\bm{w}^*) - L(\bm{w}^k)} + (\alpha^k)^2 B^2,
    \end{aligned}
  \end{equation*}
  where the first inequality follows from assumption~\ref{ass_gbound} and the second on from the properties of inner product and~\eqref{eq:convex_estimate}. Summing this expression for all~$k$ and using~\eqref{eq:nec_cond} leads to
  \begin{equation*}
    \limsup_{k \rightarrow \infty} \; \norm{\bm{w}^k - \bm{w}^*}^2
      \le \norm{\bm{w}^0 - \bm{w}^*}^2 + 2C + 2 \sum_{k=0}^\infty \alpha^k (L(\bm{w}^*) - L(\bm{w}^k)) + \sum_{k=0}^ \infty (\alpha^k)^2 B^2.
\end{equation*}
  Using assumption~\ref{ass_alpha2} results in the existence of some~$\hat{C}$ such that
  \begin{equation}\label{eq:general_bound}
  \limsup_{k \rightarrow \infty} \;\norm{\bm{w}^k - \bm{w}^*}^2 + 2\sum_{k=0}^\infty \alpha^k \Brac{L(\bm{w}^k) - L(\bm{w}^*)} \le 2 \hat{C}.
  \end{equation}
  Since~$\alpha^k > 0$ and~$L(\bm{w}^k) \ge L(\bm{w}^*)$ as~$\bm{w}^*$ is a global minimum of~$L$, we infer that sequence~$\{\bm{w}^k\}$ is bounded and~\eqref{eq:general_bound} implies
  \begin{equation*}
    \sum_{k=0}^\infty \alpha^k \Brac{L(\bm{w}^k) - L(\bm{w}^*)} \le \hat{C}.
  \end{equation*}
  Since~$L(\bm{w}^k) - L(\bm{w}^*) \ge 0$, due to assumption~\ref{ass_alpha1} we obtain
  \begin{equation*}
    \lim_{k \to \infty} L(\bm{w}^k) = L(\bm{w}^*),
  \end{equation*}
  which implies the theorem statement.
\end{proof}

\subsection{Proof of Theorem~\ref{thm:sgd}}\label{app:sgd2}

For the proof, we will consider a general surrogate which satisfies:
\begin{enumerate}[label={(S\arabic*)}]
  \item \label{surr_basic1} $l(s)\ge 0$ for all~$s\in\R$, $l(0)=1$ and~$l(s)\to 0$ as~$s\to-\infty$;
  \item \label{surr_basic2} $l$ is convex and strictly increasing function on~$(s_0,\infty)$, where~$s_0:=\sup\{s \mid l(s)=0\}$;
  \item \label{surr_ratio} $\nicefrac{l'}{l}$ is a decreasing function on~$(s_0,\infty)$;
  \item \label{surr_der1} $l'$ is a bounded function;
  \item \label{surr_der2} $l'$ is a Lipschitz continuous function with Lipschitz constant~$D$.
\end{enumerate}
All these reguirements are satisfied for the surrogate logistic or by the Huber loss, which is the hinge surrogate which is smoothened on an~$\eps$-neighborhood of zero.

\sgd*
\begin{proof}[Proof of Theorem~\ref{thm:sgd} on page~\pageref{thm:sgd}]
  We intend to apply Theorem~\ref{thm:convergence} and thus, we need to verify its assumptions. Assumption~\ref{ass_convex} is satisfied as~$L$ is convex due to Theorem~\ref{thm:convex}. Assumption~\ref{ass_gbound} follows directly from Lemma~\ref{lemma:bound_g}. Assumptions~\ref{ass_alpha1},\ref{ass_alpha2} and~\ref{ass_alpha3} are imposed directly in the statement of this theorem. It remains to verify~\eqref{eq:nec_cond}.

  For simplicity, we will do so only for~$\vartheta = 1$ and for~$2$ minibatches of the same size. However, the proof would be identical for other values. This implies that there are some~$\Imb^k$ and~$\Imb^{k+1}$ which are pairwise disjoint, they cover all samples and that~$\Imb^k = \Imb^{k+2}$ for all~$k$. The assumptions imply that the number of positive samples in each minibatch equal to~$\nmbpos^k = \nicefrac{\npos}{2}$, where~$\npos$ is the total number of positive samples.

  First we estimate the difference between~$s_i^k$ defined in~\eqref{eq:defin_z} and~$\bm{x}_i^\top \bm{w}^k$. For any~$i \in \Imb^k$ we have
  \begin{equation*}
    s_i^k = \bm{x}_i^\top \bm{w}^k
  \end{equation*}
  and since we have two disjoint minibatches, due to the construction~\eqref{eq:defin_z} we get
  \begin{equation}\label{eq:sgd_estimate_z1}
    \begin{aligned}
      s_i^{k-1}
          = s_i^{k-2}
        & = \bm{x}_i^\top \bm{w}^{k-2} \\
        & = \bm{x}_i^\top \Brac{\bm{w}^k + \alpha^{k-2}g(\bm{w}^{k-2}) + \alpha^{k-1} g(\bm{w}^{k-1})} \\
        & = \bm{x}_i^\top \bm{w}^k + \alpha^{k-2}\bm{x}_i^\top g(\bm{w}^{k-2}) + \alpha^{k-1}\bm{x}_i^\top g(\bm{w}^{k-1}).
    \end{aligned}
  \end{equation}
  Similarly, due to the construction~\eqref{eq:defin_z}, for~$i \notin \Imb^k$ we have
  \begin{equation}\label{eq:sgd_estimate_z2}
    s_i^k
    = s_i^{k-1}
    = \bm{x}_i^\top \bm{w}^{k-1}
    = \bm{x}_i^\top (\bm{w}^k+\alpha^{k-1}g(\bm{w}^{k-1}))
    = \bm{x}_i^\top \bm{w}^k + \alpha^{k-1}\bm{x}_i^\top g(\bm{w}^{k-1}).
  \end{equation}
  Recall that we already verified~\ref{ass_convex}-\ref{ass_alpha3}. Combining~\ref{ass_gbound} with~\eqref{eq:sgd_estimate_z1} and~\eqref{eq:sgd_estimate_z2} yields the existence of some~$C_2$ such that for all~$i \in \I$ we have
  \begin{equation}\label{eq:estimate_diff_z}
    \begin{aligned}
      \norm{s_i^k - \bm{x}_i^\top \bm{w}^k} &\le C_2\alpha^{k-1}, \\
      \norm{s_i^{k-1} - \bm{x}_i^\top \bm{w}^k} &\le C_2\Brac{\alpha^{k-1}+\alpha^{k-2}}. \\
    \end{aligned}
  \end{equation}
  This also immediately implies
  \begin{equation}\label{eq:estimate_diff_t}
    \begin{aligned}
      \norm{t^k - t(\bm{w}^k)}     & \le C_2\alpha^{k-1}, \\
      \norm{t^{k-1} - t(\bm{w}^k)} & \le C_2\Brac{\alpha^{k-1}+\alpha^{k-2}}. \\
    \end{aligned}
  \end{equation}
  Since~$l'$ is Lipschitz continuous with Lipschitz constant~$D$ according to~\ref{surr_der2}, due to~\eqref{eq:estimate_diff_z} and~\eqref{eq:estimate_diff_t} we get
  \begin{equation}\label{eq:sgd_lipschitz1}
    \begin{aligned}
      \norm{l'(t^k-s_i^k) - l'(t(\bm{w}^k)-\bm{x}_i^\top \bm{w}^k)}
        & \le D \norm{t^k-s_i^k - t(\bm{w}^k)+ \bm{x}_i^\top \bm{w}^k}
        \le  2C_2 D \alpha^{k-1}.
    \end{aligned}
  \end{equation}
  In an identical way we can show
  \begin{equation}\label{eq:sgd_lipschitz2}
    \begin{aligned}
      \norm{l'(t^{k-1}-s_i^{k-1}) - l'(t(\bm{w}^k)-\bm{x}_i^\top \bm{w}^k)}
        & \le 2C_2D\Brac{\alpha^{k-1}+\alpha^{k-2}}, \\
      \norm{l'(s_i^k-t^k) - l'(\bm{x}_i^\top \bm{w}^k-t(\bm{w}^k))}
        & \le 2C_2D\alpha^{k-1}, \\
      \norm{l'(s_i^{k-1}-t^{k-1}) - l'(\bm{x}_i^\top \bm{w}^k-t(\bm{w}^k))}
        & \le 2C_2D\Brac{\alpha^{k-1}+\alpha^{k-2}}.
    \end{aligned}
  \end{equation}
  Now we need to estimate the distance between~$\nabla t(\bm{w}^k)$ and~$\nabla t^k$. From~\eqref{eq:update_nablat} and~\eqref{eq:update_a}, we have
  \begin{equation*}
    \nabla t^k
      = \frac{\sum_{i \in \Imb^k} l'(s_i^k - t^k)\bm{x}_i + \sum_{i \in \Imb^{k-1}} l'(s_i^{k-1} - t^{k-1}) \bm{x}_i}{\sum_{i \in \I} l'(s_i^k - t^k)}.
  \end{equation*}
  Moreover, using Theorem~\ref{thm:derivative} and the fact that we have only two minibatches and therefore for any~$k$ we have~$\I = \Imb^k \cup \Imb^{k-1}$, we get
  \begin{equation*}
    \nabla t(\bm{w}^k)
      = \frac{\sum_{i \in \Imb^k} l'(\bm{x}_i^\top \bm{w}^k - t(\bm{w}^k))\bm{x}_i + \sum_{i \in \Imb^{k-1}} l'(\bm{x}_i^\top \bm{w}^k - t(\bm{w}^k))\bm{x}_i}{\sum_{i \in \I} l'(\bm{x}_i^\top \bm{w}^k - t(\bm{w}^k))}.
  \end{equation*}
  From Lemma~\ref{lemma:bound_zero} we deduce that the denominators in the relations above are bounded away from zero uniformly in~$k$. Assumption~\ref{ass_alpha2} implies ~$\alpha^k \to 0$. This allows us to use Lemma~\ref{lemma:ratio} which together with~\eqref{eq:sgd_lipschitz2} implies that there is some~$C_3$ such that for all sufficiently large~$k$ we have
  \begin{equation}\label{eq:sgd_nablat_diff}
    \norm{\nabla t^k - \nabla t(\bm{w}^k)} \le C_3\Brac{\alpha^{k-1} + \alpha^{k-2}}.
  \end{equation}
  Using the assumptions above, we can simplify the terms for~$g(\bm{w}^k)$ and~$\nabla L(\bm{w}^k)$ to
  \begin{equation*}
    \begin{aligned}
      g(\bm{w}^k)
        & = \frac{2}{\npos} \sum_{i \in \Imbpos^k} l'(t^k - s_i^k)(\nabla t^k - \bm{x}_i), \\
      g(\bm{w}^{k+1})
        & = \frac{2}{\npos} \sum_{i \in \Imbpos^{k+1}} l'(t^{k+1}-s_i^{k+1})(\nabla t^{k+1} - \bm{x}_i), \\
      \nabla L(\bm{w}^k)
        & = \frac{1}{\npos} \sum_{i \in \Ipos} l'(t(\bm{w}^k) - \bm{x}_i^\top \bm{w}^k)(\nabla t(\bm{w}^k) - \bm{x}_i), \\
      \nabla L(\bm{w}^{k+1})
        & = \frac{1}{\npos} \sum_{i \in \Ipos} l'(t(\bm{w}^{k+1}) - \bm{x}_i^\top \bm{w}^{k+1})(\nabla t(\bm{w}^{k+1}) - \bm{x}_i).
    \end{aligned}
  \end{equation*}
  Due to the assumptions, we have~$\Ipos = \Imbpos^k \cup \Imbpos^{k+1}$ and~$\emptyset = \Imbpos^k \cap \Imbpos^{k+1}$, which allows us to write
  \begin{subequations}\label{eq:sgd_sum}
    \begin{align}
    \label{eq:sgd_sum1}
    \npos & \Brac{g(\bm{w}^k) + g(\bm{w}^{k+1}) - \nabla f(\bm{w}^k)-\nabla f(\bm{w}^{k+1})}\\
    \label{eq:sgd_sum2}
    & = \sum_{i \in \Imbpos^k} l'(t^k - s_i^k)(\nabla t^k - \bm{x}_i) - \sum_{i \in \Imbpos^k} l'(t(\bm{w}^k) - \bm{x}_i^\top \bm{w}^k)(\nabla t(\bm{w}^k) - \bm{x}_i) \\
    \label{eq:sgd_sum3}
    & + \sum_{i \in \Imbpos^k} l'(t^k - s_i^k)(\nabla t^k - \bm{x}_i) - \sum_{i \in \Imbpos^k} l'(t(\bm{w}^{k+1}) - \bm{x}_i^\top \bm{w}^{k+1})(\nabla t(\bm{w}^{k+1}) - \bm{x}_i)\\
    \label{eq:sgd_sum4}
    & + \sum_{i \in \Imbpos^{k+1}} l'(t^{k+1} - s_i^{k+1})(\nabla t^{k+1} - \bm{x}_i) - \sum_{i\in \Imbpos^{k+1}}l'(t(\bm{w}^k) - \bm{x}_i^\top \bm{w}^k)(\nabla t(\bm{w}^k) - \bm{x}_i) \\
    \label{eq:sgd_sum5}
    & + \sum_{i \in \Imbpos^{k+1}} l'(t^{k+1} - s_i^{k+1})(\nabla t^{k+1} - \bm{x}_i)  - \sum_{i \in \Imbpos^{k+1}} l'(t(\bm{w}^{k+1}) - \bm{x}_i^\top \bm{w}^{k+1})(\nabla t(\bm{w}^{k+1}) - \bm{x}_i).
    \end{align}
  \end{subequations}
  Then relations~\eqref{eq:sgd_nablat_diff} and~\eqref{eq:sgd_lipschitz1} applied to Lemma~\ref{lemma:product} imply
  \begin{multline*}
    \norm{\sum_{i \in \Imbpos^k} l'(t^k - s_i^k)(\nabla t^k - \bm{x}_i) - \sum_{i \in \Imbpos^k} l'(t(\bm{w}^k) - \bm{x}_i^\top \bm{w}^k)(\nabla t(\bm{w}^k) - \bm{x}_i)}\\
      \le C_4 \Brac{\alpha^{k-1} + \alpha^{k-2}}
  \end{multline*}
  for some~$C_4$, which gives a bound for~\eqref{eq:sgd_sum2}. Bound for~\eqref{eq:sgd_sum5} is obtained by increasing~$k$ by one. Bounds for~\eqref{eq:sgd_sum3} and~\eqref{eq:sgd_sum4} can be find similarly using~\eqref{eq:sgd_lipschitz2}. Altogether, we showed
  \begin{equation}\label{eq:nec_cond3}
    \norm{g(\bm{w}^k) + g(\bm{w}^{k+1}) - \nabla L(\bm{w}^k) - \nabla L(\bm{w}^{k+1})}
      \le C_1(\alpha^{k-2} + \alpha^{k-1} + \alpha^{k} + \alpha^{k+1})
  \end{equation}
  for some~$C_1$. We now estimate
  \begin{equation}\label{eq:proof_est1}
    \begin{aligned}
      \alpha^k
      & \inner{ g(\bm{w}^{k})-\nabla L(\bm{w}^{k})}{\bm{w}^*-\bm{w}^{k}} + \alpha^{k+1}\inner{ g(\bm{w}^{k+1})-\nabla L(\bm{w}^{k+1})}{\bm{w}^*-\bm{w}^{k+1}} \\
      & = \inner{ g(\bm{w}^{k})-\nabla L(\bm{w}^{k})}{\alpha^k(\bm{w}^*-\bm{w}^{k})}
        + \inner{ g(\bm{w}^{k+1})-\nabla L(\bm{w}^{k+1})}{\alpha^{k+1}(\bm{w}^*-\bm{w}^{k+1})} \\
      & = \inner{ g(\bm{w}^{k})-\nabla L(\bm{w}^{k}) + g(\bm{w}^{k+1})-\nabla L(\bm{w}^{k+1})}{\alpha^k(\bm{w}^*-\bm{w}^{k})} \\
      & + \inner{ g(\bm{w}^{k+1})-\nabla L(\bm{w}^{k+1})}{\alpha^{k+1}(\bm{w}^*-\bm{w}^{k+1})-\alpha^k(\bm{w}^*-\bm{w}^{k})}.
    \end{aligned}
  \end{equation}
  To estimate the second part of the right hand side of~\eqref{eq:proof_est1}, we make use of Lemma~\ref{lemma:bound_g} to obtain the existence of some~$C_5$ such that
  \begin{equation}\label{eq:proof_est2}
    \begin{aligned}
    & \inner{ g(\bm{w}^{k+1})
    -\nabla L(\bm{w}^{k+1})}{\alpha^{k+1}(\bm{w}^*-\bm{w}^{k+1})-\alpha^k(\bm{w}^*-\bm{w}^{k})} \\
    & \le 2B\norm{\alpha^{k+1}(\bm{w}^*-\bm{w}^{k+1})-\alpha^k(\bm{w}^*-\bm{w}^{k})} \\
    & = 2B\norm{\alpha^{k+1}(\bm{w}^*-\bm{w}^k+\alpha^kg(\bm{w}^k))-\alpha^k(\bm{w}^*-\bm{w}^{k})} \\
    & = 2B\norm{(\alpha^{k+1}-\alpha^k)\bm{w}^* + (\alpha^k-\alpha^{k+1})\bm{w}^k +\alpha^k\alpha^{k+1} g(\bm{w}^k)} \\
    & \le C_5 \norm{\alpha^{k+1}-\alpha^k} + C_5(\alpha^k)^2 + C_5(\alpha^{k+1})^2.
    \end{aligned}
  \end{equation}
  In the last inequality we used the inequality~$2ab\le a^2+b^2$. To estimate the first part of the right hand side of~\eqref{eq:proof_est1}, we can apply~\eqref{eq:nec_cond3} together with the boundedness of~$\{\bm{w}^k\}$ to obtain the existence of some~$C_6$ such that
  \begin{multline}\label{eq:proof_est3}
    \inner{ g(\bm{w}^{k}) -\nabla L(\bm{w}^{k}) + g(\bm{w}^{k+1})-\nabla L(\bm{w}^{k+1})}{\alpha^k(\bm{w}^* - \bm{w}^{k})} \\
      \le C_6(\alpha^{k-2})^2 + C_6(\alpha^{k-1})^2 + C_6(\alpha^{k})^2 + C_6(\alpha^{k+1})^2.
  \end{multline}
  Plugging~\eqref{eq:proof_est2} and~\eqref{eq:proof_est3} into~\eqref{eq:proof_est1} and summing the terms yields~\eqref{eq:nec_cond}. Then the assumptions of Theorem~\ref{thm:convergence} are verified and the theorem statement follows.
\end{proof}

\subsection{Auxiliary results}\label{app:sgd3}

\begin{lemma}\label{lemma:bound_zero}
  Let~$l$ satisfy~\ref{surr_basic1}-\ref{surr_ratio}. Then there exists some~$\hat{C} > 0$ such that for all~$k$ we have
  \begin{equation*}
    \begin{aligned}
      \hat{C} \le & \sum_{i \in \I} l'(s_i^k - t^k), \\
      \hat{C} \le & \sum_{i \in \I} l'(\bm{x}_i^\top \bm{w}^k - t(\bm{w}^k)).
    \end{aligned}
  \end{equation*}
\end{lemma}
\begin{proof}
  First, we will find an upper bound of~$s_i^k-t^k$. Fix any index~$i_0$. Since~$l$ is nonnegative due to~\ref{surr_basic1}, equation~\eqref{eq:update_t} implies
  \begin{equation*}
    n\tau = \sum_{i \in \I} l(s_i^k - t^k) \ge l(s_{i_0}^k - t^k).
  \end{equation*}
  Moreover, as~$l$ is a strictly increasing function due to~\ref{surr_basic2} and~$n\tau>0$, this means 
  \begin{equation}\label{eq:sigma_bound}
    l^{-1}(n\tau) \ge s_{i_0}^k-t^k.
  \end{equation}
  Since~$i_0$ was an arbitrary index, it holds true for all indices. Then~\ref{surr_ratio} which leads to a further estimate
  \begin{equation*}
    \begin{aligned}
    \sum_{i \in \I} l'(s_i^k - t^k)
      & = \sum_{i\in \I} l(s_i^k-t^k) \frac{l'(s_i^k-t^k)}{l(s_i^k-t^k)} \\
      & \ge \sum_{i \in \I} l(s_i^k - t^k) \frac{l'(l^{-1}(n\tau))}{l(l^{-1}(n\tau))} \\
      & = n\tau \frac{l'(l^{-1}(n\tau))}{l(l^{-1}(n\tau))} \\
      & = l'(l^{-1}(n\tau)),
    \end{aligned}
  \end{equation*}
  where the inequality follows from~\eqref{eq:sigma_bound} and the following equality from~\eqref{eq:update_t}. Due to~\ref{surr_basic2} we obtain that~$l'(l^{-1}(n\tau))$ is a positive number, which finishes the proof of the first part. The second part can be obtained in an identical way.
\end{proof}

\pagebreak

\begin{lemma}\label{lemma:bound_g}
  Let~$l$ satisfy~\ref{surr_basic1}-\ref{surr_der1}. Then there exists some~$B$ such that for all~$k$ we have
  \begin{equation*}
    \begin{aligned}
      \norm{\nabla L(\bm{w}^k)} & \le B, \\
      \norm{g(\bm{w}^k)} & \le B.
    \end{aligned}
  \end{equation*}
\end{lemma}
\begin{proof}
  Due to~\ref{surr_der1} the derivative~$l'$ is bounded by some~$\hat{B}$. Then Theorem~\ref{thm:derivative} and Lemma~\ref{lemma:bound_zero} imply
  \begin{equation*}
    \norm{\nabla t(\bm{w}^k)}
      \le \frac{\hat{B} \sum_{i \in \I} \norm{\bm{x}_i}}{\sum_{i \in \I} l'(\bm{x}_i^\top \bm{w} - t(\bm{w}))}
      \le \frac{\hat{B}}{\hat{C}} \sum_{i\in \I} \norm{\bm{x}_i},
  \end{equation*}
  which is independent of~$k$. Then~\eqref{eq:derivatives} and again the boundedness of~$l'$ imply the existence of some~$B$ such that~$\norm{\nabla L(\bm{w}^k)} \le B$ for all~$k$. The proof for~$g(\bm{w}^k)$ can be performed identically.
\end{proof}

\begin{lemma}\label{lemma:ratio}
  Consider uniformly bounded positive sequences~$c_1^k,$~$c_2^k,$~$d_1^k,$~$d_2^k,$~$\alpha^k$ and positive constants~$C_1$,~$C_2$ such that for all~$k$ we have
  \begin{equation*}
    \begin{aligned}
      \norm{c_1^k-c_2^k} & \le C_1\alpha^k, \quad &
      \norm{d_1^k-d_2^k} & \le C_1\alpha^k, \quad &
      d_1^k & \ge C_2, \quad &
      d_2^k & \ge C_2.
    \end{aligned}
  \end{equation*}
  If~$\alpha^k \to 0$, then there exists a constant~$C_3$ such that for all sufficiently large~$k$ we have
  \begin{equation*}
    \norm{\frac{c_1^k}{d_1^k} - \frac{c_2^k}{d_2^k}} \le C_3\alpha^k.
  \end{equation*}
\end{lemma}

\begin{proof}
  Since~$d_1^k$ and~$d_2^k$ are bounded away from zero and since~$\alpha^k \to 0$, we have
  \begin{equation*}
    \norm{\frac{c_1^k}{d_1^k} - \frac{c_2^k}{d_2^k}}
      \le \max\Brac[c]{\frac{c_1^k}{d_1^k} - \frac{c_1^k+C_1\alpha^k}{d_1^k-C_1\alpha^k}, \; \frac{c_1^k}{d_1^k} - \frac{c_1^k-C_1\alpha^k}{d_1^k+C_1\alpha^k}}.
  \end{equation*}
  The first term can be estimated as
  \begin{equation*}
    \norm{\frac{c_1^k}{d_1^k} - \frac{c_1^k+C_1\alpha^k}{d_1^k-C_1\alpha^k}}
    = \norm{\frac{(c_1^k+d_1^k)C_1\alpha^k}{d_1^k(d_1^k-C_1\alpha^k)}}
    \le \frac{(c_1^k+d_1^k)C_1\alpha^k}{C_2|d_1^k-C_1\alpha^k|}.
  \end{equation*}
  Since~$\alpha^k\to 0$ by assumption, for large~$k$ we have~$\norm{d_1^k-C_1\alpha^k}\ge \frac{1}{2}C_2$. Since the sequences are uniformly bounded, the statement follows.
\end{proof}

\pagebreak

\begin{lemma}\label{lemma:product}
  Consider scalars~$a_i,$~$c_i$ and vectors~$b_i,$~$d_i.$ If there is some~$\hat{C}$ such that~$\norm{a_i} \le \hat{C}$ and~$\norm{d_i} \le \hat{C}$, then
  \begin{equation*}
    \norm{\sum_{i=1}^n a_ib_i - \sum_{i=1}^n c_id_i}
      \le \hat{C}\sum_{i=1}^n \Brac{\norm{a_i-c_i} + \norm{b_i-d_i}}.
  \end{equation*}
\end{lemma}
\begin{proof}
  It is simple to verify
  \begin{equation*}
    \norm{\sum_{i=1}^n a_ib_i - \sum_{i=1}^n c_id_i} \le \sum_{i=1}^n \norm{d_i}\norm{a_i-c_i} + \sum_{i=1}^n \norm{a_i}\norm{b_i-d_i},
  \end{equation*}
  from which the statement follows.
\end{proof}
\chapter{Appendix for Chapter~\ref{chap: dual}}

In this chapter we provide proofs and additional results for the Chapter~\ref{chap: dual}. In the first part, we introduce concept of conjugate functions. In the second part, we derive dual formulation to the formulations from Table~\ref{tab: summary formulations}. Finally, the last part focuses on how to efficiently solve these dual formulations.

\section{Convex Conjugate}
\begin{definition}[Convex conjugate~\cite{boyd2004convex}]\label{def: conjugate}
  Let~$l \colon \R^n \to \R.$ The function~$l^{\star} \colon \R^n \to \R,$ defined as
  \begin{equation*}
    l^{\star} (\bm{y})
      =  \sup_{\bm{x} \in \domain l} \{\bm{y}^{\top}\bm{x} - l(\bm{x})\}
      = -\inf_{\bm{x} \in \domain l} \{l(\bm{x}) - \bm{y}^{\top}\bm{x}\}.
  \end{equation*}
  is called conjugate function of~$l.$
\end{definition}
Recall the hinge loss and quadratic hinge loss function defined in Notation~\ref{not: surrogates} as follows
\begin{equation*}
  \begin{aligned}
    l_{\text{hinge}}(s) & = \max\Brac[c]{0, 1 + s}, \\
    l_{\text{quadratic}}(s) & = \Brac{\max\Brac[c]{0, 1 + s}}^2.\\
  \end{aligned}
\end{equation*}
The conjugate for the hinge loss can be found in~\cite{shnlev2014accelerated} and has the following form
\begin{equation}\label{eq: conjugate hinge}
  l_{\text{hinge}}^{\star}(y) =
  \begin{cases}
    -y & \text{if } y \in [0, 1], \\
    \infty & \text{otherwise.}
  \end{cases}  
\end{equation}
Similarly, the conjugate for the quadratic hinge is defuined in~\cite{kanamori2013conjugate} as
\begin{equation}\label{eq: conjugate quadratic hinge}
  l_{\text{quadratic}}^{\star}(y) =
  \begin{cases}
    \frac{y^2}{4} - y & \text{if } y \geq 0, \\
    \infty & \text{otherwise.}
  \end{cases}
\end{equation}

\section{Dual formulations}

In this section, we show how to derive the dual formulations to the formulations from Table Table~\ref{tab: summary formulations}. 

\subsection{Ranking Problems}

In this section, we derive the dual formulation of \TopPushK. Table~\ref{tab: summary formulations} shows, that \TopPush is a special is a special case of the \TopPushK for~$K = 1.$ Therefore, it is sufficient to show the dual form only for \TopPushK. Firstly, we introduce the alternative form of the \TopPushK.

\begin{lemma}[\TopPushK alternative formulation.]\label{lem: TopPushK primal alternative}
  The problem~\eqref{eq:TopPushK primal} can be equivalently written as follows
  \begin{maxi}{\bm{w}, t, \bm{y}, \bm{z}}{
    \frac{1}{2} \norm{\bm{w}}_{2}^{2}+ C \sum_{i = 1}^{\npos} l(y_i)
    }{\label{eq: TopPushK primal alternative}}{}
    \addConstraint{y_i}{= t + \frac{1}{K} \sum_{j = 1}^{\nneg} z_j - \bm{w}^\top \bm{x}^+_i, \quad}{i = 1, \; 2, \ldots, \; \npos.}
    \addConstraint{z_j}{\geq \bm{w}^\top \bm{x}^-_j - t,}{j = 1, \; 2, \ldots, \; \nneg}
    \addConstraint{z_j}{\geq 0,}{j = 1, \; 2, \ldots, \; \nneg}
  \end{maxi}
\end{lemma}
\begin{proof}
  Firstly, we rewrite the formula for the decision threshold from~\eqref{eq:TopPushK primal}using the Lemma~1 from~\cite{ogryczak2003minimizing}
  \begin{equation*}
    \sum_{j = 1}^{K} s^{-}_{[j]} = \min_{t} \Brac[c]{Kt + \sum_{j = 1}^{\nneg} \max\{0, \; s^-_j - t\}}.
  \end{equation*}
  Substituing this formula into the objective function from~\eqref{eq:TopPushK primal}, we get
  \begin{align*}
    \sum_{i = 1}^{\npos} l\Brac{\frac{1}{K}\sum_{j = 1}^{K} s^{-}_{[j]} - s^+_{i}}
      & = \sum_{i = 1}^{\npos} l\Brac{ \frac{1}{K} \min_{t} \Brac[c]{Kt + \sum_{j = 1}^{\nneg} \max\Brac[c]{0, \; s^-_j - t}} - s^+_{i}} \\
      & = \min_{t} \; \sum_{i = 1}^{\npos} l\Brac{t + \frac{1}{K} \sum_{j = 1}^{\nneg} \max\Brac[c]{0, \; s^-_j - t} - s^+_{i}}.
  \end{align*}
  where the last equality follows from the fact, that the surrogate function is~$l$ is non-decreasing. The max operator can be replaced using auxiliary variable~$\bm{z} \in \R^{\nneg}$ which for all~$j = 1, \; 2, \ldots, \; \nneg$ fullfills~$z _j \geq s^-_j - t$ and at the same time~$z _j \geq 0.$ Moreover, we introduce new variable~$\bm{y} \in \R^{\nneg}$ defined for all~$i = 1, \; 2, \ldots, \; \npos$ as
  \begin{equation*}
    y_i = t + \frac{1}{K} \sum_{j = 1}^{\nneg} z_j - s^+_i.
  \end{equation*}
  Altogether, we get the formulation~\eqref{eq: TopPushK primal alternative}, where we use the fact, that we have linear model and therefore~$s^-_j = \bm{w}^\top \bm{x}^-_j$ for all~$j = 1, \; 2, \ldots, \; \nneg$ and ~$s^+_i = \bm{w}^\top \bm{x}^+_i$ for all~$i = 1, \; 2, \ldots, \; \npos$.
\end{proof}

\pagebreak

\begin{theorem}[Dual formulation of \TopPush and \TopPushK ]\label{thm: TopPushK dual}
  Consider \TopPushK formulation~\eqref{eq: toppush surrogate} with linear model, surrogate function~$l$ and Notation~\ref{not: kernel matrix}. Then the corresponding dual problem has the following form
  \begin{maxi!}{\bm{\alpha}, \bm{\beta}}{
    - \frac{1}{2} \vecab^\top \Kneg \vecab - C \sum_{i = 1}^{\npos} l^{\star}\Brac{\frac{\alpha_i}{C}}
    }{\label{eq: TopPushK dual}}{\label{eq: TopPushK dual L}}
    \addConstraint{\sum_{i = 1}^{\npos} \alpha_i}{= \sum_{j = 1}^{\nneg} \beta_j \label{eq: TopPushK dual c1}}
    \addConstraint{0 \leq \beta_j}{\leq \frac{1}{K} \sum_{i = 1}^{\npos} \alpha_i, \quad j = 1, 2, \ldots, \nneg, \label{eq: TopPushK dual c2}}
  \end{maxi!}
  where~$l^{\star}$ is conjugate function of~$l.$ If~$K = 1,$ the upper bound in the second constrainet vanishes due to the first constraint and we get the dual form of \TopPush.
\end{theorem}
\begin{proof}
  In Lemma~\ref{lem: TopPushK primal alternative} we derived alternative fomrulation of \TopPushK with Lagrangian in the following form
  \begin{align*}
    \mathcal{L}(\bm{w}, t, \bm{y}, \bm{z}; \bm{\alpha}, \bm{\beta}, \bm{\gamma})
     & = \frac{1}{2} \norm{\bm{w}}_{2}^{2}
       + C \sum_{i = 1}^{\npos} l(y_i)
       + \sum_{i = 1}^{\npos} \alpha_i \Brac{t + \frac{1}{K} \sum_{j = 1}^{\nneg} z_j - \bm{w}^\top \bm{x}^+_i - y_i} \\
     & + \sum_{j = 1}^{\nneg} \beta_j \Brac{\bm{w}^\top \bm{x}^-_j - t - z_j}
       + \sum_{j = 1}^{\nneg} \gamma_j z_j,
  \end{align*}
  with feasibility conditions~$\beta_j \geq 0$ and~$\gamma_j \geq 0$ for all~$j = 1, \; 2, \ldots, \; \nneg.$ Then the corresponding dual objective function reads
  \begin{equation*}
    g(\bm{\alpha}, \bm{\beta}, \bm{\gamma})
      = \min_{\bm{w}, t, \bm{y}, \bm{z}} \; \mathcal{L}(\bm{w}, t, \bm{z}; \bm{\alpha}, \bm{\beta}, \bm{\gamma}),
  \end{equation*}
  Since the Lagrangian~$\mathcal{L}$ is separable in primal variables, it can be minimized with respect to each variable separately, i.e., the dual function can be rewritten as follows
  \begin{equation}\label{eq: TopPushK dual function}
    \begin{aligned}
      g(\bm{\alpha}, \bm{\beta}, \bm{\gamma})
        & = \min_{\bm{w}} \; \frac{1}{2} \norm{\bm{w}}_{2}^{2}
          - \bm{w}^{\top} \Brac{\sum_{i = 1}^{\npos} \alpha_i \bm{x}^+_i - \sum_{j = 1}^{\nneg} \beta_j \bm{x}^-_j} \\
        & + \min_{t} \; t \Brac{\sum_{i = 1}^{\npos} \alpha_i - \sum_{j = 1}^{\nneg} \beta_j} \\
        & + \min_{\bm{y}} \; C \sum_{i = 1}^{\npos} \Brac{l(y_i) - \frac{\alpha_i}{C}y_i} \\
        & + \min_{\bm{z}} \; \sum_{j = 1}^{\nneg} \Brac{\sum_{i = 1}^{\npos} \alpha_i - \beta_j - \gamma_j}z_j
    \end{aligned}
  \end{equation}
  From optimality conditions with respect to~$\bm{w}$ we deduce 
  \begin{equation*}
    \bm{w}
        = \sum_{i = 1}^{\npos} \alpha_i \bm{x}^+_i - \sum_{j = 1}^{\nneg} \beta_j \bm{x}^-_j
        = \Matrix{\X^+ \\ - \X^-}^\top \vecab,
  \end{equation*}
  where we use Notation~\ref{not: kernel matrix}. Using this relation, we get the first part of the objective function~\eqref{eq: TopPushK dual L} 
  \begin{equation*}
    \frac{1}{2} \norm{\bm{w}}_{2}^{2} - \bm{w}^{\top} \Brac{\sum_{i = 1}^{\npos} \alpha_i \bm{x}^+_i - \sum_{j = 1}^{\nneg} \beta_j \bm{x}^-_j}
      = - \frac{1}{2} \norm{\bm{w}}_{2}^{2}
      = - \frac{1}{2} \bm{w}^{\top} \bm{w}
      = - \frac{1}{2} \vecab^{\top} \Kneg \vecab,
  \end{equation*}
  where~$\Kneg$ is defined in Notation~\ref{not: kernel matrix}. Optimality condition with respect to~$t$ reads 
  \begin{equation*}
    \sum_{i = 1}^{\npos} \alpha_i - \sum_{j = 1}^{\nneg} \beta_j = 0,
  \end{equation*}
  and implies constrain in~\eqref{eq: TopPushK dual c1}. Similarly, Optimality condition with respect to~$\bm{z}$ reads for all $j = 1, \; 2, \ldots, \; \nneg$ as 
  \begin{equation*}
    \frac{1}{K} \sum_{i = 1}^{\npos} \alpha_i - \beta_j - \gamma_j = 0.
  \end{equation*}
  Plugging the feasibility condition~$\gamma_j \geq 0$ into this equality and combining it with the feasibility conditions~$\beta_j \geq 0$ yields constraint~\eqref{eq: TopPushK dual c2}. Finally, minimization of the Lagrangian with respect to~$\bm{y}$ yields for all $i = 1, \; 2, \ldots, \; \npos$ 
  \begin{equation*}
    C \min_{y_i} \Brac{l(y_i) - \frac{\alpha_i}{C} y_i} = - C l^{\star} \Brac{\frac{\alpha_i}{C}}.
  \end{equation*}
  where the equality follows from Definition~\ref{def: conjugate}. Plugging this back into the Lagrange function yields the second part of the objective function~\eqref{eq: TopPushK dual L}, which finishes the proof for \TopPushK. For \TopPush, we have~$K = 1.$ From~\eqref{eq: TopPushK dual c1} and non-negativity of~$\beta_j$ we deduce, that the upper bound in constraint~\eqref{eq: TopPushK dual c2} is always fulfilled and therefore can be ommited, which finishes the proof. 
\end{proof}

\subsection{Accuracy at the Top}

In Section~\ref{sec: aatp} we derived three problem formulations that fall into our framework~\eqref{eq: aatp surrogate}. Namely: \Grill, \TopMeanK and \PatMat. We focus only on \TopMeanK and \PatMat formulations, since as showed in Chapter~\ref{chap: linear}, these two formulations are convex for linear model.

\begin{theorem}[Dual formulation of \TopMeanK]\label{thm: TopMeanK dual}
  Consider \TopMeanK formulation~\eqref{eq: topmeank} with linear model, surrogate function~$l$ and Notation~\ref{not: kernel matrix}. Then the corresponding dual problem has the following form
  \begin{maxi*}{\bm{\alpha}, \bm{\beta}}{
    - \frac{1}{2} \vecab^\top \Kall \vecab - C \sum_{i = 1}^{\npos} l^{\star}\Brac{\frac{\alpha_i}{C}}
    }{}{}
    \addConstraint{\sum_{i = 1}^{\npos} \alpha_i}{= \sum_{j = 1}^{\nall} \beta_j}
    \addConstraint{0 \leq \beta_j}{\leq \frac{1}{K} \sum_{i = 1}^{\npos} \alpha_i, \quad j = 1, 2, \ldots, \nall,}
  \end{maxi*}
  where~$l^{\star}$ is conjugate function of~$l$ and~$K = \nall \tau.$
\end{theorem}
\begin{proof}
  \TopMeanK formulation is similar to the \TopPushK and therefore also dual formulations are similar. The main difference is, that the decision threshold for \TopMeanK is computed from all socres and not only from the negative ones as for \TopPushK. Due to that, the dual variable~$\bm{\beta}$ has different size and the kernel matrix has slightly different form as can be seen in Notation~\ref{not: kernel matrix}. Besides that dual formulations of \TopMeanK and \TopMeanK are identical and the proof of Theorem~\ref{thm: TopMeanK dual} is almost identical to the proof of Theorem~\ref{thm: TopPushK dual}.
\end{proof}

\begin{theorem}[Dual formulation of \PatMat]\label{thm: PatMat dual}
  Consider \PatMat formulation~\eqref{eq: patmat} with linear model, surrogate function~$l$ and Notation~\ref{not: kernel matrix}. Then the corresponding dual problem has the following form
  \begin{maxi!}{\bm{\alpha}, \bm{\beta}, \delta}{
    - \frac{1}{2} \vecab^\top \Kall \vecab
    - C \sum_{i = 1}^{\npos} l^{\star}\Brac{\frac{\alpha_i}{C}}
    - \delta \sum_{j = 1}^{\nall} l^{\star} \Brac{\frac{\beta_j}{\delta\vartheta}}
    - \delta \nall \tau
    }{\label{eq: PatMat dual}}{\label{eq: PatMat dual L}}
    \addConstraint{\sum_{i = 1}^{\npos} \alpha_i}{= \sum_{j = 1}^{\nall} \beta_j \label{eq: PatMat dual c1}}
    \addConstraint{\delta }{\geq 0, \label{eq: PatMat dual c2}}
  \end{maxi!}
  where~$l^{\star}$ is conjugate function of~$l$ and~$\vartheta > 0$ is a scaling parameter.
\end{theorem}
\begin{proof}
  Let us first realize tha \PatMat formulation~\eqref{eq: patmat} with linear model is equivalent to
  \begin{mini*}{\bm{w}, t, \bm{y}, \bm{z}}{
    \frac{1}{2} \norm{\bm{w}}_{2}^{2}+ C \sum_{i = 1}^{\npos} l(y_i)
    }{}{}
    \addConstraint{\sum_{j = 1}^{\nall} l(\vartheta z_i)}{\leq \nall \tau}{}
    \addConstraint{y_i}{= t - \bm{w}^\top \bm{x}^+_i,}{i = 1, \; 2, \ldots, \; \npos.}
    \addConstraint{z_j}{= \bm{w}^\top \bm{x}_j - t, \quad}{j = 1, \; 2, \ldots, \; \nall}
  \end{mini*}
  Corresponding Lagrangian is in the following form
  \begin{align*}
    \mathcal{L}(\bm{w}, t, \bm{y}, \bm{z}; \bm{\alpha}, \bm{\beta}, \delta)
    & = \frac{1}{2} \norm{\bm{w}}_{2}^{2}
      + C \sum_{i = 1}^{\npos} l(y_i)
      + \sum_{i = 1}^{\npos} \alpha_i (t - \bm{w}^{\top}\bm{x}^+_{i} - y_i) \\
    & + \sum_{j = 1}^{\nall} \beta_j(\bm{w}^{\top}\bm{x}_j - t - z_j)
      + \delta \Brac{\sum_{j = 1}^{\nall} l(\vartheta z_j) - \nall \tau}.
  \end{align*}
  with feasibility condition~$\delta \geq 0.$ Then the corresponding dual objective function reads
  \begin{equation*}
    g(\bm{\alpha}, \bm{\beta}, \delta)
      = \min_{\bm{w}, t, \bm{y}, \bm{z}} \; \mathcal{L}(\bm{w}, t, \bm{y}, \bm{z}; \bm{\alpha}, \bm{\beta}, \delta),
  \end{equation*}
  Since the Lagrangian~$\mathcal{L}$ is separable in primal variables, it can be minimized with respect to each variable separately, i.e., the dual function can be rewritten as follows
  \begin{align*}
    g(\bm{\alpha}, \bm{\beta}, \delta)
      & = \min_{\bm{w}} \; \frac{1}{2} \norm{\bm{w}}_{2}^{2}
        - \bm{w}^{\top} \Brac{\sum_{i = 1}^{\npos} \alpha_i \bm{x}^+_i - \sum_{j = 1}^{\nall} \beta_j \bm{x}_j} \\
      & + \min_{t} \; t \Brac{\sum_{i = 1}^{\npos} \alpha_i - \sum_{j = 1}^{\nall} \beta_j} \\
      & + \min_{\bm{y}} \; C \sum_{i = 1}^{\npos} \Brac{l(y_i) - \frac{\alpha_i}{C}y_i} \\
      & + \min_{\bm{z}} \; \delta \sum_{j = 1}^{\nall} \Brac{l(\vartheta z_j) - \frac{\beta_j}{\delta}z_j} \\
      & - \delta \nall \tau.
  \end{align*}
  Note that resulting dual function is very similar to the dual function~\eqref{eq: TopPushK dual function} for \TopPushK, i.e. minimization of the Lagrangian with respect to~$\bm{w}$,~$t$ and~$\bm{y}$ yields similar results. From optimality conditions with respect to~$\bm{w}$ we deduce 
  \begin{equation*}
    \bm{w}
        = \sum_{i = 1}^{\npos} \alpha_i \bm{x}^+_i - \sum_{j = 1}^{\nall} \beta_j \bm{x}_j
        = \Matrix{\X^+ \\ - \X}^\top \vecab,
  \end{equation*}
  where we use Notation~\ref{not: kernel matrix}. Using this relation, we get the first part of the objective function~\eqref{eq: PatMat dual L} 
  \begin{equation*}
    \frac{1}{2} \norm{\bm{w}}_{2}^{2} - \bm{w}^{\top} \Brac{\sum_{i = 1}^{\npos} \alpha_i \bm{x}^+_i - \sum_{j = 1}^{\nall} \beta_j \bm{x}_j}
      = - \frac{1}{2} \norm{\bm{w}}_{2}^{2}
      = - \frac{1}{2} \bm{w}^{\top} \bm{w}
      = - \frac{1}{2} \vecab^{\top} \Kall \vecab,
  \end{equation*}
  where~$\Kall$ is defined in Notation~\ref{not: kernel matrix}. Optimality condition with respect to~$t$ reads 
  \begin{equation*}
    \sum_{i = 1}^{\npos} \alpha_i - \sum_{j = 1}^{\nall} \beta_j = 0,
  \end{equation*}
  and implies constrain in~\eqref{eq: PatMat dual c1}. The optimality condition with respect to~$\bm{y}$ is identical to the one in the proof of Theorem~\ref{thm: TopPushK dual}. Finally, inimization of the Lagrangian with respect to~$\bm{z}$ yields for all $j = 1, \; 2, \ldots, \; \nall$ 
  \begin{equation*}
    \delta \min_{\bm{z}} \; \Brac{l(\vartheta z_j) - \frac{\beta_j}{\delta\vartheta } \vartheta z_j} = - \delta l^{\star} \Brac{\frac{\beta_i}{\delta\vartheta }},
  \end{equation*}
  where the equality follows from Definition~\ref{def: conjugate}. Plugging this back into the Lagrange function yields the second part of the objective function~\eqref{eq: PatMat dual L}, which finishes the proof.
\end{proof}

\subsection{Hypothesis Testing}

In Section~\ref{sec: Neyman-Pearson} we derived three problem formulations that fall into our framework~\eqref{eq: aatp surrogate}. Namely: \GrillNP, \tauFPL and \PatMatNP. Similarly to the previous section, we focus only on \tauFPL and \PatMatNP. Since \tauFPL is a special case of \TopPushK for~$K = \nneg \tau,$ the dual formulation is identical to the one in~\ref{thm: TopPushK dual}.

\begin{theorem}[Dual formulation of \PatMatNP]\label{thm: PatMatNP dual}
  Consider \PatMatNP formulation~\eqref{eq: patmat np} with linear model, surrogate function~$l$ and Notation~\ref{not: kernel matrix}. Then the corresponding dual problem has the following form
  \begin{maxi*}{\bm{\alpha}, \bm{\beta}, \delta}{
    - \frac{1}{2} \vecab^\top \Kneg \vecab
    - C \sum_{i = 1}^{\npos} l^{\star}\Brac{\frac{\alpha_i}{C}}
    - \delta \sum_{j = 1}^{\nneg} l^{\star} \Brac{\frac{\beta_j}{\delta\vartheta}}
    - \delta \nneg \tau
    }{}{}
    \addConstraint{\sum_{i = 1}^{\npos} \alpha_i}{= \sum_{j = 1}^{\nneg} \beta_j}
    \addConstraint{\delta }{\geq 0,}
  \end{maxi*}
  where~$l^{\star}$ is conjugate function of~$l$ and~$\vartheta > 0$ is a scaling parameter.
\end{theorem}
\begin{proof}
  \PatMatNP formulation is similar to the \PatMat and therefore also dual formulations are similar. The main difference is, that the decision threshold for \PatMatNP is computed from all socres and not only from the negative ones as for \PatMat. Due to that, the dual variable~$\bm{\beta}$ has different size and the kernel matrix has slightly different form as can be seen in Notation~\ref{not: kernel matrix}. Besides that dual formulations of \PatMatNP and \PatMat are identical and the proof of Theorem~\ref{thm: PatMatNP dual} is almost identical to the proof of Theorem~\ref{thm: PatMat dual}.
\end{proof}

\section{Coordinate descent}

In previous sections we derived dual formulations of formulations from Table~\ref{tab: summary formulations}. We showed that dual formulations of \TopPush, \TopPushK, \TopMeanK and \tauFPL are very similary and can be written in general form summarized in Theorem~\ref{thm: Top dual}. Similarly, dual formulations of \PatMat and \PatMatNP are very similary and can be written in general form summarized in Theorem~\ref{thm: Pat dual}. In this section, we show concrete form of both dual formulations when the hinge loss or quadratic hinge loss function is used as surrogate. Moreover, we show coordinate descent algorithm that can be used to solve these formulations.

Consider dual formulation from Theorem~\ref{thm: Top dual} or~\ref{thm: Pat dual} and fixed feasible dual variables~$\bm{\alpha},$~$\bm{\beta}.$ Let us define vector of scores~$\bm{s}$ by
\begin{equation}\label{eq: dual scores}
  \bm{s} = \K \vecab.
\end{equation}
Our goal is to derive coordinate algorithm that can be used to solve this dual problem. Due to the constraint~\eqref{eq: top dual formulation c1} in each step of such algorithm we have to update at least two coordinates of vectors~$\bm{\alpha},$~$\bm{\beta}.$ Moreover, if we want to update two coordinates at once, there are only three update rules which modify two coordinates of~$\bm{\alpha},$~$\bm{\beta}$ and which satisfy constraints~\eqref{eq: top dual formulation c1} and keep~\eqref{eq: dual scores} satisfied. The first one updates two components of~$\bm{\alpha}$
\begin{subequations}\label{eq: update rules}
  \begin{align}\label{eq: update rule a,a}
    \alphak & \to \alphak + \Delta, & \quad
    \alphal & \to \alphal - \Delta, & \quad
    \bm{s} & \to \bm{s} + \Brac{\K_{\bullet, k} - \K_{\bullet, l}}\Delta,
  \end{align}
  where~$K_{\bullet, i}$ denotes~$i$-th column of~$\K.$ Note that the update rule for~$\bm{s}$ does not use matrix multiplication but only vector addition. The second rule updates one component of~$\bm{\alpha}$ and one component of~$\bm{\beta}$ 
  \begin{align}\label{eq: update rule a,b}
    \alphak & \to \alphak + \Delta, & \quad
    \betal  & \to \betal  + \Delta, & \quad
    \bm{s} & \to \bm{s} + \Brac{\K_{\bullet, k} + \K_{\bullet, l}}\Delta,
  \end{align}
  and the last one updates two components of~$\bm{\beta}$
  \begin{align}\label{eq: update rule b,b}
    \betak & \to \betak + \Delta, & \quad
    \betal & \to \betal - \Delta, & \quad
    \bm{s}  & \to \bm{s} + \Brac{\K_{\bullet, k} - \K_{\bullet, l}}\Delta.
  \end{align}
\end{subequations}
These three update rules hold true for any surrogate function. However, the calculation of the optimal~$\Delta$ depends on the used problem formulation and surrogate function. In the following subsections, we show the closed-form formula for~$\Delta$ for dual formulations from Theorem~\ref{thm: Top dual} and~\ref{thm: Pat dual} with hinge loss and quadratic hinge loss function as surrogate.

\begin{notation}\label{not: dual update rules}
  To avoid confusion, we use two different notation for indices. The length of the vector of scores~$\bm{s}$ is always~$\npos + \nall$ or~$\npos + \nneg.$ Then we use indices~$k,$~$l$ to denote
  \todo[inline]{DEfine new notation for indices}
\end{notation}

\subsection{Dual formulation from Theorem~\ref{thm: Top dual}}

We start with dual formulation from Theorem~\ref{thm: Top dual}. Then using any of the update rules~\eqref{eq: update rules}, the dual formulation can be rewritten as a rewritten as a quadratic one-dimensional problem with respect to~$\Delta$
\begin{maxi*}{\Delta}{
  -\frac{1}{2} a(\bm{\alpha}, \bm{\beta}) \Delta^2
  - b(\bm{\alpha}, \bm{\beta}) \Delta
  - c(\bm{\alpha}, \bm{\beta})
  }{}{}
  \addConstraint{\Delta_{lb}(\bm{\alpha}, \bm{\beta})}{\leq \Delta \leq \Delta_{ub}(\bm{\alpha}, \bm{\beta})}
\end{maxi*}
where~$a,$~$b,$~$c,$~$\Delta_{lb},$~$\Delta_{ub}$ are constants with respect to~$\Delta.$ The optimal solution to this problem is
\begin{equation}\label{eq: Delta optimal}
  \Delta^{\star} = \clip{\Delta_{lb}}{\Delta_{ub}}{\gamma},
\end{equation}
where~$-\nicefrac{b}{a}.$ Since we assume one of the update rule~\eqref{eq: update rules}, the constrain~\eqref{eq: top dual formulation c1} is always satisfied after the update.

\subsection*{Hinge loss}

Plugging the conjugate~\eqref{eq: conjugate hinge} of the hinge loss into the dual formulation from Theorem~\ref{thm: Top dual} yields
\begin{maxi!}{\bm{\alpha}, \bm{\beta}}{
  - \frac{1}{2} \vecab^\top \K \vecab
  + \sum_{i = 1}^{\npos} \alpha_i
  }{\label{eq: Top dual hinge}}{\label{eq: Top dual hinge L}}
  \addConstraint{\sum_{i = 1}^{\npos} \alpha_i}{= \sum_{j = 1}^{\ntil} \beta_j
  \label{eq: Top dual hinge c1}}
  \addConstraint{0 \leq \alpha_i}{\leq C,}{i = 1, 2, \ldots, \npos
  \label{eq: Top dual hinge c2}}
  \addConstraint{0 \leq \beta_j}{\leq \frac{1}{K} \sum_{i = 1}^{\npos} \alpha_i, \quad}{j = 1, 2, \ldots, \ntil,
  \label{eq: Top dual hinge c3}}
\end{maxi!}
Moreover, for~$K = 1,$ the upper limit in~\eqref{eq: Top dual hinge c3} is always satisfied due to~\eqref{eq: Top dual hinge c1} and the problem can be simplified. The following three lemmas provide formulas for optimal~$\Delta$ for each of update rules~\eqref{eq: update rules}.

\begin{lemma}[Update rule~\eqref{eq: update rule a,a} for problem~\eqref{eq: Top dual hinge}]
  Consider problem~\eqref{eq: Top dual hinge}, update rule~\eqref{eq: update rule a,a} and~$1 \leq k \leq \npos$ and~$1 \leq l \leq \npos.$ Then the optimal solution~$\Delta^{\star}$ is given by~\eqref{eq: Delta optimal} where
  \begin{align*}
    \Delta_{lb} & = \max\{- \alphak,\; \alphal - C\}, \\
    \Delta_{ub} & = \min\{C - \alphak,\; \alphal \}, \\
    \gamma & = -\frac{s_k - s_l}{\K_{kk} + \K_{ll} - \K_{kl} - \K_{lk}}.
  \end{align*}
\end{lemma}

\begin{proof}
  Constraint~\eqref{eq: Top dual hinge c1} is always satisfied from the definition of the update rule~\eqref{eq: update rule a,a}. Constraint~\eqref{eq: Top dual hinge c3} is always satisfied since no~$\beta_j$ was updated and the sum of all~$\alpha_i$ did not change. Constraint~\eqref{eq: Top dual hinge c2} reads
  \begin{align*}
    0 \leq \alphak + \Delta \leq C
    & \quad \implies \quad
    - \alphak \leq \Delta \leq C - \alphak \\
    0 \leq \alphal - \Delta \leq C
    & \quad \implies \quad
    \alphal - C \leq \Delta \leq \alphal
  \end{align*}
  which gives the lower and upper bound of~$\Delta.$ Using the update rule~\eqref{eq: update rule a,a}, objective~\eqref{eq: Top dual hinge L} can be rewritten as a quadratic function with respect to~$\Delta$ as
  \begin{equation*}
    - \frac{1}{2} \Brac[s]{\K_{kk} + \K_{ll} - \K_{kl} - \K_{lk}} \Delta^2 - \Brac[s]{s_k - s_l} \Delta - c(\bm{\alpha}, \bm{\beta}).
  \end{equation*}
  Finally, the optimal solution~$\Delta^{\star}$ is given by~\eqref{eq: Delta optimal}.
  \todo{inline}[Add details about objective function]
\end{proof}

\begin{lemma}[Update rule~\eqref{eq: update rule a,b} for problem~\eqref{eq: Top dual hinge}]
  Consider problem~\eqref{eq: Top dual hinge}, update rule~\eqref{eq: update rule a,b} and~$1 \leq k \leq \npos$ and~$\npos + 1 \leq l \leq \ntil.$ Let us define~$\hat{l} = l - \npos$ and
  \begin{equation*}
    \beta_{\max} = \max_{j \in \{1, 2, \ldots, \ntil \} \setminus \{\hat{l}\}} \beta_j.
  \end{equation*}
  Then the optimal solution~$\Delta^{\star}$ is given by~\eqref{eq: Delta optimal} where
  \begin{align*}
    \Delta_{lb} & = 
      \begin{cases*}
        \max \Brac[c]{- \alphak, \;  -\betal} & K = 1, \\
        \max \Brac[c]{- \alphak, \;  -\betal, \; K\beta_{\max} - \sum_{i = 1}^{\npos} \alpha_i} & \textrm{otherwise},
      \end{cases*} \\
    \Delta_{ub} & = 
      \begin{cases*}
          C - \alphak & K = 1, \\
          \min \Brac[c]{C - \alphak, \; \frac{1}{K-1}\Brac{\sum_{i = 1}^{\npos} \alpha_i - K \betal}}  & \textrm{otherwise}.
      \end{cases*} \\
    \gamma & = - \frac{s_k + s_l - 1}{\K_{kk} + \K_{ll} + \K_{kl} + \K_{lk}}.
  \end{align*}
\end{lemma}

\begin{proof}
  Constraint~\eqref{eq: Top dual hinge c1} is always satisfied from the definition of the update rule~\eqref{eq: update rule a,b}. Constraint~\eqref{eq: Top dual hinge c2} reads
  \begin{equation*}
    0 \leq \alphak + \Delta \leq C
    \quad \implies \quad
    - \alphak \leq \Delta \leq C - \alphak.
  \end{equation*}
  Using the definition of~$\beta_{\max},$ constraint~\eqref{eq: Top dual hinge c3} for any~$K \geq 2$ reads
  \begin{align*}
    0 \leq \beta_{\max} \leq \frac{1}{K} \sum_{i = 1}^{\npos} \alpha_i + \frac{\Delta}{K} 
    & \quad \implies \quad
    K\beta_{\max} - \sum_{i = 1}^{\npos} \alpha_i \leq \Delta \\
    0 \leq \betal + \Delta \leq \frac{1}{K} \sum_{i = 1}^{\npos} \alpha_i + \frac{\Delta}{K}
    & \quad \implies \quad
    -\betal \leq \Delta \quad \land \quad \Delta \leq \frac{1}{K-1}\Brac{\sum_{i = 1}^{\npos} \alpha_i - K \betal}
  \end{align*}
  Combination of these bounds yealds the lower bound~$\Delta_{lb}$ and upper bound~$\Delta_{ub}.$ If~$K = 1,$ the upper bounds in~\eqref{eq: Top dual hinge c3} is always satisfied due to~\eqref{eq: Top dual hinge c1} and the lower and upper bound of~$\Delta$ can be simplified. Using the update rule~\eqref{eq: update rule a,b}, objective~\eqref{eq: Top dual hinge L} can be rewritten as a quadratic function with respect to~$\Delta$ as
  \begin{equation*}
    - \frac{1}{2} \Brac[s]{\K_{kk} + \K_{ll} + \K_{kl} + \K_{lk}} \Delta^2 - \Brac[s]{s_k + s_l - 1} \Delta - c(\bm{\alpha}, \bm{\beta}).
  \end{equation*}
  Finally, the optimal solution~$\Delta^{\star}$ is given by~\eqref{eq: Delta optimal}.
  \todo{inline}[Add details about objective function]
\end{proof}

\begin{lemma}[Update rule~\eqref{eq: update rule b,b} for problem~\eqref{eq: Top dual hinge}]
  Consider problem~\eqref{eq: Top dual hinge}, update rule~\eqref{eq: update rule b,b} and~$\npos + 1 \leq k \leq \ntil$ and~$\npos + 1 \leq l \leq \ntil.$ Let us define~$\hat{k} = k - \npos$ and~$\hat{l} = l - \npos.$ Then the optimal solution~$\Delta^{\star}$ is given by~\eqref{eq: Delta optimal} where
  \begin{align*}
    \Delta_{lb} & = 
      \begin{cases*}
        - \betak & K = 1, \\
        \max \Brac[c]{- \betak,\; \betal - \frac{1}{K} \sum_{i = 1}^{\npos} \alpha_i} & \textrm{otherwise},
      \end{cases*} \\
    \Delta_{ub} & = 
      \begin{cases*}
        \betal & K = 1, \\
        \min \Brac[c]{\frac{1}{K} \sum_{i = 1}^{\npos} \alpha_i - \betak,\; \betal} & \textrm{otherwise}.
      \end{cases*} \\
    \gamma & = -\frac{s_k - s_l}{\K_{kk} + \K_{ll} - \K_{kl} - \K_{lk}}.
  \end{align*}
\end{lemma}

\begin{proof}
  Constraint~\eqref{eq: Top dual hinge c1} is always satisfied from the definition of the update rule~\eqref{eq: update rule b,b}. Constraint~\eqref{eq: Top dual hinge c2} is also always satisfied since no~$\alpha_i$ is updated. Constraint~\eqref{eq: Top dual hinge c3} for any~$K \geq 2$ reads
  \begin{align*}
    0 \leq \betak + \Delta \leq \frac{1}{K} \sum_{i = 1}^{\npos} \alpha_i 
    & \quad \implies \quad
    -\betak \leq \Delta \leq \frac{1}{K} \sum_{i = 1}^{\npos} \alpha_i - \betak \\
    0 \leq \betal - \Delta \leq \frac{1}{K} \sum_{i = 1}^{\npos} \alpha_i
    & \quad \implies \quad
    \betal - \frac{1}{K} \sum_{i = 1}^{\npos} \alpha_i \leq \Delta \leq \betal
  \end{align*}
  which gives the lower and upper bound of~$\Delta.$ If~$K = 1,$ the upper bounds in~\eqref{eq: Top dual hinge c3} is always satisfied due to~\eqref{eq: Top dual hinge c1} and the lower and upper bound of~$\Delta$ can be simplified. Using the update rule~\eqref{eq: update rule b,b}, objective~\eqref{eq: Top dual hinge L} can be rewritten as a quadratic function with respect to~$\Delta$ as
  \begin{equation*}
    - \frac{1}{2} \Brac[s]{\K_{kk} + \K_{ll} - \K_{kl} - \K_{lk}} \Delta^2 - \Brac[s]{s_k - s_l} \Delta - c(\bm{\alpha}, \bm{\beta}).
  \end{equation*}
  Finally, the optimal solution~$\Delta^{\star}$ is given by~\eqref{eq: Delta optimal}.
  \todo{inline}[Add details about objective function]
\end{proof}

\subsection*{Quadratic hinge loss}

Plugging the conjugate~\eqref{eq: conjugate hinge} of the quadratic hinge loss into the dual formulation from Theorem~\ref{thm: Top dual} yields
\begin{maxi!}{\bm{\alpha}, \bm{\beta}}{
  - \frac{1}{2} \vecab^\top \K \vecab
  + \sum_{i = 1}^{\npos} \alpha_i
  - \frac{1}{4C} \sum_{i = 1}^{\npos} \alpha_i^2
  }{\label{eq: Top dual quadratic}}{\label{eq: Top dual quadratic L}}
  \addConstraint{\sum_{i = 1}^{\npos} \alpha_i}{= \sum_{j = 1}^{\ntil} \beta_j
  \label{eq: Top dual quadratic c1}}
  \addConstraint{0 \leq \alpha_i}{,}{i = 1, 2, \ldots, \npos
  \label{eq: Top dual quadratic c2}}
  \addConstraint{0 \leq \beta_j}{\leq \frac{1}{K} \sum_{i = 1}^{\npos} \alpha_i, \quad}{j = 1, 2, \ldots, \ntil,
  \label{eq: Top dual quadratic c3}}
\end{maxi!}
Moreover, for~$K = 1,$ the upper limit in~\eqref{eq: Top dual quadratic c3} is always satisfied due to~\eqref{eq: Top dual quadratic c1} and the problem can be simplified. The following three lemmas provide formulas for optimal~$\Delta$ for each of update rules~\eqref{eq: update rules}.

\begin{lemma}[Update rule~\eqref{eq: update rule a,a} for problem~\eqref{eq: Top dual quadratic}]
  Consider problem~\eqref{eq: Top dual quadratic}, update rule~\eqref{eq: update rule a,a} and~$1 \leq k \leq \npos$ and~$1 \leq l \leq \npos.$ Then the optimal solution~$\Delta^{\star}$ is given by~\eqref{eq: Delta optimal} where
  \begin{align*}
    \Delta_{lb} & = -\alphak, &
    \Delta_{ub} & = \alphal, &
    \gamma & = -\frac{s_k - s_l + \frac{1}{2C}(\alphak - \alphal)}{\K_{kk} + \K_{ll} - \K_{kl} - \K_{lk} + \frac{1}{C}}.
  \end{align*}
\end{lemma}

\begin{proof}
  Constraint~\eqref{eq: Top dual quadratic c1} is always satisfied from the definition of the update rule~\eqref{eq: update rule a,a}. Constraint~\eqref{eq: Top dual quadratic c3} is also always satisfied since no~$\beta_j$ was updated and the sum of all~$\alpha_i$ did not change. Constraint~\eqref{eq: Top dual quadratic c2} reads
  \begin{align*}
    0 \leq \alphak + \Delta
    & \quad \implies \quad
    - \alphak \leq \Delta \\
    0 \leq \alphal - \Delta
    & \quad \implies \quad
    \Delta \leq \alphal
  \end{align*}
  which gives the lower and upper bound of~$\Delta.$ Using the update rule~\eqref{eq: update rule a,a}, objective~\eqref{eq: Top dual quadratic L} can be rewritten as a quadratic function with respect to~$\Delta$ as
  \begin{equation*}
    - \frac{1}{2} \Brac[s]{\K_{kk} + \K_{ll} - \K_{kl} - \K_{lk} + \frac{1}{C}} \Delta^2 - \Brac[s]{s_k - s_l + \frac{1}{2C}(\alphak - \alphal)} \Delta - c(\bm{\alpha}, \bm{\beta}).
  \end{equation*}
  Finally, the optimal solution~$\Delta^{\star}$ is given by~\eqref{eq: Delta optimal}.
  \todo{inline}[Add details about objective function]
\end{proof}

\begin{lemma}[Update rule~\eqref{eq: update rule a,b} for problem~\eqref{eq: Top dual quadratic}]
  Consider problem~\eqref{eq: Top dual quadratic}, update rule~\eqref{eq: update rule a,b} and~$1 \leq k \leq \npos$ and~$\npos + 1 \leq l \leq \ntil.$ Let us define~$\hat{l} = l - \npos$ and
  \begin{equation*}
    \beta_{\max} = \max_{j \in \{1, 2, \ldots, \ntil \} \setminus \{\hat{l}\}} \beta_j.
  \end{equation*}
  Then the optimal solution~$\Delta^{\star}$ is given by~\eqref{eq: Delta optimal} where
  \begin{align*}
    \Delta_{lb} & = 
      \begin{cases*}
        \max \Brac[c]{- \alphak,\;  -\betal} & K = 1, \\
        \max \Brac[c]{- \alphak,\;  -\betal, \; K\beta_{\max} - \sum_{i = 1}^{\npos} \alpha_i} & \textrm{otherwise},
      \end{cases*} \\
    \Delta_{ub} & = 
      \begin{cases*}
        + \infty & K = 1, \\
        \frac{1}{K-1}\Brac{\sum_{i = 1}^{\npos} \alpha_i - K \betal} & \textrm{otherwise},
      \end{cases*} \\
    \gamma & = -\frac{s_k + s_l - 1 + \frac{1}{2C} \alphak}{\K_{kk} + \K_{ll} + \K_{kl} + \K_{lk} + \frac{1}{2C}}.
  \end{align*}
\end{lemma}

\begin{proof}
  Constraint~\eqref{eq: Top dual quadratic c1} is always satisfied from the definition of the update rule~\eqref{eq: update rule a,b}. Constraint~\eqref{eq: Top dual quadratic c2} reads
  \begin{equation*}
    0 \leq \alphak + \Delta
    \quad \implies \quad
    - \alphak \leq \Delta.
  \end{equation*}
  Using the definition of~$\beta_{\max},$ constraint~\eqref{eq: Top dual quadratic c3} for any~$K \geq 2$ reads
  \begin{align*}
    0 \leq \beta_{\max} \leq \frac{1}{K} \sum_{i = 1}^{\npos} \alpha_i + \frac{\Delta}{K} 
    & \quad \implies \quad
    K\beta_{\max} - \sum_{i = 1}^{\npos} \alpha_i \leq \Delta \\
    0 \leq \betal + \Delta \leq \frac{1}{K} \sum_{i = 1}^{\npos} \alpha_i + \frac{\Delta}{K}
    & \quad \implies \quad
    -\betal \leq \Delta \quad \land \quad \Delta \leq \frac{1}{K-1}\Brac{\sum_{i = 1}^{\npos} \alpha_i - K \betal}
  \end{align*}
  Combination of these bounds yealds the lower bound~$\Delta_{lb}$ and upper bound~$\Delta_{ub}.$ If~$K = 1,$ the upper bounds in~\eqref{eq: Top dual quadratic c3} is always satisfied due to~\eqref{eq: Top dual quadratic c1} and the lower and upper bound of~$\Delta$ can be simplified. Using the update rule~\eqref{eq: update rule a,b}, objective~\eqref{eq: Top dual quadratic L} can be rewritten as a quadratic function with respect to~$\Delta$ as
  \begin{equation*}
    - \frac{1}{2} \Brac[s]{\K_{kk} + \K_{ll} + \K_{kl} + \K_{lk} + \frac{1}{2C}} \Delta^2 - \Brac[s]{s_k + s_l - 1 + \frac{1}{2C} \alphak} \Delta - c(\bm{\alpha}, \bm{\beta}).
  \end{equation*}
  Finally, the optimal solution~$\Delta^{\star}$ is given by~\eqref{eq: Delta optimal}.
  \todo{inline}[Add details about objective function]
\end{proof}

\begin{lemma}[Update rule~\eqref{eq: update rule b,b} for problem~\eqref{eq: Top dual quadratic}]
  Consider problem~\eqref{eq: Top dual quadratic}, update rule~\eqref{eq: update rule b,b} and~$\npos + 1 \leq k \leq \ntil$ and~$\npos + 1 \leq l \leq \ntil.$ Let us define~$\hat{k} = k - \npos$ and~$\hat{l} = l - \npos.$ Then the optimal solution~$\Delta^{\star}$ is given by~\eqref{eq: Delta optimal} where
  \begin{align*}
    \Delta_{lb} & = 
      \begin{cases*}
        -\betak & K = 1, \\
        \max \Brac[c]{- \betak,\; \betal - \frac{1}{K} \sum_{i = 1}^{\npos} \alpha_i} & \textrm{otherwise},
      \end{cases*} \\
    \Delta_{ub} & = 
      \begin{cases*}
        \betal & K = 1, \\
        \min \Brac[c]{\betal,\; \frac{1}{K} \sum_{i = 1}^{\npos} \alpha_i - \betak} & \textrm{otherwise},
      \end{cases*} \\
    \gamma & = -\frac{s_k - s_l}{\K_{kk} + \K_{ll} - \K_{kl} - \K_{lk}}.
  \end{align*}
\end{lemma}

\begin{proof}
  Constraint~\eqref{eq: Top dual quadratic c1} is always satisfied from the definition of the update rule~\eqref{eq: update rule b,b}. Constraint~\eqref{eq: Top dual quadratic c2} is also always satisfied since no~$\alpha_i$ is updated. Constraint~\eqref{eq: Top dual quadratic c3} for any~$K \geq 2$ reads
  \begin{align*}
    0 \leq \betak + \Delta \leq \frac{1}{K} \sum_{i = 1}^{\npos} \alpha_i 
    & \quad \implies \quad
    -\betak \leq \Delta \leq \frac{1}{K} \sum_{i = 1}^{\npos} \alpha_i - \betak \\
    0 \leq \betal - \Delta \leq \frac{1}{K} \sum_{i = 1}^{\npos} \alpha_i
    & \quad \implies \quad
    \betal - \frac{1}{K} \sum_{i = 1}^{\npos} \alpha_i \leq \Delta \leq \betal
  \end{align*}
  which gives the lower and upper bound of~$\Delta.$ If~$K = 1,$ the upper bounds in~\eqref{eq: Top dual quadratic c3} is always satisfied due to~\eqref{eq: Top dual quadratic c1} and the lower and upper bound of~$\Delta$ can be simplified. Using the update rule~\eqref{eq: update rule b,b}, objective~\eqref{eq: Top dual quadratic L} can be rewritten as a quadratic function with respect to~$\Delta$ as
  \begin{equation*}
    - \frac{1}{2} \Brac[s]{\K_{kk} + \K_{ll} - \K_{kl} - \K_{lk}} \Delta^2 - \Brac[s]{s_k - s_l} \Delta - c(\bm{\alpha}, \bm{\beta}).
  \end{equation*}
  Finally, the optimal solution~$\Delta^{\star}$ is given by~\eqref{eq: Delta optimal}.
  \todo{inline}[Add details about objective function]
\end{proof}

\subsection{Dual formulation from Theorem~\ref{thm: Pat dual}}

In the beginning of this subsection we derived problem~\eqref{eq: Pat dual quadratic}. As in the proof of Theorem~\ref{thm:Update rule TopPushK with quadratic loss}, we show, that for each of update rules~\eqref{eq:Update rules} and for fixed~$\bm{\alpha},$~$\bm{\beta},$~$\delta,$ this problem can be rewritten as a simple one-dimensional quadratic problem with bound constraints. In this case, however, we have to also consider the third primal variable~$\delta.$ For fixed~$\bm{\alpha}$ and~$\bm{\beta},$, maximizing objective function~(\ref{eq: Pat dual quadratic L1}-\ref{eq: Pat dual quadratic L2}) with respect to~$\delta$ leads to the
\begin{align*}
  \maximize{\delta}
    & - (n\tau) \delta - \Brac{\frac{1}{4\vartheta_2^2} \sum_{j = 1}^{\nneg} \beta_j^2} \frac{1}{\delta}, \\
  \st
    & \delta \geq 0.
\end{align*}
The solution of this problem equals to
\begin{equation}\label{eq:PatMat dual quadratic optimal delta}
  \delta^* = \sqrt{\frac{1}{4\vartheta_2^2 n \tau} \sum_{j = 1}^{n} \beta_j^2}.
\end{equation}
In the following list, we discuss each of update rules~\eqref{eq:Update rules}:

\subsection*{Hinge loss}

Similarly to the previous section, Plugging the conjugate~\eqref{eq: conjugate hinge} of the hinge loss into the dual formulation from Theorem~\ref{thm: Pat dual} yields
\begin{maxi!}{\bm{\alpha}, \bm{\beta}, \delta}{
  - \frac{1}{2} \vecab^\top \K \vecab
  + \sum_{i = 1}^{\npos} \alpha_i
  + \frac{1}{\vartheta} \sum_{j = 1}^{\ntil} \beta_j 
  - \delta \ntil \tau
  }{\label{eq: Pat dual hinge}}{\label{eq: Pat dual hinge L}}
  \addConstraint{\sum_{i = 1}^{\npos} \alpha_i}{= \sum_{j = 1}^{\ntil} \beta_j \label{eq: Pat dual hinge c1}}
  \addConstraint{0 \leq \alpha_i}{\leq C,}{i = 1, 2, \ldots, \npos \label{eq: Pat dual hinge c2}}
  \addConstraint{0 \leq \beta_j}{\leq \delta \vartheta, \quad}{j = 1, 2, \ldots, \ntil \label{eq: Pat dual hinge c3}}
  \addConstraint{\delta }{\geq 0,  \label{eq: Pat dual hinge c4}}
\end{maxi!}
This is again a convex quadratic problem. The following three lemmas provide formulas for optimal~$\Delta$ for each of update rules~\eqref{eq: update rules}.

\begin{lemma}[Update rule~\eqref{eq: update rule a,a} for problem~\eqref{eq: Pat dual hinge}]
  Consider problem~\eqref{eq: Pat dual hinge}, update rule~\eqref{eq: update rule a,a} and~$1 \leq k \leq \npos$ and~$1 \leq l \leq \npos.$ Then the optimal solution~$\Delta^{\star}$ is given by~\eqref{eq: Delta optimal} where
  \begin{align*}
    \Delta_{lb} & = \min\{- \alphak,\; \alphal - C\}, \\
    \Delta_{ub} & = \max\{C - \alphak,\; \alphal\}, \\
    \gamma & = -\frac{s_k - s_l}{\K_{kk} + \K_{ll} - \K_{kl} - \K_{lk}}, \\
    \delta^{\star} & = \delta.
  \end{align*}
\end{lemma}

\begin{proof}
  \todo[inline]{Add proof}
\end{proof}

\begin{lemma}[Update rule~\eqref{eq: update rule a,b} for problem~\eqref{eq: Pat dual hinge}]
  Consider problem~\eqref{eq: Pat dual hinge}, update rule~\eqref{eq: update rule a,b} and~$1 \leq k \leq \npos$ and~$\npos + 1 \leq l \leq \ntil.$ Let us define~$\hat{l} = l - \npos$ and
  \begin{equation*}
    \beta_{\max} = \max_{j \in \{1, 2, \ldots, \ntil \} \setminus \{\hat{l}\}} \beta_j.
  \end{equation*}
  Then the optimal solution~$\Delta^{\star}$ is given by~\eqref{eq: Delta optimal} where the bounds~$\Delta_{lb},$~$\Delta_{ub}$ and~$\delta$ is equal to one of the two following possibilities which maximizes the original objective:
  \begin{enumerate}
    \item If~$\betal + \Delta^{\star} \leq \beta_{\max}$, then
    \begin{align*}
      \Delta_{lb} & = \max\{- \alphak,\; -\betal \}, \\
      \Delta_{ub} & = \min\{C - \alphak,\; \beta_{\max} - \betal \}, \\
      \gamma & = -\frac{s_k + s_l - 1 - \frac{1}{\vartheta}}{\K_{kk} + \K_{ll} + \K_{kl} + \K_{lk}}, \\
      \delta^{*} & = \frac{\beta_{\max}}{\vartheta}.
    \end{align*}
    \item If~$\betal + \Delta^{\star} \geq \beta_{\max}$, then
    \begin{align*}
      \Delta_{lb} & = \max\{- \alphak,\; -\betal \}, \\
      \Delta_{ub} & = C - \alphak, \\
      \gamma & = -\frac{s_k + s_l - 1 - \frac{1 - \ntil \tau}{\vartheta}}{\K_{kk} + \K_{ll} + \K_{kl} + \K_{lk}}, \\
      \delta^{*} & = \frac{\betal + \Delta^{\star}}{\vartheta}.
    \end{align*}
  \end{enumerate}
\end{lemma}

\begin{proof}
  \todo[inline]{Add proof}
\end{proof}

\begin{lemma}[Update rule~\eqref{eq: update rule b,b} for problem~\eqref{eq: Pat dual hinge}]
  Consider problem~\eqref{eq: Pat dual hinge}, update rule~\eqref{eq: update rule b,b} and~$\npos + 1 \leq k \leq \ntil$ and~$\npos + 1 \leq l \leq \ntil.$ Let us define~$\hat{k} = k - \npos,$~$\hat{l} = l - \npos$ and
  \begin{equation*}
    \beta_{\max} = \max_{j \in \{1, 2, \ldots, \ntil \} \setminus \{\hat{k}, \hat{l}\}} \beta_j.
  \end{equation*}
  Then the optimal solution~$\Delta^{\star}$ is given by~\eqref{eq: Delta optimal} where the bounds~$\Delta_{lb},$~$\Delta_{ub}$ and~$\delta$ is equal to one of the three following possibilities which maximizes the original objective:
  \begin{enumerate}
    \item If~$\beta_{\max} \geq \max\{\betak + \Delta^{\star}, \betal - \Delta^{\star}\}$, then
    \begin{align*}
      \Delta_{lb} & = \max\{- \betak,\; \betal - \beta_{\max} \}, \\
      \Delta_{ub} & = \min\{\beta_{\max} - \betak, \; \betal \}, \\
      \gamma & = -\frac{s_k - s_l}{\K_{kk} + \K_{ll} - \K_{kl} - \K_{lk}}, \\
      \delta^{*} & = \frac{\beta_{\max}}{\vartheta}.
    \end{align*}
    \item If~$\betak + \Delta^{\star} \geq \max\{\beta_{\max} , \betal - \Delta^{\star}\}$, then
    \begin{align*}
      \Delta_{lb} & = \max\{- \betak,\; \frac{1}{2}(\betal - \betak)\}, \\
      \Delta_{ub} & = \betal, \\
      \gamma & = -\frac{s_k - s_l + \frac{\ntil \tau}{\vartheta}}{\K_{kk} + \K_{ll} - \K_{kl} - \K_{lk}}, \\
      \delta^{*} & = \frac{\betak + \Delta}{\vartheta}.
    \end{align*}
    \item If~$\betal - \Delta^{\star} \geq \max\{\betak + \Delta^{\star}, \beta_{\max}\}$, then
    \begin{align*}
      \Delta_{lb} & = -\betak, \\
      \Delta_{ub} & = \min\{\frac{1}{2}(\betak - \betal),\; \betal \}, \\
      \gamma     & = -\frac{s_k - s_l - \frac{\ntil \tau}{\vartheta}}{\K_{kk} + \K_{ll} - \K_{kl} - \K_{lk}}, \\
      \delta^{*} & = \frac{\betak - \Delta^{\star}}{\vartheta}.
    \end{align*}
  \end{enumerate}
\end{lemma}

\begin{proof}
  \todo[inline]{Add proof}
\end{proof}

\subsection*{Quadratic hinge loss}

Plugging the conjugate~\eqref{eq: conjugate quadratic hinge} of the quadratic hinge loss into the dual formulation from Theorem~\ref{thm: Pat dual} yields
\begin{maxi!}{\bm{\alpha}, \bm{\beta}, \delta}{
  - \frac{1}{2} \vecab^\top \K \vecab
  + \sum_{i = 1}^{\npos} \alpha_i
  - \frac{1}{4C} \sum_{i = 1}^{\npos} \alpha_i^2
  }{\label{eq: Pat dual quadratic}}{\label{eq: Pat dual quadratic L1}}
  \breakObjective{
    + \frac{1}{\vartheta} \sum_{j = 1}^{\ntil} \beta_j 
    - \frac{1}{4 \delta \vartheta^2} \sum_{j = 1}^{\ntil} \beta_j^2
    - \delta \ntil \tau \label{eq: Pat dual quadratic L2}
  }
  \addConstraint{\sum_{i = 1}^{\npos} \alpha_i}{= \sum_{j = 1}^{\ntil} \beta_j
  \label{eq: Pat dual quadratic c1}}
  \addConstraint{\alpha_i}{\geq 0,}{i = 1, 2, \ldots, \npos
  \label{eq: Pat dual quadratic c2}}
  \addConstraint{\beta_j}{\geq 0,}{j = 1, 2, \ldots, \ntil
  \label{eq: Pat dual quadratic c3}}
  \addConstraint{\delta }{\geq 0,
  \label{eq: Pat dual quadratic c4}}
\end{maxi!}
This is again a convex quadratic problem. The following theorem provides a formula for the optimal step~$\Delta^\star$ for the update rule~\eqref{eq:Update rules}. Note that we do not perform a joint minimization in~$(\alphak, \; \beta_l, \; \delta)$ but perform a minimization with respect to~$(\alphak, \; \beta_l)$, update these two values and then optimize the objective with respect to~$\delta$. 

\begin{lemma}[Update rule~\eqref{eq: update rule a,a} for problem~\eqref{eq: Pat dual quadratic}]
  Consider problem~\eqref{eq: Pat dual quadratic}, update rule~\eqref{eq: update rule a,a} and~$1 \leq k \leq \npos$ and~$1 \leq l \leq \npos.$ Then the optimal solution~$\Delta^{\star}$ is given by~\eqref{eq: Delta optimal} where
  \begin{align*}
    \Delta_{lb} & = -\alphak, \\
    \Delta_{ub} & = \alphal, \\
    \gamma & = -\frac{s_k - s_l + \frac{1}{2C}(\alphak - \alphal)}{\K_{kk} + \K_{ll} - \K_{kl} - \K_{lk} + \frac{1}{C}}, \\
    \delta^{\star}  & = \delta.
  \end{align*}
\end{lemma}

\begin{proof}
  For update rule~\eqref{eq: update rule a,a} and any~$1\leq k, l \leq \npos$, constraint~\eqref{eq: Pat dual quadratic c3} is satisfied since no~$\beta_j$ was updated. Constraint~\eqref{eq: Pat dual quadratic c2} reads~$-\alphak \leq \Delta \leq \alphal$ while objective~(\ref{eq: Pat dual quadratic L1}-\ref{eq: Pat dual quadratic L2}) can be rewritten as
  \begin{equation*}
    - \frac{1}{2} \Brac[s]{\K_{kk} + \K_{ll} - \K_{kl} - \K_{lk} + \frac{1}{C\vartheta_1^2}} \Delta^2 - \Brac[s]{s_k - s_l + \frac{1}{2C\vartheta_1^2}(\alphak - \alphal)} \Delta + c(\bm{\alpha}, \bm{\beta}).
  \end{equation*}
  Since optimal~$\delta$ is given by~\eqref{eq:PatMat dual quadratic optimal delta} and no~$\beta_j$ was updated, the optimal~$\delta$ does not change.
\end{proof}

\begin{lemma}[Update rule~\eqref{eq: update rule a,b} for problem~\eqref{eq: Pat dual quadratic}]
  Consider problem~\eqref{eq: Pat dual quadratic}, update rule~\eqref{eq: update rule a,b} and~$1 \leq k \leq \npos$ and~$\npos + 1 \leq l \leq \ntil.$ Let us define~$\hat{l} = l - \npos.$ Then the optimal solution~$\Delta^{\star}$ is given by~\eqref{eq: Delta optimal} where
  \begin{align*}
    \Delta_{lb} & = \max\{- \alphak, - \betal\}, \\
    \Delta_{ub} & = +\infty, \\
    \gamma      & = -\frac{s_k + s_l  - 1 + \frac{\alphak}{2C} - \frac{1}{\vartheta_2} + \frac{\betal}{2 \delta \vartheta_2^2}}{\K_{kk} + \K_{ll} + \K_{kl} + \K_{lk} + \frac{1}{2C} + \frac{1}{2 \delta \vartheta_2^2}}, \\
    \delta^{\star}  & = \sqrt{\delta^2 + \frac{1}{4 \vartheta_2 \ntil \tau}({\Delta^{\star}}^2 + 2 \Delta^{\star} \betal)}.
  \end{align*}
\end{lemma}

\begin{proof}
  For update rule~\eqref{eq: update rule a,b} with~$1 \leq k \leq \npos$ and~$\npos + 1 \leq l \leq n$ we define~$\hat{l} = l - \npos.$ In this case, constraints~(\ref{eq: Pat dual quadratic c2},\ref{eq: Pat dual quadratic c3}) can be written in a simple form~$\Delta \geq \max \{- \alphak, - \betal\}$ and~$\Delta$ has no upper bound. Objective~(\ref{eq: Pat dual quadratic L1}-\ref{eq: Pat dual quadratic L2})  can be rewritten as
  \begin{equation*}
    \begin{split}
      - \frac{1}{2} \Brac[s]{\K_{kk} + \K_{ll} + \K_{kl} + \K_{lk} + \frac{1}{2C\vartheta_1^2} + \frac{1}{2\delta\vartheta_2^2}} \Delta^2 \dots \qquad\qquad \\ 
      - \Brac[s]{s_k + s_l - \frac{1}{\vartheta_1} - \frac{1}{\vartheta_2} + \frac{\alphak}{2C\vartheta_1^2} + \frac{\betal}{2\delta\vartheta_2^2}} \Delta + c(\bm{\alpha}, \bm{\beta}).
    \end{split}
  \end{equation*}
  We know that the optimal~$\delta^*$ is given by~\eqref{eq:PatMat dual quadratic optimal delta}, then
  \begin{equation*}
    \delta^*
    = \sqrt{\frac{1}{4\vartheta_2^2 n \tau} \Brac{\sum_{j\neq \hat{l}} \beta_j^2 + (\betal + \Delta^\star)^2}}
    = \sqrt{\delta^2 + \frac{1}{4\vartheta_2^2 n \tau} (\Delta^{\star2} + 2\Delta^\star \betal)}.
  \end{equation*}
\end{proof}

\begin{lemma}[Update rule~\eqref{eq: update rule b,b} for problem~\eqref{eq: Pat dual quadratic}]
  Consider problem~\eqref{eq: Pat dual quadratic}, update rule~\eqref{eq: update rule b,b} and~$\npos + 1 \leq k \leq \ntil$ and~$\npos + 1 \leq l \leq \ntil.$ Let us define~$\hat{k} = k - \npos$ and~$\hat{l} = l - \npos.$ Then the optimal solution~$\Delta^{\star}$ is given by~\eqref{eq: Delta optimal} where
  \begin{align*}
    \Delta_{lb} & = - \betak, \\
    \Delta_{ub} & = \betal, \\
    \gamma      & = -\frac{s_k - s_l + \frac{1}{2\delta \vartheta_2^2}(\betak - \betal)}{\K_{kk} + \K_{ll} - \K_{kl} - \K_{lk} + \frac{1}{\delta \vartheta_2^2}}, \\
    \delta^{\star}  & = \sqrt{\delta^2 + \frac{1}{2 \vartheta_2 \ntil \tau}({\Delta^{\star}}^2 + \Delta^{\star} (\betak - \betal))}.
  \end{align*}
\end{lemma}

\begin{proof}
  For update rule~\eqref{eq: update rule b,b} with~$\npos + 1\leq k,l \leq \npos + \nneg$ we define~$\hat{k} = k - \npos,$~$\hat{l} = l - \npos.$ Since no~$\alpha_i$ was updated, constraint~\eqref{eq: Pat dual quadratic c2} is always satisfied. Constraint~\eqref{eq: Pat dual quadratic c3} can be written in a simple form~$-\betak \leq \Delta \leq \betal$ and objective~(\ref{eq: Pat dual quadratic L1}-\ref{eq: Pat dual quadratic L2})  can be rewritten as
  \begin{equation*}
    - \frac{1}{2} \Brac[s]{\K_{kk} + \K_{ll} - \K_{kl} - \K_{lk} + \frac{1}{2\delta\vartheta_2^2}} \Delta^2 - \Brac[s]{s_k - s_l + \frac{\betak - \betal}{\delta\vartheta_2^2}} \Delta + c(\bm{\alpha}, \bm{\beta}).
  \end{equation*}
  We know that the optimal~$\delta^*$ is given by~\eqref{eq:PatMat dual quadratic optimal delta}, then
  \begin{equation*}
    \delta^*
    = \sqrt{\frac{1}{4\vartheta_2^2 n \tau} \Brac{\sum_{j \notin \{\hat{l}, \hat{k}\}} \beta_j^2 + (\betak + \Delta^\star)^2 + (\betal - \Delta^\star)^2}} 
    = \sqrt{\delta + \frac{1}{2\vartheta_2^2 n \tau} (\Delta^{\star2} + \Delta^\star (\betak - \betal))}.
  \end{equation*}
\end{proof}
\chapter{Appendix for Chapter~\ref{chap: deep}}

\section{Code online}

To promote reproducibility, we share all our code online. We follow the NeurIPS instructions which allow sharing only anonymized repositories. We provide one respository with the code\footnote{\texttt{https://anonymous.4open.science/r/AccuracyAtTop-7562}} and one repository with numerical experiments.\footnote{\texttt{https://anonymous.4open.science/r/AccuracyAtTop\_DeepTopPush-834E}}

\section{Proofs}

\lemmacovergencedeep*
\begin{proof}[Proof of Lemma~\ref{lemma:convergence} on page~\pageref{lemma:convergence}]
  If~$j^*$ is unique, then the true threshold~$t$ is a differentiable function. The differentiability of~$L$ and~$\hat L$ follows from the chain rule. If~$\hat j=j^*$ holds, then the sampled gradient equals to
  \begin{equation}\label{eq:grad_min_aux}
    \nabla \hat L(\bm{w})= \frac{1}{\nmbpos}\sum_{i\in \Imbpos}l'(t-z_i)\big(\nabla_w f(\bm{w};\bm{x}_{j^*}) - \nabla_w f(\bm{w};\bm{x}_i) \big).
  \end{equation}
  The summands are identical to the ones in \eqref{eq:grad1}. Since the sum is performed with respect to positive samples, the threshold is computed from negative samples, the lemma statement follows.
\end{proof}

\thmcovergencedeep*
\begin{proof}[Proof of Theorem~\ref{theorem:convergence} on page~\pageref{theorem:convergence}]
  The law of total expectation implies
  \begin{equation*}
    \begin{aligned}
      \EE \nabla \hat L(\bm{w})
      & = \PP(\hat j=j^*)\EE(\nabla \hat L(\bm{w}) \mid \hat j=j^*) \\
      & \qquad + \PP(\hat j\neq j^*)\EE(\nabla \hat L(\bm{w}) \mid \hat j\neq j^*),
    \end{aligned}
  \end{equation*}
  from where the statement follows due to definiton \eqref{eq:defin_bias} and Lemma~\ref{lemma:convergence}.
\end{proof}


\section{Theorem~\ref{theorem:convergence} for Rec@K}

The assumption of Theorem~\ref{theorem:convergence} requires that the threshold is computing from negative samples and the objective for positive samples. This does not hold for Rec@K. We will show that we can obtain a similar result even for this case.

The proof of Theorem~\ref{theorem:convergence} is based on Lemma~\ref{lemma:convergence}. We will now obtain the variant of Lemma~\ref{lemma:convergence} for Rec@K. First, we realize that if the threshold index~$j^*$ corresponds to a negative sample, the computation will not change and therefore
\begin{equation*}
  \EE\Brac{\nabla \hat L(\bm w) \mid \hat j=j^*\text{ is an index of a negative sample}}
  =  \nabla L(\bm w).
\end{equation*}
On the other hand, when~$j^*$ corresponds to a positive sample, it needs to be always present in the minibatch selection and there are effectively only~$\nmbpos-1$ positive samples in the minibatch. Then
\begin{equation*}
  \EE\Brac{\nabla \hat L(\bm w) \mid \hat j=j^*\text{ is an index of a positive sample}}
  = \frac{\nmbpos-1}{\nmbpos}\nabla L(\bm w).
\end{equation*}
Denote now~$p$ the probability that the threshold corresponds to a positive sample. Then we have
\begin{equation*}
  \begin{aligned}
    \EE\Brac{\nabla \hat L(\bm w) \mid \hat j=j^*}
    & = (1-p)\nabla L(\bm w) + p\frac{\nmbpos-1}{\nmbpos}\nabla L(\bm w) \\
    & = \nabla L(\bm w) - \frac{p}{\nmbpos}\nabla L(\bm w).
\end{aligned}
\end{equation*}

Theorem~\ref{theorem:convergence} will then be modified into
\begin{equation*}
  \begin{aligned}
    \bias(\bm w)
    & = \PP(\hat j\neq j^*) \Brac{\nabla L(\bm w) - \EE\Brac{\nabla \hat L(\bm w) \mid \hat j\neq j^*}} \\
    & \qquad - \PP(\hat j= j^*)\frac{p}{\nmbpos}\nabla L(\bm w).
  \end{aligned}
\end{equation*}
We changed the result by adding the last term. Usually the training set contains much less positive than negative samples. This implies that~$p$ is assumed to be small and the extra term is small as well. Thefore, this change should have a negligible impact on the theorem implications.


% ------------------------------------------------------------------------------
% Bibliography
% ------------------------------------------------------------------------------
\cleardoublepage
\phantomsection
\addcontentsline{toc}{chapter}{Bibliography}
\bibliography{references.bib}
\bibliographystyle{unsrt}

\end{document}
