% ------------------------------------------------------------------------------
% Document settings
% ------------------------------------------------------------------------------
\documentclass{Thesis}
\usepackage[ddmmyyyy]{datetime}

\usepackage{xspace}
\usepackage{thmtools}
\usepackage{thm-restate}

% ------------------------------------------------------------------------------
% Math declarations
% ------------------------------------------------------------------------------
\newcommand{\brac}[2][r]{%
  \ifx r#1 (       #2 )       \else
  \ifx c#1 \{      #2 \}      \else
  \ifx s#1 [       #2 ]       \else
  \ifx v#1 \vert   #2 \vert   \else
  \ifx a#1 \langle #2 \rangle \else
  \ifx t#1 \lceil  #2 \rceil  \else
  \ifx b#1 \lfloor #2 \rfloor \else
  \ifx n#1 \|      #2 \|      \else
  \mathrm{Illegal~option}%
  \fi\fi\fi\fi\fi\fi\fi\fi
}
\newcommand{\Brac}[2][r]{%
  \ifx r#1 \left(       #2 \right)       \else
  \ifx c#1 \left\{      #2 \right\}      \else
  \ifx s#1 \left[       #2 \right]       \else
  \ifx v#1 \left\vert   #2 \right\vert   \else
  \ifx a#1 \left\langle #2 \right\rangle \else
  \ifx t#1 \left\lceil  #2 \right\rceil  \else
  \ifx b#1 \left\lfloor #2 \right\rfloor \else
  \ifx n#1 \left\|      #2 \right\|      \else
  \mathrm{Illegal~option}%
  \fi\fi\fi\fi\fi\fi\fi\fi
}

\usepackage{pifont}% http://ctan.org/pkg/pifont
\newcommand{\cmark}{{\color{mgreen} \ding{51}}}%
\newcommand{\xmark}{{\color{mred} \ding{55}}}

\newcommand{\R}{\mathbb{R}}
\newcommand{\N}{\mathbb{N}}
\newcommand{\Xc}{\mathcal{X}}

\newcommand{\norm}[1]{\Brac[n]{#1}}
\newcommand{\minimize}{\operatorname*{minimize}}
\newcommand{\maximize}{\operatorname*{maximize}}
\newcommand{\st}{\operatorname{subject\ to}}
\newcommand{\argmin}{\operatorname*{argmin}}
\newcommand{\eps}{{\varepsilon}}

% models
\newcommand{\TopPush}{\emph{TopPush}\xspace}
\newcommand{\TopPushK}{\emph{TopPushK}\xspace}
\newcommand{\tauFPL}{{\emph{$\tau$-FPL}}\xspace}
\newcommand{\TopMeanK}{\emph{TopMeanK}\xspace}
\newcommand{\PatMat}{\emph{Pat}\&\emph{Mat}\xspace}
\newcommand{\PatMatNP}{{\emph{Pat}\&\emph{Mat-NP}}\xspace}
\newcommand{\Grill}{\emph{Grill}\xspace}
\newcommand{\GrillNP}{\emph{Grill-NP}\xspace}

% counts and rates
\newcommand{\tp}{\textnormal{tp}}
\newcommand{\tn}{\textnormal{tn}}
\newcommand{\fp}{\textnormal{fp}}
\newcommand{\fn}{\textnormal{fn}}
\newcommand{\tps}{\overline{\textnormal{tp}}}
\newcommand{\tns}{\overline{\textnormal{tn}}}
\newcommand{\fps}{\overline{\textnormal{fp}}}
\newcommand{\fns}{\overline{\textnormal{fn}}}

% ------------------------------------------------------------------------------
% Tikz
% ------------------------------------------------------------------------------
\usepackage{tikz}
\usepackage{pgfplots}
\usepackage{pgfplotstable}

\usetikzlibrary{shapes,arrows,positioning,calc} 
\pgfplotsset{compat=newest}
\usepgfplotslibrary{groupplots}

% ------------------------------------------------------------------------------
% Affiliation
% ------------------------------------------------------------------------------
\title{General Framework for Classicifcation at the Top}
\subtitle{Dissertation}

\author{Ing. Václav Mácha}
\branch{Matematické inženýrství}
\academicyear{2021/2022}
\date{1. prosince 2021}
\supervisor{doc. Ing Václav Šmídl, Ph.D.}
\supervisorspec{Mgr. Lukáš Adam, Ph.D.}

\acknowledgment{Thanks thanks thanks thanks thanks thanks thanks thanks thanks thanks thanks thanks thanks thanks thanks thanks thanks thanks thanks thanks thanks thanks thanks thanks thanks thanks}

\titleCZE{Title title title title title title}
\thesistype{Disertační práce}
\abstractCZE{Abstract abstract abstract abstract abstract abstract abstract abstract abstract abstract abstract abstract abstract abstract abstract abstract abstract abstract abstract abstract abstract abstract abstract abstract abstract abstract abstract abstract abstract abstract abstract abstract abstract abstract abstract abstract abstract abstract abstract abstract abstract abstract abstract abstract abstract abstract abstract abstract abstract abstract abstract abstract abstract abstract abstract abstract abstract abstract abstract abstract abstract abstract abstract abstract abstract abstract abstract abstract abstract abstract abstract abstract abstract abstract abstract abstract}
\keywordsCZE{Keywords keywords keywords keywords keywords keywords keywords keywords keywords keywords keywords keywords keywords}

\titleENG{Title title title title title title}
\abstractENG{Abstract abstract abstract abstract abstract abstract abstract abstract abstract abstract abstract abstract abstract abstract abstract abstract abstract abstract abstract abstract abstract abstract abstract abstract abstract abstract abstract abstract abstract abstract abstract abstract abstract abstract abstract abstract abstract abstract abstract abstract abstract abstract abstract abstract abstract abstract abstract abstract abstract abstract abstract abstract abstract abstract abstract abstract abstract abstract abstract abstract abstract abstract abstract abstract abstract abstract abstract abstract abstract abstract abstract abstract abstract abstract abstract abstract}
\keywordsENG{Keywords keywords keywords keywords keywords keywords keywords keywords keywords keywords keywords keywords keywords}


% ------------------------------------------------------------------------------
% Document
% ------------------------------------------------------------------------------
\begin{document}

\maketitle

\chapter*{How to}

Text text text text text text text text text text text text text text text text text text text text text text text text text text text text text text text text text text text text text text text text text text text text text text text text text text text text text text text text text text text text text text text text text text text text text text text text text text text text text text text text text text text text text text text text text text text text text text text text text text text text text text text text text text text text text text text text text text text text text text text

\begin{theorem}
  Theorem theorem  theorem theorem theorem theorem theorem theorem theorem theorem theorem theorem theorem theorem theorem theorem theorem theorem theorem theorem theorem theorem theorem theorem theorem theorem theorem theorem theorem theorem theorem theorem theorem theorem theorem theorem theorem theorem theorem theorem
\end{theorem}

\begin{definition}
  Definition definition definition definition definition definition definition definition definition definition definition definition definition definition definition definition definition definition definition definition definition definition definition definition definition definition definition definition definition definition definition definition definition definition definition definition
\end{definition}

Text text text text text text text text text text text text text text text text text text text text text text text text text text text text text text text text text text text text text text text text text text text text text text text text text text text text text text text text text text text text text text text text text text text text text text text text text text text text text text text text text text text text text text text text text text text text text text text text text text text text text text text text text text text text text text text text text text text text text text text

\section{Ranking Problems}

Text \cite{adam2019machine,agarwal2011infinite} text text text text text text text text text text text text text text text text text text text text text text text text text text text text text text text text text text text text text text text text text text text text text text text text text text text text text text text text text text text text text text text text text text text text text text text text text text text text text text text text text text text text text text text text text text text text text text text text text text text text text text text text text text text text text text text text text text text text text text

\begin{definition}
  Definition definition definition definition definition definition definition definition definition definition definition definition definition definition definition definition definition definition definition definition definition definition definition definition definition definition definition definition definition definition definition definition definition definition definition definition
\end{definition}

\section{Accuracy At the Top}

Text text text text text text text text text text text text text text text text text text text text text text text text text text text text text text text text text text text text text text text text text text text text text text text text text text text text text text text text text text text text text text text text text text text text text text text text text text text text text text text text text text text text text text text text text text text text text text text text text text text text text text text text text text text text text text text text text text text text text text text

\section{Hypothesis Testing}

Text text text text text text text text text text text text text text text text text text text text text text text text text text text text text text text text text text text text text text text text text text text text text text text text text text text text text text text text text text text text text text text text text text text text text text text text text text text text text text text text text text text text text text text text text text text text text text text text text text text text text text text text text text text text text text text text text text text text text text text

\begin{theorem}\label{thm: theorem 1}
  Theorem theorem  theorem theorem theorem theorem theorem theorem theorem theorem theorem theorem theorem theorem theorem theorem theorem theorem theorem theorem theorem theorem theorem theorem theorem theorem theorem theorem theorem theorem theorem theorem theorem theorem theorem theorem theorem theorem theorem theorem
\end{theorem}

\begin{proof}
  Proof proof proof proof proof proof proof proof proof proof proof proof proof proof proof proof proof proof proof proof proof proof proof proof proof proof proof proof proof proof proof proof proof proof proof proof proof proof proof proof proof proof proof proof proof proof proof proof proof proof proof proof proof proof proof proof proof proof proof proof proof proof proof proof proof proof proof proof proof proof proof proof proof proof proof proof proof
\end{proof}

Text text text text text text text text text text text text text text text text text text text text text text text text text text text text text text text text text text text text text text text text text text text text text text text text text text text text text text text text text text text text text text text text text text text text text text text text text text text text text text text text text text text text text text text text text text text text text text text text text text text text text text text text text text text text text text text text text text text text text text text

\begin{proof}[Proof of theorem \ref{thm: theorem 1}]
  Proof proof proof proof proof proof proof proof proof proof proof proof proof proof proof proof proof proof proof proof proof proof proof proof proof proof proof proof proof proof proof proof proof proof proof proof proof proof proof proof proof proof proof proof proof proof proof proof proof proof proof proof proof proof proof proof proof proof proof proof proof proof proof proof proof proof proof proof proof proof proof proof proof proof proof proof proof
\end{proof}

Text text text text text text text text text text text text text text text text text text text text text text text text text text text text text text text text text text text text text text text text text text text text text text text text text text text text text text text text text text text text text text text text text text text text text text text text text text text text text text text text text text text text text text text text text text text text text text text text text text text text text text text text text text text text text text text text text text text text text text text

\begin{example}
  Example example example example example example example example example example example example example example example example example example example example example example example example example example example example example example example example example example example example example example example example example example example example example example example example example example example example example example example example example example example example example example example example example example example example example example example example example example example example example example example example example example
\end{example}

\chapter{Introduction}

Many binary classification problems focus on separating the dataset by a linear hyperplane $\wb^\top \xb - t$. A sample $\xb$ is deemed to be positive or relevant (depending on the application) if its score $\wb^\top \xb$ is above a threshold $t$. Multiple problem categories belong to this framework:s

\chapter{Linear Classification at the Top}

\chapter{Non-Linear Classification at the Top}

% ------------------------------------------------------------------------------
% Bibliography
% ------------------------------------------------------------------------------
\cleardoublepage
\phantomsection
\addcontentsline{toc}{chapter}{Bibliography}
\bibliography{references.bib}
\bibliographystyle{unsrt}

\end{document}