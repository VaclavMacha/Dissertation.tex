\documentclass[10pt, aspectratio=169]{beamer}
\usetheme[
  sectionpage=progressbar,
  numbering=fraction,
  progressbar=none,
  block=fill,
]{metropolis} % Use metropolis theme

\usepackage[ddmmyyyy]{datetime}

% ------------------------------------------------------------------------------
% Packages
% ------------------------------------------------------------------------------
% Page setting
\usepackage[explicit]{titlesec}
\usepackage{sectsty}
\usepackage{fancyhdr}

% Text options
\usepackage{lmodern}
\usepackage[T1]{fontenc}
\usepackage[utf8]{inputenc}
\usepackage{xspace}

\usepackage{amsfonts}
\usepackage{dsfont}
\usepackage{pifont}

\usepackage[color=myred!50]{todonotes}

% Graphics and colors
\usepackage{graphicx}
\usepackage{import}
\usepackage{graphics}
\usepackage{xcolor}

\definecolor{myred}{RGB}{150,0,0}  
\definecolor{mygreen}{RGB}{0,150,0}
\definecolor{myblue}{RGB}{0, 101, 189}
\definecolor{myyellow}{RGB}{220, 206, 0}
\definecolor{myorange}{RGB}{255, 153, 51}
\definecolor{mycyan}{RGB}{51, 204, 204}
\definecolor{mypurple}{RGB}{204, 0, 153}

\newcommand{\doccol}{\color{myblue}}

% Hyperrefs
\usepackage{hyperref}
\hypersetup{
  pdfusetitle,
  unicode = true,
  bookmarks = true,
  bookmarksnumbered = false,
  bookmarksopen = true,
  breaklinks = false,
  pdfborderstyle = {},
  backref = false,
  colorlinks = true,
  linkcolor = myblue,
  urlcolor = myred,
  citecolor = mygreen,
}

% enumerate and itemize
\usepackage{enumitem}

% Appendix
\usepackage[title, titletoc]{appendix}

% Captions
\usepackage{caption}
\usepackage{subcaption}

\captionsetup[figure]{position = bottom}
\captionsetup[table]{position = bottom}

% Figures

% Tables
\usepackage{booktabs}
\usepackage{nicematrix}

\renewcommand{\arraystretch}{1.5}

% Algorithms
\usepackage{algorithm}
\usepackage{algorithmicx}
\usepackage{algpseudocode}

% Math
\usepackage{amsmath}
\usepackage{amsthm}
\usepackage{amssymb}
\usepackage{mathtools}
\usepackage{nicefrac}
\usepackage{bm}
\usepackage{thmtools}
\usepackage{thm-restate}
\usepackage{optidef}

% Theorems
\usepackage[framemethod=TikZ]{mdframed}
\usepackage{xifthen}

% Tikz and pfgplots
\usepackage{tikz}
\usepackage{pgfplots}
\usepackage{pgfplotstable}

\usetikzlibrary{shapes}
\usetikzlibrary{arrows}
\usetikzlibrary{automata}
\usetikzlibrary{positioning}
\usetikzlibrary{calc}
\usetikzlibrary{intersections}

\pgfplotsset{compat=newest}
\usepgfplotslibrary{groupplots}
\usepgfplotslibrary{fillbetween}

% ------------------------------------------------------------------------------
% Math declarations
% ------------------------------------------------------------------------------
\newcommand{\Brac}[2][r]{%
  \ifx r#1 \left(       #2 \right)       \else
  \ifx c#1 \left\{      #2 \right\}      \else
  \ifx s#1 \left[       #2 \right]       \else
  \ifx v#1 \left\vert   #2 \right\vert   \else
  \ifx a#1 \left\langle #2 \right\rangle \else
  \ifx t#1 \left\lceil  #2 \right\rceil  \else
  \ifx b#1 \left\lfloor #2 \right\rfloor \else
  \ifx n#1 \left\|      #2 \right\|      \else
  \mathrm{Illegal~option}%
  \fi\fi\fi\fi\fi\fi\fi\fi
}

\newcommand{\clip}[4][s]{
  \ifx s#1 \mathrm{clip}_{\Brac[s]{#2,\; #3}}\Brac{#4} \else
  \ifx u#1 \mathrm{clip}_{\left[#2,\; #3\right)}\Brac{#4} \else
  \ifx l#1 \mathrm{clip}_{\left(#2,\; #3\right]}\Brac{#4} \else
  \mathrm{Illegal~option}%
  \fi\fi\fi
}

\newcommand{\yesmark}{\textcolor{mygreen}{\ding{51}}}%
\newcommand{\nomark}{\textcolor{myred}{\ding{55}}}
\newcommand{\good}[1]{\textcolor{mygreen}{#1}}
\newcommand{\bad}[1]{\textcolor{myred}{#1}}

\newcommand{\R}{\mathbb{R}}
\newcommand{\N}{\mathbb{N}}
\newcommand{\X}{\mathbb{X}}
\newcommand{\Xc}{\mathcal{X}}
\newcommand{\D}{\mathcal{D}}

\newcommand{\I}{\mathcal{I}}
\newcommand{\Itil}{\tilde{\mathcal{I}}}
\newcommand{\Ineg}{\I_{-}}
\newcommand{\Ipos}{\I_{+}}

\newcommand{\Imb}{\I_{\text{mb}}}
\newcommand{\Imbneg}{\I_{\text{mb},-}}
\newcommand{\Imbpos}{\I_{\text{mb},+}}

\newcommand{\nall}{n}
\newcommand{\nneg}{n_{-}}
\newcommand{\npos}{n_{+}}
\newcommand{\ntil}{\tilde{n}}

\newcommand{\nmb}{n_{\text{mb}}}
\newcommand{\nmbneg}{n_{\text{mb},-}}
\newcommand{\nmbpos}{n_{\text{mb},+}}

\newcommand{\K}{\mathbb{K}}
\newcommand{\Kall}{\K^{\pm}}
\newcommand{\Kneg}{\K^{-}}

\newcommand{\updateaa}{\Delta_{\alpha,\alpha}}
\newcommand{\updateab}{\Delta_{\alpha,\beta}}
\newcommand{\updatebb}{\Delta_{\beta,\beta}}

\newcommand{\alphak}{\alpha_{\hat{k}}}
\newcommand{\alphal}{\alpha_{\hat{l}}}
\newcommand{\betak}{\beta_{\hat{k}}}
\newcommand{\betal}{\beta_{\hat{l}}}

\newcommand{\norm}[1]{\Brac[n]{#1}}
\newcommand{\abs}[1]{|#1|}
\newcommand{\inner}[2]{\Brac[a]{#1, \; #2}}
\newcommand{\dd}[1]{\mathop{}\!\mathrm{d}#1}
\newcommand{\minimize}[1]{\ifthenelse{\isempty{#1}}{\operatorname*{minimize}\quad}{\operatorname*{minimize}_{#1}\quad}}
\newcommand{\maximize}[1]{\ifthenelse{\isempty{#1}}{\operatorname*{maximize}\quad}{\operatorname*{maximize}_{#1}\quad}}
\newcommand{\st}{\operatorname{subject\ to}}
\newcommand{\argmin}{\operatorname*{argmin}}
\newcommand{\eps}{{\varepsilon}}

\newcommand{\Imin}{I_{\rm mb}}
\newcommand{\Iminh}{I_{\rm mb}^{\rm enh}}
\newcommand{\nmin}{n_{\rm mb}}
\newcommand{\II}{\mathds{1}}
\newcommand{\Iverson}[1]{\mathds{1}_{\Brac[s]{#1}}}

\newcommand{\EE}{\mathbb{E}}
\newcommand{\PP}{\mathbb{P}}
\newcommand{\bias}{\operatorname{bias}}

\newcommand{\Matrix}[1]{\begin{pmatrix} #1 \end{pmatrix}}
\newcommand{\Set}[2]{\Brac[c]{#1 \; \middle\vert \; #2}}
\newcommand{\domain}{\operatorname*{dom}}

\newcommand{\repeatloop}{\texttt{repeat}\xspace}
\newcommand{\forloop}{\texttt{for}\xspace}

\newcommand{\vecab}{\Matrix{\bm{\alpha} \\ \bm{\beta}}}

% models
\newcommand{\AccatTop}{\emph{Accuracy at the Top}\xspace}
\newcommand{\TopPush}{\emph{TopPush}\xspace}
\newcommand{\TopPushK}{\emph{TopPushK}\xspace}
\newcommand{\tauFPL}{{\emph{$\tau$-FPL}}\xspace}
\newcommand{\TopMeanK}{\emph{TopMeanK}\xspace}
\newcommand{\PatMat}{\emph{Pat}\&\emph{Mat}\xspace}
\newcommand{\PatMatNP}{{\emph{Pat}\&\emph{Mat-NP}}\xspace}
\newcommand{\Grill}{\emph{Grill}\xspace}
\newcommand{\GrillNP}{\emph{Grill-NP}\xspace}
\newcommand{\DeepTopPush}{\emph{DeepTopPush}\xspace}
\newcommand{\TFCO}{\emph{TFCO}\xspace}
\newcommand{\APPerf}{\emph{Ap-Perf}\xspace}
\newcommand{\BaseLine}{\emph{BaseLine}\xspace}

% counts and rates
\DeclareMathOperator{\tp}{tp}
\DeclareMathOperator{\tn}{tn}
\DeclareMathOperator{\fp}{fp}
\DeclareMathOperator{\fn}{fn}
\DeclareMathOperator{\tpr}{tpr}
\DeclareMathOperator{\tnr}{tnr}
\DeclareMathOperator{\fpr}{fpr}
\DeclareMathOperator{\fnr}{fnr}

\DeclareMathOperator{\tps}{\overline{tp}}
\DeclareMathOperator{\tns}{\overline{tn}}
\DeclareMathOperator{\fps}{\overline{fp}}
\DeclareMathOperator{\fns}{\overline{fn}}

\DeclareMathOperator{\accuracy}{acc}
\DeclareMathOperator{\baccuracy}{bacc}
\DeclareMathOperator{\precision}{precision}
\DeclareMathOperator{\recall}{recall}
\DeclareMathOperator{\pratrec}{Precision@Recall}
\DeclareMathOperator{\postop}{pos@top}

\setbeamercolor{normal text}{fg=black}
\setbeamercolor{alerted text}{fg=myblue}
\setbeamercolor{example text}{fg=myblue}

\setbeamercolor{palette primary}{fg=white, bg=myblue}
\setbeamercolor{titlelike}{fg=myblue, bg=myblue}

% Affiliation
\title{General Framework for \\ Classification at the Top}
\date{\today}
\author{\textbf{Václav Mácha}}
\institute{
  Czech Technical University in Prague \\
  Faculty of Nuclear Sciences and Physical Engineering \\
  Department of Mathematics
}


\begin{document}
\maketitle

\section{Motivation}

\begin{frame}{Binary classification}
  \begin{itemize}
    \item<1-> Two group of samples:
    \begin{itemize}
      \item negative samples with label~$y = 0,$
      \item positive samples with label~$y = 1.$
    \end{itemize}
    \item<2-> Classifier usually has two parts:
    \begin{itemize}
      \item model~$f$ maps a sample~$\bm{x}$ to its classification score~$s \in \R,$
      \item decision threshold~$t \in \R$ decides whether a sample is classified as positive or not
      \begin{equation*}
        \hat{y} = \begin{cases}
          1 & \quad \text{if } s \geq t, \\
          0 & \quad \text{otherwise.}
        \end{cases}
      \end{equation*}
    \end{itemize}
    \item<2-> Usually, the model returns the probability that the sample is from the positive class and the threshold is 0.5.
    \item<3-> \textbf{Goal:} find classifier which predicts labels for unknown samples with the lowest possible error.
  \end{itemize}
\end{frame}

\begin{frame}{Confusion matrix}
  \begin{center}
  \resizebox{0.65\columnwidth}{!}{% 
  \begin{NiceTabular}{cccccc}[cell-space-limits = 7pt]
    && \Block[draw=black, line-width=2pt, rounded-corners]{1-2}{
      \textbf{Predicted label}
    } \\
    && $\hat{y} = 0$
    &  $\hat{y} = 1$
    && \Block{1-1}{\textbf{Row total:}} \\
    \Block[draw=black, line-width=2pt, rounded-corners]{2-1}{
      \rotate \textbf{Actual} \\ \textbf{label}
    }
    & $y = 0$
    & \Block[draw=mygreen, fill=mygreen!50, rounded-corners]{1-1}{
      true \\ negatives \\ (\textbf{tn})
    }
    & \Block[draw=myred, fill=myred!50, rounded-corners]{1-1}{
      false \\ positives \\ (\textbf{fp})
    }
    & $\rightarrow$
    & \Block[draw=black, rounded-corners]{1-1}{all \\ negatives \\ ($\nneg$)} \\
    & $y = 1$
    & \Block[draw=myred, fill=myred!50, rounded-corners]{1-1}{
      false \\ negatives \\ (\textbf{fn})
    }
    & \Block[draw=mygreen, fill=mygreen!50, rounded-corners]{1-1}{
      true \\ positives \\ (\textbf{tp})
    }
    & $\rightarrow$
    & \Block[draw=black, rounded-corners]{1-1}{all \\ positives \\ ($\npos$)} \\
    && $\downarrow$
    &  $\downarrow$ \\
    \Block{1-2}{\textbf{Column} \\ \textbf{total:}}
    && \Block[draw=black, rounded-corners]{1-1}{all predicted \\ negatives}
    & \Block[draw=black, rounded-corners]{1-1}{all predicted \\ positives}
  \end{NiceTabular}
  }
  \end{center}
\end{frame}

\begin{frame}{Binary Classification}
  \begin{itemize}
    \item General form of binary classification
    \begin{mini*}{\bm{w}, t}{
      \frac{1}{\nneg} \sum_{i \in \Ineg} \Iverson{s_i \geq t} + \frac{1}{\npos} \sum_{i \in \Ipos} \Iverson{s_i < t}
      }{}{}
      \addConstraint{s_i}{= f(\bm{x}_i; \bm{w}), \quad}{i \in \I,}
    \end{mini*}
    \item $\I = \Ineg \cup \Ipos$ is a set of indices of all sample where
    \begin{equation*}
      \begin{aligned}
        \Ineg & = \Set{i}{i \in \{1, 2, \ldots, n\} \; \land \; y_i = 0}, \\
        \Ipos & = \Set{i}{i \in \{1, 2, \ldots, n\} \; \land \; y_i = 1},
      \end{aligned}
    \end{equation*}
    \item $\Iverson{\cdot{}}$ is Iverson function defined by
    \begin{equation*}
      \Iverson{x} = \begin{cases}
        0 & \quad \text{if } x \text{ is false}, \\
        1 & \quad \text{if } x \text{ is true}.
      \end{cases}
    \end{equation*}
  \end{itemize}
\end{frame}

\begin{frame}{ROC curves}
  \begin{center}
    \includegraphics{../images/confusion_rates.pdf}
  \end{center}
\end{frame}

\begin{frame}{ROC curves}
  \begin{center}
    \includegraphics[width=\linewidth, height=0.9\textheight, keepaspectratio]{
      ../images/roc_space_presentation.pdf
    }
  \end{center}
\end{frame}

\begin{frame}{Classifier 1 is better ... or not?}
  \begin{itemize}
    \item<1-> If the goal is to classify all samples with the lowest possible error then \textbf{Classifier 1} is better.
    \item<2-> What if some samples are more relevant than others?
    \begin{itemize}
      \item Search engines: relevant results should be on the first few pages.
      \item Malware detection: false-alarms are disruptive to the user.
      \item Expensive post-processing: development of new drugs.
    \end{itemize}
    \item<3-> In such cases, \textbf{Classifier 2} may be better.
    \item<4-> \textbf{Classifier 1} maximizes accuracy
    \begin{equation*}
      \accuracy(\bm{s}) = \frac{1}{\nall} \sum_{i \in \I} \Iverson{y_i = \hat{y}_i}.
    \end{equation*}
    \item<4-> \textbf{Classifier 2} maximizes the number of positive samples at the top
    \begin{equation*}
      \postop(\bm{s}) = \frac{1}{\npos} \sum_{i \in \Ipos} \Iverson{s_i \geq \max_{j \in \Ineg}s_j}.
    \end{equation*}
  \end{itemize}
\end{frame}

\begin{frame}{ROC curves}
  \begin{center}
    \includegraphics[width=\linewidth, height=0.9\textheight, keepaspectratio]{
      ../images/standard_aatp_comparison.pdf
    }
  \end{center}
\end{frame}

\begin{frame}{ROC curves}
  \begin{center}
    \includegraphics[width=\linewidth, height=0.9\textheight, keepaspectratio]{
      ../images/roc_space_log.pdf
    }
  \end{center}
\end{frame}

\section{Classification at the Top}

\begin{frame}{Problem formulation}
  \begin{itemize}
    \item<1-> \textbf{Goal:} classify correctly only the most relevant samples.
    \item<2-> The most relevant samples are samples with the highest scores.
    \item<3-> General formulation
    \begin{mini*}{\bm{w}}{
      \frac{1}{\nneg} \sum_{i \in \Ineg} \Iverson{s_i \geq t} + \frac{1}{\npos} \sum_{i \in \Ipos} \Iverson{s_i < t}
    }{}{}
      \addConstraint{s_i}{= f(\bm{x}_i; \bm{w}), \quad}{i \in \I}
      \addConstraint{t}{= G\Brac{\bm{s}, \bm{y}},}
    \end{mini*}
    where threshold~$t$ is a function of all scores.
    \item<4-> Difficult problem: constrained, discontinuous, non-convex, and non-decomposable.
  \end{itemize}
\end{frame}

\begin{frame}{ROC curves}
  \begin{center}
    \includegraphics[width=\linewidth, height=0.9\textheight, keepaspectratio]{
      ../images/surrogates.pdf
    }
  \end{center}
\end{frame}

\begin{frame}{Surrogate approximation}
  \begin{itemize}
    \item General surrogate formulation
    \begin{mini*}{\bm{w}}{
      \frac{1}{2} \norm{\bm{w}}^2 + \frac{1}{\npos} \sum_{i \in \Ipos} l(t - s_i)
    }{}{}
      \addConstraint{s_i}{= f(\bm{x}_i; \bm{w}), \quad}{i \in \I}
      \addConstraint{t}{= G\Brac{\bm{s}, \bm{y}}.}
    \end{mini*}
    \item<2-> \TopPush maximizes the number of positive samples at the top 
    \begin{equation*}
      t = \max_{j \in \Ineg} s_j.
    \end{equation*}
    \item<3-> \PatMatNP maximizes true-positive rate with fixed false-positive rate 
    \begin{equation*}
      t \;\; \text{solves} \;\; \frac{1}{\nneg}\sum_{i \in \Ineg} l\Brac{\vartheta(s_i - t)} = \tau.
    \end{equation*}
  \end{itemize}
\end{frame}

\section{Classification at the Top: \\ Linear Model}

\begin{frame}
  \begin{itemize}
    \item Linear model~$f(\bm{x}; \bm{w}) = \bm{w}^{\top} \bm{x}.$
    \item General surrogate formulation with linear model
    \begin{mini*}{\bm{w}}{
      \frac{1}{2} \norm{\bm{w}}^2 + \frac{1}{\npos} \sum_{i \in \Ipos} l(t - s_i)
    }{}{}
      \addConstraint{s_i}{= \bm{w}^{\top} \bm{x}_i, \quad}{i \in \I}
      \addConstraint{t}{= G\Brac{\bm{s}, \bm{y}}.}
    \end{mini*}
    \item Properties that we are interested in:
    \begin{itemize}
      \item Convexity of the objective function.
      \item Robustness to outliers.
    \end{itemize}
  \end{itemize}
\end{frame}

\begin{frame}{Convexity of the objective function}
  \begin{block}{Theorem}
    If the threshold $t$ is a convex function of the weights $\bm{w},$ then function
    \begin{equation*}
      L(\bm{w}) = \frac{1}{2} \norm{\bm{w}}^2 + \frac{1}{\npos} \sum_{i \in \Ipos} l(t - s_i)
    \end{equation*}
    is convex.
  \end{block}
  \begin{itemize}
    \item Both formulations \TopPush and \PatMatNP have convex thresholds.
    \item Both formulations are convex and continuous.
  \end{itemize}
\end{frame}

\begin{frame}{When convexity is not enough...}
  \begin{center}
    \includegraphics[width=\linewidth, height=0.9\textheight, keepaspectratio]{
      ../images/toppush_convergence.pdf
    }    
  \end{center}
\end{frame}

\begin{frame}{How to solve it?}
  \begin{itemize}
    \item Using gradient descent
    \begin{equation*}
      \bm{w}^{k+1} \gets \bm{w}^k - \alpha^k \cdot \nabla L(\bm{w}^k),
    \end{equation*}
    where~$\alpha^k > 0$ is a learning rate, and~$\nabla L(\bm{w}^k)$ is a gradient of the objective function
    \begin{equation*}
      \nabla L(\bm{w})
        = \bm{w} + \frac{1}{\npos} \sum_{i \in \Ipos} l'\Brac{t(\bm{w}) - f(\bm{x}_i; \bm{w})}\Brac{\nabla t(\bm{w}) - \nabla f(\bm{x}_i; \bm{w})}.
    \end{equation*}
  \end{itemize}
\end{frame}

\section{Classification at the Top: \\ Non-linear Model}

\begin{frame}{???}
  \begin{itemize}
    \item General surrogate formulation with non-linear model
    \begin{mini*}{\bm{w}}{
      \frac{1}{2} \norm{\bm{w}}^2 + \frac{1}{\npos} \sum_{i \in \Ipos} l(t - s_i)
    }{}{}
      \addConstraint{s_i}{= \bm{w}^{\top} \bm{x}_i, \quad}{i \in \I}
      \addConstraint{t}{= G\Brac{\bm{s}, \bm{y}}.}
    \end{mini*}
    \item<2-> Disadvantages:
    \begin{itemize}
      \item Objective function is not convex.
      \item Non-linear models are usually large and expensive to train.
    \end{itemize}
    \item<3-> What to do if the dataset is too large to fit in memory?
    \item<4-> Stochastic gradient descent: the gradient is computed only on a small subset of all data called minibatch.
  \end{itemize}
\end{frame}

\begin{frame}{Issues when passing to mini-batches}
  \begin{itemize}
    \item Problems:
    \begin{itemize}
      \item The threshold is a function of all scores $\rightarrow$ the loss function is non-decomposable.
      \item As a result, stochastic gradient descent provides a biased gradient estimate.
    \end{itemize}
  \end{itemize}
  \begin{center}
    \includegraphics[width=\linewidth, height=0.6\textheight, keepaspectratio]{
      ../images/deep_threshold_bias.pdf
    }
  \end{center}
\end{frame}

\begin{frame}{How to reduce bias?}
  \begin{itemize}
    \item Increase size of minibatch $\rightarrow$ \PatMatNP.
    \item Add threshold from last minibatch $\rightarrow$ \DeepTopPush.
  \end{itemize}
  \begin{center}
    \includegraphics[width=\linewidth, height=0.75\textheight, keepaspectratio]{
      ../images/deep_thresholds.pdf
    }
  \end{center}
\end{frame}

\section{How does it work?}

\begin{frame}{Steganography}
  \begin{itemize}
    \item Dataset: 205 579 samples, 9\% of samples are positive.
    \item Each sample consists of 22 510 features.
    \item Only linear model.
    \item \textbf{Goals:}
    \begin{itemize}
      \item Maximize the true-positive rate at extremely low levels of the false-positive rate.
    \end{itemize}
  \end{itemize}
\end{frame}

\begin{frame}{Steganography}
  \begin{center}
    \includegraphics[width=\linewidth, height=0.9\textheight, keepaspectratio]{
      ../images/stego_nsft5.pdf
    }
  \end{center}
\end{frame}

\begin{frame}{AVAST: Malware detection}
  \begin{itemize}
    \item Dataset: 6 580 166 samples, 87\% of samples are positive.
    \item Hierarchical data structure:
    \begin{itemize}
      \item Each sample is a JSON file, which may consist of other JSON files.
      \item Each sample is of a different size (from 1 KB to 2.5 MB).
      \item \DeepTopPush and \PatMatNP used as an extension for hierarchical multi-instance learning (HMIL).
    \end{itemize}
    \item \textbf{Goals:}
    \begin{itemize}
      \item Maximize the true-positive rate at extremely low levels of the false-positive rate.
      \item The false-positive rate must be as low as possible to avoid disruptive false alarms for the end-user.
    \end{itemize}
  \end{itemize}
\end{frame}

\begin{frame}{AVAST: Malware detection}
  \begin{center}
    \includegraphics[width=\linewidth, height=0.9\textheight, keepaspectratio]{
      ../images/malware_detection.pdf
    }
  \end{center}
\end{frame}

\begin{frame}
  \vfill \par
  \begin{center}
    \textbf{\Large{Thank you for your attention.}}
  \end{center}\par
  \vfill
\end{frame}

\end{document}
